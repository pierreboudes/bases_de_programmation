\begin{tabular}[c]{|c|c|c|c|c|c|c|c|}
\hline
Cycles & CP & instruction & r0& r1& 2& 3& 14\\ \hline
INIT & 1 & & ? & ? & 10
 & 15
 & ?
 \\ \hline1 &\textbf{4} & \commentaire{Saut a l'adresse 4
} saut 4
 & & & & & \\ \hline
2 & 5 & \commentaire{Lecture de la donnée d'adresse 2 dans le registre 0
} lecture 2 r0
 & 10 & & & & \\ \hline
3 & 6 & \commentaire{Lecture de la donnée d'adresse 3 dans le registre 1
} lecture 3 r1
 & & 15 & & & \\ \hline
4 & 7 & \commentaire{Inversion du signe de la valeur du registre 0
} inverse r0
 & -10 & & & & \\ \hline
5 & 8 & \commentaire{Ajout de la valeur du registre 0 au registre 1
} add r0 r1
 & & 5 & & & \\ \hline
6 &\textbf{11} & \commentaire{Si la valeur (5) du registre 1 est positive, saute a l'adresse 11
} sautpos r1 11
 & & & & & \\ \hline
7 & 12 & \commentaire{Lecture de la donnée d'adresse 3 dans le registre 0
} lecture 3 r0
 & 15 & & & & \\ \hline
8 & 13 & \commentaire{Écriture du registre 0 à l'adresse 14
} ecriture r0 14
 & & & & & 15
 \\ \hline
9 & 14 & \commentaire{Fin du processus.
} stop
 & & & & & \\ \hline
\end{tabular}