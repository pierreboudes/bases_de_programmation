\begin{tabular}[c]{|c|c|c|c|c|c|c|c|c|c|c|c|c|c|c|c|c|c|c|}
\hline
Cycles & CP & instruction & r0& r1& r2& r3& r4& 19& 21& 22& 23& 24& 25& 26& 27& 28& 29& 31\\ \hline
INIT & 1 & & ? & ? & ? & ? & ? & 1
 & 23
 & 29
 & 34
 & -23
 & -40
 & 45
 & 1
 & 0
 & 19
 & ?
 \\ \hline1 & 2 & \commentaire{Lecture de la donnée d'adresse 21 dans le registre 0
} lecture 21 r0
 & 23 & & & & & & & & & & & & & & & \\ \hline
2 & 3 & \commentaire{Lecture de la donnée d'adresse 23 dans le registre 3
} lecture *r0 r3
 & & & & 34 & & & & & & & & & & & & \\ \hline
3 & 4 & \commentaire{Lecture de la donnée d'adresse 19 dans le registre 2
} lecture 19 r2
 & & & 1 & & & & & & & & & & & & & \\ \hline
4 & 5 & \commentaire{Lecture de la donnée d'adresse 22 dans le registre 1
} lecture 22 r1
 & & 29 & & & & & & & & & & & & & & \\ \hline
5 & 6 & \commentaire{Ajout de la valeur du registre 2 au registre 0
} add r2 r0
 & 24 & & & & & & & & & & & & & & & \\ \hline
6 & 7 & \commentaire{Ajout de la valeur du registre 2 au registre 1
} add r2 r1
 & & 30 & & & & & & & & & & & & & & \\ \hline
7 & 8 & \commentaire{Inversion du signe de la valeur du registre 1
} inverse r1
 & & -30 & & & & & & & & & & & & & & \\ \hline
8 & 9 & \commentaire{Ajout de la valeur du registre 0 au registre 1
} add r0 r1
 & & -6 & & & & & & & & & & & & & & \\ \hline
9 & 10 & \commentaire{Si la valeur (-6) du registre 1 est positive, saute a l'adresse 17
} sautpos r1 17
 & & & & & & & & & & & & & & & & \\ \hline
10 & 11 & \commentaire{Pas d'operation
} noop
 & & & & & & & & & & & & & & & & \\ \hline
11 & 12 & \commentaire{Lecture de la donnée d'adresse 24 dans le registre 4
} lecture *r0 r4
 & & & & & -23 & & & & & & & & & & & \\ \hline
12 & 13 & \commentaire{Inversion du signe de la valeur du registre 4
} inverse r4
 & & & & & 23 & & & & & & & & & & & \\ \hline
13 & 14 & \commentaire{Ajout de la valeur du registre 3 au registre 4
} add r3 r4
 & & & & & 57 & & & & & & & & & & & \\ \hline
14 &\textbf{4} & \commentaire{Si la valeur (57) du registre 4 est positive, saute a l'adresse 4
} sautpos r4 4
 & & & & & & & & & & & & & & & & \\ \hline
15 & 5 & \commentaire{Lecture de la donnée d'adresse 22 dans le registre 1
} lecture 22 r1
 & & 29 & & & & & & & & & & & & & & \\ \hline
16 & 6 & \commentaire{Ajout de la valeur du registre 2 au registre 0
} add r2 r0
 & 25 & & & & & & & & & & & & & & & \\ \hline
17 & 7 & \commentaire{Ajout de la valeur du registre 2 au registre 1
} add r2 r1
 & & 30 & & & & & & & & & & & & & & \\ \hline
18 & 8 & \commentaire{Inversion du signe de la valeur du registre 1
} inverse r1
 & & -30 & & & & & & & & & & & & & & \\ \hline
19 & 9 & \commentaire{Ajout de la valeur du registre 0 au registre 1
} add r0 r1
 & & -5 & & & & & & & & & & & & & & \\ \hline
20 & 10 & \commentaire{Si la valeur (-5) du registre 1 est positive, saute a l'adresse 17
} sautpos r1 17
 & & & & & & & & & & & & & & & & \\ \hline
21 & 11 & \commentaire{Pas d'operation
} noop
 & & & & & & & & & & & & & & & & \\ \hline
22 & 12 & \commentaire{Lecture de la donnée d'adresse 25 dans le registre 4
} lecture *r0 r4
 & & & & & -40 & & & & & & & & & & & \\ \hline
23 & 13 & \commentaire{Inversion du signe de la valeur du registre 4
} inverse r4
 & & & & & 40 & & & & & & & & & & & \\ \hline
24 & 14 & \commentaire{Ajout de la valeur du registre 3 au registre 4
} add r3 r4
 & & & & & 74 & & & & & & & & & & & \\ \hline
25 &\textbf{4} & \commentaire{Si la valeur (74) du registre 4 est positive, saute a l'adresse 4
} sautpos r4 4
 & & & & & & & & & & & & & & & & \\ \hline
26 & 5 & \commentaire{Lecture de la donnée d'adresse 22 dans le registre 1
} lecture 22 r1
 & & 29 & & & & & & & & & & & & & & \\ \hline
27 & 6 & \commentaire{Ajout de la valeur du registre 2 au registre 0
} add r2 r0
 & 26 & & & & & & & & & & & & & & & \\ \hline
28 & 7 & \commentaire{Ajout de la valeur du registre 2 au registre 1
} add r2 r1
 & & 30 & & & & & & & & & & & & & & \\ \hline
29 & 8 & \commentaire{Inversion du signe de la valeur du registre 1
} inverse r1
 & & -30 & & & & & & & & & & & & & & \\ \hline
30 & 9 & \commentaire{Ajout de la valeur du registre 0 au registre 1
} add r0 r1
 & & -4 & & & & & & & & & & & & & & \\ \hline
31 & 10 & \commentaire{Si la valeur (-4) du registre 1 est positive, saute a l'adresse 17
} sautpos r1 17
 & & & & & & & & & & & & & & & & \\ \hline
32 & 11 & \commentaire{Pas d'operation
} noop
 & & & & & & & & & & & & & & & & \\ \hline
33 & 12 & \commentaire{Lecture de la donnée d'adresse 26 dans le registre 4
} lecture *r0 r4
 & & & & & 45 & & & & & & & & & & & \\ \hline
34 & 13 & \commentaire{Inversion du signe de la valeur du registre 4
} inverse r4
 & & & & & -45 & & & & & & & & & & & \\ \hline
35 & 14 & \commentaire{Ajout de la valeur du registre 3 au registre 4
} add r3 r4
 & & & & & -11 & & & & & & & & & & & \\ \hline
36 & 15 & \commentaire{Si la valeur (-11) du registre 4 est positive, saute a l'adresse 4
} sautpos r4 4
 & & & & & & & & & & & & & & & & \\ \hline
37 & 16 & \commentaire{Lecture de la donnée d'adresse 26 dans le registre 3
} lecture *r0 r3
 & & & & 45 & & & & & & & & & & & & \\ \hline
38 &\textbf{4} & \commentaire{Saut a l'adresse 4
} saut 4
 & & & & & & & & & & & & & & & & \\ \hline
39 & 5 & \commentaire{Lecture de la donnée d'adresse 22 dans le registre 1
} lecture 22 r1
 & & 29 & & & & & & & & & & & & & & \\ \hline
40 & 6 & \commentaire{Ajout de la valeur du registre 2 au registre 0
} add r2 r0
 & 27 & & & & & & & & & & & & & & & \\ \hline
41 & 7 & \commentaire{Ajout de la valeur du registre 2 au registre 1
} add r2 r1
 & & 30 & & & & & & & & & & & & & & \\ \hline
42 & 8 & \commentaire{Inversion du signe de la valeur du registre 1
} inverse r1
 & & -30 & & & & & & & & & & & & & & \\ \hline
43 & 9 & \commentaire{Ajout de la valeur du registre 0 au registre 1
} add r0 r1
 & & -3 & & & & & & & & & & & & & & \\ \hline
44 & 10 & \commentaire{Si la valeur (-3) du registre 1 est positive, saute a l'adresse 17
} sautpos r1 17
 & & & & & & & & & & & & & & & & \\ \hline
45 & 11 & \commentaire{Pas d'operation
} noop
 & & & & & & & & & & & & & & & & \\ \hline
46 & 12 & \commentaire{Lecture de la donnée d'adresse 27 dans le registre 4
} lecture *r0 r4
 & & & & & 1 & & & & & & & & & & & \\ \hline
47 & 13 & \commentaire{Inversion du signe de la valeur du registre 4
} inverse r4
 & & & & & -1 & & & & & & & & & & & \\ \hline
48 & 14 & \commentaire{Ajout de la valeur du registre 3 au registre 4
} add r3 r4
 & & & & & 44 & & & & & & & & & & & \\ \hline
49 &\textbf{4} & \commentaire{Si la valeur (44) du registre 4 est positive, saute a l'adresse 4
} sautpos r4 4
 & & & & & & & & & & & & & & & & \\ \hline
50 & 5 & \commentaire{Lecture de la donnée d'adresse 22 dans le registre 1
} lecture 22 r1
 & & 29 & & & & & & & & & & & & & & \\ \hline
51 & 6 & \commentaire{Ajout de la valeur du registre 2 au registre 0
} add r2 r0
 & 28 & & & & & & & & & & & & & & & \\ \hline
52 & 7 & \commentaire{Ajout de la valeur du registre 2 au registre 1
} add r2 r1
 & & 30 & & & & & & & & & & & & & & \\ \hline
53 & 8 & \commentaire{Inversion du signe de la valeur du registre 1
} inverse r1
 & & -30 & & & & & & & & & & & & & & \\ \hline
54 & 9 & \commentaire{Ajout de la valeur du registre 0 au registre 1
} add r0 r1
 & & -2 & & & & & & & & & & & & & & \\ \hline
55 & 10 & \commentaire{Si la valeur (-2) du registre 1 est positive, saute a l'adresse 17
} sautpos r1 17
 & & & & & & & & & & & & & & & & \\ \hline
56 & 11 & \commentaire{Pas d'operation
} noop
 & & & & & & & & & & & & & & & & \\ \hline
57 & 12 & \commentaire{Lecture de la donnée d'adresse 28 dans le registre 4
} lecture *r0 r4
 & & & & & 0 & & & & & & & & & & & \\ \hline
58 & 13 & \commentaire{Inversion du signe de la valeur du registre 4
} inverse r4
 & & & & & 0 & & & & & & & & & & & \\ \hline
59 & 14 & \commentaire{Ajout de la valeur du registre 3 au registre 4
} add r3 r4
 & & & & & 45 & & & & & & & & & & & \\ \hline
60 &\textbf{4} & \commentaire{Si la valeur (45) du registre 4 est positive, saute a l'adresse 4
} sautpos r4 4
 & & & & & & & & & & & & & & & & \\ \hline
61 & 5 & \commentaire{Lecture de la donnée d'adresse 22 dans le registre 1
} lecture 22 r1
 & & 29 & & & & & & & & & & & & & & \\ \hline
62 & 6 & \commentaire{Ajout de la valeur du registre 2 au registre 0
} add r2 r0
 & 29 & & & & & & & & & & & & & & & \\ \hline
63 & 7 & \commentaire{Ajout de la valeur du registre 2 au registre 1
} add r2 r1
 & & 30 & & & & & & & & & & & & & & \\ \hline
64 & 8 & \commentaire{Inversion du signe de la valeur du registre 1
} inverse r1
 & & -30 & & & & & & & & & & & & & & \\ \hline
65 & 9 & \commentaire{Ajout de la valeur du registre 0 au registre 1
} add r0 r1
 & & -1 & & & & & & & & & & & & & & \\ \hline
66 & 10 & \commentaire{Si la valeur (-1) du registre 1 est positive, saute a l'adresse 17
} sautpos r1 17
 & & & & & & & & & & & & & & & & \\ \hline
67 & 11 & \commentaire{Pas d'operation
} noop
 & & & & & & & & & & & & & & & & \\ \hline
68 & 12 & \commentaire{Lecture de la donnée d'adresse 29 dans le registre 4
} lecture *r0 r4
 & & & & & 19 & & & & & & & & & & & \\ \hline
69 & 13 & \commentaire{Inversion du signe de la valeur du registre 4
} inverse r4
 & & & & & -19 & & & & & & & & & & & \\ \hline
70 & 14 & \commentaire{Ajout de la valeur du registre 3 au registre 4
} add r3 r4
 & & & & & 26 & & & & & & & & & & & \\ \hline
71 &\textbf{4} & \commentaire{Si la valeur (26) du registre 4 est positive, saute a l'adresse 4
} sautpos r4 4
 & & & & & & & & & & & & & & & & \\ \hline
72 & 5 & \commentaire{Lecture de la donnée d'adresse 22 dans le registre 1
} lecture 22 r1
 & & 29 & & & & & & & & & & & & & & \\ \hline
73 & 6 & \commentaire{Ajout de la valeur du registre 2 au registre 0
} add r2 r0
 & 30 & & & & & & & & & & & & & & & \\ \hline
74 & 7 & \commentaire{Ajout de la valeur du registre 2 au registre 1
} add r2 r1
 & & 30 & & & & & & & & & & & & & & \\ \hline
75 & 8 & \commentaire{Inversion du signe de la valeur du registre 1
} inverse r1
 & & -30 & & & & & & & & & & & & & & \\ \hline
76 & 9 & \commentaire{Ajout de la valeur du registre 0 au registre 1
} add r0 r1
 & & 0 & & & & & & & & & & & & & & \\ \hline
77 &\textbf{17} & \commentaire{Si la valeur (0) du registre 1 est positive, saute a l'adresse 17
} sautpos r1 17
 & & & & & & & & & & & & & & & & \\ \hline
78 & 18 & \commentaire{Écriture du registre 3 à l'adresse 31
} ecriture r3 31
 & & & & & & & & & & & & & & & & 45
 \\ \hline
79 & 19 & \commentaire{Fin du processus.
} stop
 & & & & & & & & & & & & & & & & \\ \hline
\end{tabular}