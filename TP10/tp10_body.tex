% -*- coding: utf-8 -*-

% \renewcommand{\labelitemi}{$\bullet$}

% \newcommand{\commentaire}[1]{}

\entete{Travaux pratiques 10 : un struct craquant}


\begin{correction}
  Note aux chagé de TP : j'utiliserai ce jeu pour évoquer la notion
  d'invariant de boucle pour le dernier cours, la stratégie gagnante
  (toujours rendre la tablette carrée) sera explicitée à ce moment là
  (ne pas trop les aider à la trouver !).

  Il est important que les étudiants apprennent à manipuler les
  struct. On déploiera donc des trésors de pédagogie pour les forcer à
  représenter une tablette par un struct plutôt que par deux
  d'entiers (les étudiants qui ne sont pas sensibles aux tablettes aux
  noisettes, au chocolat noir ou au lait, devront donc être convaincus
  par d'autres moyens : essayer l'allégé, le bio, le culturisme).
\end{correction}
% L'objectif de ce TP est de terminer votre menu, et d'y ajouter
% quelques fonctionnalités en utilisant des fonctions que vous aurez à
% déclarer et/ou à définir.

%\vspace{-1cm}
\section{Le jeu de la tablette de chocolat}

Une tablette (ou plaquette) de chocolat est une matrice de carreaux de
chocolat que l'ont peut couper selon les lignes ou les colonnes qui
séparent les carreaux. La hauteur et la largeur se comptent en nombre
de carreaux. On a peint en vert un carreau dans le coin d'une
tablette. Deux joueurs coupent tour à tour la tablette, au choix,
selon une ligne ou une colonne (pas nécessairement près du
bord). L'objectif est de ne pas être le joueur qui se retrouve avec le
carreau vert (la tablette de hauteur 1 et de largeur 1).

Nous allons programmer ce jeu en faisant jouer à l'ordinateur le rôle
de l'opposant.

\subsection{Type utilisateur}

Définir un type utilisateur qui servira à représenter la tablette
(faire simple !).

\begin{correction}
\begin{verbatim}
struct tablette_s 
{
  int l; /* largeur (nombre de colonnes) */
  int c; /* hauteur (nombre de lignes) */
}
\end{verbatim}
\end{correction}

\subsection{Le tour de jeu et l'arbitre}
Le joueur commencera la partie, puis l'opposant (l'ordinateur) jouera
et ainsi de suite. À chaque fois l'une des deux dimensions de la tablette sera
modifiée, jusqu'à ce qu'il y ait un perdant. Vous adopterez la
structure suivante pour le programme :
\begin{itemize}
\item Une variable sera utilisée pour déterminer à qui vient le tour de
  jouer (joueur ou opposant);
\item Une boucle réalisera le tour de jeu;
\item Une fonction \C{partie\_perdue()} prenant en argument la tablette
  et renvoyant vrai lorsque celle-ci se réduit au carré vert, servira
  à construire la condition de sortie de la boucle;
\item Après quoi un affichage annoncera au joueur s'il a gagné ou
  perdu.
\item À l'intérieur de la boucle, selon le tour, il sera fait appel à
  une fonction \C{joueur()} ou bien à une fonction \C{opposant()} (ces
  fonctions doivent modifier la tablette).
\end{itemize}

Vous êtes libre du choix de la taille initiale de la tablette.

\subsection{Affichage}

Réaliser une fonction d'affichage de la tablette qui sera appelée au
début de chaque tour de jeu. Vous pouvez dans un premier temps vous
contenter d'un affichage numérique de ses dimensions, et par la suite
améliorer votre programme avec un affichage plus graphique.

\subsection{Les joueurs}

Le joueur étant l'humain entre le clavier et l'écran, il faudra lui
faire saisir le choix de son coup, en deux temps : couper des colonnes
ou couper des lignes ? Combien ? À vous de déterminer la manière de
réaliser cette saisie.

Vous pouvez commencer à tester le fonctionnement de votre programme en
utilisant la fonction \C{joueur()} également pour l'opposant.

L'opposant jouera au hasard. Pour cela vous pouvez vous aider d'une
fonction \C{nombre\_aleatoire} qui tire au hasard un nombre entre
zéro et son argument (un entier positif). Cette fonction utilisera
\C{rand}. N'oubliez pas d'initialiser
le générateur de nombres aléatoires dans le \C{main()} ni d'inclure
les bons fichiers d'en-tête (fonctionnalités supplémentaires).

Tester votre programme. Avez-vous prévu les cas où l'un des deux
joueurs n'a pas le choix ? Est-ce que les coups invalides sont rendus
impossibles ? Vous pouvez réutiliser et améliorer des fonctions vues
précédemment (\C{choix\_utilisateur()}, par exemple).

\section{Améliorations}

\subsection{Tablette aléatoire}
Dans votre programme la tablette est initialisée au
départ. Pouvez-vous utiliser une fonction sans argument,
\C{tablette\_aleatoire} qui renvoie une tablette aléatoire ?

\subsection{L'opposant parfait}
Pouvez-vous trouver la meilleure manière de jouer (cela dépend de la
tablette de départ) ? Si oui, faîtes en profiter votre opposant. Vous
pouvez également le faire se tromper de temps en temps, en introduisant
pour cela une dose de hasard...

\subsection{Plusieurs tablettes}
On peut utiliser plusieurs tablettes pour ce jeu : chaque joueur
choisit sur quelle tablette jouer son tour et le perdant sera celui
recevant pour son tour toutes les tablettes à la dimension $1\times
1$.

\subsection{Intégration au menu}
Comme pour \emph{deviner un nombre}, nous voulons maintenant que \emph{croque
tablette} soit un jeu disponible dans notre menu.

\begin{correction}
{
\footnotesize
\listinginput{1}{chocolat.c}
}
\end{correction}


%%% Local Variables: 
%%% mode: latex
%%% TeX-master: "tp10"
%%% End: 
