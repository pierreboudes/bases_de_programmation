% -*- coding: utf-8 -*-
\entete{Travaux Pratiques 1 : Programmation en mini-assembleur}
%\vspace{-.8cm}

Il est demandé, pendant ce TP, d'écrire des programmes simples en
mini-assembleur et de les exécuter sur un processeur simulé en \C{amil} (\emph{assembleur miniature pour l'informatique de
  licence}).  Le simulateur peut être utilisé en ligne ici :\\ 
\url{http://www-lipn.univ-paris13.fr/~boudes/amilweb/}

\section*{Prise en main : le terminal}

%\subsection{Le terminal}

Faîtes vous aider par votre chargé de TP pour ajouter un lanceur de
terminal dans votre barre d'outils ou votre barre de menu, ainsi qu'un
lanceur pour l'éditeur de texte (\C{gedit}).

Dans un terminal, taper les lignes de commandes suivantes :

\noindent
\begin{tabular}[c]{lp{10cm}}
% \verb|~boudes/pub/EI/init.sh| & Exécuter un script qui modifiera
% légèrement votre environnement de travail pour l'adapter au TP.\\
% \multicolumn{2}{c}{\emph{Fermer le terminal et le rouvrir pour que les changements soient
% pris en compte.}}\\
\verb|mkdir TP1| & Créer un répertoire TP1.\\
\verb|cd TP1| & Entrer dans le répertoire TP1.\\
% \verb|amil &| & Lancer \C{amil}, le simulateur de mini-assembleur en
% \emph{tâche de fond} (en tâche de fond le terminal ne restera pas
% bloqué jusqu'à l'arrêt du simulateur).
\multicolumn{2}{p{0.95\linewidth}}{\emph{Dans la suite, vous gagnerez à taper vos programme dans l'éditeur de texte, les
charger dans le simulateur par copier/coller, et à \textbf{les
enregistrer} dans le répertoire TP1 (sous des noms comme
\C{exercice1.txt}, \C{exercice2\_1.txt}, etc.).}}\\
\verb|gedit exercice1.txt &| & Lance l'édition d'un fichier \C{exercice1.txt}, en \emph{tâche de fond}. 
\end{tabular}


\paragraph{Astuces du terminal.}
la touche de tabulation vous permet de compléter votre saisie quand
vous tapez une commande dans le terminal. La touche flèche vers le
haut rappelle une ligne de commande tapée précédemment. 

\section{Initialisation de la mémoire}

Soit la case mémoire, $x$, d'adresse 10 et la case mémoire, $y$,
d'adresse 11. Écrire et exécuter le programme qui initialise $x$ à $7
\times 2$ et $y$ à $x - 1$. 

\begin{correction}
\begin{listing}{1}
valeur 7 r0
mult 2 r0   
ecriture r0 10
valeur -1 r1
add r1 r0
ecriture r0 11
stop
\end{listing}
\end{correction}

\section{Exécution conditionnelle d'instructions}

À l'aide de l'instruction \verb|sautpos|, écrire les programmes
correspondant aux algorithmes suivants et les exécuter avec amil sur
un exemple, afin de tester leur correction :
\begin{enumerate}
\item Soient la valeur $a$ à l'adresse 20, $b$ à l'adresse $21$. Si $a
  < b$ alors écrire $a$ à l'adresse $22$ sinon écrire $b$ à l'adresse
  $22$.
\begin{correction}

\begin{itemize}
\item Calcul de $a - b$ :
\begin{listing}{1}
lecture 20 r0
lecture 21 r1
soustr r1 r0     # r0 vaut a - b
\end{listing}
\item Si $a < b$ (c'est à dire si $a - b < 0$)
\begin{listing}{4}
sautpos r0 8
\end{listing}
\item Alors écrire $a$ à l'adresse $22$
\begin{listing}{5}
lecture 20 r2   
ecriture r2 22
saut 10
\end{listing}
\item Sinon écrire $b$ à l'adresse $22$
\begin{listing}{8}
lecture 21 r2     
ecriture r2 22
stop
\end{listing}
\item Données :
\begin{listing}{20}
3  # a
2  # b
?  # résultat
\end{listing}
\end{itemize}

\end{correction}
\item Soient trois cases mémoires contenant trois entiers. Calculer et écrire le minimum de ces trois entiers en mémoire.
\end{enumerate}

\begin{correction}
  
\begin{itemize}
\item Soient $a$, $b$ et $c$ trois entiers
\begin{listing}{20}
12 # a    
23 # b
6  # c
?  # min
\end{listing}
\item $\text{min}$ est initialise à $a$ (par défaut)
\begin{listing}{1}
lecture 20 r0
ecriture r0 23
\end{listing}
\item Si $b < \text{min}$ alors $\text{min}$ vaut $b$
\begin{listing}{3}
lecture 21 r0
lecture 23 r1
inverse r1
add r0 r1    # r1 vaut b - min
sautpos r1 9 
ecriture r0 23
\end{listing}
\item Si $c < \text{min}$ alors $\text{min}$ vaut $c$
\begin{listing}{9}
lecture 22 r0
lecture 23 r1
inverse r1
add r0 r1 # r1 vaut c - min
sautpos r1 15
ecriture r0 23
\end{listing}
\item $\text{min}$ contient le minimum de $a$, $b$ et $c$ 
\begin{listing}{15}
stop
\end{listing}
\end{itemize}
\end{correction}

\section{Boucles d'instructions}

\begin{enumerate}
\item Avec l'instruction \verb|saut|, écrire un programme qui ne
  termine jamais.
\begin{correction}
\begin{listing}{1}
saut 1
stop # <- jamais atteint
\end{listing}
\end{correction}

\item Avec l'instruction \verb|sautpos|, écrire un programme qui ne termine jamais.
\begin{correction}
\begin{listing}{1}
valeur 0 r0
sautpos r0 2
stop # <- jamais atteint
\end{listing}    
\end{correction}
\end{enumerate}

On dit de ces programmes qu'ils bouclent à l'infini.

%%% Local Variables: 
%%% mode: latex
%%% TeX-master: "tp1"
%%% End: 
