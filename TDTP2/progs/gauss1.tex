\begin{tabular}[c]{l||c|c|c|c|c|c|c|c|}
\hline
 \emph{Instructions} & Cycles & CP& r0& r1& r2& r3& 15& 16\\ \hline
\hfill Initialisation & 0 & 1 & ? & ? & ? & ? & 3
 & ?
 \\ \hline \commentaire{Initialisation du registre 1 à 0
} \C{valeur 0 r1
} & 1 & 2  & & 0 & & & &\\ \hline
 \commentaire{Initialisation du registre 2 à 0
} \C{valeur 0 r2
} & 2 & 3  & & & 0 & & &\\ \hline
 \commentaire{Saut à l'adresse 7
} \C{saut 7
} & 3 & \textbf{7} & & & & & &\\ \hline
 \commentaire{Lecture de la donnée d'adresse 15 dans le registre 0
} \C{lecture 15 r0
} & 4 & 8  & 3 & & & & &\\ \hline
 \commentaire{Soustrait la valeur du registre 2 au registre 0
} \C{soustr r2 r0
} & 5 & 9  & 3 & & & & &\\ \hline
 \commentaire{Si la valeur (3) du registre 0 est positive, saute à l'adresse 4
} \C{sautpos r0 4
} & 6 & \textbf{4} & & & & & &\\ \hline
 \commentaire{Ajout de la valeur du registre 2 au registre 1
} \C{add r2 r1
} & 7 & 5  & & 0 & & & &\\ \hline
 \commentaire{Initialisation du registre 3 à 1
} \C{valeur 1 r3
} & 8 & 6  & & & & 1 & &\\ \hline
 \commentaire{Ajout de la valeur du registre 3 au registre 2
} \C{add r3 r2
} & 9 & 7  & & & 1 & & &\\ \hline
 \commentaire{Lecture de la donnée d'adresse 15 dans le registre 0
} \C{lecture 15 r0
} & 10 & 8  & 3 & & & & &\\ \hline
 \commentaire{Soustrait la valeur du registre 2 au registre 0
} \C{soustr r2 r0
} & 11 & 9  & 2 & & & & &\\ \hline
 \commentaire{Si la valeur (2) du registre 0 est positive, saute à l'adresse 4
} \C{sautpos r0 4
} & 12 & \textbf{4} & & & & & &\\ \hline
 \commentaire{Ajout de la valeur du registre 2 au registre 1
} \C{add r2 r1
} & 13 & 5  & & 1 & & & &\\ \hline
 \commentaire{Initialisation du registre 3 à 1
} \C{valeur 1 r3
} & 14 & 6  & & & & 1 & &\\ \hline
 \commentaire{Ajout de la valeur du registre 3 au registre 2
} \C{add r3 r2
} & 15 & 7  & & & 2 & & &\\ \hline
 \commentaire{Lecture de la donnée d'adresse 15 dans le registre 0
} \C{lecture 15 r0
} & 16 & 8  & 3 & & & & &\\ \hline
 \commentaire{Soustrait la valeur du registre 2 au registre 0
} \C{soustr r2 r0
} & 17 & 9  & 1 & & & & &\\ \hline
 \commentaire{Si la valeur (1) du registre 0 est positive, saute à l'adresse 4
} \C{sautpos r0 4
} & 18 & \textbf{4} & & & & & &\\ \hline
 \commentaire{Ajout de la valeur du registre 2 au registre 1
} \C{add r2 r1
} & 19 & 5  & & 3 & & & &\\ \hline
 \commentaire{Initialisation du registre 3 à 1
} \C{valeur 1 r3
} & 20 & 6  & & & & 1 & &\\ \hline
 \commentaire{Ajout de la valeur du registre 3 au registre 2
} \C{add r3 r2
} & 21 & 7  & & & 3 & & &\\ \hline
 \commentaire{Lecture de la donnée d'adresse 15 dans le registre 0
} \C{lecture 15 r0
} & 22 & 8  & 3 & & & & &\\ \hline
 \commentaire{Soustrait la valeur du registre 2 au registre 0
} \C{soustr r2 r0
} & 23 & 9  & 0 & & & & &\\ \hline
 \commentaire{Si la valeur (0) du registre 0 est positive, saute à l'adresse 4
} \C{sautpos r0 4
} & 24 & \textbf{4} & & & & & &\\ \hline
 \commentaire{Ajout de la valeur du registre 2 au registre 1
} \C{add r2 r1
} & 25 & 5  & & 6 & & & &\\ \hline
 \commentaire{Initialisation du registre 3 à 1
} \C{valeur 1 r3
} & 26 & 6  & & & & 1 & &\\ \hline
 \commentaire{Ajout de la valeur du registre 3 au registre 2
} \C{add r3 r2
} & 27 & 7  & & & 4 & & &\\ \hline
 \commentaire{Lecture de la donnée d'adresse 15 dans le registre 0
} \C{lecture 15 r0
} & 28 & 8  & 3 & & & & &\\ \hline
 \commentaire{Soustrait la valeur du registre 2 au registre 0
} \C{soustr r2 r0
} & 29 & 9  & -1 & & & & &\\ \hline
 \commentaire{Si la valeur (-1) du registre 0 est positive, saute à l'adresse 4
} \C{sautpos r0 4
} & 30 & 10  & & & & & &\\ \hline
 \commentaire{Écriture du registre 1 à l'adresse 16
} \C{ecriture r1 16
} & 31 & 11  & & & & & & 6
\\ \hline
 \commentaire{Fin du processus.
} \C{stop
} & 32 & 12  & & & & & &\\ \hline
\end{tabular}