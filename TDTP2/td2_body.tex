% -*- coding: utf-8 -*-

\newcommand{\commentaire}[1]{}

\entete{Travaux dirigés 2 : la structure de contrôle \C{for}}

L'objectif de ce TD TP est de vous familiariser avec la notion
d'itération en programmation. On parle communément de
\emph{boucle}. Cette notion sera illustrée sur des problèmes de comptage
et de répétition d'actions. 


\section{Itération : l'instruction \C{for}}

%\subsection{Rappel}

Soit le programme suivant :
{\small
\begin{listing}{1}
/* déclaration de fonctionalités supplémentaires */
#include <stdlib.h> /* EXIT_SUCCESS */
#include <stdio.h> /* printf */

/* déclaration constantes et types utilisateurs */

/* déclaration de fonctions utilisateurs */

/* fonction principale */
int main()
{
    /* déclaration et initialisation variables */
    int i; /* variable de boucle */

    for(i = 0; i < 5; i = i + 1)
    {
        printf("i = %d\n",i);
    }
    /* i >= 5 */

    printf("i vaut %d après l'exécution de la boucle.\n",i);

    return EXIT_SUCCESS;
}

/* definitions des fonctions utilisateurs */
\end{listing}
}

\begin{enumerate}
\item Quelle est la signification de chaque argument du \verb|for|~?
  Quelles instructions composent le corps de la boucle~?
  \begin{correction}
    Vu en cours. Tout est expression en C (avec effet de bord) mais on
    ment ici et on considère que ce sont des instructions. Ne sert a
    rien de les embrouiller et en plus c une mauvaise pratique de les
    utiliser comme expressions (source d'erreurs, de confusion). 
\begin{verbatim}
for(instruction1; expression_booléenne; instruction2)
{
    corps de la boucle : le bloc défini par la séquence d'instructions entre {}
}
\end{verbatim}
    \begin{itemize}
    \item instruction1 : sert à initialiser la variable de boucle

    \item expression\_booléenne : (s'évalue à vrai ou faux) si vrai
      exécute le corps de boucle, sinon passe a l'instruction suivant
      la boucle. On parle parfois de la garde de la boucle. 

    \item instruction2 : prépare l'itération suivante; il est
      important de comprendre que cette instruction doit forcément
      modifier la valeur de (au moins une variable, on ne leur parle
      que de la var de boucle) la variable intervenant dans
      l'expression booléenne SINON la val de expression\_booléenne est
      constante et on ne sort jamais de la boucle 
    \end{itemize}
  \end{correction}
\item Faire la trace du programme. Qu'affiche le programme~?
\begin{correction}
L'exécution de $i = i + 1$ est mise à la ligne 18 pour montrer que c à
la fin du corps de la boucle, mm si c pas très clair.
\begin{verbatim}
ligne          | i | affichage (sortie/écriture à l'écran)
-------------------------------------------------
initialisation | ? |
15             | 0 |
17             |   | i = 0
18             | 1 | 
17             |   | i = 1
18             | 2 | 
17             |   | i = 2
18             | 3 | 
17             |   | i = 3
18             | 4 | 
17             |   | i = 4
18             | 5 | 
21             |   | i vaut 5 après l'exécution de la boucle.
\end{verbatim}

Il affiche :
\begin{verbatim}
i = 0
i = 1
i = 2
i = 3
i = 4
i vaut 5 après l'exécution de la boucle.
\end{verbatim}
\end{correction}
\item Modifiez le programme afin que la séquence affichée soit
  exactement (faire cinq programmes) :
  \begin{itemize}
  \item \verb|0 1 2 3 4|,
  \item \verb|1 2 3 4|,
  \item \verb|1 2 3 4 5|,
  \item \verb|1 3 5|
  \item puis, enfin \verb|(0,0) (1,1) (2,2)|.
  \end{itemize}
  \begin{correction}
    Ici, les corrections peuvent se faire plus ou moins rapidement, en
    fonction de la compréhension des étudiants. Chaque séquence
    demande la modif d'un ou de plusieurs arguments afin d'insister
    sur leur rôle. 

Pour \verb|1 2 3 4 5|, c'est soit \verb|i < 6| soit \verb|i <=5|. Ca dépend du problème.
\begin{verbatim}
/* correction (0,0) (1,1) (2,2)*/
/* déclaration de fonctionalités supplémentaires */
#include <stdlib.h> /* EXIT_SUCCESS */
#include <stdio.h> /* printf */

/* déclaration constantes et types utilisateurs */

/* déclaration de fonctions utilisateurs */

/* fonction principale */
int main()
{
    /* déclaration et initialisation variables */
    int i; /* var. de boucle */

    for(i = 0; i < 3; i = i + 1)
    {
        printf("(%d,%d) ", i, i);
    }
    /* i >= 3 */

    printf("\n");

    return EXIT_SUCCESS;
}

/* definitions des fonctions utilisateurs */
\end{verbatim}

  \end{correction}
\item Modifiez le programme afin que la séquence affichée soit~:
  \begin{itemize}
  \item \verb|0 1 2 0 1 2|,
  \item puis \verb|0 1 2 0 1 2 3|.
  \end{itemize}
  De combien de boucles avez-vous besoin~? De combien de variables de
  boucles~?

  \begin{correction}
    2 boucles à la suite, 1 seule variable (on la réutilise)
  \end{correction}

\item Modifiez le programme afin que la séquence affichée soit :
  \begin{itemize}
  \item \verb|(0,0) (0,1) (0,2) (1,0) (1,1) (1,2) (2,0) (2,1) (2,2)|.
  \end{itemize}
  
  De combien de boucles avez-vous besoin~? De combien de variables de
  boucles~? Quelle est la différence de structuration
  des boucles entre le point 4 et le point 5~?

  \begin{correction}
    C'est un produit cartésien : $\{0,1,2\} \times \{0,1,2\}$. Deux boucles
    imbriquées, deux variables. La différence, c'est 1) deux à la suite; 2)
    deux imbriquées 

\begin{verbatim}
/* déclaration de fonctionalités supplémentaires */
#include <stdlib.h> /* EXIT_SUCCESS */
#include <stdio.h> /* printf */

/* déclaration constantes et types utilisateurs */

/* déclaration de fonctions utilisateurs */

/* fonction principale */
int main()
{
    /* déclaration et initialisation variables */
    int i; /* var. de boucle */
    int j; /* var. de boucle */

    for(i = 0; i < 3; i = i + 1)
    {
        for(j = 0; j < 3; j = j + 1)
        {
            printf("(%d,%d) ", i, j);
        }
        /* j >= 3 */
    }
    /* i >= 3 */

    /* passe a la ligne pour faire joli */
    printf("\n");

    return EXIT_SUCCESS;
}

/* definitions des fonctions utilisateurs */
\end{verbatim}
\end{correction}

\end{enumerate}

\subsection{Exercice type : calcul de $\Sigma_1^n i$}

Écrire un programme qui calcule et affiche la somme des entiers de 1 à
$n$ : $\Sigma_1^n i$, où $n$ est un entier quelconque (tester avec
différentes valeurs). 

\begin{correction}
\begin{verbatim}
/* déclaration de fonctionalités supplémentaires */
#include <stdlib.h> /* EXIT_SUCCESS */
#include <stdio.h> /* printf */

/* déclaration constantes et types utilisateurs */

/* déclaration de fonctions utilisateurs */

/* fonction principale */
int main()
{
    /* déclaration et initialisation variables */
    int n = 4;
    int somme = 0; /* élément neutre pour l'addition */
    int i; /* var. de boucle */

    for(i = 1; i <= n; i = i + 1) /* i allant de 1 à n */
    {
        /* ajoute i à la somme partielle */
        somme = somme + i;
    }
    /* i > n */

    /* somme vaut 0 + 1 + ... + n */
    printf("somme = %d\n",somme);

    return EXIT_SUCCESS;
}

/* definitions des fonctions utilisateurs */
\end{verbatim}
\end{correction}

Comment feriez vous pour écrire le même programme en assembleur (amil)
?
\begin{correction}
Le plus facile est d'appliquer le schéma se traduction du for (vu en
cours) au for
précédent (en laissant la variable $i$ dans un registre ou en la
stockant dans la mémoire).  Voici quelques (anciennes) corrections
plus détaillées (attention, elles partent de i = 0) mais il ne faut pas passer plus que cinq-dix minutes
sur cet exercice. Il suffit de laisser les étudiants y réfléchir un peu, puis
donner la structure du programme.

La correction suivante reprend la traduction en assembleur (\C{gcc
  -S}) d'un \C{while (...) {...}}, ou d'une boucle \C{for( i = 0; i <=
  n; i = i + 1)} en C. L'algorithme consiste en copier la formule de
la somme :
\begin{itemize}
\item la somme vaut $0$,  $i$ vaut $0$.
\item Tant que $i \leq n$, ajouter $x_i$ à la somme puis ajouter $1$ à
  $i$.
\end{itemize}

Le programme est alors :

\listinginput{1}{progs/gauss1.ail}
\begin{figure}[tbp]
  \centering \begin{tabular}[c]{l||c|c|c|c|c|c|c|c|}
\hline
 \emph{Instructions} & Cycles & CP& r0& r1& r2& r3& 15& 16\\ \hline
\hfill Initialisation & 0 & 1 & ? & ? & ? & ? & 3
 & ?
 \\ \hline \commentaire{Initialisation du registre 1 à 0
} \C{valeur 0 r1
} & 1 & 2  & & 0 & & & &\\ \hline
 \commentaire{Initialisation du registre 2 à 0
} \C{valeur 0 r2
} & 2 & 3  & & & 0 & & &\\ \hline
 \commentaire{Saut à l'adresse 7
} \C{saut 7
} & 3 & \textbf{7} & & & & & &\\ \hline
 \commentaire{Lecture de la donnée d'adresse 15 dans le registre 0
} \C{lecture 15 r0
} & 4 & 8  & 3 & & & & &\\ \hline
 \commentaire{Soustrait la valeur du registre 2 au registre 0
} \C{soustr r2 r0
} & 5 & 9  & 3 & & & & &\\ \hline
 \commentaire{Si la valeur (3) du registre 0 est positive, saute à l'adresse 4
} \C{sautpos r0 4
} & 6 & \textbf{4} & & & & & &\\ \hline
 \commentaire{Ajout de la valeur du registre 2 au registre 1
} \C{add r2 r1
} & 7 & 5  & & 0 & & & &\\ \hline
 \commentaire{Initialisation du registre 3 à 1
} \C{valeur 1 r3
} & 8 & 6  & & & & 1 & &\\ \hline
 \commentaire{Ajout de la valeur du registre 3 au registre 2
} \C{add r3 r2
} & 9 & 7  & & & 1 & & &\\ \hline
 \commentaire{Lecture de la donnée d'adresse 15 dans le registre 0
} \C{lecture 15 r0
} & 10 & 8  & 3 & & & & &\\ \hline
 \commentaire{Soustrait la valeur du registre 2 au registre 0
} \C{soustr r2 r0
} & 11 & 9  & 2 & & & & &\\ \hline
 \commentaire{Si la valeur (2) du registre 0 est positive, saute à l'adresse 4
} \C{sautpos r0 4
} & 12 & \textbf{4} & & & & & &\\ \hline
 \commentaire{Ajout de la valeur du registre 2 au registre 1
} \C{add r2 r1
} & 13 & 5  & & 1 & & & &\\ \hline
 \commentaire{Initialisation du registre 3 à 1
} \C{valeur 1 r3
} & 14 & 6  & & & & 1 & &\\ \hline
 \commentaire{Ajout de la valeur du registre 3 au registre 2
} \C{add r3 r2
} & 15 & 7  & & & 2 & & &\\ \hline
 \commentaire{Lecture de la donnée d'adresse 15 dans le registre 0
} \C{lecture 15 r0
} & 16 & 8  & 3 & & & & &\\ \hline
 \commentaire{Soustrait la valeur du registre 2 au registre 0
} \C{soustr r2 r0
} & 17 & 9  & 1 & & & & &\\ \hline
 \commentaire{Si la valeur (1) du registre 0 est positive, saute à l'adresse 4
} \C{sautpos r0 4
} & 18 & \textbf{4} & & & & & &\\ \hline
 \commentaire{Ajout de la valeur du registre 2 au registre 1
} \C{add r2 r1
} & 19 & 5  & & 3 & & & &\\ \hline
 \commentaire{Initialisation du registre 3 à 1
} \C{valeur 1 r3
} & 20 & 6  & & & & 1 & &\\ \hline
 \commentaire{Ajout de la valeur du registre 3 au registre 2
} \C{add r3 r2
} & 21 & 7  & & & 3 & & &\\ \hline
 \commentaire{Lecture de la donnée d'adresse 15 dans le registre 0
} \C{lecture 15 r0
} & 22 & 8  & 3 & & & & &\\ \hline
 \commentaire{Soustrait la valeur du registre 2 au registre 0
} \C{soustr r2 r0
} & 23 & 9  & 0 & & & & &\\ \hline
 \commentaire{Si la valeur (0) du registre 0 est positive, saute à l'adresse 4
} \C{sautpos r0 4
} & 24 & \textbf{4} & & & & & &\\ \hline
 \commentaire{Ajout de la valeur du registre 2 au registre 1
} \C{add r2 r1
} & 25 & 5  & & 6 & & & &\\ \hline
 \commentaire{Initialisation du registre 3 à 1
} \C{valeur 1 r3
} & 26 & 6  & & & & 1 & &\\ \hline
 \commentaire{Ajout de la valeur du registre 3 au registre 2
} \C{add r3 r2
} & 27 & 7  & & & 4 & & &\\ \hline
 \commentaire{Lecture de la donnée d'adresse 15 dans le registre 0
} \C{lecture 15 r0
} & 28 & 8  & 3 & & & & &\\ \hline
 \commentaire{Soustrait la valeur du registre 2 au registre 0
} \C{soustr r2 r0
} & 29 & 9  & -1 & & & & &\\ \hline
 \commentaire{Si la valeur (-1) du registre 0 est positive, saute à l'adresse 4
} \C{sautpos r0 4
} & 30 & 10  & & & & & &\\ \hline
 \commentaire{Écriture du registre 1 à l'adresse 16
} \C{ecriture r1 16
} & 31 & 11  & & & & & & 6
\\ \hline
 \commentaire{Fin du processus.
} \C{stop
} & 32 & 12  & & & & & &\\ \hline
\end{tabular}
  \caption{Somme des entiers de $0$ à $3$}
  \label{fig:gauss1}
\end{figure}


La boucle incrémente $i$. Dans amil, à cause de la difficulté à écrire
des tests tels que $i \leq n$, il serait plus rapide sur cet exercice
de décrémenter $n$ jusqu'à $0$ comme à la question précédente :

\begin{itemize}
\item $x$ vaut $n$, la somme vaut $0$
\item Tant que $x \geq 0$, ajouter $x$ à la somme et décrementer $x$.
\end{itemize}

Cette réponse est correcte et on peut encourager dans un premier temps
les étudiants qui cherchent dans cette direction, mais, pour les
préparer à la suite du cours, il faut leur donner la correction qui
incrémente l'indice.

Parmi les autres programmes corrects possibles, des étudiants peuvent
même penser à la formule de Gauss $\sum^n_{i = 0} i = \frac{n (n +
  1)}{2}$. 


Solution de Gauss (trace figure~\ref{fig:gauss3}).

\listinginput{1}{progs/gauss3.ail}
    \begin{figure}[tbp]
      \centering
      \begin{tabular}[c]{l||c|c|c|c|c|c|}
\hline
 \emph{Instructions} & Cycles & CP& r0& r1& 15& 16\\ \hline
\hfill Initialisation & 0 & 1 & ? & ? & 3
 & ?
 \\ \hline \commentaire{Lecture de la donnée d'adresse 15 dans le registre 0
} \C{lecture 15 r0
} & 1 & 2  & 3 & & &\\ \hline
 \commentaire{Lecture de la donnée d'adresse 15 dans le registre 1
} \C{lecture 15 r1
} & 2 & 3  & & 3 & &\\ \hline
 \commentaire{Ajout de la valeur 1 au registre 1
} \C{add 1 r1
} & 3 & 4  & & 4 & &\\ \hline
 \commentaire{Multiplie la valeur du registre 1 par celle du registre 0
} \C{mult r0 r1
} & 4 & 5  & & 12 & &\\ \hline
 \commentaire{Divise la valeur du registre 1 par 2
} \C{div 2 r1
} & 5 & 6  & & 6 & &\\ \hline
 \commentaire{Écriture du registre 1 à l'adresse 16
} \C{ecriture r1 16
} & 6 & 7  & & & & 6
\\ \hline
 \commentaire{Fin du processus.
} \C{stop
} & 7 & 8  & & & &\\ \hline
\end{tabular}
      \caption{Somme des entiers de $0$ à $3$ (3)}
      \label{fig:gauss3}
    \end{figure}
\end{correction}



% \newcommand{\TPshortName}{TP4}
% % -*- coding: utf-8 -*-

Vous allez mettre tous vos programmes écrits dans ce TP dans le
répertoire \TPshortName. 

\begin{newenu}
\item À partir du début de votre arborescence, créez le répertoire
\TPshortName : \C{mkdir \TPshortName} 
\item Allez dans ce répertoire pour y mettre des fichiers : 
  \C{cd \TPshortName}
\end{newenu}

L'étape suivante est à répéter pour chaque nouveau programme (exo1, exo2 etc..) :
\begin{lastenu}
\item Créez un nouveau fichier source pour le langage C ou une nouvelle copie d'un programme existant.
  \begin{description}
  \item[Création]   \verb|gedit exo1.c &| (vous pouvez utiliser
    \C{emacs} ou \C{kwrite} au lieu de \C{gedit})
\item[Copie] Il est plus rapide de repartir d'une copie de votre programme \C{bonjour.c} du TP2 pour éviter de retaper tout le squelette. Dans le terminal : \\ \C{cp ../TP2/bonjour.c exo1.c} \\ \verb|gedit exo1.c &|\\ Vous pouvez-aussi ouvrir \verb+bonjour.c+ et utiliser la fonction \emph{Enregistrer sous...} de votre éditeur mais attention à enregistrer la nouvelle copie dans le bon répertoire. 
\end{description}
\end{lastenu}

Vous pouvez utiliser à tout moment la commande \C{ls} (list directory)
pour voir la liste des fichiers d'un répertoire. 


Les trois étapes suivantes seront à répéter autant de fois que nécessaire pour la mise au point de chaque programme (apprenez à utiliser les raccourcis clavier). 
\begin{lastenu}
\item Après avoir fini d'écrire votre programme, enregistrez le.
\item Créez un programme exécutable à partir de votre fichier source :\\
  \verb|gcc -Wall exo1.c -o exo1.exe|
\item Quand l'étape précédente a réussi (il faut lire attentivement les
  messages affichés), exécutez le programme pour
  vérifier qu'il fonctionne : \verb|exo1.exe| (ou
  \verb|./exo1.exe|).
\end{lastenu}


\section{Affichage de figures géométriques}

% Les exercices suivants utilisent le caractère \verb|*| (étoile) pour dessiner des figures géométriques simples, appelées figures d'étoiles.

\subsection{Exercice type : affichage d'un rectangle d'étoiles}

Écrire un programme qui, étant données deux variables,
\verb|longueur| et \verb|largeur|, initialisées à des valeurs
strictement positives quelconques,affiche  un rectangle d'étoiles ayant pour
longueur \verb|longueur| étoiles et largeur \verb|largeur|
étoiles. Exemple :
\begin{verbatim}
Affichage d'un rectangle d'etoiles de longueur 6 et largeur 3.
******
******
******
\end{verbatim}

\begin{correction}
Durée 3/4 d'heure ?

Les algos sont à faire (les extraire du code).

Vous pouvez dans un premier temps supprimer la boucle la plus imbriquée en leur demandant d'afficher un rectangle de longueur exactement ``************''.

\begin{verbatim}
/* declaration de fonctionnalites supplementaires */
#include <stdlib.h> /* EXIT_SUCCESS */
#include <stdio.h> /* printf */

/* declaration constantes et types utilisateurs */

/* declaration de fonctions utilisateurs */

/* fonction principale */
int main()
{
    /* declaration et initialisation variables */
    int largeur = 3; /* largeur du rectangle en nb d'etoiles */
    int longueur = 6; /* longueur du rectangle en nb d'etoiles */
    int i; /* var. de boucle */
    int j; /* var. de boucle */

    printf("Affichage d'un rectangle d'etoiles de longueur %d et largeur %d.\n",longueur,largeur);

    for(i = 0;i < largeur;i = i + 1) /* chaque ligne d'étoiles */
    {
        /* affiche longueur etoiles */
        for(j = 0;j < longueur;j = j + 1) /* chaque colonne d'etoiles */
        {
            /* affiche une etoile */
            printf("*");
        }
        /* j >= longueur */

        /* passe a la ligne suivante */
        printf("\n");
    }
    /* i >= largeur */

    return EXIT_SUCCESS;
}

/* definitions des fonctions utilisateurs */
\end{verbatim}
\end{correction}

\subsection{Exercice type : affichage d'un demi-carré d'étoiles}

Écrire un programme qui affiche, étant donnée la variable,
\verb|cote|, initialisée à une valeur quelconque, un demi-carré d'étoiles (triangle rectangle isocèle) ayant pour
longueur de côté \verb|cote| étoiles. Exemple :
\begin{verbatim}
Affichage d'un demi-carre d'etoiles de cote 5.
*
**
***
****
*****
\end{verbatim}

\begin{correction}
Durée 3/4 d'heure ?

Les algos sont à faire (les extraire du code).

\begin{verbatim}
/* declaration de fonctionnalites supplementaires */
#include <stdlib.h> /* EXIT_SUCCESS */
#include <stdio.h> /* printf */

/* declaration constantes et types utilisateurs */

/* declaration de fonctions utilisateurs */

/* fonction principale */
int main()
{
    /* declaration et initialisation variables */
    int cote = 2; /* cote du demi-carré en nb d'etoiles */
    int i; /* var. de boucle */
    int j; /* var. de boucle */

    printf("Affichage d'un demi-carre d'etoiles de cote %d.\n",cote);

    for(i = 1;i <= cote;i = i + 1) /* chaque numero de ligne d'étoiles */
    {
        /* affiche autant d'etoiles que le numero de ligne */
        for(j = 0;j < i;j = j + 1) /* chaque colonne d'etoiles */
        {
            /* affiche une etoile */
            printf("*");
        }
        /* j >= i */

        /* passe a la ligne suivante */
        printf("\n");
    }
    /* i > cote */

    return EXIT_SUCCESS;
}

/* definitions des fonctions utilisateurs */
\end{verbatim}
\end{correction}

\section{Exercices facultatifs}
\subsection{Affichage d'un demi-carré droit d'étoiles}


Écrire un programme qui affiche un demi-carré droit d'étoiles de côté spécifié par l'utilisateur. Exemple d'exécution :
\begin{small}
\begin{verbatim}
Entrer la taille du demi-carré :
5
Affichage d'un demi-carre droit d'etoiles de cote 5.
    *
   **
  ***
 ****
*****
\end{verbatim}
\end{small}

\begin{correction}
\begin{verbatim}
/* declaration de fonctionnalites supplementaires */
#include <stdlib.h> /* EXIT_SUCCESS */
#include <stdio.h> /* printf, scanf */

/* declaration constantes et types utilisateurs */

/* declaration de fonctions utilisateurs */

/* fonction principale */
int main()
{
    int cote; /* cote du demi-carré droit en nb d'etoiles a saisir par l'utilisateur*/
    int i; /* var. de boucle */
    int j; /* var. de boucle */

    /* saisie cote (optionnel) */
    printf("Entrer la taille du demi-carré :\n");
    scanf("%d", &cote);

    /* affichage du demi-carre droit */
    printf("Affichage d'un demi-carre droit d'etoiles de cote %d.\n",cote);

    for(i = 1;i <= cote;i = i + 1) /* chaque numero de ligne d'étoiles */
    {
        /* affiche les blancs */
        for(j = 0;j < cote - i;j = j + 1) /* chaque colonne de blancs */
        {
            /* affiche un blanc */
            printf(" ");
        }
        /* j >= cote - i */

        /* affiche autant d'etoiles que le numero de ligne */
        for(j = 0;j < i;j = j + 1) /* chaque colonne d'etoiles */
        {
            /* affiche une etoile */
            printf("*");
        }
        /* j >= i */

        /* passe a la ligne suivante */
        printf("\n");
    }
    /* i > cote */

    return EXIT_SUCCESS;
}

/* definition de fonctions utilisateurs */

\end{verbatim}
\end{correction}

\subsection{Calcul de la somme d'une série d'entiers saisie par l'utilisateur}

Écrire un programme qui demande à l'utilisateur combien d'entiers
composent sa série, lit la série d'entiers et affiche la somme des
valeurs de la série. 

\paragraph{Indication :} l'instruction 
\verb+scanf("%d", &a)+
permet de réaliser une saisie utilisateur d'un entier dont la valeur
sera affectée à la variable \texttt{a} (comme toute variable, \texttt{a} doit
être préalablement déclarée).

\begin{correction}
\begin{verbatim}
/* declaration de fonctionnalites supplementaires */
#include <stdlib.h> /* EXIT_SUCCESS */
#include <stdio.h> /* printf, scanf */

/* declaration constantes et types utilisateurs */

/* declaration de fonctions utilisateurs */

/* fonction principale */
int main()
{
    int n; /* taille de la serie a saisir par l'utilisateur*/
    int elt; /* un element de la serie a saisir par l'utilsateur */
    int somme = 0; /* somme de la serie a calculer */

    int i; /* var. de boucle */

    /* demande la taille de la serie a l'utilisateur */
    printf("Combien d'elements dans la série ? ");
    scanf("%d", &n);

    /* saisie serie (n entiers) et calcul incremental de la somme */
    for(i = 0; i < n; i = i + 1) /* chaque entier de la serie */
    {
        /* saisir sa valeur */
        scanf("%d", &elt);

        /* l'ajoute a la somme partielle */
        somme = somme + elt;
    }
    /* i >= n */

    /* somme contient la somme des elements de la serie. */
    printf("La somme des valeurs de cette serie est : %d\n",somme);

    return EXIT_SUCCESS;
}

/* definitions des fonctions utilisateurs */
\end{verbatim}
\end{correction}


%%% Local Variables: 
%%% mode: latex
%%% TeX-master: "td2"
%%% End: 
