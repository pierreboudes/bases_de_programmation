% -*- coding: utf-8 -*-

\newcommand{\commentaire}[1]{}

\entete{Éléments d'informatique, rattrapage}


\begin{correction}
  \textbf{IMPORTANT dans 2.2 ``bientôt l'été'': indiquer aux étudiants
    dans les salles d'examen qu'ils doivent définir une variable
    entière \C{saison} dont les constantes symboliques PRINTEMPS, ETE,
    AUTRE représentent les valeurs possibles.}

RDV salle C102 pour remettre les copies à la fin de l'examen puis
salle C213 pour les corrections de copies.

\end{correction}
\begin{description}
\item[Durée :] 3 heures.
\item[Documents autorisés :] Aucun.
\item[Recommandations :] Un barème vous est donné à
titre indicatif afin de vous permettre de gérer votre temps. La
notation prendra en compte à la fois la syntaxe et la sémantique de
vos programmes, c'est-à-dire qu'ils doivent compiler correctement. Une
fois votre programme écrit, il est fortement recommandé de le faire tourner à la
main sur un exemple pour s'assurer de sa correction.
\end{description}

\section{Somme des éléments d'un tableau (\textit{4 points})}

Nous voulons écrire un programme qui calcule la somme des éléments
d'un tableau d'entiers. Une partie du programme est déjà écrite, et
Pippo pense qu'il ne
reste plus qu'à écrire la partie qui calcule effectivement la
somme. Voici son programme :
\begin{listing}{1}
#include <stdlib.h> /* EXIT_SUCCESS */
#include <stdio.h> /* printf, scanf */

/* déclaration constantes et types utilisateurs */
#define TAILLE 3

/* Fonction principale */
int main()
{
  int t[TAILLE] = {12,10,15}; /* tableau à sommer */



  /* calcul de la somme (À FAIRE) */





  /* affichage du résultat */
  printf("La somme est %d\n", somme);

  /* valeur fonction */
  return EXIT_SUCCESS;
}
\end{listing}

\begin{enumerate}
\item La compilation échoue. Expliquer pourquoi, quelle étape de la
  compilation échoue précisément et ce qu'il manque pour que la compilation réussisse. 
\item Expliquer le fonctionnement de l'instruction \C{\#define}.
\item Compléter le programme pour qu'il calcule effectivement la somme
  des éléments du tableau (pour une taille de tableau arbitraire).
\end{enumerate}

Un exemple de sortie du programme pour le tableau donné dans le code est :
\begin{verbatim}
La somme est 37
\end{verbatim}

\begin{correction}
\begin{enumerate}
\item (1.5pt) dont 1 pt pour ``déclaration de la variable \C{somme}'' + 0.5 pt pour ``analyse
  sémantique'' ou 0.25 pt si ``avant la génération du code objet'' et/ou
  ``avant l'édition de lien''.
\item 1 pt si juste. 0.5 pt si seulement notion de constantes et pas de mise en
  relation avec la substitution au moment de la compilation.
\item 1.5pt pour le code juste (compilation correction quel que soit
  TAILLE). Sinon, maxi 1 pt :
  \begin{itemize}
  \item pas de boucle -> 0
  \item 0.5pt une (et une seule) boucle (for, while accepté)
  \item 0,5pt la boucle parcours effectivement un tableau de taille TAILLE
  \item pas de points en moins ou en plus pour la déclaration de la
    variable de boucle (redite sous une autre forme de la première
    question) ou pour l'initialisation de somme à zéro.
  \end{itemize}
\end{enumerate}
\end{correction}

\section{Sans fonctions}

Il est demandé de résoudre les deux problèmes suivants sans définir de
fonctions utilisateurs.  L'ensemble du code sera à écrire dans la
fonction principale \verb|main|.

\subsection{ For ou while (\textit{4 points})}

Écrire le programme qui :
\begin{itemize}
\item demande à l'utilisateur d'entrer les valeurs d'un tableau
  d'entiers de taille \C{N} (une constante symbolique)
\item puis cherche si le tableau contient l'entier zéro et affiche
  l'indice de la première case nulle s'il y en a une ou \C{Pas de case
    nulle} sinon.
\end{itemize}
Deux exemples d'exécution sont les suivants (pour \verb|N| valant 4) :
\begin{small}
\begin{verbatim}
Saisissez 4 entiers : 8 31 1 -12
Pas de case nulle

Saisissez 4 entiers : 2 -5 4 0
Case nulle d'indice 3
\end{verbatim}
\end{small}

\begin{correction}
  \begin{itemize}
\item 1 point. pour la saisie correcte on ne compte pas faux si pas de
  \&.  Constante symbolique comptée à part.
\item 2.5pt. Deux structures admises pour la seconde partie : soit utiliser un
booleen ``trouve'' comme dans d'autres exercices du meme type, soit
utiliser le depassement de borne de boucle comme test. Dans les deux
cas while puis if -> 1pt. Si ils se debrouillent pour ça fonctionne
correctement avec un for au lieu du while ok. Ensuite 0.5 pour une
boucle qui termine. 1pt pour l'affichage correct : 0.5 point pour le
cas sans zéro, 0.5 point pour les cas avec zéro. Compter faux les cas
où il y a plusieurs affichages.
\item 0.5pt si définit et utilise correctement la constante symbolique
  (si elle n'est que définie et qu'ils ne traitent pas les boucles
  compter juste quand même par contre si les boucles sont là et que la
  taille du tableau n'est pas donnée par N compter faux !).
\end{itemize}
\begin{small}
\begin{verbatim}
#include <stdlib.h> /* EXIT_SUCCESS */
#include <stdio.h> /* printf, scanf */

/* déclaration constantes et types utilisateurs */
#define N 4 /* taille du tableau utilisateur */

/* déclaration de fonctions utilisateurs */

int main()
{
    int tab[N]; /* tableau à initialiser par l'utilisateur */
   int i; /* var. de boucle */

    /* demande saisie de N entiers*/
    printf("Saisissez %d entiers : ", N);

    /* saisie des elts du tableau (N entiers) */
    for(i = 0; i < N; i = i + 1) /* chaque case du tableau */
    {
        /* saisie valeur */
        scanf("%d",&tab[i]);
    }
    /* i >= N */

    /* recherche zero */
    i = 0;
    while ((i < N) && (tab[i] != 0)) 
    {
       i = i + 1;
    }
    /* i >= N  ou bien tab[i] == 0 */
    if (i < N)
    {
        printf("Cas nulle d'indice %d\n", i);
    }
    else
    {
        printf("Pas de case nulle\n");
    }

    return EXIT_SUCCESS;
}
\end{verbatim}
\end{small}
\end{correction}

\subsection{Bientôt l'été (\textit{5 points})}

Écrire un programme qui demande à l'utilisateur de saisir une date dans l'année et affiche :
\begin{itemize}
\item \C{C'est le printemps}, si la date est au printemps;
%  et que l'été ne
% viendra pas avant la fin du mois courant;
% \item \C{C'est le printemps et ce sera l'été dans x jours} où \{x} est une
% valeur numérique valant le nombre de jours avant l'été si l'été
% arrivera dans le mois courant;
\item \C{Déjà l'été} si la date du jour est en été:
\item \C{Ce n'est ni le printemps, ni l'été}, si la date n'est ni au printemps ni en été;
\end{itemize}

Vous utiliserez trois constantes symboliques pour représenter les
trois périodes différentes de l'année : \C{PRINTEMPS}, \C{ETE},
\C{AUTRE}. 

Vous séparerez clairement la partie saisie, la partie
calcul de la saison et la partie affichage.

La date sera saisie sous la forme de deux entiers : l'un entre 1 et 31
(inclus) pour le jour dans le mois et l'autre entre 1 et 12 (inclus)
pour le numéro du mois. On suppose que la saisie est une date
correcte.

Le printemps commence le 20 mars, l'été le 21 juin, l'automne le 22
septembre.

Exemples d'exécutions :
\begin{small}
\begin{verbatim}
Date : 21 10
Ce n'est ni le printemps, ni l'ete

Date : 11 6
C'est le printemps

Date : 21 6
C'est l'ete
\end{verbatim}
\end{small}
  \begin{correction}
   \begin{itemize}
    \item 1pt saisie (sur-récompensé).
    \item 4 points pour calcul et affichage en simulant le programme de l'etudiant. Si ce
      programme est trop compliqué (des boucles ou autre) c'est zéro à
      cette partie.
      \begin{itemize}
      \item 0.5pt si fonctionne pour mois < 3  (jour quelconque)
        \item 0.5pt si fonctionne pour mois > 9  (jour quelconque)
      \item 0.5pt si mois 4,5 au printemps  (jour quelconque)
      \item 0.5pt si mois 7,8 en été  (jour quelconque)
        \item 1.5pt : 0,5 point par traitement correct, à un jour près, des
          jours de transition dans les mois 3,6,9.  (1-19 mars autre et 21-31
          mars printemps etc...)
\item 0.5 point si chacun des trois jours de début de saison sont bien
  dans la saison (20 mars an printemps, 21 juin été, 22 septembre).
      \end{itemize}
    \item Si correct mais affichage au milieu des tests ou saisie au
      milieu des tests enlever 0.5 pt.
      \item 0.5 pt en moins si il y a des variables non declarées, pas
        de points en moins pour les scanf, 0.5 point en moins si pas
        de constantes symboliques.
    \end{itemize}
\begin{small}
\begin{verbatim}
#include <stdlib.h> /* EXIT_SUCCESS */
#include <stdio.h> /* printf, scanf */

/* déclaration constantes et types utilisateurs */
#define TAILLE 3
#define PRINTEMPS 1
#define ETE 2
#define AUTRE 42

/* Fonction principale */
int main()
{
    int jour = 0;
    int mois = 0;
    int saison = AUTRE;
    
    /* saisie */
    printf("Saisir la date : ");
    scanf("%d", &jour);
    scanf("%d", &mois);


    /* calcul de la saison (on peut faire plus cours en initialisant saison a AUTRE) */
    if (mois < 3)
    {
        saison = AUTRE;
    }
    if (mois == 3) {
        if (jour < 20) {
            saison = AUTRE;
        }
        else 
        {
            saison = PRINTEMPS;
        }
    }
    if ((mois > 3) && (mois < 5))
    {
        saison = PRINTEMPS;
    }
    if (mois == 6) 
    {
        if (jour < 21) 
        {
            saison = PRINTEMPS;     
        }
        else
        {
            saison = ETE;
        }
    }
    if ((mois > 6) && (mois < 9)) 
    {
        saison = ETE;
    }
    if (mois == 9)
    {
        if (jour < 22) {
            saison = ETE;
        }
        saison = AUTRE;
    }
    if (mois > 9) 
    {
        saison = AUTRE;
    }


    /* affichage */
    if (saison == AUTRE)
    {
        printf("Ce n'est ni le printemps, ni l'ete\n");
    }
    if (saison == PRINTEMPS)
    {
        printf("C'est le printemps\n");
    }
    if (saison == ETE)
    {
        printf("C'est l'ete\n");
    }
    /* valeur fonction */
    return EXIT_SUCCESS;
}
\end{verbatim}
\end{small}
  \end{correction}

\section{Trace d'un programme avec fonctions (\textit{3 points})}

Simulez l'exécution du programme suivant, en réalisant sa \textbf{trace}, comme cela a été vu en TD et en cours.

\begin{small}
\begin{listing}{1}
#include <stdlib.h> /* EXIT_SUCCESS */
#include <stdio.h> /* printf, scanf */

/* declarations constantes et types utilisateurs */

/* declarations de fonctions utilisateurs */
int mystere(int n);

int main()
{
    int x = 2;
    int res;
    
    res = mystere(x);
    printf("mystere(%d) = %d\n", x, res);
    
    return EXIT_SUCCESS;
}

/* definitions de fonctions utilisateurs */
int mystere(int n)
{
   if (n > 1) 
   {
       return n * mystere(n - 1);
   }
   return 1;
}
\end{listing}
\end{small}

\begin{correction}

  \begin{itemize}
  \item 0.5pt pour l'identification des variables du main et leur
    initialisation
  \item 0.5pt pour le premier appel de fonction (avec ou sans titre)
    ligne 14 et identification de l'unique variable (n).
  \item 0.5 pt pour le passage par valeur (colonne n initialisée à 2)
    dans cet appel.
  \item 0.5pt pour la hauteur de pile (appel imbriqué).
  \item 0.5 pt si le if est correctement suivi dans les deux cas (faux
    si un seul appel).
  \item 0.5 pt si res est affecté après l'appel initial a mystere
    (effet de l'affectaction), même si pas le bon entier (si pas
    valeur entière c'est zéro). 
  \item 0.5 pt si l'affichage est correct (et donc res vaut bien 2).
  \end{itemize}
    \paragraph{Trace.}
 \begin{table}[h]
      %\begin{center}
\small
        \setlength{\unitlength}{\tabcolsep}
          \begin{tabular}[t]{|c|c|c|l|}
          \multicolumn{4}{l}{\C{main()}}\\ \hline
          ligne & x & res &  Affichage\\ \hline
          initialisation  & 2 & ? & \\ \hline
          14 & & & 
          \multicolumn{1}{r|}{
            \put(1,0){\noindent
              \begin{tabular}[t]{|c|c|l|}
                \multicolumn{3}{l}{\C{mystere(2)}}\\ \hline
                ligne & n & Affichage \\ \hline
                initialisation  & 2 & \\ \hline
                25 &  & \multicolumn{1}{r|}{
                  \put(1,0){\noindent
                    \begin{tabular}[t]{|c|c|l|}
                      \multicolumn{3}{l}{\C{mystere(1)}}\\ \hline
                      ligne & n & Affichage \\ \hline
                      initialisation  & 1 & \\ \hline
                     27  &\multicolumn{2}{|l|}{renvoie 1}\\ \hline                
                    \end{tabular}
                  }
                }\\ \hline 
                25  &\multicolumn{2}{|l|}{renvoie 2}\\ \hline                
              \end{tabular}
            }
          }\\ \hline 
          14 & & 2 &  \\ \hline
          15 & &  &  mystere(2) = 2\\ \hline
          17 &\multicolumn{3}{|l|}{renvoie EXIT\_SUCCESS}\\ \hline
        \end{tabular}
      %\end{center}
  \end{table}
\end{correction}

\section{Écriture de fonctions (\textit{4 $\times$ 1 point})}

\begin{enumerate}
\item Écrire une fonction qui prend en entrée un entier $n$ et calcule
  la somme :\label{item:somme}
  \begin{gather*}
    \sum^n_{k = 1} \frac{1}{k^2}
  \end{gather*}

\item Écrire la procédure \verb|rect| qui n'a pas d'entrée et
  affiche le motif suivant :\label{item:rect}
\begin{verbatim}
*****
*   *
*****
\end{verbatim}
\item Écrire la procédure \verb|rectangle| qui prend en entrée un
  entier \C{largeur} et un entier \C{hauteur} et affiche un rectangle plein
  d'étoiles de \C{hauteur} lignes sur \C{largeur} colonnes. \label{item:rectangle}

\item Écrire une fonction \verb|volume_cylindre| qui prend en entrée
  un rayon réel \C{r} et une hauteur réelle \C{h} et renvoie le volume du cylindre
  de rayon \C{r}  et de hauteur \C{h}. vous ferez appel à une
  fonction \C{calcul\_surface}, supposée déjà définie, qui calcule la
  surface d'un cercle à partir de son rayon.
\end{enumerate}

\paragraph{Bonus.}
Écrire une procédure \C{boite} qui affiche un rectangle de \C{hauteur}
lignes sur \C{largeur} colonnes, comme la fonction \C{rectangle}
(question~3), mais creux, comme dans l'exemple de la question~2.

\begin{correction}
Difficultées diverses mais c'est voulu, on compte :
0.5 pt prototype correct et 0.5 pt code correct (ou 0.25 si
quasi-correct ie boucles dans questions \ref{item:somme} et \ref{item:rectangle}).
\begin{verbatim}
double sommation(int n)
{
  int k;
  double somme = 0;
  for (i = 1; i <= n; i = i + 1)
  {
    somme = somme + 1.0 / (k*k); /* Attention au double : 1.0 */
  }
  return somme;
}

void rect()
{
  printf("*****\n");
  printf("*   *\n");
  printf("*****\n");
}

void rectangle(int hauteur, int largeur)
{
  int i; /* variable de boucle ligne */
  int j; /* variable de boucle colonne */

  for (i = 0; i < hauteur; i = i + 1)/* pour chaque ligne */
  { 
    /* afficher une ligne d'etoiles */     
    for (j = 0; j < largeur; j = j + 1) /* pour chaque colonne */
    {
      printf("*"); /* afficher etoile */
    }
    printf("\n"); /* fin de la ligne */
  }
}

double volume_cylindre(double r, double h)
{
  return h * calcul_surface(r);
}
\end{verbatim}

Pour la question bonus on compte 2 points si rigoureusement
juste. Sinon rien.

\begin{verbatim}
void boite(int hauteur, int largeur)
{
  int i; /* variable de boucle ligne */
  int j; /* variable de boucle colonne */

  for (i = 0; i < hauteur; i = i + 1)/* pour chaque ligne */
  { 
    for (j = 0; j < largeur; j = j + 1) /* pour chaque colonne */
    {
      if ((0 == i) 
          || (hauteur - 1 == i) 
          || (0 == j)
          || (largeur - 1 == j))
      {
         printf("*"); /* bord : etoile */
      }
      else
      {
        printf(" "); /* centre : espace */
      }
    }
    printf("\n"); /* fin de la ligne */
  }
}
\end{verbatim}

\end{correction}