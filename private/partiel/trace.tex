\documentclass[landscape, 12pt]{article}

\usepackage[scale=0.85,vmargin=1.2cm ]{geometry}
\usepackage{latexsym,amssymb,amsmath}
\usepackage[utf8]{inputenc}
\usepackage[T1]{fontenc}
\usepackage[francais]{babel}
\newcommand{\C}[1]{{\upshape\texttt{#1}}}
\begin{document}
\thispagestyle{empty}

  \begin{table}[h]
      \begin{center}
\renewcommand{\arraystretch}{1.4}
  Nom : \hfill Prénom : \hfill N$^{\circ}$ de la carte d'étudiant : \hfill \phantom{.}\\ \hrulefill \\
       \begin{tabular}[t]{|r|c|c|c|c|c|l|}
          \multicolumn{7}{l}{\C{main()} (l'utilisateur saisit $1$, puis $2$, puis $3$)}\\ \hline
          \textbf{numéro de ligne} & \hspace{3em} & \hspace{3em} & \hspace{3em} & \hspace{3em} & \hspace{3em} &  \textbf{Affichage (sortie écran)}  \hspace{15em} \\ \hline
          initialisation  & & & & & & \\ \hline
           & & & & & & \\ \hline
           & & & & & & \\ \hline
           & & & & & & \\ \hline
           & & & & & & \\ \hline
           & & & & & & \\ \hline
           & & & & & & \\ \hline
           & & & & & & \\ \hline
           & & & & & & \\ \hline
           & & & & & & \\ \hline
           & & & & & & \\ \hline
           & & & & & & \\ \hline
           & & & & & & \\ \hline
           & & & & & & \\ \hline
           & & & & & & \\ \hline
           & & & & & & \\ \hline
           & & & & & & \\ \hline
           & & & & & & \\ \hline
           & & & & & & \\ \hline
           & & & & & & \\ \hline
           & & & & & & \\ \hline
       \end{tabular}
\end{center}
        \caption{Réponse à la question E. Trace du programme de la figure 2 (nombre de lignes vierges approximatif).}
        \label{simulation}
\end{table}
\end{document}
