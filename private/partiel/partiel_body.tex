% -*- coding: utf-8 -*-

\newcounter{questioncount}
\setcounter{questioncount}{0}
\newcommand{\question}{\addtocounter{questioncount}{1}\paragraph{Question \Alph{questioncount}.}}
\newcommand{\commentaire}[1]{}
\newcommand{\pt}[1]{\fbox{$#1 \operatorname{pt}$}}

\entete{Éléments d'informatique : partiel de mi-semestre}
\vspace{-1cm}
\begin{description}
\item[Durée :] 3 heures.
\item[Documents autorisés :] Aucun.
\item[Recommandations :] Un barème vous est donné à titre indicatif
  afin de vous permettre de gérer votre temps. Vous pouvez traiter les
  questions dans l'ordre de votre choix, en faisant bien apparaître la
  numérotation alphabétique. Les premières questions sont plus faciles que les
  dernières.  Il est recommandé de faire tourner à la main sur un exemple
  vos programmes pour s'assurer qu'il sont corrects.
\end{description}

\setlength{\textfloatsep}{1em}

\section{Étude de programmes et questions de cours (\emph{8 points})}

\subsection{Questions de cours}
\question Rappeler les cinq étapes de la compilation. \bareme{1}


\begin{correction}
Les cinq étapes de la compilation sont :
  \begin{itemize}
  \item l'analyse lexicale,
\item l'analyse syntaxique,
\item l'analyse sémantique,
\item la génération du code,
\item l'édition de liens.
  \end{itemize}
  \begin{baremeenv}
Si tout juste \pt{1}, sinon :
    \begin{itemize}
\item S'il ne manque que génération du code \pt{0.75}, sinon:
    \item \pt{0.5} s'il manque une ou deux étapes ou s'il n'en manque
      aucune mais qu'elles sont dans le désordre (quel que soit le
      niveau de désordre). Sinon :
    \item \pt{0.25} si il y a au moins dans cet ordre : analyse syntaxique
      et analyse sémantique.
    \end{itemize}
  \end{baremeenv}
\end{correction}
\question Expliquer le fonctionnement de l'instruction
\C{\#define}. Vous pouvez prendre comme exemple le programme de la
figure~\ref{fig:max}. \bareme{1}

\begin{correction}
  L'instruction \verb|#define| s'adresse au préprocesseur du
  compilateur, elle instruit le préprocesseur de l'existence d'un nom
  (par convention noté en majuscule) dans la suite du texte du
  programme et d'une expression qui devra se substituer à ce nom avant
  de commencer la compilation. Ce nom est appelé constante symbolique,
  il est en général plus clair pour le programmeur que l'expression
  qui le remplacera et facilite ainsi une relecture non ambiguë du
  programme (deux constantes symboliques définiront des noms
  différents pour des rôles différents, même si les expressions de
  substitution sont
  identiques).

Dans le programme de la
figure~\ref{fig:max}, ligne~5, l'instruction \verb|#define| définit une
constante symbolique de nom \C{TAILLE} qui sera substituée par
l'expression \C{3}. Cette constante est utilisée pour référer à la
taille du tableau \C{t} dans la suite du programme.

 \begin{baremeenv}
\pt{1} si constante symbolique, préprocesseur,  TAILLE remplacé
      par 3. \pt{0.5} si seulement notion de constantes et pas de mise en
      relation avec la substitution au moment de la compilation.
\end{baremeenv}
\end{correction}

\begin{figure}[htb]
\begin{small}
\begin{listing}{1}
#include <stdlib.h> /* EXIT_SUCCESS */
#include <stdio.h> /* printf, scanf */

/* déclaration constantes et types utilisateurs */
#define TAILLE 3

/* Fonction principale */
int main()
{
  int t[TAILLE] = {12, 15, 10}; /* tableau */



  /* calcul du maximum (À FAIRE) */





  /* affichage du résultat */
  printf("Le maximum est %d\n", maximum);

  /* valeur fonction */
  return EXIT_SUCCESS;
}
\end{listing}
\end{small}
  \caption{Calcul du maximum d'un tableau (à compléter)}
  \label{fig:max}
\end{figure}

\subsection{Maximum des éléments d'un tableau}

Nous voulons écrire un programme qui calcule le maximum des éléments
d'un tableau d'entiers positifs. Une partie du programme est déjà écrite, figure~\ref{fig:max}, et
Pippo pense qu'il ne
reste plus qu'à écrire la partie qui calcule effectivement le
maximum. 

\question Avant de continuer Pippo veut tester son programme, mais la compilation échoue. Expliquer pourquoi, quelle étape de la
  compilation échoue précisément et ce qu'il manque pour que la
  compilation réussisse. \bareme{1}

  \begin{correction}

La compilation échoue car la variable \C{maximum} est utilisée sans
avoir été déclarée. C'est à l'analyse sémantique que cette erreur est
détectée puisque c'est l'étape où les objets manipulés par le
programme et les actions sur ces objets sont analysés.  
    \begin{baremeenv}
      \pt{1} dont \pt{0.5} pour \emph{déclaration de la variable
      \C{maximum}}  plus \pt{0.5} pour \emph{analyse sémantique} ou
      \pt{0.25} si \emph{avant la génération du code objet} et/ou \emph{avant
      l'édition de lien}.
    \end{baremeenv}
  \end{correction}
\question Compléter le programme pour qu'il calcule effectivement le maximum
  des éléments du tableau (pour n'importe quel tableau contenant des
  entiers positifs et de taille \C{TAILLE} fixée arbitrairement). \bareme{2}

  \paragraph{Exemple.} Pour le tableau donné dans le code, une fois
  terminé, le programme affichera :

\begin{small}
\begin{verbatim}
Le maximum est 15
\end{verbatim}
\end{small}

\begin{correction}
Voir figure~\ref{fig:maxcorr}.
  \begin{figure}[htbp]
\small
  \begin{listing}{11}
    int maximum = 0; /* maximum initial: plus petite valeur possible */    
    int i; /* variable de boucle */

    /* calcul du maximum */
    for (i = 0; i < TAILLE; i = i + 1) /* pour chaque element de t */    
    {
        if (maximum < t[i]) /* t[i] > max(t[0], ... , t[i - 1]) */
        {
            maximum = t[i]; /* t[i] = max(t[0], ..., t[i]) */ 
        }
    }
  \end{listing}
   \caption{Complément de programme}
    \label{fig:maxcorr}
  \end{figure}

\begin{baremeenv}
\pt{2.5}  (barème augmenté)  pour le code juste (compilation correction quel que soit
  TAILLE). Sinon, maxi \pt{2} :
  \begin{itemize}
  \item pas de boucle on met \pt{0} quelque soit le contenu.
  \item pas de points en moins ou en plus pour la déclaration de la
    variable de boucle (redite sous une autre forme de la première
    question) ou pour l'initialisation de maximum à zéro ou à la
    première case du tableau.
  \item \pt{0.5} une (et une seule boucle (for, while accepté)
 \item \pt{0.5} la boucle parcours effectivement un tableau de taille
    TAILLE (TAILLE - 1 admis si l'initialisation du maximum est faite à \C{t[0]}).
\item \pt{0.5} pour un if imbriqué dans la boucle.
\item \pt{0.5} si le if est correct, c'est à dire s'il réalise le bon test (comparaison avec le maximum
  courant) et change la valeur du maximum courant en conséquent.
  \end{itemize}
\end{baremeenv}
\end{correction}


\subsection{Trace d'un programme} 

\question Simuler l'exécution du programme figure~\ref{pourtrace}, en réalisant sa
\textbf{trace}.\bareme{3}

\paragraph{Rappel.} La trace représente l'exécution du programme par
un tableau à $n + 2$ colonnes : la première colonne contient le numéro
de la ligne exécutée; les $n$ colonnes suivantes, les variables du
programme ($n$ à déterminer) et la dernière colonne, l'affichage
réalisé par le programme à l'écran. La trace commence par une ligne
nommant les colonnes, puis la ligne reportant les valeurs initiales
des variables, après quoi seules les lignes modifiant l'état mémoire
du programme ou l'affichage sont à reporter dans le
tableau. L'utilisateur doit saisir trois entiers : considerer, comme
indiqué en commentaires, qu'il saisit $1$ puis $2$ puis $3$.
\begin{figure}[htbp]
\begin{small}
\begin{listing}{1}
#include <stdlib.h> /* EXIT_SUCCESS */
#include <stdio.h> /* printf, scanf */

/* declarations de constantes et types utilisateurs */

/* declarations de fonctions utilisateurs */

/* fonction principale */
int main()
{
    int terme;
    int raison;
    int n;
    int i;
    int serie = 0;
    
    printf("Entrez le premier terme, la raison puis le nombre de termes : ");
    scanf("%d",&terme);  /* 1 est saisi par l'utilisateur */
    scanf("%d",&raison); /* 2 est saisi par l'utilisateur */
    scanf("%d",&n);      /* 3 est saisi par l'utilisateur */
    
    for (i = 0; i < n; i = i + 1)
    {
        serie = serie + terme;
        terme = terme * raison;
    }

    printf("somme des %d premiers termes de la suite : %d\n", n, serie);

    return EXIT_SUCCESS;
}
\end{listing}
\end{small}
  \caption{Faire la trace du programme}
  \label{pourtrace}
\end{figure}

\begin{correction}
Voir table~\ref{fig:trace}.

  \begin{sidewaystable}
  \centering
         \begin{tabular}[t]{|r|c|c|c|c|c|l|}
          \multicolumn{7}{l}{\C{main()} (l'utilisateur saisit $1$, puis $2$, puis $3$)}\\ \hline
          \textbf{numéro de ligne} & \C{terme} &\C{raison} & \C{n} &
          \C{i} & \C{serie} & \textbf{Affichage (sortie écran)}  \\ \hline
          initialisation  & ? & ? & ? & ? &  0& \\ \hline
          17 & & & & & &Entrez le premier terme, la raison, puis le
          nombre de termes :\\ \hline
          18 &  1& & & & & \\ \hline
          19&  & 2 & & & &\\ \hline
          20 &  &  & 3 & & &\\ \hline
          22 &  &  & & 0 & &\\ \hline
          24 &  &  & &  & 1 &\\ \hline
          25 & 2 &  & &  & &\\ \hline
          26 &  &  & & 1 & &\\ \hline
          24 &  &  & &  & 3 &\\ \hline
          25 & 4 &  & &  & &\\ \hline
          26 &  &  & & 2 & &\\ \hline
          24 &  &  & &  & 7 &\\ \hline
          25 & 8 &  & &  & &\\ \hline
          26 &  &  & & 3 & &\\ \hline
          28 &  &  & &  & & somme des 3 premiers termes de la suite : 7\\ \hline
          30 &  \multicolumn{6}{l|}{retourne \C{EXIT\_SUCCESS}} \\
          \hline
         \end{tabular}
  \caption{Trace du programme de la figure~\ref{pourtrace}}
  \label{fig:trace}
\end{sidewaystable}
  \begin{baremeenv}
      Compter :
      \begin{itemize}
      \item \pt{0.5} pour une ligne d'en-tête juste, autrement dit pour
        les colonnes (0 sinon)
      \item \pt{0.5} pour une ligne d'initialisation juste (0 sinon)
      \item aucune pénalité si la dernière ligne est absente (renvoie
        ....).
      \item \pt{2} pour le corps du main juste :
        \begin{itemize}
        \item \pt{-0,5} de pénalités si des erreurs dans la partie saisie
        \item \pt{-0,5}  de pénalités si des erreurs dans l'un ou l'autre
          ou les deux affichages.
        \item \pt{-1} de pénalité sur le for : si la boucle ne passe pas exactement
          par trois étapes avec i = 0, 1, 2 ou bien si la dernière
          affectation de \C{i} (à 3) n'est pas faite.
        \item Pour toute autre ligne erronée (ligne 24 ou 25 ne
          compter qu'une seule fois)
          compter \pt{-0.5} de pénalité.
        \end{itemize}
     \item Si des valeurs de variables non modifiées apparaissent sur
        certaines lignes compter également une pénalité de \pt{-0.5}.
      \end{itemize}
    \end{baremeenv}
\end{correction}

\section{Jours d'automne (\textit{4 points})}

Cette année votre premier semestre se déroule sur trois saisons :
l'été qui se termine le 22 septembre, l'automne qui commence le 23
septembre et se termine le 21 décembre et l'hiver qui commence le 22
décembre.

Soient deux variables entières \C{jour} et \C{mois}, représentant une
date entre le premier septembre (1/9) et le 31 décembre (31/12) et que
vous initialiserez à des valeurs de votre choix, prises dans cet
intervalle (votre programme doit fonctionner pour n'importe quel
choix correct de date).

\question Écrire un programme qui :
\begin{enumerate}
\item définit trois constantes symboliques pour représenter par trois
  nombres différents les trois
  saisons : \C{ETE}, \C{AUTOMNE}, \C{HIVER}. \bareme{0.5}
\item affiche la date représentée par les variables \C{jour} et \C{mois} \bareme{0.5}
\item trouve la saison correspondant à cette date \bareme{2}
\item affiche : \bareme{1}
\begin{itemize}
\item \C{C'est encore l'été}, si la date est en été;
\item \C{C'est l'automne}, si la date est en automne;
\item \C{Déjà l'hiver}, si la date du jour est en hiver.
\end{itemize}
\end{enumerate}


Vous séparerez clairement le calcul de la saison et la partie
affichage dans votre programme. Pour cela vous utiliserez une variable
entière \C{saison} dont les valeurs possibles seront : \C{ETE},
\C{AUTOMNE}, \C{HIVER}.


\paragraph{Exemples d'exécutions :}
\begin{small}
\begin{verbatim}
Date : 20 10
c'est l'automne

Date : 5 9
C'est encore l'ete

Date :  23 12
Deja l'hiver
\end{verbatim}
\end{small}
  \begin{correction}
Voir figure~\ref{automne}.
\begin{baremeenv}
   \begin{itemize}
\item \pt{0.5} pour les trois define avec trois valeurs quelconques mais
  différentes.
    \item \pt{0.5} pour l'affichage jour/mois
    \item \pt{3} pour calcul et affichage. Si ce
      programme est trop compliqué (des boucles ou autre) c'est zéro à
      cette partie sans regarder plus loin. Sinon :
      \begin{itemize}
\item \pt{1} si séparation calcul / affichage et que l'affichage en fonction
  de la valeur de la variable saison est correct (a des variations de
  syntaxe pret).
      \item \pt{0.5} si fonctionne pour mois 10 et 11  (jour quelconque)
     \item \pt{0.5} par transition dans le mois correctes (en décembre et
       en septembre), \pt{0.25} si c'est à un jour pret.
     \end{itemize}
     \item \pt{-0.5} s'il y a des variables non declarées, ou
       si il n'y a pas de main.
\item aucune pénalité pour le traitement des dates mals formatées ou
  en dehors de l'intervalle.
\item pas de penalité si oublie d'un include ou du EXIT\_SUCCESS.
   \end{itemize}
\end{baremeenv}
\begin{figure}[htbp]
 \begin{small}
\begin{verbatim}
#include <stdlib.h> /* EXIT_SUCCESS */
#include <stdio.h> /* printf, scanf */

/* 1) déclaration constantes et types utilisateurs */
#define ETE 1
#define AUTOMNE 2
#define HIVER 3

/* Fonction principale */
int main()
{
    int jour = 21;
    int mois = 9;
    int saison = AUTOMNE;
    
    /* 2 affichage */
    printf("Date : %d %d\n", jour, mois);


    /*3 calcul de la saison */
    if ((9 == mois) && (jour < 23)) 
    {
        saison = ETE;
    }
    if ((12 == mois) && (jour > 21)) 
    {
        saison = HIVER;
    }

    /* 3' optionnel (non demandé) */
    if ((mois < 9) || (mois > 12))
    {
      saison = 0;
      printf("Il n'y a plus de saisons !\n");
    }

    /*4 affichage */
    if (ETE == saison)
    {
        printf("C'est encore l'ete\n");
    }
    if (AUTOMNE == saison)
    {
        printf("C'est l'automne\n");
    }
    if (HIVER == saison)
    {
        printf("Deja l'hiver\n");
    }


   /* valeur fonction */
    return EXIT_SUCCESS;
}
\end{verbatim}
\end{small}
  \caption{Jours d'automne}
  \label{automne}
\end{figure}
  \end{correction}

\section{For ou while ?}
\subsection{Puissances successives d'un entier (\emph{4 points})}

\question Écrire un programme qui demande un entier $n$ à l'utilisateur et
calcule puis affiche $n^{10}$ ($n$ puissance dix). \bareme{2}
\paragraph{Exemple d'exécution.} 
\begin{small}
\begin{verbatim}
Donner un entier : 2
2 exposant 10 = 1024
\end{verbatim}
\end{small}


\begin{correction}
Voir figure~\ref{fig:puissance}.
  \begin{figure}[htbp]
    \begin{small}
      \begin{verbatim}
#include <stdlib.h> /* EXIT_SUCCESS */
#include <stdio.h> /* printf, scanf */
/* declarations de constantes et types utilisateurs */

/* declarations de fonctions utilisateurs */

/* fonction principale */
int main()
{
    int n = 0; /* base */
    int produit = 1;
    int i; /* var de boucle */

    /* saisie */
    printf("Donner un entier : ");
    scanf("%d", &n);

    /* calcul */
    for (i = 0; i < 10; i = i + 1) /* faire 10 fois 
                                     la multiplication par n */
    {
        produit = produit * n;
    }

    /* affichage */
    printf("%d exposant %d = %d\n", n, 10, produit);

    return EXIT_SUCCESS;
}
\end{verbatim}
    \end{small}
    \caption{Puissance dix}
    \label{fig:puissance}
  \end{figure}
  \begin{baremeenv}
    \begin{itemize}
    \item \pt{0.5} pour la saisie meme si oublie du \&
    \item \pt{1.5} pour le reste, zéro si pas de boucle sinon :
      \begin{itemize}
      \item enlever \pt{0.5} si c'est un while
      \item enlever \pt{0.5} par problème de déclaration, d'initialisation,
        ou d'include manquant. 
\item enlever \pt{0.5} si erreur sur le nombre d'étapes de boucle.
      \item on ne demande pas un affichage élaboré.
      \end{itemize}
    \end{itemize}
  \end{baremeenv}
\end{correction}

\question Comment modifier simplement votre programme pour qu'il
affiche le calcul des puissances successives de l'entier $n$ saisi,
de $n^0$ jusqu'à $n^{10}$ ?\bareme{1}


\paragraph{Exemple d'exécution.} 
\begin{small}
\begin{verbatim}
Donner un entier : 2
2 exposant 0 = 1
2 exposant 1 = 2
2 exposant 2 = 4
...
2 exposant 10 = 1024
\end{verbatim}
\end{small}
\begin{correction}
  Il suffit de placer l'affichage dans la boucle, au début du bloc.
\begin{small}
\begin{verbatim}
    /* calcul et affichage */
    for (i = 0; i < 10; i = i + 1) /* faire 10 fois 
                                     la multiplication par n */
    {
        /* affichage */
        printf("%d exposant %d = %d\n", n, i, produit);
        produit = produit * n;
    }
    /* affichage */
    printf("%d exposant %d = %d\n", n, 10, produit);
\end{verbatim}
\end{small}
  \begin{baremeenv}
    Compter \pt{0} pour toute autre solution. Pénalité de \pt{-0.5} si
    n'affiche pas l'exposant ou pas la bonne valeur
    d'exposant. Ne pas enlever de point si l'affichage ne va que
    jusqu'à l'exposant 9.
  \end{baremeenv}
\end{correction}


\question Comment modifier simplement votre programme pour qu'il
affiche le calcul des puissances successives de l'entier $n$
uniquement à partir de l'exposant $5$ ?\bareme{1}

\paragraph{Exemple d'exécution.} 
\begin{small}
\begin{verbatim}
Donner un entier : 2
2 exposant 5 = 32
2 exposant 6 = 64
...
2 exposant 10 = 1024
\end{verbatim}
\end{small}

\begin{correction}
il suffit d'encapsuler l'affichage de l'intérieur de boucle dans un
test :
\begin{small}
\begin{verbatim}
    /* calcul et affichage */
    for (i = 0; i < 10; i = i + 1) /* faire 10 fois 
                                     la multiplication par n */
    {
        /* affichage */
        if (i > 4) 
        {
            printf("%d exposant %d = %d\n", n, i, produit);
        }
        produit = produit * n;
    }
    /* affichage */
    printf("%d exposant %d = %d\n", n, 10, produit);
\end{verbatim}
\end{small}
  \begin{baremeenv}
    Autre solution admise : deux boucles (i allant de 0 à 4 puis i
    allant de 5 à 9)  à la suite l'une de l'autre. Sinon zéro.
 \end{baremeenv}
\end{correction}



\subsection{Unicité des éléments d'un tableau (\emph{4 points})}

Nous disposons d'un tableau $t$ de $N$ entiers. Utiliser une constante
symbolique pour $N$. Nous souhaitons
savoir si chaque entier apparaissant dans le tableau n'y apparaît
qu'une seule fois, autrement dit on veut savoir si chaque entier
est unique dans le tableau.

\question Écrire un programme qui, étant donné un tableau initalisé $t$,
  teste si le premier élément du tableau est unique et affiche
  \C{Vrai} si c'est le cas, \C{Faux} sinon.\bareme{2}
  \begin{correction} 
Voir figure~\ref{fig:unicite1}.
    \begin{figure}[htbp]
      \begin{small}
        \begin{listing}{1}
#include <stdlib.h> /* EXIT_SUCCESS */
#include <stdio.h> /* printf, scanf */

/* déclaration constantes et types utilisateurs */
#define N 3
#define TRUE 1
#define FALSE 0

/* Fonction principale */
int main()
{
  int t[N] = {10,10,15};
  int unique = TRUE;
  int i; /* var de boucle */

  i = 1;
  while (unique && (i < N)) /* tant que t[0] est unique */
  {
    if (t[i] == t[0]) /* t[0] n'est pas unique */
    {
      unique = FALSE;
    }
    i = i + 1; /* case suivante */
  }

  /* affichage du résultat */
  if (unique)
  {
    printf("Vrai\n");
  }
  else
  {
    printf("Faux\n");
  }
  /* valeur fonction */
  return EXIT_SUCCESS;
}
\end{listing}
      \end{small}
      \caption{Unicité des éléments d'un tableau}
      \label{fig:unicite1}
    \end{figure}

    \begin{baremeenv}
    \pt{2} si arrive au résultat avec une seule boucle (for ou
    while). L'affichage doit être correct mais on accepte les
    variantes : ''machin n'est
    pas unique / "machin est unique'' etc.

    Sinon maximum \pt{1,5} : \pt{0,5} s'il y a une seule boucle; \pt{0,5}
    de plus si c'est un while; ajouter \pt{0,5} si utilise une
    variable booléenne (déclarée et initialisée, et les define TRUE
    et FALSE).      
    \end{baremeenv}
  \end{correction}


\question Écrire un programme qui étant donné un tableau initialisé $t$, teste
  si tous les éléments sont uniques et affiche \C{Vrai} si c'est le
  cas, \C{Faux} sinon.\bareme{2}

  \begin{correction} 
Voir figure~\ref{fig:unicite2}.

    \begin{figure}[htbp]
      \begin{small}
        \begin{listing}{1}
#include <stdlib.h> /* EXIT_SUCCESS */
#include <stdio.h> /* printf, scanf */

/* déclaration constantes et types utilisateurs */
#define N 4
#define TRUE 1
#define FALSE 0

/* Fonction principale */
int main()
{
  int t[N] = {10,12,15,12};
  int unique = TRUE;
  int i; /* var de boucle */
  int j; /* var de boucle */

  j = 0; /* premiere case */
  while (unique && (j < N)) /* pour chaque case, verifier que la case
  est unique */
  { 
    i = j + 1; /* case suivant t[j] */
    while (unique && (i < N)) /* tant que t[j] est unique */
    {
      if (t[i] == t[j]) /* t[j] n'est pas unique */
      {
        unique = FALSE;
      }
      i = i + 1; /* case suivante */
    }
    j = j + 1; /* case suivante */
  }
  /* affichage du résultat */
  if (unique)
  {
    printf("Vrai\n");
  }
  else
  {
    printf("Faux\n");
  }
  /* valeur fonction */
  return EXIT_SUCCESS;
}
\end{listing}
      \end{small}
      \caption{Unicité de tous les éléments du tableau}
\label{fig:unicite2}
    \end{figure}

   \begin{baremeenv}
   \pt{2} si arrive au résultat avec deux boucles (avec affichage mais
   variantes acceptée). 

    Sinon maximum \pt{1,5} : \pt{0,5} s'il y a deux boucles avec
   variables de boucles différentes (quelles
   que soient les boucles et les bornes); \pt{0,5} si elles sont
   toutes deux des while; \pt{0,5} s'il a un algorithme correct en
   français en commentaire ou dans le texte; \pt{0,5} si utilise une
   variable boolénnee ou deux variables booléennes. 
   \end{baremeenv}
  \end{correction}

\subsection*{Question bonus (\emph{2 points})}
 Cette question est difficile et elle sera notée de manière très stricte. Il vaut
beaucoup mieux traiter toutes les autres questions avant de chercher à y
répondre. 
 \question Écrire un programme qui :
\begin{enumerate}
\item  demande à l'utilisateur d'entrer un
nombre $n$ dont on supposera qu'il est strictement supérieur à $1$.
\item tant que $n$ est différent de $1$, répète l'étape suivante :

  \hfill\fbox{\parbox{0.8\linewidth}{
   calculer une nouvelle valeur de $n$ ainsi :
    \begin{itemize}
    \item si $n$ est pair, le diviser par deux
    \item si $n$ est impair, le multiplier par trois et lui ajouter un
    \end{itemize}
    et afficher la nouvelle valeur de $n$.}}

\item affiche le nombre de fois où l'étape précédente a été répétée,
et la plus grande valeur prise par $n$ au cours de ce calcul.
\end{enumerate}
Ce programme termine-t-il toujours ?

\begin{correction}
  Cette question bonus sera traitée en TD à la fin du semestre.
\end{correction}
%\newpage
%\hfill\thetotalpointsint


%%% Local Variables: 
%%% mode: latex
%%% TeX-master: "partiel"
%%% End: 