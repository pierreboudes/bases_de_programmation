\documentclass[12pt,a4paper]{article}
\usepackage{latexsym,amssymb,amsmath}
%\usepackage[latin1]{inputenc}
\usepackage[utf8]{inputenc}
\usepackage[T1]{fontenc}
\usepackage[francais]{babel}

\usepackage[dvips]{graphicx}
\usepackage{listings}
\lstset{numbers=left, numberstyle=\tiny, stepnumber=1, numbersep=5pt}
%\lstset{language=C, showstringspaces=false}


%%%% Moreverb (inclusion de fichiers sources etc.)
\usepackage{moreverb}
\usepackage{rotating}


%\begin{verbatimtab}[tab width]
%...   <-respect des tabulations
% \end{verbatimtab}
%\verbatiminput[tab width]{file name}
%\begin{listing}[interval]{start line}
%...   <-numerotation
%\end{listing}
%\begin{listingcont}
%...   <-continuer un listing
%\end{listingcont}
%\listinginput[interval]{start line}{file name}

\topmargin      0mm
\oddsidemargin  -5mm
\evensidemargin -5mm
\textheight     253.5mm
\textwidth      170.0mm
\columnsep        4.1mm
\parindent        1.0em
\headsep          0mm
\headheight       0pt
\lineskip         0pt
\normallineskip   0pt
\def\baselinestretch{1}

\sloppy
\hbadness=10000

\newcommand{\C}[1]{{\upshape\texttt{#1}}}
\usepackage{version}
\newenvironment{correction}{\paragraph{Correction .}}{}

\newcommand{\commentaire}[1]{}
%\excludeversion{correction}
\newcommand{\entete}[1]{
        \noindent
        \rule{\linewidth}{0.5mm}\\
        \noindent
        Universit\'e Paris-Nord \hfill CP2I 1\\
        \noindent
        Institut Galil\'ee \hfill Ann\'ee 2008-2009\\
        Eléments d'Informatique - $1^{\text{er}}$ semestre
        \begin{center} {\bf\Large {#1}}
        \end{center}
        \rule{\linewidth}{0.5mm}
}



\newcommand{\question}[1]{\par\noindent\refstepcounter{section}
  \hspace{-.25cm}\fbox{\normalfont\Large\bfseries{Question
      \arabic{section}}} \hfill barême~: {#1} pt\bigskip}

\pagestyle{empty}

\begin{document}
\entete{Devoir sur table \no 1}

\question{2}
\label{q:exec_proc}

Décrire brièvement le cycle d'exécution du processeur dans une
machine de Von Neumann.

\bigskip





\question{2}
\label{q:exec_SE}

Décrire brièvement le cycle d'exécution d'un système d'exploitation
mono-tâche.



\bigskip


\question{3}
\label{q:echange}

Les questions suivantes concernent une machine de Von Neumann dont le
jeu d'instructions est donné à la fin du sujet. Il s'agit du jeu
d'instructions du simulateur AMIL.

\begin{enumerate}
\item Décrire (en français) un algorithme en trois étapes qui échange
  les contenus des mémoires 100 et 101.
\item Ecrire le programme correspondant en utilisant le jeu
  d'instructions d'AMIL. Commenter le programme afin de le rendre
  compréhensible.
\end{enumerate}



\begin{correction}
  \lstinputlisting[lastline=5]{echange.ail}

  \begin{tabular}[c]{l||c|c|c|c|c|c|}
\hline
 \emph{Instructions} & Cycles & CP& r0& r1& 100& 101\\ \hline
\hfill INIT & 0 & 1 & ? & ? & 18
 & 124
 \\ \hline \commentaire{Lecture de la donnée d'adresse 100 dans le registre 0
} \C{lecture 100 r0
} & 1 & 2  & 18 & & &\\ \hline
 \commentaire{Lecture de la donnée d'adresse 101 dans le registre 1
} \C{lecture 101 r1
} & 2 & 3  & & 124 & &\\ \hline
 \commentaire{Écriture du registre 1 à l'adresse 100
} \C{ecriture r1 100
} & 3 & 4  & & & 124
 &\\ \hline
 \commentaire{Écriture du registre 0 à l'adresse 101
} \C{ecriture r0 101
} & 4 & 5  & & & & 18
\\ \hline
 \commentaire{Fin du processus.
} \C{stop
} & 5 & 6  & & & &\\ \hline
\end{tabular}
\end{correction}

\bigskip

\question{5}
\label{q:allumage}

On considère un appartement équipé d'un chauffage électrique, et d'un
abonnement avec l'option \textit{Heures Pleines / Heures Creuses},
qui permet de payer moins chère l'électricité la nuit. Par soucis
d'économie, on souhaite que le chauffage se mette en route
automatiquement pendant les heures creuses et s'éteigne pendant les
heures pleines.

\bigskip

On suppose que le début des heures pleines est inscrit dans la
mémoire à l'adresse 100, que le début des heures creuses se trouve à
l'adresse 101 et que l'heure actuelle se trouve à l'adresse 102 (les
heures sont des entiers compris entre 0 et 24). On souhaite que la
machine AMIL décide de l'état du chauffage, et écrive à
l'adresse 103 la valeur 1 s'il faut que les radiateurs soient (ou restent)
allumés, ou bien la valeur 0 s'il doivent être (ou rester) éteints.

\bigskip


\begin{enumerate}
\item Décrire (en français) un algorithme répondant à ce souhait.
\item Ecrire et commenter le programme correspondant  en utilisant le jeu
  d'instructions d'AMIL.
\item Faire les traces d'exécution du programme~:
  \begin{enumerate}
  \item lorsque les cases mémoires 100, 101 et 102 contiennent
    respectivement les valeurs 7,~22,~et~1~;
  \item lorsque les cases mémoires 100, 101 et 102 contiennent
    respectivement les valeurs 7,~22~et~12~;
  \item lorsque les cases mémoires 100, 101 et 102 contiennent
    respectivement les valeurs 7,~22~et~23~!
  \end{enumerate}
\end{enumerate}

\begin{correction}
  \lstinputlisting[lastline=15]{chauffage.ail}

  \begin{tabular}[c]{l||c|c|c|c|c|c|c|}
\hline
 \emph{Instructions} & Cycles & CP& r0& r1& 100& 102& 103\\ \hline
\hfill INIT & 0 & 1 & ? & ? & 7
 & 1
 & ?
 \\ \hline \commentaire{Lecture de la donnée d'adresse 100 dans le registre 0
} \C{lecture 100 r0
} & 1 & 2  & 7 & & & &\\ \hline
 \commentaire{Inversion du signe de la valeur du registre 0
} \C{inverse r0
} & 2 & 3  & -7 & & & &\\ \hline
 \commentaire{Lecture de la donnée d'adresse 102 dans le registre 1
} \C{lecture 102 r1
} & 3 & 4  & & 1 & & &\\ \hline
 \commentaire{Ajout de la valeur du registre 1 au registre 0
} \C{add r1 r0
} & 4 & 5  & -6 & & & &\\ \hline
 \commentaire{Si la valeur (-6) du registre 0 est positive, saute à l'adresse 7
} \C{sisaut r0 7
} & 5 & 6  & & & & &\\ \hline
 \commentaire{Saut à l'adresse 14
} \C{saut 14
} & 6 & \textbf{14} & & & & &\\ \hline
 \commentaire{Initialisation du registre 0 à 1
} \C{init 1 r0
} & 7 & 15  & 1 & & & &\\ \hline
 \commentaire{Écriture du registre 0 à l'adresse 103
} \C{ecriture r0 103
} & 8 & 16  & & & & & 1
\\ \hline
 \commentaire{Fin du processus.
} \C{stop
} & 9 & 17  & & & & &\\ \hline
\end{tabular}

  \bigskip

  \begin{tabular}[c]{l||c|c|c|c|c|c|c|c|}
\hline
 \emph{Instructions} & Cycles & CP& r0& r1& 100& 101& 102& 103\\ \hline
\hfill INIT & 0 & 1 & ? & ? & 7
 & 22
 & 12
 & ?
 \\ \hline \commentaire{Lecture de la donnée d'adresse 100 dans le registre 0
} \C{lecture 100 r0
} & 1 & 2  & 7 & & & & &\\ \hline
 \commentaire{Inversion du signe de la valeur du registre 0
} \C{inverse r0
} & 2 & 3  & -7 & & & & &\\ \hline
 \commentaire{Lecture de la donnée d'adresse 102 dans le registre 1
} \C{lecture 102 r1
} & 3 & 4  & & 12 & & & &\\ \hline
 \commentaire{Ajout de la valeur du registre 1 au registre 0
} \C{add r1 r0
} & 4 & 5  & 5 & & & & &\\ \hline
 \commentaire{Si la valeur (5) du registre 0 est positive, saute à l'adresse 7
} \C{sisaut r0 7
} & 5 & \textbf{7} & & & & & &\\ \hline
 \commentaire{Lecture de la donnée d'adresse 101 dans le registre 0
} \C{lecture 101 r0
} & 6 & 8  & 22 & & & & &\\ \hline
 \commentaire{Inversion du signe de la valeur du registre 0
} \C{inverse r0
} & 7 & 9  & -22 & & & & &\\ \hline
 \commentaire{Ajout de la valeur du registre 1 au registre 0
} \C{add r1 r0
} & 8 & 10  & -10 & & & & &\\ \hline
 \commentaire{Si la valeur (-10) du registre 0 est positive, saute à l'adresse 14
} \C{sisaut r0 14
} & 9 & 11  & & & & & &\\ \hline
 \commentaire{Initialisation du registre 0 à 0
} \C{init 0 r0
} & 10 & 12  & 0 & & & & &\\ \hline
 \commentaire{Écriture du registre 0 à l'adresse 103
} \C{ecriture r0 103
} & 11 & 13  & & & & & & 0
\\ \hline
 \commentaire{Fin du processus.
} \C{stop
} & 12 & 14  & & & & & &\\ \hline
\end{tabular}

  \bigskip

  \begin{tabular}[c]{l||c|c|c|c|c|c|c|c|}
\hline
 \emph{Instructions} & Cycles & CP& r0& r1& 100& 101& 102& 103\\ \hline
\hfill INIT & 0 & 1 & ? & ? & 7
 & 22
 & 23
 & ?
 \\ \hline \commentaire{Lecture de la donnée d'adresse 100 dans le registre 0
} \C{lecture 100 r0
} & 1 & 2  & 7 & & & & &\\ \hline
 \commentaire{Inversion du signe de la valeur du registre 0
} \C{inverse r0
} & 2 & 3  & -7 & & & & &\\ \hline
 \commentaire{Lecture de la donnée d'adresse 102 dans le registre 1
} \C{lecture 102 r1
} & 3 & 4  & & 23 & & & &\\ \hline
 \commentaire{Ajout de la valeur du registre 1 au registre 0
} \C{add r1 r0
} & 4 & 5  & 16 & & & & &\\ \hline
 \commentaire{Si la valeur (16) du registre 0 est positive, saute à l'adresse 7
} \C{sisaut r0 7
} & 5 & \textbf{7} & & & & & &\\ \hline
 \commentaire{Lecture de la donnée d'adresse 101 dans le registre 0
} \C{lecture 101 r0
} & 6 & 8  & 22 & & & & &\\ \hline
 \commentaire{Inversion du signe de la valeur du registre 0
} \C{inverse r0
} & 7 & 9  & -22 & & & & &\\ \hline
 \commentaire{Ajout de la valeur du registre 1 au registre 0
} \C{add r1 r0
} & 8 & 10  & 1 & & & & &\\ \hline
 \commentaire{Si la valeur (1) du registre 0 est positive, saute à l'adresse 14
} \C{sisaut r0 14
} & 9 & \textbf{14} & & & & & &\\ \hline
 \commentaire{Initialisation du registre 0 à 1
} \C{init 1 r0
} & 10 & 15  & 1 & & & & &\\ \hline
 \commentaire{Écriture du registre 0 à l'adresse 103
} \C{ecriture r0 103
} & 11 & 16  & & & & & & 1
\\ \hline
 \commentaire{Fin du processus.
} \C{stop
} & 12 & 17  & & & & & &\\ \hline
\end{tabular}


\end{correction}

\bigskip


\question{4}
\label{q:factorielle}

On appelle $n$ l'entier stocké à l'adresse 100 de la mémoire.
\begin{enumerate}
\item Décrire en français l'algorithme permettant de placer $(n!)$ à
  l'adresse 101 de la mémoire.
\item Ecrire et commenter le programme correspondant en utilisant le jeu
  d'instructions d'AMIL.
\item Faire la trace d'exécution du programme lorsque $n$ vaut $3$.
\end{enumerate}


\begin{correction}

  \lstinputlisting[lastline=11]{factorielle.ail}

  \begin{tabular}[c]{l||c|c|c|c|c|c|c|}
\hline
 \emph{Instructions} & Cycles & CP& r0& r1& r2& 100& 101\\ \hline
\hfill INIT & 0 & 1 & ? & ? & ? & 3
 & ?
 \\ \hline \commentaire{Initialisation du registre 1 à 1
} \C{init 1 r1
} & 1 & 2  & & 1 & & &\\ \hline
 \commentaire{Initialisation du registre 2 à 1
} \C{init 1 r2
} & 2 & 3  & & & 1 & &\\ \hline
 \commentaire{Lecture de la donnée d'adresse 100 dans le registre 0
} \C{lecture 100 r0
} & 3 & 4  & 3 & & & &\\ \hline
 \commentaire{Inversion du signe de la valeur du registre 0
} \C{inverse r0
} & 4 & 5  & -3 & & & &\\ \hline
 \commentaire{Ajout de la valeur du registre 1 au registre 0
} \C{add r1 r0
} & 5 & 6  & -2 & & & &\\ \hline
 \commentaire{Si la valeur (-2) du registre 0 est positive, saute à l'adresse 10
} \C{sisaut r0 10
} & 6 & 7  & & & & &\\ \hline
 \commentaire{Ajout de la valeur 1 au registre 1
} \C{add 1 r1
} & 7 & 8  & & 2 & & &\\ \hline
 \commentaire{Multiplie la valeur du registre 2 par celle du registre 1
} \C{mult r1 r2
} & 8 & 9  & & & 2 & &\\ \hline
 \commentaire{Saut à l'adresse 3
} \C{saut 3
} & 9 & \textbf{3} & & & & &\\ \hline
 \commentaire{Lecture de la donnée d'adresse 100 dans le registre 0
} \C{lecture 100 r0
} & 10 & 4  & 3 & & & &\\ \hline
 \commentaire{Inversion du signe de la valeur du registre 0
} \C{inverse r0
} & 11 & 5  & -3 & & & &\\ \hline
 \commentaire{Ajout de la valeur du registre 1 au registre 0
} \C{add r1 r0
} & 12 & 6  & -1 & & & &\\ \hline
 \commentaire{Si la valeur (-1) du registre 0 est positive, saute à l'adresse 10
} \C{sisaut r0 10
} & 13 & 7  & & & & &\\ \hline
 \commentaire{Ajout de la valeur 1 au registre 1
} \C{add 1 r1
} & 14 & 8  & & 3 & & &\\ \hline
 \commentaire{Multiplie la valeur du registre 2 par celle du registre 1
} \C{mult r1 r2
} & 15 & 9  & & & 6 & &\\ \hline
 \commentaire{Saut à l'adresse 3
} \C{saut 3
} & 16 & \textbf{3} & & & & &\\ \hline
 \commentaire{Lecture de la donnée d'adresse 100 dans le registre 0
} \C{lecture 100 r0
} & 17 & 4  & 3 & & & &\\ \hline
 \commentaire{Inversion du signe de la valeur du registre 0
} \C{inverse r0
} & 18 & 5  & -3 & & & &\\ \hline
 \commentaire{Ajout de la valeur du registre 1 au registre 0
} \C{add r1 r0
} & 19 & 6  & 0 & & & &\\ \hline
 \commentaire{Si la valeur (0) du registre 0 est positive, saute à l'adresse 10
} \C{sisaut r0 10
} & 20 & \textbf{10} & & & & &\\ \hline
 \commentaire{Écriture du registre 2 à l'adresse 101
} \C{ecriture r2 101
} & 21 & 11  & & & & & 6
\\ \hline
 \commentaire{Fin du processus.
} \C{stop
} & 22 & 12  & & & & &\\ \hline
\end{tabular}

\end{correction}

\bigskip
\question{4}
\label{q:prod_scal}

\newcommand{\Z}{\mathbb{Z}}

Soient deux vecteurs de dimension $n$, dont les composantes sont des
entiers relatifs~: 

$$\vec{u}=\left(
\begin{array}[c]{c}
  u_1\\
  u_2\\
  \dots\\
  u_n
\end{array}
\right),\ \vec{v}=\left(
\begin{array}[c]{c}
  v_1\\
  v_2\\
  \dots\\
  v_n
\end{array}
\right) \text{ avec } u_1\dots u_n, v_1\dots v_n \in \Z.$$

Le but de cet exercice est de faire calculer leur produit scalaire
($\vec{u}\ .\ \vec{v}=\sum_{i=1}^n u_iv_i$) par le simulateur AMIL.

\bigskip

On suppose que la dimension $n$ des deux vecteurs est stockée à
l'adresse 100 de la mémoire.

Les composantes $u_1$, $u_2$, \dots, $u_n$ de $\vec{u}$ sont stockées
aux adresses $101,\ 102,\ \dots,\ 100+n$.

Les composantes $v_1$, $v_2$, \dots, $v_n$ de $\vec{v}$ sont stockées
aux adresses $100+n+1,\  100+n+2,\ \dots,\ 100+2n$.



\bigskip

\begin{enumerate}
\item Décrire en français un algorithme qui calcule le produit
  scalaire $\vec{u}\ .\ \vec{v}$ et qui stocke le résultat à l'adresse
  99 de la mémoire.
\item Ecrire le programme correspondant en utilisant le jeu
  d'instructions d'AMIL. Commenter le programme autant que possible.
  Préciser en particulier le rôle de chacun des registres que le
  programme utilise.
\end{enumerate}



\begin{correction}

  \lstinputlisting[lastline=18]{scalaire.ail}

  \begin{tabular}[c]{l||c|c|c|c|c|c|c|c|c|c|c|c|c|c|c|}
\hline
 \emph{Instructions} & Cycles & CP& r0& r1& r2& r3& r4& 99& 100& 101& 102& 103& 104& 105& 106\\ \hline
\hfill INIT & 0 & 1 & ? & ? & ? & ? & ? & ?
 & 3
 & 5
 & 3
 & 1
 & -3
 & 5
 & 0
 \\ \hline \commentaire{Initialisation du registre 0 à 0
} \C{init 0 r0
} & 1 & 2  & 0 & & & & & & & & & & & &\\ \hline
 \commentaire{Initialisation du registre 1 à 0
} \C{init 0 r1
} & 2 & 3  & & 0 & & & & & & & & & & &\\ \hline
 \commentaire{Lecture de la donnée d'adresse 100 dans le registre 2
} \C{lecture 100 r2
} & 3 & 4  & & & 3 & & & & & & & & & &\\ \hline
 \commentaire{Inversion du signe de la valeur du registre 2
} \C{inverse r2
} & 4 & 5  & & & -3 & & & & & & & & & &\\ \hline
 \commentaire{Ajout de la valeur du registre 0 au registre 2
} \C{add r0 r2
} & 5 & 6  & & & -3 & & & & & & & & & &\\ \hline
 \commentaire{Si la valeur (-3) du registre 2 est positive, saute à l'adresse 17
} \C{sisaut r2 17
} & 6 & 7  & & & & & & & & & & & & &\\ \hline
 \commentaire{Ajout de la valeur 1 au registre 0
} \C{add 1 r0
} & 7 & 8  & 1 & & & & & & & & & & & &\\ \hline
 \commentaire{Initialisation du registre 2 à 100
} \C{init 100 r2
} & 8 & 9  & & & 100 & & & & & & & & & &\\ \hline
 \commentaire{Ajout de la valeur du registre 0 au registre 2
} \C{add r0 r2
} & 9 & 10  & & & 101 & & & & & & & & & &\\ \hline
 \commentaire{Lecture de la donnée d'adresse 101 dans le registre 3
} \C{lecture *r2 r3
} & 10 & 11  & & & & 5 & & & & & & & & &\\ \hline
 \commentaire{Lecture de la donnée d'adresse 100 dans le registre 4
} \C{lecture 100 r4
} & 11 & 12  & & & & & 3 & & & & & & & &\\ \hline
 \commentaire{Ajout de la valeur du registre 4 au registre 2
} \C{add r4 r2
} & 12 & 13  & & & 104 & & & & & & & & & &\\ \hline
 \commentaire{Lecture de la donnée d'adresse 104 dans le registre 4
} \C{lecture *r2 r4
} & 13 & 14  & & & & & -3 & & & & & & & &\\ \hline
 \commentaire{Multiplie la valeur du registre 3 par celle du registre 4
} \C{mult r4 r3
} & 14 & 15  & & & & -15 & & & & & & & & &\\ \hline
 \commentaire{Ajout de la valeur du registre 3 au registre 1
} \C{add r3 r1
} & 15 & 16  & & -15 & & & & & & & & & & &\\ \hline
 \commentaire{Saut à l'adresse 3
} \C{saut 3
} & 16 & \textbf{3} & & & & & & & & & & & & &\\ \hline
 \commentaire{Lecture de la donnée d'adresse 100 dans le registre 2
} \C{lecture 100 r2
} & 17 & 4  & & & 3 & & & & & & & & & &\\ \hline
 \commentaire{Inversion du signe de la valeur du registre 2
} \C{inverse r2
} & 18 & 5  & & & -3 & & & & & & & & & &\\ \hline
 \commentaire{Ajout de la valeur du registre 0 au registre 2
} \C{add r0 r2
} & 19 & 6  & & & -2 & & & & & & & & & &\\ \hline
 \commentaire{Si la valeur (-2) du registre 2 est positive, saute à l'adresse 17
} \C{sisaut r2 17
} & 20 & 7  & & & & & & & & & & & & &\\ \hline
 \commentaire{Ajout de la valeur 1 au registre 0
} \C{add 1 r0
} & 21 & 8  & 2 & & & & & & & & & & & &\\ \hline
 \commentaire{Initialisation du registre 2 à 100
} \C{init 100 r2
} & 22 & 9  & & & 100 & & & & & & & & & &\\ \hline
 \commentaire{Ajout de la valeur du registre 0 au registre 2
} \C{add r0 r2
} & 23 & 10  & & & 102 & & & & & & & & & &\\ \hline
 \commentaire{Lecture de la donnée d'adresse 102 dans le registre 3
} \C{lecture *r2 r3
} & 24 & 11  & & & & 3 & & & & & & & & &\\ \hline
 \commentaire{Lecture de la donnée d'adresse 100 dans le registre 4
} \C{lecture 100 r4
} & 25 & 12  & & & & & 3 & & & & & & & &\\ \hline
 \commentaire{Ajout de la valeur du registre 4 au registre 2
} \C{add r4 r2
} & 26 & 13  & & & 105 & & & & & & & & & &\\ \hline
 \commentaire{Lecture de la donnée d'adresse 105 dans le registre 4
} \C{lecture *r2 r4
} & 27 & 14  & & & & & 5 & & & & & & & &\\ \hline
 \commentaire{Multiplie la valeur du registre 3 par celle du registre 4
} \C{mult r4 r3
} & 28 & 15  & & & & 15 & & & & & & & & &\\ \hline
 \commentaire{Ajout de la valeur du registre 3 au registre 1
} \C{add r3 r1
} & 29 & 16  & & 0 & & & & & & & & & & &\\ \hline
 \commentaire{Saut à l'adresse 3
} \C{saut 3
} & 30 & \textbf{3} & & & & & & & & & & & & &\\ \hline
 \commentaire{Lecture de la donnée d'adresse 100 dans le registre 2
} \C{lecture 100 r2
} & 31 & 4  & & & 3 & & & & & & & & & &\\ \hline
 \commentaire{Inversion du signe de la valeur du registre 2
} \C{inverse r2
} & 32 & 5  & & & -3 & & & & & & & & & &\\ \hline
 \commentaire{Ajout de la valeur du registre 0 au registre 2
} \C{add r0 r2
} & 33 & 6  & & & -1 & & & & & & & & & &\\ \hline
 \commentaire{Si la valeur (-1) du registre 2 est positive, saute à l'adresse 17
} \C{sisaut r2 17
} & 34 & 7  & & & & & & & & & & & & &\\ \hline
 \commentaire{Ajout de la valeur 1 au registre 0
} \C{add 1 r0
} & 35 & 8  & 3 & & & & & & & & & & & &\\ \hline
 \commentaire{Initialisation du registre 2 à 100
} \C{init 100 r2
} & 36 & 9  & & & 100 & & & & & & & & & &\\ \hline
 \commentaire{Ajout de la valeur du registre 0 au registre 2
} \C{add r0 r2
} & 37 & 10  & & & 103 & & & & & & & & & &\\ \hline
 \commentaire{Lecture de la donnée d'adresse 103 dans le registre 3
} \C{lecture *r2 r3
} & 38 & 11  & & & & 1 & & & & & & & & &\\ \hline
 \commentaire{Lecture de la donnée d'adresse 100 dans le registre 4
} \C{lecture 100 r4
} & 39 & 12  & & & & & 3 & & & & & & & &\\ \hline
 \commentaire{Ajout de la valeur du registre 4 au registre 2
} \C{add r4 r2
} & 40 & 13  & & & 106 & & & & & & & & & &\\ \hline
 \commentaire{Lecture de la donnée d'adresse 106 dans le registre 4
} \C{lecture *r2 r4
} & 41 & 14  & & & & & 0 & & & & & & & &\\ \hline
 \commentaire{Multiplie la valeur du registre 3 par celle du registre 4
} \C{mult r4 r3
} & 42 & 15  & & & & 0 & & & & & & & & &\\ \hline
 \commentaire{Ajout de la valeur du registre 3 au registre 1
} \C{add r3 r1
} & 43 & 16  & & 0 & & & & & & & & & & &\\ \hline
 \commentaire{Saut à l'adresse 3
} \C{saut 3
} & 44 & \textbf{3} & & & & & & & & & & & & &\\ \hline
 \commentaire{Lecture de la donnée d'adresse 100 dans le registre 2
} \C{lecture 100 r2
} & 45 & 4  & & & 3 & & & & & & & & & &\\ \hline
 \commentaire{Inversion du signe de la valeur du registre 2
} \C{inverse r2
} & 46 & 5  & & & -3 & & & & & & & & & &\\ \hline
 \commentaire{Ajout de la valeur du registre 0 au registre 2
} \C{add r0 r2
} & 47 & 6  & & & 0 & & & & & & & & & &\\ \hline
 \commentaire{Si la valeur (0) du registre 2 est positive, saute à l'adresse 17
} \C{sisaut r2 17
} & 48 & \textbf{17} & & & & & & & & & & & & &\\ \hline
 \commentaire{Écriture du registre 1 à l'adresse 99
} \C{ecriture r1 99
} & 49 & 18  & & & & & & 0
 & & & & & & &\\ \hline
 \commentaire{Fin du processus.
} \C{stop
} & 50 & 19  & & & & & & & & & & & & &\\ \hline
\end{tabular}

\end{correction}




\subsection*{Rappel du jeu d'instructions du simulateur AMIL}
\noindent
\begin{tabular}[c]{lp{13.2cm}}
  \C{stop} & Arrête l'exécution du programme.\\
  \C{noop} & N'effectue aucune opération.\\
  \C{saut i} & Met le compteur ordinal à la valeur $i$.\\
  \C{sisaut ri j} & Si la valeur contenue dans le registre i est positive ou nulle, met le compteur ordinal à la valeur $j$.\\
  \C{init x ri} & Initialise le registre $i$ avec la valeur $x$.\\
  \C{lecture i rj} & Charge, dans le registre $j$, le contenu de la mémoire d'adresse $i$.\\
  \C{lecture *ri rj} & Charge, dans le registre $j$, le contenu de la mémoire dont l'adresse est la valeur du registre $i$.\\
  \C{ecriture ri j} & Écrit le contenu du registre $i$ dans la mémoire d'adresse $j$.\\
  \C{ecriture ri *rj} & Écrit le contenu du registre $i$ dans la mémoire dont  l'adresse est la valeur du registre $j$.\\
  \C{inverse ri} & Inverse le signe du contenu du registre $i$.\\
  \C{add x rj} & Ajoute $x$ au contenu du registre $j$.\\
  \C{add ri rj} & Ajoute la valeur du registre $i$ à celle du registre $j$.\\
  \C{mult}, \C{div}, \C{et} & Même syntaxe que pour \C{add} mais pour la multiplication, la division entière et le et bit à bit.
\end{tabular}

\end{document}
