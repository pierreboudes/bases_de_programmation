\begin{tabular}[c]{l||c|c|c|c|c|c|c|}
\hline
 \emph{Instructions} & Cycles & CP& r0& r1& r2& 100& 101\\ \hline
\hfill INIT & 0 & 1 & ? & ? & ? & 3
 & ?
 \\ \hline \commentaire{Initialisation du registre 1 à 1
} \C{init 1 r1
} & 1 & 2  & & 1 & & &\\ \hline
 \commentaire{Initialisation du registre 2 à 1
} \C{init 1 r2
} & 2 & 3  & & & 1 & &\\ \hline
 \commentaire{Lecture de la donnée d'adresse 100 dans le registre 0
} \C{lecture 100 r0
} & 3 & 4  & 3 & & & &\\ \hline
 \commentaire{Inversion du signe de la valeur du registre 0
} \C{inverse r0
} & 4 & 5  & -3 & & & &\\ \hline
 \commentaire{Ajout de la valeur du registre 1 au registre 0
} \C{add r1 r0
} & 5 & 6  & -2 & & & &\\ \hline
 \commentaire{Si la valeur (-2) du registre 0 est positive, saute à l'adresse 10
} \C{sisaut r0 10
} & 6 & 7  & & & & &\\ \hline
 \commentaire{Ajout de la valeur 1 au registre 1
} \C{add 1 r1
} & 7 & 8  & & 2 & & &\\ \hline
 \commentaire{Multiplie la valeur du registre 2 par celle du registre 1
} \C{mult r1 r2
} & 8 & 9  & & & 2 & &\\ \hline
 \commentaire{Saut à l'adresse 3
} \C{saut 3
} & 9 & \textbf{3} & & & & &\\ \hline
 \commentaire{Lecture de la donnée d'adresse 100 dans le registre 0
} \C{lecture 100 r0
} & 10 & 4  & 3 & & & &\\ \hline
 \commentaire{Inversion du signe de la valeur du registre 0
} \C{inverse r0
} & 11 & 5  & -3 & & & &\\ \hline
 \commentaire{Ajout de la valeur du registre 1 au registre 0
} \C{add r1 r0
} & 12 & 6  & -1 & & & &\\ \hline
 \commentaire{Si la valeur (-1) du registre 0 est positive, saute à l'adresse 10
} \C{sisaut r0 10
} & 13 & 7  & & & & &\\ \hline
 \commentaire{Ajout de la valeur 1 au registre 1
} \C{add 1 r1
} & 14 & 8  & & 3 & & &\\ \hline
 \commentaire{Multiplie la valeur du registre 2 par celle du registre 1
} \C{mult r1 r2
} & 15 & 9  & & & 6 & &\\ \hline
 \commentaire{Saut à l'adresse 3
} \C{saut 3
} & 16 & \textbf{3} & & & & &\\ \hline
 \commentaire{Lecture de la donnée d'adresse 100 dans le registre 0
} \C{lecture 100 r0
} & 17 & 4  & 3 & & & &\\ \hline
 \commentaire{Inversion du signe de la valeur du registre 0
} \C{inverse r0
} & 18 & 5  & -3 & & & &\\ \hline
 \commentaire{Ajout de la valeur du registre 1 au registre 0
} \C{add r1 r0
} & 19 & 6  & 0 & & & &\\ \hline
 \commentaire{Si la valeur (0) du registre 0 est positive, saute à l'adresse 10
} \C{sisaut r0 10
} & 20 & \textbf{10} & & & & &\\ \hline
 \commentaire{Écriture du registre 2 à l'adresse 101
} \C{ecriture r2 101
} & 21 & 11  & & & & & 6
\\ \hline
 \commentaire{Fin du processus.
} \C{stop
} & 22 & 12  & & & & &\\ \hline
\end{tabular}