\begin{tabular}[c]{l||c|c|c|c|c|c|c|c|c|c|c|c|c|c|c|}
\hline
 \emph{Instructions} & Cycles & CP& r0& r1& r2& r3& r4& 99& 100& 101& 102& 103& 104& 105& 106\\ \hline
\hfill INIT & 0 & 1 & ? & ? & ? & ? & ? & ?
 & 3
 & 5
 & 3
 & 1
 & -3
 & 5
 & 0
 \\ \hline \commentaire{Initialisation du registre 0 à 0
} \C{init 0 r0
} & 1 & 2  & 0 & & & & & & & & & & & &\\ \hline
 \commentaire{Initialisation du registre 1 à 0
} \C{init 0 r1
} & 2 & 3  & & 0 & & & & & & & & & & &\\ \hline
 \commentaire{Lecture de la donnée d'adresse 100 dans le registre 2
} \C{lecture 100 r2
} & 3 & 4  & & & 3 & & & & & & & & & &\\ \hline
 \commentaire{Inversion du signe de la valeur du registre 2
} \C{inverse r2
} & 4 & 5  & & & -3 & & & & & & & & & &\\ \hline
 \commentaire{Ajout de la valeur du registre 0 au registre 2
} \C{add r0 r2
} & 5 & 6  & & & -3 & & & & & & & & & &\\ \hline
 \commentaire{Si la valeur (-3) du registre 2 est positive, saute à l'adresse 17
} \C{sisaut r2 17
} & 6 & 7  & & & & & & & & & & & & &\\ \hline
 \commentaire{Ajout de la valeur 1 au registre 0
} \C{add 1 r0
} & 7 & 8  & 1 & & & & & & & & & & & &\\ \hline
 \commentaire{Initialisation du registre 2 à 100
} \C{init 100 r2
} & 8 & 9  & & & 100 & & & & & & & & & &\\ \hline
 \commentaire{Ajout de la valeur du registre 0 au registre 2
} \C{add r0 r2
} & 9 & 10  & & & 101 & & & & & & & & & &\\ \hline
 \commentaire{Lecture de la donnée d'adresse 101 dans le registre 3
} \C{lecture *r2 r3
} & 10 & 11  & & & & 5 & & & & & & & & &\\ \hline
 \commentaire{Lecture de la donnée d'adresse 100 dans le registre 4
} \C{lecture 100 r4
} & 11 & 12  & & & & & 3 & & & & & & & &\\ \hline
 \commentaire{Ajout de la valeur du registre 4 au registre 2
} \C{add r4 r2
} & 12 & 13  & & & 104 & & & & & & & & & &\\ \hline
 \commentaire{Lecture de la donnée d'adresse 104 dans le registre 4
} \C{lecture *r2 r4
} & 13 & 14  & & & & & -3 & & & & & & & &\\ \hline
 \commentaire{Multiplie la valeur du registre 3 par celle du registre 4
} \C{mult r4 r3
} & 14 & 15  & & & & -15 & & & & & & & & &\\ \hline
 \commentaire{Ajout de la valeur du registre 3 au registre 1
} \C{add r3 r1
} & 15 & 16  & & -15 & & & & & & & & & & &\\ \hline
 \commentaire{Saut à l'adresse 3
} \C{saut 3
} & 16 & \textbf{3} & & & & & & & & & & & & &\\ \hline
 \commentaire{Lecture de la donnée d'adresse 100 dans le registre 2
} \C{lecture 100 r2
} & 17 & 4  & & & 3 & & & & & & & & & &\\ \hline
 \commentaire{Inversion du signe de la valeur du registre 2
} \C{inverse r2
} & 18 & 5  & & & -3 & & & & & & & & & &\\ \hline
 \commentaire{Ajout de la valeur du registre 0 au registre 2
} \C{add r0 r2
} & 19 & 6  & & & -2 & & & & & & & & & &\\ \hline
 \commentaire{Si la valeur (-2) du registre 2 est positive, saute à l'adresse 17
} \C{sisaut r2 17
} & 20 & 7  & & & & & & & & & & & & &\\ \hline
 \commentaire{Ajout de la valeur 1 au registre 0
} \C{add 1 r0
} & 21 & 8  & 2 & & & & & & & & & & & &\\ \hline
 \commentaire{Initialisation du registre 2 à 100
} \C{init 100 r2
} & 22 & 9  & & & 100 & & & & & & & & & &\\ \hline
 \commentaire{Ajout de la valeur du registre 0 au registre 2
} \C{add r0 r2
} & 23 & 10  & & & 102 & & & & & & & & & &\\ \hline
 \commentaire{Lecture de la donnée d'adresse 102 dans le registre 3
} \C{lecture *r2 r3
} & 24 & 11  & & & & 3 & & & & & & & & &\\ \hline
 \commentaire{Lecture de la donnée d'adresse 100 dans le registre 4
} \C{lecture 100 r4
} & 25 & 12  & & & & & 3 & & & & & & & &\\ \hline
 \commentaire{Ajout de la valeur du registre 4 au registre 2
} \C{add r4 r2
} & 26 & 13  & & & 105 & & & & & & & & & &\\ \hline
 \commentaire{Lecture de la donnée d'adresse 105 dans le registre 4
} \C{lecture *r2 r4
} & 27 & 14  & & & & & 5 & & & & & & & &\\ \hline
 \commentaire{Multiplie la valeur du registre 3 par celle du registre 4
} \C{mult r4 r3
} & 28 & 15  & & & & 15 & & & & & & & & &\\ \hline
 \commentaire{Ajout de la valeur du registre 3 au registre 1
} \C{add r3 r1
} & 29 & 16  & & 0 & & & & & & & & & & &\\ \hline
 \commentaire{Saut à l'adresse 3
} \C{saut 3
} & 30 & \textbf{3} & & & & & & & & & & & & &\\ \hline
 \commentaire{Lecture de la donnée d'adresse 100 dans le registre 2
} \C{lecture 100 r2
} & 31 & 4  & & & 3 & & & & & & & & & &\\ \hline
 \commentaire{Inversion du signe de la valeur du registre 2
} \C{inverse r2
} & 32 & 5  & & & -3 & & & & & & & & & &\\ \hline
 \commentaire{Ajout de la valeur du registre 0 au registre 2
} \C{add r0 r2
} & 33 & 6  & & & -1 & & & & & & & & & &\\ \hline
 \commentaire{Si la valeur (-1) du registre 2 est positive, saute à l'adresse 17
} \C{sisaut r2 17
} & 34 & 7  & & & & & & & & & & & & &\\ \hline
 \commentaire{Ajout de la valeur 1 au registre 0
} \C{add 1 r0
} & 35 & 8  & 3 & & & & & & & & & & & &\\ \hline
 \commentaire{Initialisation du registre 2 à 100
} \C{init 100 r2
} & 36 & 9  & & & 100 & & & & & & & & & &\\ \hline
 \commentaire{Ajout de la valeur du registre 0 au registre 2
} \C{add r0 r2
} & 37 & 10  & & & 103 & & & & & & & & & &\\ \hline
 \commentaire{Lecture de la donnée d'adresse 103 dans le registre 3
} \C{lecture *r2 r3
} & 38 & 11  & & & & 1 & & & & & & & & &\\ \hline
 \commentaire{Lecture de la donnée d'adresse 100 dans le registre 4
} \C{lecture 100 r4
} & 39 & 12  & & & & & 3 & & & & & & & &\\ \hline
 \commentaire{Ajout de la valeur du registre 4 au registre 2
} \C{add r4 r2
} & 40 & 13  & & & 106 & & & & & & & & & &\\ \hline
 \commentaire{Lecture de la donnée d'adresse 106 dans le registre 4
} \C{lecture *r2 r4
} & 41 & 14  & & & & & 0 & & & & & & & &\\ \hline
 \commentaire{Multiplie la valeur du registre 3 par celle du registre 4
} \C{mult r4 r3
} & 42 & 15  & & & & 0 & & & & & & & & &\\ \hline
 \commentaire{Ajout de la valeur du registre 3 au registre 1
} \C{add r3 r1
} & 43 & 16  & & 0 & & & & & & & & & & &\\ \hline
 \commentaire{Saut à l'adresse 3
} \C{saut 3
} & 44 & \textbf{3} & & & & & & & & & & & & &\\ \hline
 \commentaire{Lecture de la donnée d'adresse 100 dans le registre 2
} \C{lecture 100 r2
} & 45 & 4  & & & 3 & & & & & & & & & &\\ \hline
 \commentaire{Inversion du signe de la valeur du registre 2
} \C{inverse r2
} & 46 & 5  & & & -3 & & & & & & & & & &\\ \hline
 \commentaire{Ajout de la valeur du registre 0 au registre 2
} \C{add r0 r2
} & 47 & 6  & & & 0 & & & & & & & & & &\\ \hline
 \commentaire{Si la valeur (0) du registre 2 est positive, saute à l'adresse 17
} \C{sisaut r2 17
} & 48 & \textbf{17} & & & & & & & & & & & & &\\ \hline
 \commentaire{Écriture du registre 1 à l'adresse 99
} \C{ecriture r1 99
} & 49 & 18  & & & & & & 0
 & & & & & & &\\ \hline
 \commentaire{Fin du processus.
} \C{stop
} & 50 & 19  & & & & & & & & & & & & &\\ \hline
\end{tabular}