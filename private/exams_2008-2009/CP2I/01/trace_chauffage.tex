\begin{tabular}[c]{l||c|c|c|c|c|c|c|}
\hline
 \emph{Instructions} & Cycles & CP& r0& r1& 100& 102& 103\\ \hline
\hfill INIT & 0 & 1 & ? & ? & 7
 & 1
 & ?
 \\ \hline \commentaire{Lecture de la donnée d'adresse 100 dans le registre 0
} \C{lecture 100 r0
} & 1 & 2  & 7 & & & &\\ \hline
 \commentaire{Inversion du signe de la valeur du registre 0
} \C{inverse r0
} & 2 & 3  & -7 & & & &\\ \hline
 \commentaire{Lecture de la donnée d'adresse 102 dans le registre 1
} \C{lecture 102 r1
} & 3 & 4  & & 1 & & &\\ \hline
 \commentaire{Ajout de la valeur du registre 1 au registre 0
} \C{add r1 r0
} & 4 & 5  & -6 & & & &\\ \hline
 \commentaire{Si la valeur (-6) du registre 0 est positive, saute à l'adresse 7
} \C{sisaut r0 7
} & 5 & 6  & & & & &\\ \hline
 \commentaire{Saut à l'adresse 13
} \C{saut 13
} & 6 & \textbf{13} & & & & &\\ \hline
 \commentaire{Initialisation du registre 0 à 1
} \C{init 1 r0
} & 7 & 14  & 1 & & & &\\ \hline
 \commentaire{Écriture du registre 0 à l'adresse 103
} \C{ecriture r0 103
} & 8 & 15  & & & & & 1
\\ \hline
 \commentaire{Fin du processus.
} \C{stop
} & 9 & 16  & & & & &\\ \hline
\end{tabular}