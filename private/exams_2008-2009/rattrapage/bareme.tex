\documentclass[12pt,a4paper]{article}

%%%% PACKAGES
\usepackage[scale=0.74,vmargin=1.4cm ]{geometry}

\usepackage{latexsym,amssymb,amsmath}
\usepackage[utf8]{inputenc}
\usepackage[T1]{fontenc}
\usepackage[francais]{babel}
\usepackage{textcomp} %pour euro
\usepackage{ifpdf}
\ifpdf
\usepackage[pdftex]{graphicx}
\else
\usepackage[dvips]{graphicx}
\usepackage{pstricks, pst-tree, pst-node}
\fi 
\usepackage{calc}
\usepackage{moreverb}
\usepackage{version}
\usepackage{url}
\usepackage{multicol}
\usepackage{listings}
\usepackage{tikz}
\usetikzlibrary{shapes,arrows}
\usepackage{rotating}
\usepackage{subfigure}

\lstset{language=C, showstringspaces=false}
\usepackage{enumerate}

%%%% COMMANDES
\newcommand{\entete}[1]{%
\noindent
\rule{\linewidth}{0.5mm}\\
\noindent
Universit\'e Paris-Nord \hfill Ann\'ee 2012-2013\\
\noindent
Algorithmique et programmation \hfill DUT R\&T 1\\
\rule{\linewidth}{0.5mm}
\begin{center}
\Large\bf #1
\end{center}
\vspace{0.5cm}
}

\newcommand{\C}[1]{{\upshape\texttt{#1}}}
\newenvironment{correction}{\paragraph{Correction.}\hrulefill\begin{itshape}}{\end{itshape}\hrulefill}
\newenvironment{cpii}{}{} % pour les liens spécifiques au CP2I
\excludeversion{cpii}
\newenvironment{baremeenv}{\paragraph{Barème.} \begin{itshape}}{\end{itshape}}
%\excludeversion{baremeenv}

% Hack tres moche pour des enumerate qui continuent la numérotation...
% j'ai deja fait ca ailleurs et mieux mais ou ?
\newcounter{saveenumi}
\newenvironment{newenu}
{\begin{enumerate}}
{\setcounter{saveenumi}{\value{enumi}}\end{enumerate}}
\newenvironment{lastenu}
{\begin{enumerate}\setcounter{enumi}{\value{saveenumi}}}
{\setcounter{saveenumi}{\value{enumi}}\end{enumerate}}

\newcommand{\bbbn}{\ensuremath{\mathbb{N}}}
\newcommand{\bbbr}{\ensuremath{\mathbb{R}}}

\newcommand{\carriagereturn}{\C{$\backslash$n}}
%%% Local Variables: 
%%% mode: latex
%%% TeX-master: t
%%% End: 

\usepackage{multicol}
\begin{document}

\newcommand{\bexo}[1]{\section*{Exercice #1}}
\newcommand{\bquestion}[1]{\subsection*{Question #1}}
\newcommand{\breponse}[1]{\noindent Réponse : #1}
\newcommand{\bnote}[1]{#1}
\title{Barème}
%\maketitle
\bexo{1}
\bquestion{1}
\breponse{}\verb+int quotient_par_soustraction(int a, int b)+

\begin{itemize}
\item 1 pt si juste et au bon endroit

\item Enlever 0,5 pt par
    erreur de type (double au lieu de int, void au lieu de int pour le
    retour) pas de point en moins sur les noms des variables

  \item Enlever 0,5 pt
    si pas au bon endroit dans le programme
\end{itemize}


\bquestion{2}
\breponse{La compilation fonctionne mais l'édition de lien échoue car la fonction \C{quotient...} n'a pas de définition}

\begin{itemize} 
\item \bnote{1 point} 

\item \bnote{0,25 point si pas de
    justification} 
\end{itemize}

\bquestion{3 sur 2 pt}
\begin{itemize}
\item \bnote{Si la méthode emploie la division 0pt}
\item \bnote{S'il n'y a pas de boucle ou trop de boucles : 0pt}
\item \bnote{Si c'est par une autre méthode que la soustraction (sans employer la division) : on ne donne que 0,5 pt.}
\item \bnote{Si exactement une boucle et méthode par soustraction et le bon quotient est trouvé (à des erreurs syntaxiques marginales, type oublie de point virgule) : 2 points  (pas de points en moins si c'est un for () dans cet exo)}
\item \bnote{Si il manque un élément structurel important autre que la boucle pour que ça fonctionne (valeur de retour, condition de boucle) on enlève 1 pt}
\item \bnote{Si le quotient trouvé est q - 1 ou q + 1 : on enlève 0,25 pt}
\item \bnote{Si la valeur retournée n'est pas déclarée : on enlève 0,25 pt}
\end{itemize}


\bexo{2.1 sur 3,5 pt}
\begin{itemize}
\item \bnote{Si squelette de programme correct : 0,25pt}
\item \bnote{Si \C{define TAILLE 4} (ou autre entier) : 0,25 pt}
\item \bnote{Si ça ne fonctionne pas du tout, mais que la structure du programme respecte : boucle de saisie, saisie, boucle de comptage, on donne 1pt.}
\item \bnote{Sinon}
\item \bnote{1,5 pt pour la saisie des entiers (affichage, scanf dans boucle for, affichage, scanf)}
\item \bnote{1,5 pt pour le nombre d'occurrences calculé par une boucle for}
\item \bnote{On enlève 0,5 pt par boucle while au lieu de for}
\item \bnote{Si le nombre d'occurrences est calculé à un près on enlève 0,25pt}
\item \bnote{Si les déclarations sont fausse ou inexistantes : on enlève 0,5pt si ça concerne le tableau (et d'autres variables), 0,25 pt si une seule variable autre que le tableau}
\item \bnote{Si il y a une erreur de syntaxe importante dans un scanf : on enlève 0,5pt}
\item \bnote{Si il y a une erreur de syntaxe importante dans le printf du résultat : on enlève 0,5pt}
\end{itemize}

\bexo{2.2 sur 3,5 pt}
\begin{itemize}
\item \bnote{Si squelette de programme correct : 0,25pt}
\item \bnote{Si il n'y a pas de boucle ou trop de boucles 0pt pour le reste}
\item \bnote{Si utilise un tableau et fonctionne : seulement 1 pt}
\item \bnote{Si ça ne fonctionne pas du tout, on note la structure : c'est  une boucle (0,5pt) dont le corps fait de la saisie (0,5pt) et du comptage avec if (0,5pt).}
\item \bnote{Si le programme fait ce qui est demandé avec une unique boucle on part sur 3,25pt}
\item \bnote{Si les déclarations sont fausse ou inexistantes : on enlève 0,5pt si ça concerne plusieurs variables, 0,25 pt si une seule variable.}
\item \bnote{Si la boucle n'est ni un while ni un do while (donc un for) : on enlève 0,5pt}
\item \bnote{Si la condition de boucle est fausse on enlève 1pt ou 0,5 pt si c'est presque bon}
\item \bnote{Pas de comptage on enlève 1,5.}
\item \bnote{Si les résultats du comptage sont faux, on enlève 0,5pt}
\item \bnote{Si il y a une erreur de syntaxe importante dans un scanf : on enlève 0,5pt}
\item \bnote{Si il y a une erreur de syntaxe importante dans le printf du résultat : on enlève 0,5pt}
\end{itemize}

\bexo{3 sur 4 pt}
\begin{itemize}
\item \bnote{2 pt pour le tableau du main, 2pt pour le tableau de puissance(4, 3)}
\item \bnote{On enlève 1 point si le for ne se termine pas sur un i valant 3}.
\item \bnote{On enlève 0,5 pt par colonne ayant une erreur : numéro de ligne faux, variable avec une mauvaise valeur ou  variable oubliée, affichage erroné, ...}
\item \bnote{On enlève 0,5 pt si la valeur de retour est incorrecte.}
\end{itemize}

\bexo{4 sur 5 pt}
\begin{itemize}
\item \bnote{Un point par exercice tout juste. On admet les prototypes avec seulement $n$ comme avec $n$ et $m$ aux deux dernières questions.}
\item \bnote{0,25 si seul le prototype de la fonction est juste.}
\item \bnote{si le prototype est faux : on mets 0 et on arrête là pour les trois premières questions on note sur 0,5 pt maximum les deux dernières.}
\item \bnote{Pour chacune des deux dernières question, si le calcule est faux mais qu'il y a deux for imbriqués on ajoute 0,25pt.}
\end{itemize}
\end{document}
