% -*- coding: utf-8 -*-

\newcommand{\commentaire}[1]{}

\entete{Partiel d'Éléments d'Informatique}

\begin{description}
\item[Durée :] 3 heures.
\item[Documents autorisés :] Aucun.
\item[Recommandations :] Un barême vous est donné à
titre indicatif afin de vous permettre de gérer votre temps. La
notation prendra en compte à la fois la syntaxe et la sémantique de
vos programmes, c'est-à-dire qu'ils doivent compiler correctement. Une
fois votre programme écrit, il est recommandé de le faire tourner à la
main sur un exemple pour s'assurer de sa correction.
\end{description}

\section{Compilation, édition de liens et exécution d'un programme (\textit{5 points})}
\label{tableau_sans_zero}

Nous étudions un programme qui travaille sur un tableau d'entiers déjà
initialisé. Il supprime de ce tableau tous les zéros et met sa
taille à jour pour prendre en compte les suppressions. Le programme est le suivant :

\begin{small}
\begin{listing}{1}
#include <stdlib.h> /* EXIT_SUCCESS */

/* déclaration constantes et types utilisateurs */

/* déclaration de fonctions utilisateurs */

int main()
{
    /* un tableau et sa taille */
    int tab[5] = {0,1,0,2,3};
    int taille = 5;
    int nouvelle_taille = 0; /* la nouvelle taille du tableau modifié */
    int i; /* var boucle */
 
    /* suppression des zéros dans tab */
    for(i = 0;i < taille;i = i + 1)
    {
        if(tab[i] != 0) /* valeur à conserver */
        {
            tab[nouvelle_taille] = tab[i];
            nouvelle_taille = nouvelle_taille + 1;
        }
    }
    /* i >= taille */
    /* les zéros ont été supprimés,
       mise à jour de la taille */
    taille = nouvelle_taille;
 
    /* affichage du tableau pour vérifier la suppression */
    printf("taille : %d\n",taille);
    for(i = 0;i < taille;i = i + 1)
    {
        printf("%d ",tab[i]);
    }
    /* i >= taille */
    printf("\n");
 
    return EXIT_SUCCESS;
}

/* definitions des fonctions utilisateurs */
\end{listing}
\end{small}

\begin{enumerate}
\item Cependant, ce programme est erroné. Expliquez succintement pourquoi le programme ci-dessus est incomplet et ce
  qu'il manque pour que la compilation réussisse. Corrigez le programme pour qu'il compile.
\begin{correction}
  Le programme utilise la fonction \verb|printf|, pour l'affichage du tableau à l'écran, qui est une fonctionnalité disponible dans la bibliothèque \verb|stdio|. Or, ses fonctionnalités ne sont pas déclarées par la directive du pré-processeur : \verb|#include <stdio.h> /* printf */|. La compilation va échouer à la première occurrence de \verb|printf| dans le programme, c'est-à-dire à la ligne 30.
\end{correction}
\item Après la compilation, décrivez succintement, en vos propres termes, ce qu'il se passe lors de la création de l'exécutable par l'éditeur de liens.

  \begin{correction}
    L'édition de liens transforme le code objet résultant de la
    compilation en fichier exécutable. Durant cette étape, le code
    objet des fonctions externes (bibliothèques) est ajouté à
    l'exécutable et le point d'entrée dans le programme est fixé (sur
    le main).
  \end{correction}

\item Afin de bien comprendre la sémantique du programme, simulez
  l'exécution du programme, en réalisant sa \textbf{trace} : l'exécution
  du programme est représenté par un tableau à $n + 2$ colonnes : la
  première colonne étant le numéro de la ligne exécutée, les $n$
  colonnes suivantes, les colonnes des $n$ variables du programme ($n$ à
  déterminer) et la dernière colonne, la colonne servant à
  l'affichage à l'écran du programme. Seules les lignes modifiant l'état mémoire du programme sont à reporter dans le tableau.
  \begin{correction}
    La correction est donnée dans le tableau \ref{simulation}.
    \begin{table}[h]
      \begin{tiny}
	\begin{center}
		\begin{tabular}[h]{|c|c|c|c|c|c|c|c|c|c|}
                  \hline
			ligne & tab[0] & tab[1] & tab[2] & tab[3] & tab[4]& taille & nouvelle\_taille & i & Affichage (sortie écran) \\ 
		\hline
           initialisation  &0 &1 &0 &2 &3 &5 &0 &? &  \\ \hline
			16 &  &  &  &  &  &  &  &0 &  \\ \hline
                        23 &  &  &  &  &  &  &  &1 &  \\ \hline
			20 &1 &  &  &  &  &  &  &  &  \\ \hline
			21 &  &  &  &  &  &  &1 &  &  \\ \hline
			23 &  &  &  &  &  &  &  &2 &  \\ \hline
			23 &  &  &  &  &  &  &  &3 &  \\ \hline
			20 &  &2 &  &  &  &  &  &  &  \\ \hline
			21 &  &  &  &  &  &  &2 &  &  \\ \hline
			23 &  &  &  &  &  &  &  &4 &  \\ \hline
			20 &  &  &3 &  &  &  &  &  &  \\ \hline
			21 &  &  &  &  &  &  &3 &  &  \\ \hline
			23 &  &  &  &  &  &  &  &5 &  \\ \hline
			27 &  &  &  &  &  &3 &  &  &  \\ \hline
			30 &  &  &  &  &  &  &  &  & taille : 3(passage à la ligne) \\ \hline
			31 &  &  &  &  &  &  &  &0 &  \\ \hline
			30 &  &  &  &  &  &  &  &  & 1(espace) \\ \hline
			34 &  &  &  &  &  &  &  &1 &  \\ \hline
			30 &  &  &  &  &  &  &  &  & 2(espace) \\ \hline
			34 &  &  &  &  &  &  &  &2 &  \\ \hline
			30 &  &  &  &  &  &  &  &  & 3(espace) \\ \hline
			34 &  &  &  &  &  &  &  &3 &  \\ \hline
			36 &  &  &  &  &  &  &  &  & (passage à la ligne) \\ \hline
		\end{tabular}
		\caption{Trace du programme donné à l'exercice \ref{tableau_sans_zero}.}
		\label{simulation}
	\end{center}
      \end{tiny}
    \end{table}
  \end{correction}
\end{enumerate}

\section{Les tableaux (\textit{5 points})}

\subsection{Calcul du produit des éléments d'un tableau (\textit{2,5 points})}

Écrire le programme qui demande à l'utilisateur d'entrer les valeurs d'un tableau d'entiers de taille \verb|N| (une constante symbolique), et qui calcule et affiche la valeur du produit des entiers du tableau. Deux exemples d'exécution sont les suivants (pour \verb|N| valant 4) :
\begin{small}
\begin{verbatim}
Saisissez 4 entiers : 2 -3 6 -10
Le produit des éléments du tableau vaut 360.

Saisissez 4 entiers : 3 -3 4 0
Le produit des éléments du tableau vaut 0.
\end{verbatim}
\end{small}

\begin{correction}
\begin{small}
\begin{verbatim}
#include <stdlib.h> /* EXIT_SUCCESS */
#include <stdio.h> /* printf, scanf */

/* déclaration constantes et types utilisateurs */
#define TAILLE 4 /* taille du tableau utilisateur */

/* déclaration de fonctions utilisateurs */

int main()
{
    int tab[TAILLE]; /* tableau à initialiser par l'utilisateur */
    int produit = 1; /* élément neutre de la multiplication */
    int i; /* var. de boucle */

    /* demande saisie de TAILLE entiers*/
    printf("Saisissez %d entiers : ",TAILLE);

    /* saisie des elts du tableau (TAILLE entiers) */
    for(i = 0;i < TAILLE;i = i + 1) /* chaque case du tableau */
    {
        /* saisie valeur */
        scanf("%d",&tab[i]);
    }
    /* i >= TAILLE */

    /* calcul produit */
    for(i = 0;i < TAILLE;i = i + 1) /* chaque case */
    {
        /* multiplie le produit partiel par la valeur de la case */
        produit = produit * tab[i];
    }
    /* i >= TAILLE */

    /* produit contient le produit des éléments du tableau */
    printf("Le produit des éléments du tableau vaut %d.\n",produit);

    return EXIT_SUCCESS;
}

/* definitions des fonctions utilisateurs */

\end{verbatim}
\end{small}
\end{correction}
\subsection{Comptage du nombre d'éléments inférieurs à un seuil donné (\textit{2,5 points})}

Écrire un programme qui demande à l'utilisateur d'entrer les valeurs d'un tableau de réels de taille \verb|N| (une constante symbolique), puis la valeur d'un réel \verb|seuil|, et qui affiche combien d'éléments du tableau sont strictement plus petits que \verb|seuil|. Deux exemples d'exécution sont les suivants (pour \verb|N| valant 4):
\begin{small}
\begin{verbatim}
Saisissez 4 réels : 2.2 4.3 -1.1 0.33
Entrez le seuil : 1.1
Il y a 2 nombre(s) < seuil.

Saisissez 4 réels : 2.2 4.3 -1.1 0.33
Entrez le seuil : 0.33
Il y a 1 nombre(s) < seuil.
\end{verbatim}
\end{small}

\begin{correction}
\begin{small}
\begin{verbatim}
#include <stdlib.h> /* EXIT_SUCCESS */
#include <stdio.h> /* printf, scanf */

/* déclaration constantes et types utilisateurs */
#define TAILLE 4 /* taille du tableau utilisateur */

/* déclaration de fonctions utilisateurs */

int main()
{
    double tab[TAILLE]; /* tableau à initialiser par l'utilisateur */
    double seuil; /* seuil à saisir par l'utilisateur pour filtrer le tableau */
    int plus_petit = 0; /* nombre d'éléments du tableau < seuil */
    int i; /* var. de boucle */

    /* demande saisie de TAILLE réels */
    printf("Saisissez %d réels : ",TAILLE);

    /* saisie des elts du tableau (TAILLE réels) */
    for(i = 0;i < TAILLE;i = i + 1) /* chaque case du tableau */
    {
        /* saisie valeur */
        scanf("%lg",&tab[i]);
    }
    /* i >= TAILLE */
    
    /* saisie seuil */
    printf("Entrez le seuil : ");
    scanf("%lg",&seuil);
    
    /* comptage nombre < seuil */
    for(i = 0;i < TAILLE;i = i + 1) /* chaque case */
    {
        if(tab[i] < seuil) /* trouvé plus petit */
        {
            plus_petit = plus_petit + 1; /* un de plus */
        }
    }
    /* i >= TAILLE */

    printf("Il y a %d nombre(s) < seuil.\n",plus_petit);
  
    return EXIT_SUCCESS;
}

/* definitions des fonctions utilisateurs */
\end{verbatim}
\end{small}
\end{correction}
\section{Une seconde plus tôt, il était exactement... (\textit{5 points})}

Il faut écrire un programme qui demande à l'utilisateur de saisir l'heure sous la forme de 3 entiers (3 variables pour les heures, \verb|h|, les minutes,
\verb|m| et les secondes, \verb|s|) et qui affiche l'heure qu'il était 1
seconde plus tôt. Il faudra envisager tous les
cas possibles pour le changement d'heure. Deux exemples de sortie sont~:

\begin{small}
\begin{verbatim}
Introduire l'heure puis les minutes puis les secondes : 23 12 0
Une seconde plus tôt, il était exactement : 23h11m59s

Introduire l'heure puis les minutes puis les secondes : 0 0 0
Une seconde plus tôt, il était exactement : 23h59m59s
\end{verbatim}
\end{small}

\begin{enumerate}
\item En s'aidant des exemples donnés plus haut, écrire l'algorithme correspondant en français.
\begin{correction}
\begin{small}
\begin{verbatim}
- saisie de l'heure
- enlève une seconde
- si tour du cadran à l'envers des secondes alors
    - remise à 59 des secondes
    - il est une minute de moins
    - si tour du cadran à l'envers des minutes alors
        - remise à 59 des minutes
        - il est une heure de moins
        - si tour du cadran à l'envers des heures alors
            - remise à 23 des heures
- affichage de l'heure
\end{verbatim}
\end{small}
\end{correction}

\item En s'aidant éventuellement de la question précédente, écrire le programme.
  \begin{correction}
\begin{small}
\begin{verbatim}
#include <stdlib.h> /* EXIT_SUCCESS */
#include <stdio.h> /* printf, scanf */

/* déclaration constantes et types utilisateurs */

/* déclaration de fonctions utilisateurs */

/* fonction principale */
int main()
{
    /* soit l'heure représentée par 3 entiers */
    int h; /* heures */
    int m; /* minutes */
    int s; /* secondes */

    /* saisie de l'heure */
    printf("Introduire l'heure puis les minutes puis les secondes : ");
    scanf("%d",&h);
    scanf("%d",&m);
    scanf("%d",&s);
    
    /* calcul de la nouvelle heure */
    s = s - 1; /* une seconde plus tôt */

    if(s == -1) /* tour du cadran à l'envers des secondes */
    {
        /* remise à 59 */
        s = 59;

        /* une minute de moins */
        m = m - 1;

        if(m == -1) /* tour du cadran à l'envers des minutes */
        {
            /* remise à 59 */
            m = 59;

            /* une heure de moins */
            h = h - 1;

            if(h == -1) /* tour du cadran à l'envers des heures */
            {
                /* remise à 23 */
                h = 23;
            }
        }
    }
    /* h,m,s contiennent l'heure une seconde plus tôt */

    /* affichage de l'heure */
    printf("Une seconde plus tôt, il était exactement : %dh%dm%ds\n",h,m,s);

    return EXIT_SUCCESS;
}

/* definitions des fonctions utilisateurs */

\end{verbatim}
\end{small}
  \end{correction}
\end{enumerate}

\section{Affichage d'un triangle isocèle d'étoiles (\textit{5 points})}

Écrire un programme qui, étant donné un entier naturel impair \verb|base|, affiche un triangle isocèle d'étoiles, ayant pour base, \verb|base| étoiles. La valeur de la \verb|base| sera saisie par l'utilisateur et on considérera qu'il saisit bien un nombre impair. Trois exemples d'exécution sont les suivants :
\begin{small}
\begin{verbatim}
Nombre d'étoiles à la base du triangle (impair) ?
5
  *
 ***
*****
\end{verbatim}
\end{small}
\begin{small}
\begin{verbatim}
Nombre d'étoiles à la base du triangle (impair) ?
3
 *
***
\end{verbatim}
\end{small}
\begin{small}
\begin{verbatim}
Nombre d'étoiles à la base du triangle (impair) ?
1
*
\end{verbatim}
\end{small}

\begin{enumerate}
\item Quelle est la série d'entiers correspondant aux nombres d'étoiles à afficher par ligne ?
  \begin{correction}
    C'est la série des nombres impairs de 1 à \verb|base|.
  \end{correction}
\item Combien y a-t-il de blancs à afficher sur une ligne, pour un nombre $n$ d'étoiles sur la ligne ?
  \begin{correction}
    Il y a $(base - n) / 2$ blancs à afficher par ligne.
  \end{correction}
\item Écrire le programme.
  \begin{correction}
\begin{small}
\begin{verbatim}
#include <stdlib.h> /* EXIT_SUCCESS */
#include <stdio.h> /* printf, scanf */

/* déclaration constantes et types utilisateurs */

/* déclaration de fonctions utilisateurs */

int main()
{
    int base; /* un nombre impair d'étoiles à la base du triangle, saisi par l'utilisateur */
    int i; /* var. de boucle */
    int j; /* var. de boucle */

    /* saisie base (impair) */
    printf("Nombre d'etoiles à la base du triangle (impair) ?\n");
    scanf("%d",&base);

    /* affichage triangle isocèle d'étoiles */
    for(i = 1;i <= base;i = i + 2) /* chaque nombre d'étoiles à afficher (base impaire)*/
    {
        /* affiche les blancs */
        for(j = 0;j < (base - i) / 2;j = j + 1) /* (base - i) / 2 fois */
        {
            /* affiche un blanc */
            printf(" ");
        }
        /* j >= (base - i) /2 */

        /* affiche les étoiles */
        for(j = 0;j < i;j = j + 1) /* i fois */
        {
            /* affiche une étoile */
            printf("*");
        }
        /* j >= i */

        /* passe à la ligne suivante */
        printf("\n");
    }
    /* i >= base */

    return EXIT_SUCCESS;
}

/* definitions des fonctions utilisateurs */

\end{verbatim}
\end{small}
  \end{correction}
\end{enumerate}

\subsection*{Question bonus}

Compléter le programme précédent afin qu'il affiche un losange d'étoiles, \verb|base| étant le nombre impair d'étoiles sur la ligne du milieu. Deux exemple d'exécution sont les suivants :
\begin{small}
\begin{verbatim}
Base du losange ?
5
  *
 ***
*****
 ***
  *

Base du losange ?
7
   *
  ***
 *****
*******
 *****
  ***
   *
\end{verbatim}
\end{small}

