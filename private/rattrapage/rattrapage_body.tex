% -*- coding: utf-8 -*-


\newcounter{questioncount}
\setcounter{questioncount}{0}
\newcommand{\question}{\addtocounter{questioncount}{1}\paragraph{Question \Alph{questioncount}.}}
\newcommand{\commentaire}[1]{}
\newcommand{\pt}[1]{\fbox{$#1 \operatorname{pt}$}}


\setlength{\abovecaptionskip}{-10pt}



\entete{Éléments d'informatique, rattrapage}
\vspace{-1cm}
\begin{description}
\item[Durée :] 3 heures.
\item[Documents autorisés :] Aucun.
\item[Recommandations :] Un barème vous est donné à
titre indicatif afin de vous permettre de gérer votre temps. La
notation prendra en compte à la fois la syntaxe et la sémantique de
vos programmes, c'est-à-dire qu'ils doivent compiler correctement. Une
fois votre programme écrit, il est recommandé de le faire tourner à la
main sur un exemple pour s'assurer de sa correction.
\end{description}


\section{Trace d'un programme avec fonctions}

\question Simulez l'exécution du programme figure~\ref{fig:prog}, en réalisant sa
\textbf{trace}. \bareme{4}

\begin{figure}[htbp]
  \centering
\begin{small}
\listinginput{1}{pourtrace.c}
\end{small} 
  \caption{Programme pour la trace}
  \label{fig:prog}
\end{figure}

\begin{correction}
  Table~\ref{trace} page~\pageref{trace}.
  \setlength{\abovecaptionskip}{10pt}

  \begin{table}
  \begin{small} 
   \setlength{\unitlength}{\tabcolsep}
    \begin{tabular}[t]{|r|c|c|c|l|}
      \multicolumn{5}{l}{\C{main()}}\\ \hline
      ligne & n & x & res & Aff.\\ \hline
      initialisation  & 3 & 6.0 &  ? & \\ \hline
      17 &\multicolumn{4}{r|}{
        \put(1,0){\noindent
          \begin{tabular}[t]{|r|c|c|c|l|}
            \multicolumn{5}{l}{\C{foo(3, 6.0)}}\\ \hline
            ligne & n & x & i & Aff. \\ \hline
            initialisation & 3 & 6.0 & ? &  \\ \hline
            27 & &  & 0 & \\ \hline
            29  &\multicolumn{4}{r|}{
              \put(1,0){\noindent
                \begin{tabular}[t]{|r|c|l|}
                  \multicolumn{3}{l}{\C{bar(6.0)}}\\ \hline
                  ligne & x & Aff. \\ \hline
                  initialisation & 6.0 &  \\ \hline
                  36 & \multicolumn{2}{l|}{renvoie 4.0} \\ \hline
                \end{tabular}
              }}\\ \hline
            29 & & 4.0 & & \\ \hline
            30 & & & 1 &  \\ \hline
            29  &\multicolumn{4}{r|}{
              \put(1,0){\noindent
                \begin{tabular}[t]{|r|c|l|}
                  \multicolumn{3}{l}{\C{bar(4.0)}}\\ \hline
                  ligne & x & Aff. \\ \hline
                  initialisation & 4.0 &  \\ \hline
                  36 & \multicolumn{2}{l|}{renvoie 3.0} \\ \hline
                \end{tabular}
              }}\\ \hline
            29 & & 3.0 & & \\ \hline
            30 & & & 2 &  \\ \hline
            29  &\multicolumn{4}{r|}{
              \put(1,0){\noindent
                \begin{tabular}[t]{|r|c|l|}
                  \multicolumn{3}{l}{\C{bar(3.0)}}\\ \hline
                  ligne & x & Aff. \\ \hline
                  initialisation & 3.0 &  \\ \hline
                  36 & \multicolumn{2}{l|}{renvoie 2.5} \\ \hline
                \end{tabular}
              }}\\ \hline
            29 & & 2.5 & & \\ \hline
            30 & & & 3 &  \\ \hline
            31 & \multicolumn{4}{l|}{renvoie 2.5} \\ \hline
          \end{tabular}
        }}\\ \hline
      17 & &  & 2.5 & \\ \hline
      18 & & & & \verb|bar^3(6) = 2.5| \\ \hline
      20 & \multicolumn{4}{l|}{renvoie \C{EXIT\_SUCCESS}} \\ \hline
     \end{tabular}
        \caption{Trace du programme de l'exercice 2.}
        \label{trace}
\end{small}
  \end{table}
\end{correction}

\begin{baremeenv}
 Maximum \pt{4}. Si des erreurs, maximum \pt{3,5} en sommant :
  \begin{enumerate}[(a)]
\item \pt{+1} deux premières lignes de la trace du main sont correctes
  (identification des variables et leurs initialisations).
  \begin{enumerate}[$\ast$]
  \item \pt{-0,5} par variable manquante ou en trop
  \item \pt{-0,25} par initialisation manquante ou en trop
  \end{enumerate}
\item \pt{+2} Pour l'appel à foo :
  \begin{enumerate}[$\ast$]
  \item \pt{+0,5} pour foo(3, 6) ligne 17
\item \pt{+0,5}, \pt{+0,25} par colonne paramètre formel n et x bien
  initialisées à 3 et 6
\item \pt{+0,5} pour le déroulement correct de la boucle sans se
  soucier de ce qui s'y passe (trois tours, i
  = 3), \pt{0.25} si juste oubli de i = 3.
\item \pt{+0,5} pour le fait que \C{x} change de valeur au moins une
  fois ligne 29.
 \end{enumerate}
\item \pt{+1}  pour au moins un des (trois) appels à bar
  correct et dans \C{foo()}. Pénalité de \pt{-0.5} si contient un  (ou
  plusieurs) appel imbriqué.
  \begin{enumerate}[$\ast$]
  \item \pt{+0,25} pour bar(6) ligne 19.
\item \pt{+0,5} si retourne un résultat exacte.
\end{enumerate}
\item \pt{+0,5} affichage ligne 18 correct et cohérent avec les
  valeurs trouvées. \pt{0} si une erreur.
\end{enumerate}
\end{baremeenv}

\section{Sans fonctions, for ou while, arbre de décision (11 pt)}

Il est demandé de résoudre les trois problèmes suivants sans définir de
fonctions utilisateurs et en faisant le meilleur choix entre for et
while s'il y a des boucles.  L'ensemble du code sera à écrire dans la
fonction principale \verb|main|.

\subsection{Puissance}

\question Écrire le programme qui : demande à l'utilisateur d'entrer
un nombre à virgule $x$, puis un entier $n$; et qui calcule puis affiche
la valeur de $x^n$, comme dans l'exemple suivant. \bareme{2.5}

On supposera que l'entier saisi est toujours supérieur ou égal à zéro
(on ne teste pas la saisie de l'utilisateur).

\paragraph{Exemple d'exécution :}
\begin{small}
\begin{verbatim}
Saisissez un nombre decimal : 0.5
Saisissez un entier : 3
0.5 puissance 3 = 0.125
\end{verbatim}
\end{small}


\begin{correction}
  \begin{small}
\listinginput{1}{puissance.c}
  \end{small}
  
  \begin{baremeenv}
Si tout juste à l'aide d'un for : \pt{2.5} sinon (y compris si while) total maximum de
\pt{2} pris dans les points suivants :
    \begin{itemize}
      \item \pt{0.25}  déclaration (avec ou sans initialisation) d'une variable
        double ou float pour la saisie de $x$
      \item \pt{0.25}  déclaration (avec ou sans initialisation) d'une variable
        entière pour la saisie de $n$
      \item \pt{0.25} déclaration d'un accumulateur de type double ou float pour stocker le
      produit 
    \item \pt{0.25} initialisation à 1 de l'accumulateur
      \item \pt{0.25} saisie de $n$ à l'aide d'un scanf avec ou sans \&
      \item \pt{0.25} saisie de $n$ à l'aide d'un scanf avec ou sans \&
      \item \pt{0.5} boucle for pas de point en moins si oublie de
        déclaration de la variable de boucle
      \item \pt{0.25} bloc de boucle exécuté $n$ fois
    \item \pt{0.25} affichage du résultat du calcul
    \end{itemize}
  \end{baremeenv}
\end{correction}


\subsection{Moins de dette privée}

Pippo emprunte 100 brouzoufs à 6\% d'intérêts annuel. Il rembourse
par versements de 25 brouzoufs chaque année, en commençant un an après
la date d'emprunt. Combien d'années devra t'il rembourser son emprunt
?  Chaque année, on calcule d'abord les intérêts à ajouter à la dette
actuelle, puis on soustrait le montant du versement annuel.

\question Écrire un programme qui affiche le plan de remboursement de
Pippo, c'est à dire, pour chaque année : les intérêts de l'année, le
versement effectué, la dette restant à rembourser.\bareme{2.5}

Votre programme doit pouvoir être exécuté pour n'importe quelle autre
initialisation des données du prêt.



\paragraph{Exemple d'exécution :}
\begin{small}
\begin{verbatim}
annee 1, interets : 6, versement : 25, dette : 81
annee 2, interets : 4.86, versement : 25, dette : 60.86
annee 3, interets : 3.6516, versement : 25, dette : 39.5116
annee 4, interets : 2.3707, versement : 25, dette : 16.8823
annee 5, interets : 1.01294, versement : 25, dette : -7.10477
\end{verbatim}
\end{small}

\begin{correction}
  \begin{small}
\listinginput{1}{dette.c}
  \end{small}
  
  \begin{baremeenv}
Si tout juste à l'aide d'un while: \pt{2.5} sinon maxi \pt{2,25}.
    \begin{itemize}
      \item \pt{0.25}  déclaration et initialisation de la dette et
        du taux comme des double (on ne regarde pas versement).
      \item \pt{0.5} une et une seule boucle while.
      \item \pt{0.5} modification de la valeur de la dette dans la
        boucle (\pt{0.25} si mauvaise modification)
      \item \pt{0.25} déclaration et initialisation du compteur
        d'années (une erreur de 1 est acceptée)
      \item \pt{0.25} incrément du compteur d'année à chaque tour de boucle
  \item \pt{0.25} affichage dans la boucle
    \item \pt{0.25} affichage des bonnes valeurs avec les bonnes conversions
    \end{itemize}
  \end{baremeenv}
\end{correction}

\question Le dernier versement est trop important. Comment modifier
votre programme de manière à  ne rembourser que le reste à payer,
dernière année ?\bareme{1} La dernière ligne affichée sera alors :
\begin{small}
\begin{verbatim}
annee 5, interets : 1.01294, versement : 17.8952, dette : 0
\end{verbatim}
\end{small}

\begin{correction}
juste avant affichage dans le while on insère :
  \begin{small}
\begin{verbatim}
        if (dette < 0)
        {
            versement = versement + dette; /* on supprime le trop versé */
            dette = 0;
        }
\end{verbatim}
  \end{small}
  
  \begin{baremeenv}
Si tout juste : \pt{1} sinon maxi \pt{0.75}.
    \begin{itemize}
\item \pt{0.25} pour un if dans le while (ou après si condition du while
  modifiée)
\item \pt{0.25} pour un pour dette mise à zéro avant affichage
\item \pt{0.25} pour un versement égal au reste à percevoir avant affichage
\end{itemize}
  \end{baremeenv}
\end{correction}


\question Si le versement est inférieur aux intérêts, que dire de
l'exécution de ce programme ?\bareme{1}

\begin{correction}
  Si le versement est inférieur aux intérêts la première année, la
  dette va augmenter et elle ne sera jamais nulle (ou négative). Donc
  le programme va boucler et la dette atteindre une valeur
  représentant l'infinie.

  \begin{baremeenv}
    \pt{0.5} si mention du fait que le programme ne s'arrête jamais.
  \end{baremeenv}
\end{correction}


\subsection{Faut-il renoncer à rembourser la dette publique ?}

\begin{figure}[h!]
    \begin{center}
% Define block styles
\tikzstyle{decision} = [diamond, draw, fill=blue!20, 
    text width=4.5em, text badly centered, node distance=3cm, inner sep=0pt]
\tikzstyle{block} = [rectangle, draw, fill=blue!20, 
    text width=5em, text centered, rounded corners, minimum height=4em, node distance=3.5cm]
\tikzstyle{line} = [draw, -latex']
\tikzstyle{cloud} = [draw, ellipse,fill=red!20,  text width=5.7em, text centered, 
    minimum height=2em,  node distance=3.2cm]
\tikzstyle{choix} = [rectangle, inner sep=2pt,
    text centered]
    
\begin{tikzpicture}[node distance = 1.7cm, auto]
    % Place nodes
    \node [cloud] (temps) {Comment sont les marchés ?};
    \node [below of=temps] (fantome) {};
    \node [left of=fantome] (fantome2) {};
    \node [block, left of=fantome2] (panique) {Ne plus rembourser};
    \node [cloud, right of=fantome] (inquiets) {Dette $ > 0.5$
      PIB ?};
    \node [block,right of=inquiets] (confiants) {Ne rien changer};
    \node [cloud,  below left of=inquiets] (pop) {La population
      refuse l'austérité ?};
    \node [block, below left of=pop] (ouipop) {Ne plus
      rembourser};
    \node [block, below right of=pop] (nonpop) {Ne
      rien changer};
    \node [block, below right of=inquiets] (nondette) {Ne rien changer};

   % Draw edges
\path [line] (temps) -| node [near start, above, choix] {paniqués} (panique);
\path [line] (temps) -- node [near start, right, choix] {inquiets} (inquiets);
\path [line] (temps) -| node [near start, above, choix] {confiants}
(confiants);
\path [line] (inquiets) -- node [left, choix] {oui} (pop);
\path [line] (inquiets) -- node [right, choix] {non} (nondette);
\path [line] (pop) -- node [left, choix] {oui} (ouipop);
\path [line] (pop) -- node [right, choix] {non} (nonpop);
\end{tikzpicture}
    \end{center}
    \caption{Décider s'il faut continuer de rembourser la dette}
    \label{fig:ad}
\end{figure}



Un arbre de décision
est un graphe particulier où les n\oe{}uds sont 
des questions et les arêtes sont les réponses à ces questions.
Il se lit de haut en bas. On avance dans l'arbre en répondant aux
questions. Les n{\oe}uds les plus bas jouent le rôle particulier de
classes de réponse au problème initial. 

Ici, il y a deux classes de réponse : <<~Ne plus rembourser~>> et
<<~Ne rien changer~>>.
Par exemple, si les marchés sont inquiets, si la dette est
strictement supérieure à $0.5$ fois le PIB et que la population refuse
l'austérité, alors on ne rembourse plus.



Vous utiliserez quatre variables dont vous coderez les valeurs avec des
constantes symboliques. Les trois premières représentent les propriétés du jour
courant pour prendre la décision, la dernière représente la décision à
prendre :
  
\begin{itemize}
\item \verb|marches| est un entier représentant l'état des marchés,
  elle peut valoir un entier signifiant PANIQUE, ou un entier
  signifiant INQUIETUDE ou un entier signifiant CONFIANCE.
\item \verb|dette| est un nombre a virgule représentant la dette en
  proportion du PIB (une valeur de 0.5 représente 50\% du PIB).
\item \verb|refus| est un entier représentant le refus de l'austérité
  qui peut avoir la valeur usuelle
 TRUE (si la population refuse l'austérité) ou la valeur usuelle FALSE (sinon).
\item \verb|rembourser| est un entier représentant la décision à
  prendre et peut avoir la valeur TRUE (s'il faut continuer de
  rembourser) ou FALSE (s'il faut arrêter de rembourser). Sa valeur
  initiale est TRUE. 
\end{itemize}

Votre programme sera en deux parties. Une première partie concernera
la prise de décision sans affichage, en fonction
des valeurs de \verb|marches|, \verb|dette|, \verb|refus|. Vous
donnerez des valeurs de votre choix à ces trois variables mais le
programme doit fonctionner pour tous les autres choix possibles. À la fin
de cette première partie \verb|rembourser| contiendra la valeur
correcte pour la décision prise (TRUE ou FALSE). La seconde partie
exploitera la valeur de \verb|rembourser| pour effectuer l'affichage.

\question
Écrire le programme complet en distinguant bien les deux
parties.\bareme{4}

\begin{correction}
\begin{figure}
  \centering
\begin{small}
\listinginput{1}{marches.c}
\end{small} 
  \caption{Faut-il annuler la dette ? -- Corrigé.}
  \label{fig:dropdebt}
\end{figure}
Correction figure~\ref{fig:dropdebt}.
 \begin{baremeenv}
\pt{4} si tout juste. Sinon on fait un maximum de \pt{3.5} en sommant :
\begin{itemize}
\item 
  \pt{0.5} pour le codage TRUE, FALSE, PANIQUE, INQUIETUDE, CONFIANCE à l'aide de define s'il est correct (pas d'erreur de syntaxe, comme un signe égal), retirer une pénalité de 0.25 s'il a trop de constantes symboliques. S'il y a par exemple une constante symbolique pour définir la valeur d'initialisation de la distance.
\item \pt{0,5} pour des
  déclarations avec initialisations correctes utilisant les constantes
  symboliques (\pt{0.25} si sans les constantes symboliques (par
  exemple 1 au lieu de TRUE).  
\item \pt{0,5} pour une indentation correcte (si il y a matière à
  indenter !).
\item  Une pénalité de \pt{-0,25} par oubli du stdlib, du  stdio ou du
  return final (on ne sanctionne pas l'utilisation d'un 0 à la place
  du \C{EXIT\_SUCCESS}).
\item Pour l'arbre de décision: on part d'un total de \pt{2.5} et on compte \pt{0.5}
  point de pénalité par branche de l'arbre en erreur.  S'il y a des boucles c'est
  zéro ! 
\item \pt{0.5} pour test sur \C{rembourser} et affichage correct.
\end{itemize}
  \end{baremeenv}
\end{correction}


\section{Complexes en notation algébrique}
\begin{figure}
\small
\begin{listing}{1}
#include <stdlib.h> /* EXIT_SUCCESS */
#include <stdio.h> /* printf, scanf */

/* déclaration constantes et types utilisateurs */
... 

/* déclaration de fonctions utilisateurs */
struct complexe_s multiplier(struct complexe_s x, struct complexe_s y);
...

/* Fonction principale */
int main()
{
   struct complexe_s z = ... ; /* à compléter pour traduire z = - 0.5 + 0.5 i */ 
   struct complexe_s resultat;
   
   /* calcul de (z * z) + z */
   ....

   /* affichage du résultat */
   afficher(resultat);

   /* valeur fonction */
  return EXIT_SUCCESS;
}

/* définition de fonctions utilisateurs */
...
\end{listing}
  \caption{Calculs dans les complexes}
  \label{fig:deuxj}
\end{figure}

\question 
Compléter le programme de la figure~\ref{fig:deuxj} :\bareme{5}
\begin{enumerate}
\item définir le type utilisateur \verb+struct complexe_s+, pour stocker
  un nombre complexe en notation algébrique (sa partie réelle et sa
  partie imaginaire sous forme de nombres à virgule).%\bareme{1}
\item Dans le main, initialiser la
  valeur de la variable $z$ à  $- 0.5 + 0.5 i$ (il faut traduire en
  instructions C).%\bareme{0.5}
\item Dans le main, calculer $z = z^2 + z$ en faisant appel à des
  fonctions élémentaires sur les complexes, telles que \C{multiplier} (rappel : les opérations \C{+} et
  \C{*} du C ne sont pas définies sur des struct). Vous aurez besoin %\bareme{1.5}
  d'une autre fonction que \C{multiplier} pour effectuer le calcul,
  il faut la déclarer (lui donner un nom explicite). 
\item  Déclarer également la procédure \C{afficher} qui
  sert à afficher le résultat.%\bareme{0.5}
\item Définir au choix deux des trois fonctions utilisateur de ce programme.%\bareme{1.5}
\end{enumerate}

Votre programme doit fonctionner et calculer des valeurs correctes pour
n'importe quelle autre initialisation de la variable \C{z}. Vous
pouvez répondre à ces cinq questions en même temps, en donnant le code
complet de votre programme, ou signaler pour chaque question à quelles
lignes du programme de la figure~\ref{fig:deuxj} vous insérez de
nouvelles instructions.

\begin{correction}
  Correction figure~\ref{fig:deuxjcorr}.

  \begin{baremeenv}
Si tout juste: \pt{5}.
    \begin{itemize}
      \item \pt{1} déclaration correcte syntaxiquement d'une structure
      contenant exactement deux champs double ou float, (même si le nom
      n'est pas le bon).
      \item \pt{0.5} initialisation correcte de la variable $z$.
      \item \pt{1.5} calcul exact sinon maxi \pt{1} 
        \begin{itemize}
        \item \pt{0.5}  si au moins un appel syntaxiquement correct à
          \C{multiplier}.
        \item \pt{0.25} si il est fait mention d'une fonction
          d'addition (quel que soit le nom) prenant deux complexes et
          retournant deux complexes.
        \item \pt{0.5}  déclaration correcte d'une fonction d'addition avec
          même prototype que la multiplication.
        \end{itemize}
      \item \pt{0.5}  Déclaration correcte de \C{afficher}
      \item \pt{1.5} si deux définitions justes, sinon maxi \pt{1}
        avec \pt{0.75} par fonction (\pt{0.5} si valeur de retour non
        déclarée). On ne récompense pas de troisième
        fonction !
   \end{itemize}
  \end{baremeenv}
\begin{figure}
\begin{small}
  \listinginput{1}{complexes_corr.c}
\end{small}
\caption{Calculs dans les complexes -- correction}
  \label{fig:deuxjcorr}
\end{figure}
\end{correction}

\section{Autres déclarations de fonctions}
\begin{enumerate}
\item Déclarer (sans la définir) une procédure \verb|menu| qui n'a pas d'entrée et
  affiche :\label{item:menu}\bareme{0.5}
\begin{small}
\begin{verbatim}
********
* MENU *
********
\end{verbatim}
\end{small}
\item Déclarer (sans la définir) une fonction qui calcule la valeur absolue d'un réel.\label{valeurabsolue}\bareme{0.5}
\end{enumerate}

\begin{correction}
\begin{verbatim}
void menu();
double valeur_absolue(double x);
\end{verbatim}
  \begin{baremeenv}
    Compter simplement \pt{0.5} par déclaration correcte.
  \end{baremeenv}
\end{correction}

%\newpage
%\thetotalpointsint
%%% Local Variables: 
%%% mode: latex
%%% TeX-master: "rattrapage_corr"
%%% End: 