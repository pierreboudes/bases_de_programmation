# Exemple de fichier de QCM
#
# Syntaxe globale :
# Les lignes commençant par '#' sont des commentaires.
#
# Les réponses sont situées sous les questions, avec un blanc devant. La
# question se termine par un '!' seul sur sa ligne.
#
#
# Convention : mettre la bonne réponse en premier (si il y en a une seule).

@META debut=1 facile=1
Le langage C est un langage
 compilé
 lu, écrit, parlé
 interprété
 composé
!

@META debut=1 difficile=1
Une \emph{segmentation fault} est une erreur qui survient lorsque :
+le programme tente d'accèder à une partie de la mémoire qui ne lui
-est pas réservée
+ le programme source a été enregistré sur le disque dur en plusieurs
  morceaux et l'un d'entre eux ne peut pas être chargé par le
- compilateur
 la division du programme en zones homogènes échoue
+ le programme tente d'afficher des caractères sur une ligne qui va au delà
- de la largeur de la fenêtre du terminal
!

@META debut=1
+Pour déclarer une variable qui sera utilisée comme variable de boucle
- on peut utiliser l'instruction
  \verb|int k;|
  \verb|loop i;|
  \verb|int loop n;|
  \verb|int %d;|
!

@META debut=1
+Lorsqu'un programme utilise \verb|printf| ou \verb|scanf| il faut
- qu'il contienne l'instruction préprocesseur :
 \verb|#include <stdio.h>|
 \verb|#include <studio.h>|
 \verb|#include <studlib.h>|
 \verb|#appart <stdlib.h>|
!



# Un débogueur est une application qui :
#  peut aider le programmeur à détecter des erreurs dans un programme
#  peut reconstruire un fichier en langage C à partir d'un exécutable
#  signale les erreurs de syntaxe du fichier source qui empêchent la compilation 
#  peut tester la bonne terminaison de n'importe quel programme
# !

@META debut=1 facile=1
+Soient deux variables entières \verb|x| et \verb|y| initialisées à
4 et 5 respectivement. L'affichage \verb|x=4 et y=5| est obtenu avec la
-commande :
 \verb|printf("x=%d et y=%d\n",x,y);|
 \verb|printf("x=%d et y=%d\n,x,y");|
 \verb|printf("x=%x et y=%y\n");|
 \verb|printf("x=%d et y=%d\n",x y);|
!

@META debut=1
+Lorsqu'un programme utilise \verb|printf| ou \verb|scanf| il faut
- qu'il contienne l'instruction préprocesseur :
 \verb|#include <stdio.h>|
 \verb|#include <studio.h>|
 \verb|#include <studlib.h>|
 \verb|#appart <stdlib.h>|
!


#UNIX/LINUX ###############################################
@META debut=1
Sous unix (ou linux), la commande \verb|ls| permet de :
 afficher la liste de fichiers contenus dans un répertoire
 compiler un programme
 afficher le contenu d'un fichier texte
 voir des clips musicaux
!

@META debut=1
Sous unix (ou linux), la commande \verb|cd| permet de :
 changer de répertoire courant
 jouer de la musique
 ouvir un bureau partagé (common desktop)
 récupérer un programme arrêté avec la commande \verb|ab|
 détruire un fichier
!

@META debut=1
+Sous unix (ou linux), pour créer un répertoire TP4 dans le répertoire
-courant on peut utiliser la commande :
 \verb+mkdir TP4+
 \verb+new TP4+
 \verb+kwrite TP4+
 \verb+yppasswd+
!

@META debut=1 facile=1
Sur unix (ou linux), la commande \verb|mkdir| permet de :
  créer un répertoire
  changer de répertoire courant
  créer un fichier texte
  ouvrir un fichier texte
!



#ARCHI ###############################################
@META debut=1
Un registre du processeur est :
 une case mémoire interne au processeur qui sera manipulée directement lors des calculs
 une unité de calcul spécialisée de l'ordinateur
 un composant qui contient la liste des fichiers du système
 une gamme de fréquence de fonctionnement du processeur
!

@META debut=1 facile=1
Le bus système sert à :
 Transférer des données et intructions entre processeur et mémoire
 Arriver à l'heure en cours
 Écrire des données sur le dique dur
 transporter les processus du tourniquet au processeur
!

@META debut=1 facile=1
Un bit est :
 un chiffre binaire (0 ou 1)
 la longueur d'un mot mémoire
 l'instruction qui met fin à un programme
 un battement d'horloge processeur
!


@META debut=1
+Après exécution du programme :
\begin{listing}{1}
lecture 8 r0
valeur 3 r1
mult r1 r0
valeur 1 r2
add r2 r0
ecriture r0 8
stop
5
\end{listing}
-
 la case mémoire 8 contiendra 16
 la case mémoire 8 contiendra 0
 le bus explose
 le terminal affiche 8
!

#AFFECTATION ###############################################
@META debut=1
+Après exécution jusqu'à la ligne 15 du programme  C :
\begin{listing}{10}
...
int main() {
    int x = 5;

    x = 3 * x + 1;

    ...
}
\end{listing}
-
 la variable x vaut $16$
 la variable x vaut $-\frac{1}{2}$
 le programme affiche x
 le programme affiche \verb+****+
!

@META debut=1
+Après exécution jusqu'à la ligne 15 du programme  C :
\begin{listing}{10}
int main() {
    int x = 5;
    int y = 3;

    x = y;

    ...
}
\end{listing}
-
 la variable x vaut $3$
 la variable y vaut $5$
 la variable x vaut $5$ et la variable y vaut $3$
 le programme affiche "Faux''
!

@META debut=1
+Après exécution jusqu'à la ligne 15 du programme  C :
\begin{listing}{10}
int main() {
    int x = 5;
    int y;

    y = x;

    ...
}
\end{listing}
-
 la variable y vaut $5$
 la variable x vaut $0$
 la variable x vaut $5$ et la variable y vaut $0$
 le programme affiche "Faux''
!

@META debut=1
+Après exécution jusqu'à la ligne 14 du programme  C :
\begin{listing}{10}
int main() {
    int x = 5;

    printf(" x  = %d\n", 2);

    ...
}
\end{listing}
-
 le terminal affiche \verb+x = 2+
 le terminal affiche \verb+x = 5+
 le terminal affiche "Faux"
 le terminal affiche \verb+5+
!

#SYSTEME ###############################################
@META debut=1
+Sur un ordinateur avec un seul processeur, habituellement les processus sont
exécutés :
-
 tour à tour, un petit peu à chaque fois
 tous ensemble
 en parallèle, chacun dans un registre
 chacun son tour, après que le processus précédent a terminé
!

@META debut=1
+L'ordonnancement par tourniquet permet:
-
 d'entretenir l'illusion que les processus tournent en parallèle
 de ne pas perdre de temps avec la commutation de contexte
 d'afficher des ronds colorés à l'écran
 de doubler la mémoire disponible
!

@META debut=1
+La virtualisation de la mémoire permet notamment de stocker des portions inactives de la
mémoire de travail sur le disque dur. Mais on perd :
-
 en temps d'accès
 les fichiers du disque
 des processus
 certaines données de la mémoire de travail
!


#COMPILATION ##########################################

@META debut=1 facile=1
Dans la commande gcc, l'option \verb|-Wall| signifie :
 que l'on veut voir tous les avertissements
 qu'il faut indenter le fichier source
 qu'on veut changer alétoirement de fond d'écran
 qu'il faut lancer un déboggueur
!

@META debut=1 difficile=1
+Quelle étape de la compilation vient d'échouer lorsqu'on a un message
 comme celui-ci : \\
 \verb+Undefined symbols :"_prinft"+ ou \\
- \verb+référence indéfinie vers « prinft »+
 l'édition de liens
 l'analyse harmonique
 l'analyse sémantique
 l'analyse des entrées clavier
!

@META debut=1 difficile=1
+ Si cette erreur apparaît à la compilation : \\
 \verb+Undefined symbols :"_prinft"+ ou \\
 \verb+référence indéfinie vers « prinft »+
- que doit-on chercher dans le programme ?
  une faute de frappe dans un appel de fonction
  une variable non déclarée
  un caractère interdit en C
  une directive préprocesseur \verb+#include+ manquante
!

@META debut=1 facile=1
+ Si cette erreur apparaît à la compilation : \\
\verb+error: expected ‘;’ before ‘}’ token+ 
- que doit-on chercher dans le programme ?
 un point-virgule manquant
 une accolade manquante
 un point-virgule en trop
 une accolade en trop
!

@META debut=1 facile=1
Pour compiler un programme \verb|prog.c|, on utilise la ligne de commande : 
 \verb|gcc -Wall prog.c -o prog.exe|
 \verb|gcc -Wall prog.exe -o prog.c|
 \verb|gcc prog.exe -Wall -o prog.c|
 \verb|gcc prog.c -o -Wall prog.exe|
!

@META debut=1 difficile=1
+Les lignes
\begin{verbatim}
int i;
int x=0;
for(i=0,i<5,i=i+1)
{
  x=x+1;
}
\end{verbatim}
-
 comportent une erreur qui sera détectée au cours de l'analyse syntaxique
 comportent une erreur qui ne sera pas détectée
 comportent une erreur qui sera détectée au cours de l'édition de lien
 ne comportent aucune erreur
!

@META debut=1 facile=1
+Laquelle des analyses suivantes ne fait pas partie des étapes de la
-compilation :
 analyse harmonique
 analyse syntaxique
 analyse lexicale
 analyse sémantique
!

@META debut=1 facile=1
+Un programme en langage C doit comporter une et une seule définition de la
-fonction :
 main
 begin
 init
 include
!

@META debut=1 facile=1
Un fichier source est :
 un fichier texte qui sera traduit en instructions processeur
 un fichier que l'ont doit citer dans les documents produits sur l'ordinateur
 un document illisible pour les humains
 un document de référence du système
 un document qui doit être protégé
!

#PROG STRUCTUREE ########################
@META debut=1
+Une de ces manière de composer les blocs de
programmes ne fait pas partie des opérations de la programmation
-structurée :
 retourner un bloc
 sélectionner entre deux blocs à l'aide d'une condition
 répéter un bloc tant qu'une condition est vérifée
 mettre les blocs en séquence les uns à la suite des autres
!

@META debut=1
Quels calculs peut-on programmer en programmation structurée ?
 en programmation structurée on peut programmer tous les calculs programmables en langage machine
+il y a des calculs programmables en langage machine et qui ne sont
-pas programmables en programmation structurée
+il y a des calculs programmables en programmation structurée qui ne
-sont pas programmables en langage machine
 certains programmes sont de vrais plats de spaghetti
!

#IF ELSE FACILE###################
@META debut=1
+Le code suivant :
 \begin{verbatim}
 int age = 18;
 if (age < 18)
 {
     printf("Mineur\n"); 
 }
 else
 {
     printf("Majeur\n"); 
 }
 \end{verbatim}
-affichera :
 \verb|Mineur|
 \verb|Majeur|
 rien
+\verb|Mineur|\\
-\verb|Majeur|
!

@META debut=1
+Le code suivant :
 \begin{verbatim}
 int age = 20;
 if (age < 18)
 {
     printf("Mineur\n"); 
 }
 else
 {
     printf("Majeur\n"); 
 }
 \end{verbatim}
-affichera :
 \verb|Mineur|
 \verb|Majeur|
 rien
+\verb|Mineur|\\
-\verb|Majeur|
!

@META debut=1
+Le code suivant :
 \begin{verbatim}
 int age = 15;
 if (age < 18)
 {
     printf("Mineur\n"); 
 }
 else
 {
     printf("Majeur\n"); 
 }
 \end{verbatim}
-affichera :
 \verb|Mineur|
 \verb|Majeur|
 rien
+\verb|Mineur|\\
-\verb|Majeur|
!

@META debut=1
+Le code suivant :
 \begin{verbatim}
 int age = 20;
 if (age < 18)
 {
     printf("Mineur\n"); 
 }
 printf("Majeur\n"); 
\end{verbatim}
-affichera :
 \verb|Mineur|
 \verb|Majeur|
 rien
+\verb|Mineur|\\
-\verb|Majeur|
!

@META debut=1
Quel est l'opérateur de différence en C:
 \verb|<>|
 \texttt{$\neq$}
 \verb|!=|
 \verb|!|
!

#BOUCLE FOR###################
@META debut=1 difficile=1
+Le code suivant :
 \begin{verbatim}
 int somme = 0;
 int i;
 for (i = 1; i < 4; i = i + 1)
 {
   somme = somme + i;
 }
 printf("%d", somme); 
 \end{verbatim}
-affichera :
 0
 6
 1
 42
!

@META debut=1
+Le code suivant :
 \begin{verbatim}
 int i;
 for (i = 4; i > 0; i = i - 1)
 {
     printf("%d ", i); 
 }
 printf("\n");
\end{verbatim}
-affichera :
 4 3 2 1
 0 1 2 3
 0 1 2 3 4
 4 3 2 1 0
!

@META debut=1
+Le code suivant :
 \begin{verbatim}
 int i;
 for (i = 0; i < 5; i = i + 1)
 {
     printf("%d ", i); 
 }
 printf("\n");
\end{verbatim}
-affichera :
 0 1 2 3 4
 4 3 2 1
 0 1 2 3
 4 3 2 1 0
!

@META debut=1
+Le code suivant :
 \begin{verbatim}
 int i;
 for (i = 1; i < 5; i = i + 1)
 {
     printf("%d ", i); 
 }
 printf("\n");
\end{verbatim}
-affichera :
 0 1 2 3 4
 4 3 2 1
 1 2 3 4
 4 3 2 1 0
!

@META debut=1
+Le code suivant :
 \begin{verbatim}
 int i;
 for (i = 4; i >= 0; i = i - 1)
 {
     printf("%d ", i); 
 }
 printf("\n");
\end{verbatim}
-affichera :
 4 3 2 1 0
 0 1 2 3 4
 4 3 2 1
 1 2 3 4
!

@META debut=1
+Le code suivant :
 \begin{verbatim}
 int i;
 for (i = 0; i < 7; i = i + 2)
 {
     printf("%d ", i); 
 }
 printf("\n");
\end{verbatim}
-affichera :
 0 2 4 6
 0 2 4 6 8
 0 1 2 3 4 5 6
 0 1 2 3 4 5 6 7
!

@META debut=1
+Le code suivant :
 \begin{verbatim}
 int i;
 for (i = 8; i > 0; i = i - 2)
 {
     printf("%d ", i); 
 }
 printf("\n");
\end{verbatim}
-affichera :
 8 6 4 2
 0 2 4 6 8
 8 6 4 2 0
 8 2
!

@META debut=1 difficile=1
+Soit un programme contenant les lignes suivantes :
 \begin{verbatim}
 int i = 0;
 int j = 0;
 for (i = 0; i < 3; i = i + 1) 
 {
     for (j = 0; j < 5; j = j + 1) 
     {
           ...
     }
 }
 printf("j = %d\n", j); 
 \end{verbatim}
-qu'est ce qui sera affiché par ce printf ?
 \verb|j = 5|
 \verb|j = %d|
 \verb|j = 4|
 \verb|j = 0|
!

@META debut=1 difficile=1
+Soit un programme contenant les lignes suivantes :
 \begin{verbatim}
 int i = 0;
 int j = 0;
 for (i = 0; i < 0; i = i + 1) 
 {
     for (j = 0; j < 5; j = j + 1) 
     {
       ...
     }
 }
 printf("j = %d\n", j);
 ...
 }
 \end{verbatim}
-qu'est ce qui sera affiché ?
 \verb|j = 0|
 \verb|j = 5|
 \verb|j = %d|
 \verb|j = 4|
!

@META debut=1 difficile=1
+ Pour l'extrait de programme suivant: 
 \begin{verbatim} 
 int i = 0;
 int j = 0;
 for (i = 0; i < 2; i = i + 1) 
 {
     for (j = 0; j < 3; j = j + 1) 
     {
         printf("%d ", j);
     }
 }
\end{verbatim}
-qu'est ce qui sera affiché ?
 \verb|0 1 2 0 1 2|
 \verb|0 1 2 0 1 2 3|
 \verb|0 1 2 3 0 1 2|
 \verb|0 0 1 1 2 2 3|
!

@META debut=1 difficile=1
+Soit un programme contenant les lignes suivantes :
 \begin{verbatim}
 int i = 0;
 int j = 0;
 for (i = 0; i < 2; i = i + 1) 
 {
     for (j = 0; j < 3; j = j + 1) 
     {
         printf("%d ", i);
     }
 }
 \end{verbatim}
-qu'est ce qui sera affiché ?
 \verb|0 0 0 1 1 1|
 \verb|0 1 2 0 1 2|
 \verb|0 1 0 1 0 1 0 1|
 \verb|1 2 1 2 3|
!

@META debut=1 difficile=1
+ Pour l'extrait de programme suivant: 
\begin{verbatim}
 int i = 0;
 int j = 0;
 for (i = 0; i < 3; i = i + 1) 
 {
     for (j = 0; j < 2; j = j + 1) 
     {
         printf("%d ", i);
     }
 }
 printf("\n");
\end{verbatim}
-qu'est ce qui sera affiché ?
 \verb|0 0 1 1 2 2|
 \verb|0 1 2 0 1 2|
 \verb|0 1 0 1 0 1|
 \verb|1 2 3 1 2|
!


@META debut=1 difficile=1
+ Pour l'extrait de programme suivant: 
\begin{verbatim}
 int i;
 int j;
 for(i=4;i>0;i=i-1)
 {
   for(j=i;j<6;j=j+1)
   {
     printf("*");
   } 
   printf(" ");
 }
 \end{verbatim}
-qu'est ce qui sera affiché ?
 \verb|** *** **** *****|
 \verb|** ** ** ** ** **|
 \verb|**** **** **** ****|
 \verb|***** **** *** ***|
!

@META debut=1 difficile=1
+ Pour l'extrait de programme suivant: 
 \begin{verbatim} 
  int somme = 0;
  for (i = 0; i < 5; i = i + 1) 
  {
    somme = somme + i;
    i = i + 1; /* attention ! */
  }
  printf("somme = %d",somme); 
\end{verbatim}
-La valeur de somme affichée est : 
 6
 0
 10
 15
!

## tableaux #############
@META debut=1
+Pour déclarer un tableau d'entiers de taille 5, on peut utiliser
-l'instruction
 \verb|int toto[5];|
 \verb|char tableau[5];|
 \verb|int tab[] = 5;|
 \verb|int toto[taille=5];|
 \verb|int[] new tableau(5);|
!

@META debut=1
+Afin de représenter la taille d'un tableau, définir une constante
-symbolique N valant 3.
 \verb|#define N 3|
 \verb|#define taille = N|
 \verb|#define taille = 3|
 \verb|#define N =  3|
!

@META debut=1 facile=1
+Pour afficher à l'aide de \verb|printf("%d\n",tab[i]);| le contenu d'un tableau de 5 entiers initialisé au
-préalable, on utilise plutôt :
 \verb|for(i=0;i<5;i=i+1)|
 \verb|for(i=0;i<=5;i=i+1)|
 \verb|for(i=1;i<5;i=i+1)|
 \verb|for(i=1;i<=5;i=i+1)|
!

@META debut=1 difficile=1
+ Pour l'extrait de programme suivant: 
 \begin{verbatim} 
  int somme = 0;
  int serie[4] = {2, 4, 10, 4};
  for (i = 0; i < 4; i = i + 1) 
  {
    somme = somme + serie[i];
  }
  printf("somme = %d",somme); 
\end{verbatim}
-La valeur de somme affichée est : 
 20
 16
 6
 3
!

@META debut=1
+ Pour l'extrait de programme suivant: 
 \begin{verbatim} 
  int produit = 1;
  int serie[4] = {2, 2, 2, 2};
  for (i = 0; i < 4; i = i + 1) 
  {
    produit = produit * serie[i];
  }
  printf("produit = %d", produit); 
\end{verbatim}
-La valeur affichée est : 
 8
 16
 0
 4
!

@META debut=1 difficile=1
+ Pour l'extrait de programme suivant: 
 \begin{verbatim} 
  int produit = 0;
  int serie[4] = {2, 2, 2, 2};
  for (i = 0; i < 4; i = i + 1) 
  {
    produit = produit * serie[i];
  }
  printf("produit = %d", produit); 
\end{verbatim}
-La valeur affichée est : 
 0
 16
 8
 4
!


#### WHILE ET COND BOOL
@META facile=1
+Quel est le problème d'un programme comportant les lignes suivantes?
\begin{verbatim}
while (1)
{
  printf("coucou\n");
}
\end{verbatim}
-
  il comporte une boucle infinie
  il n'affiche rien
  il ne compile pas
  il risque d'afficher bonjour à la place de coucou
!

Une variable booléenne est un variable :
 qui est vraie ou fausse
 réelle positive
 jamais nulle
 à laquelle une valeur vient d'être affectée
 NaN (not a number, qui n'est pas un nombre)
! 


@META difficile=1
+On considère deux variables booléennes \verb|A| et \verb|B|
initialisées à \verb|TRUE| et \verb|FALSE| respectivement. Parmi
les expressions booléennes suivantes, laquelle a pour valeur
-\verb|TRUE| ?
 \verb+!(!A || B) == (A && !B)+
 \verb|(A == TRUE) && (B == TRUE)|
 \verb|A && B|
 \verb+(!A || B)+
!

# Céline
@META difficile=1
+On souhaite faire une boucle de contrôle de saisie: tant que l'entier
\verb|n| n'appartient pas à l'intervalle $[a..b]$, on recommence la
saisie de \verb|n|. Soit le programme suivant:
 \begin{verbatim}
 int a = 0; 
 int b = 20;
 int n;
 scanf("%d", &n); 
 while(cond) 
 {
   scanf("%d", &n);
 }
\end{verbatim}
-Quelle est la condition \verb|cond| : 
 \verb+(n<a) || (n>b)+
 \verb|(a<=n) && (n<=b)|
 \verb|a<=n<=b|
\verb|(n<=a) && (n<=b)| 
!

@META facile=1
Vous utilisez une boucle \verb|while| quand :
 vous ne connaissez pas le nombre d'itérations de la boucle à l'avance
 vous n'avez pas déclaré de fonction
 vous avez déjà fait un  \verb|for| dans le même programme principal 
 l'incrément de la variable de boucle n'est pas 1
!


######TYPES et E/S

@META facile=1
Laquelle de ces écritures correspond à la déclaration d'une variable de type caractère en langage C?
 \verb|char c;|
 \verb|char 'c';|
 \verb|char "c";|
 \verb|int char;|
!


Si $n$ est une variable entière pour demander sa valeur à l'utilisateur, on utilise plutôt :
 \verb+scanf("%d", &n);+
 \verb+printf("Valeur de n ? %d\n", n);+
 \verb+printf("Valeur de n ? %g\n", n);+
  un débogueur
!

Si x est une variable réelle (de type double) alors \verb|x = 3/2| lui affecte la valeur :
  1
  0
  0.5
  1.5
!

Le type des réels en C est :
 \verb+double+
 \verb+real+
 \verb+int+
 \verb+char+
!

L'écriture $\underline{101}$ en binaire correspond au nombre naturel :
 5
 101
 4
 3
!


L'écriture $\underline{111}$ en binaire correspond au nombre naturel :
 7
 111
 8
 3
!


#META facile=1
+Au début de la fonction \verb|main()| on place le code :
 \begin{verbatim}
 char b = 'A';
 b = b + 2;
 printf("%c\n", b);
\end{verbatim}
- Alors l'affichage sera :
 \verb|C|
 \verb|B|
 \verb|b|
 \verb|A|
!

+Au début de la fonction \verb|main()| on place le code :
 \begin{verbatim}
 char i;
 for (i = 'A'; i <= 'F'; i = i + 1)
 {
   printf("%c", i);
 }
 printf("\n");
\end{verbatim}
- Alors l'affichage sera :
 \verb|ABCDEF|
 \verb|A|
 \verb|cccccc|
 \verb|i|
!



#########FONCTIONS

@META difficile=1
+ Si cet avertissement apparaît à la compilation : \\
 \verb+warning: implicit declaration of function ‘max’+ , 
- que doit-on chercher dans le programme ?
  une fonction appelée avant sa déclaration
  une fonction déclarée mais non définie
  un désaccord entre la déclaration et la définition d'une fonction
  une directive préprocesseur \verb+#include+ manquante
!

@META difficile=1
+ Si cette erreur apparaît à la compilation : \\
  \verb+erreur: conflicting types for ‘max’+ , 
- que doit-on chercher dans le programme ?
  un désaccord entre la déclaration et la définition d'une fonction
  une fonction déclarée mais non définie
  une fonction appelée avant sa déclaration
  une directive préprocesseur \verb+#include+ manquante
!


# +Pour déclarer une fonction \verb|distance| qui prend en arguments deux
# \verb|struct point_s|
# - et renvoie la distance (réelle) entre les deux points on écrit :
#  \verb|double distance(struct point_s a, struct point_s b);|
#  \verb|int distance(struct point_s a, struct point_s b);|
#  \verb|double distance(point_s a, point_s b);|
#  \verb|void distance(struct a, struct b);|
# !

+Pour déclarer une fonction \verb|exposant| qui prend en argument un
- réel $x$ et un entier positif $n$ et renvoie la valeur de $x^n$ on écrit :
 \verb|double exposant(double x, int n);|
 \verb|int exposant(double n, int x);|
 \verb|void exposant(double x^n);|
 \verb|exposant(double x, int n, int r);|
!

+Pour déclarer une fonction \verb|factorielle| qui prend en argument un
- entier et renvoie sa factorielle on écrit :
 \verb|int factorielle(int x);|
 \verb|struct int factorielle(int n);|
 \verb|int factorielle();|
 \verb|int factorielle(double n);|
!

+Pour déclarer une fonction \verb|pgcd| qui calcule et renvoie le plus
  grand diviseur commun de deux entiers positifs passés en arguments 
- on écrit : 
 \verb|int pgcd(int y, int x);|
 \verb|int pgcd(int x, int x);|
 \verb|int pgcd(int x, y);|
 \verb|void pgcd(int x, int y);|
!

+Pour déclarer une fonction \verb|saisie_utilisateur| qui demande à
- l'utilisateur d'entrer un entier au clavier et renvoie cet entier on écrit :
 \verb|int saisie_utilisateur();|
 \verb|void saisie_utilisateur(int n);|
 \verb|void saisie_utilisateur(char c);|
 \verb|saisie_utilisateur(scanf(%d));|
!


+Pour déclarer une procédure \verb|afficher_menu| sans argument et qui
-ne renvoie rien on utilise :
 \verb|void afficher_menu();|
 \verb|double afficher_menu();|
 \verb|int afficher_menu();|
 \verb|int afficher_menu(int char);|
 \verb|char afficher_menu(printf("menu"));|
!


+Pour déclarer une procédure \verb|afficher_date| qui prend en argument un
\verb|struct date_s|
- et affiche le contenu du struct, on écrit :
 \verb|void afficher_date(struct date_s d);|
 \verb|struct date_s afficher_date(struct date_s d);|
 \verb|int afficher_date(date_s d);|
 \verb|void afficher_date(date_s d);|
!

Avant de faire appel à une fonction il est nécessaire de :
 l'avoir déclarée
 l'avoir définie
 l'avoir déclarée et définie
 avoir défini une constante symbolique de la taille de cette fonction
!

+Si \verb|carre| est une fonction prenant en entrée un entier et
renvoyant le carré de cet entier, et que n est une variable entière
-définie et initialisée, il est correct d'écrire :
 \verb+n = carre(n);+
 \verb+n = carre(int n);+
 \verb+int n = carre();+
 \verb+int carre(2);+
!

+Si \verb|pgcd| est une fonction prenant en entrée deux entiers et
-renvoyant un entier, il est correct d'écrire :
 \verb+n = pgcd(n, 3);+
 \verb+n = pgcd(int p, int q);+
 \verb+int n = pgcd();+
 \verb+int pgcd(2);+
!

+Si \verb|factorielle| est une fonction prenant en entrée un entier et
-renvoyant un entier, il est correct d'écrire :
 \verb+printf("%d", factorielle(n));+
 \verb+n = factorielle(p, q);+
 \verb+n = factorielle();+
 \verb+int factorielle(int 2);+
!

+Après la déclaration :
\verb|int mccarthy(int n);|, 
- il est correct d'écrire :
 \verb+x = mccarthy(n);+
 \verb+n = mccarthy(p, q);+
 \verb+n = mccarthy();+
 \verb+int mccarthy(int 2);+
!

@META difficile=1
+Si \verb|racine| est une fonction prenant en entrée un réel et
 renvoyant la racine carrée de cet réel, et que x est une variable réelle
-définie et initialisée, il est incorrect d'écrire :
 \verb+x - 1 = racine(x);+
 \verb+x = racine(x * x) - racine(x);+
 \verb+x = racine(racine(x)*racine(x));+
 \verb+x = racine(2/3);+
!


@META facile=1
+Vous avez déclaré préalablement un ensemble de fonctions utilisées
par votre programme principal. L'ordre dans lequel vous devez
-maintenant définir ces fonctions est l'ordre : 
 un ordre quelconque
 alphabétique 
 dans lequel vous avez déclaré ces fonction
 dans lequel ces fonctions sont appelées dans le main
!

@META facile=1
+Soit le programme principal suivant: 
 \begin{verbatim}
int main() 
{
 int a = 3;
 int b = 5;
 printf("f(a,b)=%d, a=%d, b=%d\n",f(a,b),a,b);
 return EXIT_SUCCESS;
}
\end{verbatim}
 appelant la fonction \verb|f| ainsi définie: 
 \begin{verbatim}
int f(int a, int b)
{
  a = a + b; 
  return a;
}
\end{verbatim}
-L'affichage dans le main est le suivant: 
 \verb|f(a,b)=8, a=8, b=5|
 \verb|f(a,b)=8, a=3, b=5|
 \verb|f(3,5)=8, a=3, b=5|
 \verb|f(a,b)=13, a=8, b=5|
!

@META facile=1
+Soit la fonction \verb|f| définie par :
\begin{verbatim}
int f(int a)
{
  printf("a = %d\n", a);
  if (a > 0)
  {
    return 3;
  }
  return 4;
}
\end{verbatim}
-Alors l'expression \verb|f(0)| prendra la valeur :
 4
 0
 3
!

@META facile=1
+Soit la fonction \verb|g| définie par :
\begin{verbatim}
int g(int a)
{
  printf("a = %d\n", a);
  if (1 > 0)
  {
    return 5;
  }
  return 7;
}
\end{verbatim}
-Alors l'expression \verb|g(0)| prendra la valeur :
 5
 7
 0
!

@META
+Soit la fonction \verb|f| définie par :
\begin{verbatim}
int f(int a)
{
  printf("a = %d\n", a);
  if (a > 0)
  {
    return f(a - 1) + 1;
  }
  return 4;
}
\end{verbatim}
-Alors l'expression \verb|f(1)| prendra la valeur :
 5
 4
 1
 0
!

#####STRUCT
@META facile=1
Un type struct permet de grouper plusieurs valeurs dans :
 ses champs
 ses chants
 ses cases
 ses blocs
!

+Si le code :
\begin{verbatim}
struct toto_s
{
  int n;
  double x;
};
\end{verbatim}
 précède la fonction \verb+main()+, alors on peut écrire en début de
-\verb|main()|:
 \verb|struct toto_s toto;|
 \verb|int struct toto_s = {3, -1e10};|
 \verb|int toto.n = 3;|
 \verb|toto_s struct z = {3, 0.5};|
 \verb|toto_s n, x;|
!

@META difficile=1
+Si a et b sont deux variables de type :
\begin{verbatim}
struct toto_s
{
  int n;
  double x;
};
\end{verbatim}
-Alors pour tester l'égalité de a et de b on utilise la condition :
 \verb|(a.n == b.n) && (a.x == b.x)|
 \verb|a = b|
 \verb|a == b|
 \verb|a{n, x} == b{n, x}|
!
