% -*- coding: utf-8 -*-

\newcommand{\commentaire}[1]{}

\entete{Partiel d'Éléments d'Informatique}

\begin{description}
\item[Durée :] 3 heures.
\item[Documents autorisés :] Aucun.
\item[Recommandations :] Un barème vous est donné à
titre indicatif afin de vous permettre de gérer votre temps. La
notation prendra en compte à la fois la syntaxe et la sémantique de
vos programmes, c'est-à-dire qu'ils doivent compiler correctement. Une
fois votre programme écrit, il est fortement recommandé de le faire tourner à la
main sur un exemple pour s'assurer de sa correction.
\end{description}

\section{Multiplication par additions successives (\textit{5 points})}

Nous voulons écrire une fonction \verb|mult_par_add| qui prend comme
paramètres deux entiers positifs ou nuls \verb|a| et \verb|b|, et
retourne la multiplication de \verb|a| par \verb|b|. Cette fonction
doit calculer la multiplication \textbf{par additions
  successives, sans utiliser l'opérateur de multiplication du C}. Le
programme principal est déjà écrit pour pouvoir tester la fonction~:
\begin{verbatim}
#include <stdlib.h> /* EXIT_SUCCESS */
#include <stdio.h> /* printf, scanf */

int main()
{
  int a; /* premier facteur de la multiplication */
  int b; /* deuxieme facteur de la multiplication */

  printf("Entrez les deux facteurs a et b : ");
  scanf("%d",&a);
  scanf("%d",&b);

  while((a < 0) || (b < 0)) /* pas les deux positifs ou nuls */
  {
    printf("Erreur. Entrez les deux facteurs a et b : ");
    scanf("%d",&a);
    scanf("%d",&b);
  }

  printf("la multiplication de a par b est : %d\n",mult_par_add(a,b));

  return EXIT_SUCCESS;
}
\end{verbatim}

\begin{enumerate}
\item Expliquer pourquoi le programme ci-dessus est incomplet et ce qu'il manque pour que la compilation réussisse, et puis pour que l'édition de liens réussisse. \textbf{Les réponses doivent être succintement expliquées.}
\item Faire l'ajout minimal dans le programme pour que la compilation réussisse.
\item Écrire la fonction \verb|mult_par_add|.
\end{enumerate}

Un exemple de sortie du programme est :
\begin{verbatim}
Entrez les deux facteurs a et b : 5 -1
Erreur. Entrez les deux facteurs a et b : 5 4
La multiplication de a par b est : 20
\end{verbatim}

\begin{correction}
\begin{itemize}
\item (2) 1 pt pour explication vague compil + 1 pt idem pour EdL
\item 1 pour proto correct
\item 2 pour code :
  \begin{itemize}
  \item pas de boucle -> 0
  \item 1 boucle correcte avec bonne borne (for ou while quoique c du for)
  \item 0,5 return avec expression de bon type
  \item 0,5 calcul expression correcte
  \end{itemize}
\end{itemize}
\end{correction}

\section{Résolution de problèmes \textit{(6 points)}}

Il est demandé de résoudre les deux problèmes suivants sans définir de fonctions utilisateurs.
L'ensemble du code sera à écrire dans la fonction principale \verb|main|.

\subsection{Calcul du produit des éléments d'un tableau (\textit{2,5 points})}

Écrire le programme qui demande à l'utilisateur d'entrer les valeurs d'un tableau d'entiers de taille \verb|N| (une constante symbolique), et qui calcule et affiche la valeur du produit des
 entiers du tableau. Deux exemples d'exécution sont les suivants (pour \verb|N| valant 4) :
\begin{small}
\begin{verbatim}
Saisissez 4 entiers : 2 -3 6 -10
Le produit des éléments du tableau vaut 360.

Saisissez 4 entiers : 3 -3 4 0
Le produit des éléments du tableau vaut 0.
\end{verbatim}
\end{small}
tout ou rien :
- 0.5pts : constante symbolique
- 0.5 : squelette (programme vide)
- 0.75 initialisation
- 0.75 produit (incluant init produit = 1)
\begin{correction}
\begin{small}
\begin{verbatim}
#include <stdlib.h> /* EXIT_SUCCESS */
#include <stdio.h> /* printf, scanf */

/* déclaration constantes et types utilisateurs */
#define TAILLE 4 /* taille du tableau utilisateur */

/* déclaration de fonctions utilisateurs */

int main()
{
    int tab[TAILLE]; /* tableau à initialiser par l'utilisateur */
    int produit = 1; /* élément neutre de la multiplication */
    int i; /* var. de boucle */

    /* demande saisie de TAILLE entiers*/
    printf("Saisissez %d entiers : ",TAILLE);

    /* saisie des elts du tableau (TAILLE entiers) */
    for(i = 0;i < TAILLE;i = i + 1) /* chaque case du tableau */
    {
        /* saisie valeur */
        scanf("%d",&tab[i]);
    }
    /* i >= TAILLE */

    /* calcul produit */
    for(i = 0;i < TAILLE;i = i + 1) /* chaque case */
    {
        /* multiplie le produit partiel par la valeur de la case */
        produit = produit * tab[i];
    }
    /* i >= TAILLE */

    /* produit contient le produit des éléments du tableau */
    printf("Le produit des éléments du tableau vaut %d.\n",produit);

    return EXIT_SUCCESS;
}

/* définition de fonctions utilisateurs */

\end{verbatim}
\end{small}
\end{correction}

\subsection{Une seconde plus tôt, il était exactement... (\textit{3,5 points})}

Écrire un programme qui demande à l'utilisateur de saisir l'heure sous la forme de 3 entiers (3 variables pour les heures, \verb|h|, les minutes,
\verb|m| et les secondes, \verb|s|) et qui affiche l'heure qu'il était 1
seconde plus tôt. Il faudra envisager tous les
cas possibles pour le changement d'heure. Deux exemples de sorties différentes sont~:

\begin{small}
\begin{verbatim}
Introduire l'heure puis les minutes puis les secondes : 23 12 0
Une seconde plus tôt, il était exactement : 23h11m59s

Introduire l'heure puis les minutes puis les secondes : 0 0 0
Une seconde plus tôt, il était exactement : 23h59m59s
\end{verbatim}
\end{small}
  \begin{correction}
\begin{small}
\begin{verbatim}
(0.5 bonus) 
une seconde plus TARD vu en TD. si ça -> 0
- 1 : saisie heure
- 1 : seconde OK
- 1 : minute OK
- 1 : heure OK

#include <stdlib.h> /* EXIT_SUCCESS */
#include <stdio.h> /* printf, scanf */

/* déclaration constantes et types utilisateurs */

/* déclaration de fonctions utilisateurs */

/* fonction principale */
int main()
{
    /* soit l'heure représentée par 3 entiers */
    int h; /* heures */
    int m; /* minutes */
    int s; /* secondes */

    /* saisie de l'heure */
    printf("Introduire l'heure puis les minutes puis les secondes : ");
    scanf("%d",&h);
    scanf("%d",&m);
    scanf("%d",&s);

    /* calcul de la nouvelle heure */
    s = s - 1; /* une seconde plus tôt */

    if(s == -1) /* tour du cadran à l'envers des secondes */
    {
        /* remise à 59 */
        s = 59;

        /* une minute de moins */
        m = m - 1;

        if(m == -1) /* tour du cadran à l'envers des minutes */
        {
            /* remise à 59 */
            m = 59;

            /* une heure de moins */
            h = h - 1;

            if(h == -1) /* tour du cadran à l'envers des heures */
            {
                /* remise à 23 */
                h = 23;
            }
        }
    }
    /* h,m,s contiennent l'heure une seconde plus tôt */

    /* affichage de l'heure */
    printf("Une seconde plus tôt, il était exactement : %dh%dm%ds\n",h,m,s);

    return EXIT_SUCCESS;
}

/* définition de fonctions utilisateurs */

\end{verbatim}
\end{small}
  \end{correction}

\section{Trace d'un programme avec fonctions (\textit{4 points})}

Simulez l'exécution du programme suivant, en réalisant sa \textbf{trace}, comme cela a été vu en TD et en cours.
\textbf{Rappel : } L'opérateur modulo du C \verb|%| calcule le reste de la division entière : \verb|a % b| vaut le reste de la division entière de \verb|a| par \verb|b|.

\begin{small}
\begin{listing}{1}
#include <stdlib.h> /* EXIT_SUCCESS */
#include <stdio.h> /* printf, scanf */

/* declarations constantes et types utilisateurs */

/* declarations de fonctions utilisateurs */
int mystere(int a, int b);

int main()
{
  int x = 12;
  int y = 4;
  int res;
 
  res = mystere(x, y);
  printf("mystere(%d, %d) = %d\n", x, y, res);

    return EXIT_SUCCESS;
}

/* definitions de fonctions utilisateurs */
int mystere(int a, int b)
{
   if (b == 0) 
    {
        return a;
    }
    return mystere(b, a % b);
}
\end{listing}
\end{small}

\section{Écriture de fonctions (\textit{5 $\times$ 1 point})}

\begin{enumerate}
\item Écrire la fonction \verb|factorielle| qui prend en entrée un entier positif ou nul $n$ et qui renvoie sa factorielle $n!$, avec la convention $0! = 1$.
\item Écrire la fonction \verb|combinaison| qui calcule le nombre de combinaisons possibles quand on choisit k objets parmi n objets discernables et que l'ordre dans lequel les objets sont placés (ou énumérés) n'a pas d'importance : elle prend en entrée deux entiers $k$ et $n$ et renvoie $\frac{n!}{k!(n-k)!}$. On considérera la fonction \verb|factorielle| comme disponible. 
\item Écrire la procédure \verb|affiche_menu| qui n'a pas d'entrée et affiche le menu suivant :
\begin{verbatim}
Votre choix ?
1) choix 1
2) choix 2
3) choix 3
\end{verbatim}
\item Écrire la procédure \verb|rectangle_d_etoiles| qui prend en
  entrée la longueur et la largeur en nombre d'étoiles (caractère \verb|'*'|) du rectangle,
  et qui affiche le rectangle d'étoiles.
\item Écrire la fonction qui prend en entrée un entier \verb|n| et qui renvoie : \[\sum_{i=0}^{i=n-1} (\sum_{j=0}^{j=i}i*j)\]
\end{enumerate}

\begin{correction}
0.5 points si proto juste
0.5 si corps entierement juste (tolère 1 faute syntaxique anodine, genre oubli ;)
\end{correction}
