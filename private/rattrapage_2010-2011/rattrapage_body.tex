% -*- coding: utf-8 -*-


\newcounter{questioncount}
\setcounter{questioncount}{0}
\newcommand{\question}{\addtocounter{questioncount}{1}\paragraph{Question \Alph{questioncount}.}}
\newcommand{\commentaire}[1]{}
\newcommand{\pt}[1]{\fbox{$#1 \operatorname{pt}$}}


\setlength{\abovecaptionskip}{-10pt}



\entete{Éléments d'informatique, rattrapage}
\vspace{-1cm}
\begin{description}
\item[Durée :] 3 heures.
\item[Documents autorisés :] Aucun.
\item[Recommandations :] Un barème vous est donné à
titre indicatif afin de vous permettre de gérer votre temps. La
notation prendra en compte à la fois la syntaxe et la sémantique de
vos programmes, c'est-à-dire qu'ils doivent compiler correctement. Une
fois votre programme écrit, il est recommandé de le faire tourner à la
main sur un exemple pour s'assurer de sa correction.
\end{description}


\section{Étude de programmes et questions de cours}


\subsection{Minimum des éléments d'un tableau}

Nous voulons écrire un programme qui calcule le minimum des éléments
d'un tableau d'entiers. Une partie du programme est déjà écrite, figure~\ref{fig:min}, et
Pippo pense qu'il ne
reste plus qu'à écrire la partie qui calcule effectivement le
minimum. 
\begin{figure}[htbp]
\begin{small}
\begin{listing}{1}
#include <stdlib.h> /* EXIT_SUCCESS */
#include <stdio.h> /* printf, scanf */

/* déclaration constantes et types utilisateurs */
#define TAILLE 3

/* Fonction principale */
int main()
{
  int t[TAILLE] = {12,10,15}; /* tableau */



  /* calcul du minimum (À FAIRE) */





  /* affichage du résultat */
  printf("Le minimum est %d\n", minimum);

  /* valeur fonction */
  return EXIT_SUCCESS;
}
\end{listing}
\end{small}
  \caption{Minimum à compléter}
  \label{fig:min}
\end{figure}

\question La compilation échoue. Expliquer pourquoi, quelle étape de la
  compilation échoue précisément et ce qu'il manque pour que la compilation réussisse. \bareme{1}
\question Expliquer le fonctionnement de l'instruction \C{\#define}. \bareme{1}
\question Compléter le programme pour qu'il calcule effectivement le minimum
  des éléments du tableau (pour une taille de tableau arbitraire). \bareme{2.5}

\paragraph{Exemple d'exécution} pour le tableau donné dans le code :

\begin{small}
\begin{verbatim}
Le minimum est 10
\end{verbatim}
\end{small}

\begin{correction}
Mêmes question qu'au partiel de mi-semestre.
\begin{baremeenv}
  \begin{enumerate}
\item \pt{1.5} (barème augmenté) dont \pt{1} pour ``déclaration de la variable \C{minimum}'' + \pt{0.5} pour ``analyse
  sémantique'' ou \pt{0.25} si ``avant la génération du code objet'' et/ou
  ``avant l'édition de lien''.
\item \pt{1} si constante symbolique, préprocesseur,  TAILLE remplacé
  par 3. \pt{0.5} si seulement notion de constantes et pas de mise en
  relation avec la substitution au moment de la compilation.
\item \pt{2.5} pour le code juste (compilation correction quel que soit
  TAILLE). Sinon, maxi \pt{2} :
  \begin{itemize}
  \item pas de boucle -> \pt{zéro}
  \item \pt{0.5} une (et une seule) boucle (for, while accepté)
  \item \pt{0.5} la boucle parcours effectivement un tableau de taille TAILLE
  \item pas de points en moins ou en plus pour la déclaration de la
    variable de boucle (redite sous une autre forme de la première
    question) ou pour l'initialisation de minimum à zéro.
\item \pt{0.5} pour un if imbriqué dans la boucle.
\item \pt{0.5} si le if est correct, c'est à dire s'il réalise le bon test (comparaison avec le minimum
  courant) et change la valeur du minimum courant en conséquent.
  \end{itemize}
\end{enumerate}
\end{baremeenv}
\end{correction}

\subsection{Trace d'un programme avec fonctions}

\question Simulez l'exécution du programme figure~\ref{fig:prog}, en réalisant sa
\textbf{trace}, comme cela a été vu en TD et en cours. \bareme{4}

\begin{figure}[htbp]
  \centering
\begin{small}
\begin{listing}{1}
#include <stdlib.h> /* EXIT_SUCCESS */
#include <stdio.h> /* printf, scanf */

/* declarations constantes et types utilisateurs */

/* declarations de fonctions utilisateurs */
int foo(int n);
void bar(int n);

/* fonction principale */
int main()
{
    int x = 3;
    int res;

    res = foo(x);
    printf("foo(%d) = %d\n", x, res);
    
    bar(x);

    return EXIT_SUCCESS;
}

/* definitions de fonctions utilisateurs */
int foo(int n)
{
    int i;
    int reponse = 42;
    for (i = n; i > 0; i = i - 2)
    {
        printf("%d ", i);
    }
    printf("\n");
    return reponse;
}

void bar(int n)
{
    if (n > 1) 
    {
        printf("%d ", n);
        bar(n - 2);
    }
    else
    {
        printf("%d\n", n);
    }
}
\end{listing}
\end{small} 
  \caption{Programme pour la trace}
  \label{fig:prog}
\end{figure}

\begin{correction}
  Table~\ref{trace} page~\pageref{trace}.
  \setlength{\abovecaptionskip}{10pt}

  \begin{table}
  \begin{small} 
   \setlength{\unitlength}{\tabcolsep}
    \begin{tabular}[t]{|r|c|c|l|}
      \multicolumn{4}{l}{\C{main()}}\\ \hline
      ligne & x & res & Affichage\\ \hline
      initialisation  & 3 & ? & \\ \hline
      16 &\multicolumn{3}{r|}{
        \put(1,0){\noindent
          \begin{tabular}[t]{|r|c|c|c|c|}
            \multicolumn{5}{l}{\C{foo(3)}}\\ \hline
            ligne & n & i & reponse & Affichage \\ \hline
            initialisation & 3 & ? & 42 &  \\ \hline
            29 & & 3 &  & \\ \hline
            31 & & & & 3 \\ \hline
            32 & & 1 & & \\ \hline
            31 & & & & 1 \\ \hline
            32 & & -1 & & \\ \hline 
            33 & &  & & \carriagereturn \\ \hline 
            36 & \multicolumn{4}{l|}{renvoie 42} \\ \hline
          \end{tabular}
        }}\\ \hline
      16 & & 42 & \\ \hline
      17 & & &   foo(3) = 42\carriagereturn \\ \hline
      19 &\multicolumn{3}{r|}{
        \put(1,0){\noindent
          \begin{tabular}[t]{|r|c|c|}
            \multicolumn{3}{l}{\C{bar(3)}}\\ \hline
            ligne & n  & Aff. \\ \hline
            init. & 3 &  \\ \hline
            41 &  &  3 \\ \hline
            42 &\multicolumn{2}{r|}{
              \put(1,0){\noindent
                \begin{tabular}[t]{|r|c|c|}
                  \multicolumn{3}{l}{\C{bar(1)}}\\ \hline
                  ligne & n  & Aff.\\ \hline
                  init. & 1 & \\ \hline
                  46 &  & 1\carriagereturn\\ \hline
               \end{tabular}
              }}\\ \hline
         \end{tabular}
        }}\\ \hline
     22 & \multicolumn{3}{l|}{renvoie \C{EXIT\_SUCCESS}} \\ \hline
     \end{tabular}
        \caption{Trace du programme de l'exercice 2.}
        \label{trace}
\end{small}
  \end{table}
\end{correction}

\begin{baremeenv}
 Maximum \pt{4}. Si des erreurs, maximum \pt{3,75}.
  \begin{enumerate}[(a)]
\item \pt{+1} deux premières lignes de la trace du main sont correctes
  (identification des variables et leurs initialisations).
  \begin{enumerate}[$\ast$]
  \item \pt{-0,5} par variable manquante ou en trop
  \item \pt{-0,5} par initialisation manquante ou en trop
  \end{enumerate}
\item \pt{+1,25} Pour l'appel à foo :
  \begin{enumerate}[$\ast$]
  \item \pt{+0,25} pour foo(3) ligne 16 
\item \pt{+0,25} une colonne pour le paramètre formel n bien
  initialisé à 3 
\item \pt{+0,5} pour le déroulement correct de la boucle (deux tours, i
  = -1), 0 si une erreur.
\item \pt{+0,25} pour le retour de 42 et l'affectation ligne 16.
 \end{enumerate}
\item \pt{+1,25}  pour l'appel récursif à bar :
  \begin{enumerate}[$\ast$]
  \item \pt{+0,25} pour bar(3) ligne 19 
\item \pt{+0,5} pour le déclenchement du second appel ligne 42.
\item \pt{+0,25} une colonne pour le paramètre formel n bien
  initialisé à 3 et à 1 dans le second appel.
  \item \pt{+0,25} pour une profondeur de exactement 2 (pas plus). 
\end{enumerate}
\item \pt{+0,25} pour au moins l'un des deux affichages (3 1
  \carriagereturn) de foo ou de bar correct. 
\end{enumerate}
\end{baremeenv}

\section{Sans fonctions, for ou while, if, structures}

Il est demandé de résoudre les trois problèmes suivants sans définir de
fonctions utilisateurs.  L'ensemble du code sera à écrire dans la
fonction principale \verb|main|.

\subsection{Factorielle}

\question Écrire le programme qui : demande à l'utilisateur d'entrer
un entier; calcule la factorielle de l'entier saisi; affiche le
résultat du calcul, comme dans l'exemple suivant. \bareme{2}

On supposera que l'entier saisi est toujours supérieur ou égal à zéro
(on ne teste pas la saisie de l'utilisateur). Rappel : la factorielle
d'un entier $n$, notée habituellement $n!$, vaut le produit de tous
les entiers de $1$ à $n$ : $n! = 1\times 2\times \ldots\times n$.

\paragraph{Exemple d'exécution :}
\begin{small}
\begin{verbatim}
Saisissez un entier : 5
factorielle de 5 = 120
\end{verbatim}
\end{small}


\begin{correction}
  \begin{small}
\begin{verbatim}
#include <stdlib.h> /* EXIT_SUCCESS */
#include <stdio.h> /* printf, scanf */

/* declarations constantes et types utilisateurs */


/* fonction principale */
int main()
{
    int n = 0;       /* saisie utilisateur */
    int produit = 1; /* accumulateur pour le produit */
    int i;           /* var de boucle */

    printf("Saisissez un entier : ");
    scanf("%d", &n);

    for (i = 1; i <= n; i = i + 1)
    {
        produit = produit * i;
    }

    printf("factorielle de %d = %d\n", n , produit);
    
    return EXIT_SUCCESS;
}
\end{verbatim}
  \end{small}
  
  \begin{baremeenv}
Si tout juste à l'aide d'un for : \pt{2} sinon maxi \pt{1,75}.
    \begin{itemize}
      \item \pt{0.25}  déclaration (avec ou sans initialisation) d'une variable
        entière pour la saisie
      \item \pt{0.25} saisie à l'aide d'un scanf avec ou sans 
      \item \pt{0.5} boucle for pas de point en moins si oublie de
        déclaration de la variable de boucle
      \item \pt{0.25} bloc de boucle exécuté $n$ ou  $n - 1$ fois
      \item \pt{0.25} déclaration d'un accumulateur pour stocker le
      produit
    \item \pt{0.25} initialisation à 1 de l'accumulateur
    \item \pt{0.25} affichage du résultat du calcul avec les deux
      conversions (faux ou juste)
    \end{itemize}
  \end{baremeenv}
\end{correction}

\subsection{Halte à la dette}

Pippo emprunte 100 brouzoufs à 6\% d'intérêts annuel. Il rembourse
par versements de 25 brouzoufs chaque année, en commençant un an après
la date d'emprunt. Combien d'années devra t'il rembourser son emprunt
?  Chaque année, on calcule d'abord les intérêts à ajouter à la dette
actuelle, puis on soustrait le montant du versement annuel.

\question Écrire un programme qui affiche le plan de remboursement de
Pippo, c'est à dire, pour chaque année : les intérêts de l'année, le
versement, la dette restant à rembourser.\bareme{2.5}

Votre programme doit pouvoir être exécuté pour n'importe quelle autre
initialisation des données du prêt.



\paragraph{Exemple d'exécution :}
\begin{small}
\begin{verbatim}
annee 1, interets : 6, versement : 25, dette : 81
annee 2, interets : 4.86, versement : 25, dette : 60.86
annee 3, interets : 3.6516, versement : 25, dette : 39.5116
annee 4, interets : 2.3707, versement : 25, dette : 16.8823
annee 5, interets : 1.01294, versement : 25, dette : -7.10477
\end{verbatim}
\end{small}

\begin{correction}
  \begin{small}
\begin{verbatim}
#include <stdlib.h> /* EXIT_SUCCESS */
#include <stdio.h> /* printf, scanf */

/* déclaration constantes et types utilisateurs */

/* Fonction principale */
int main()
{
    double dette = 100.0;
    double versement = 25.0;
    double tauxa = 0.06;
    int annee = 0;
    double interets;
    double total = 0.0;

    while (dette > 0)
    {
        interets = dette * tauxa;
        dette = dette + interets - versement;
        annee = annee + 1;
        printf("annee %d, interets : %g, versement : %g, dette : %g\n", 
                annee, interets, versement, dette);
        total = total + versement;
    }   

    return EXIT_SUCCESS;
}
\end{verbatim}
  \end{small}
  
  \begin{baremeenv}
Si tout juste à l'aide d'un while: \pt{2.5} sinon maxi \pt{2,25}.
    \begin{itemize}
      \item \pt{0.25}  déclaration et initialisation de la dette et
        du taux comme des double (on ne regarde pas versement).
      \item \pt{0.5} une et une seule boucle while.
      \item \pt{0.5} modification de la valeur de la dette dans la
        boucle (\pt{0.25} si mauvaise modification)
      \item \pt{0.25} déclaration et initialisation du compteur
        d'années (une erreur de 1 est acceptée)
      \item \pt{0.25} incrément du compteur d'année à chaque tour de boucle
  \item \pt{0.25} affichage dans la boucle
    \item \pt{0.25} affichage des bonnes valeurs avec les bonnes conversions
    \end{itemize}
  \end{baremeenv}
\end{correction}

\question Le dernier versement est trop important. Comment modifier
votre programme de manière à  ne rembourser que le reste à payer,
dernière année ?\bareme{1} La dernière ligne affichée sera alors :
\begin{small}
\begin{verbatim}
annee 5, interets : 1.01294, versement : 17.8952, dette : 0
\end{verbatim}
\end{small}

\begin{correction}
juste avant affichage dans le while on insère :
  \begin{small}
\begin{verbatim}
        if (dette < 0)
        {
            versement = versement + dette; /* on supprime le trop vers'e */
            dette = 0;
        }
\end{verbatim}
  \end{small}
  
  \begin{baremeenv}
Si tout juste : \pt{1} sinon maxi \pt{0.75}.
    \begin{itemize}
\item \pt{0.25} pour un if dans le while (ou après si condition du while
  modifiée)
\item \pt{0.25} pour un pour dette mise à zéro avant affichage
\item \pt{0.25} pour un versement égal au reste à percevoir avant affichage
\end{itemize}
  \end{baremeenv}
\end{correction}


\question Si le versement est inférieur aux intérêts, que dire de
l'exécution de ce programme ?\bareme{0.5}

\begin{correction}
  Si le versement est inférieur aux intérêts la première année, la
  dette va augmenter et elle ne sera jamais nulle (ou négative). Donc
  le programme va boucler et la dette atteindre une valeur
  représentant l'infinie.

  \begin{baremeenv}
    \pt{0.5} si mention du fait que le programme ne s'arrête jamais.
  \end{baremeenv}
\end{correction}

\subsection{Deux jours avant}
\begin{figure}
\small
\begin{listing}{1}
#include <stdlib.h> /* EXIT_SUCCESS */
#include <stdio.h> /* printf, scanf */

/* déclaration constantes et types utilisateurs */

/* Fonction principale */
int main()
{
  int nbjours[13] = {31, 31, 28, 31, 30, 31, 30, 31, 31, 30, 31, 30, 31};
  /* nombre de jours de chaque mois, de decembre 2010 a decembre 2011 */
    
  struct date_s demain = {1, 6, 2011}; /* exemple */
  struct date_s hier;

  /* calcul de la date d'hier */
   
  /* affichage du résultat */

  /* valeur fonction */
  return EXIT_SUCCESS;
}
\end{listing}
  \caption{Deux jours avant (à compléter)}
  \label{fig:deuxj}
\end{figure}

\question 
Compléter le programme de la figure~\ref{fig:deuxj} :
\begin{itemize}
\item définir le type utilisateur \verb+struct date_s+, pour stocker
  une date sous la forme numéro du jour, numéro du mois et année, de
  manière à ce que la valeur initiale de la variable \C{demain}
  corresponde à la date du premier juin 2011;\bareme{1}
\item à partir de la date de demain, calculer la date d'hier (deux
  jours avant demain). Vous supposerez que la date de demain
  appartient à l'année 2011. Pour connaître le nombre de jours de
  chaque mois, vous utiliserez le tableau \C{nbjours}. Ce tableau
  contient à chaque indice $i$ entre $1$ et $12$ le nombre de jours du
  $i$ème mois de l'année 2011 et à l'indice $0$ le nombre de jours du
  mois de décembre 2010;\bareme{2}
\item afficher le résultat comme dans les exemples suivants.\bareme{0.5}
\end{itemize}

Votre programme doit fonctionner et calculer la date correcte pour
n'importe quelle autre initialisation de la variable \C{demain} par
une date de l'année 2011.

\paragraph{Exemple d'exécution :}
\begin{small}
\begin{verbatim}
Demain : 1/6/2011
Hier : 30/5/2011
\end{verbatim}
\end{small}

\paragraph{Autres exemples} (en modifiant l'initialisation de
la variable \C{demain}) :
\begin{small}
\begin{verbatim}
Demain : 2/1/2011
Hier : 31/12/2010

Demain : 12/5/2011
Hier : 10/5/2011
\end{verbatim}
\end{small}
\begin{correction}
  \begin{small}
\begin{verbatim}
#include <stdlib.h> /* EXIT_SUCCESS */
#include <stdio.h> /* printf, scanf */

/* déclaration constantes et types utilisateurs */
struct date_s 
{
    int j; /* jour */
    int m; /* mois */
    int a; /* annee */
};
/* Fonction principale */
int main()
{
  int nbjours[13] = {31, 31, 28, 31, 30, 31, 30, 31, 31, 30, 31, 30, 31};
  /* nombre de jours de chaque mois, de decembre 2010 a decembre 2011 */
    
  struct date_s demain = {2, 1, 2011}; /* exemple */
  struct date_s hier;

  /* calcul de la date d'hier */
  hier = demain;
  hier.j = demain.j - 2;
  if (hier.j < 1) 
  {
      hier.m = demain.m - 1;
      hier.j = nbjours[hier.m] + hier.j;
  }
  if (0 == hier.m) 
  {
      hier.m = 12;
      hier.a = 2010;
  }

  /* affichage du résultat */
  printf("Demain : %d/%d/%d\n", demain.j, demain.m, demain.a);
  printf("Hier :   %d/%d/%d\n", hier.j, hier.m, hier.a);

  /* valeur fonction */
  return EXIT_SUCCESS;
}
\end{verbatim}
  \end{small}
  
  \begin{baremeenv}
Si tout juste: \pt{3.5} sinon maxi \pt{2,25}.
    \begin{itemize}
      \item \pt{0.5} déclaration correcte syntaxiquement d'une structure
      contenant au moins trois champs (entiers, double ou char).
      \item \pt{0.5} la déclaration de la structure est en accord avec son usage
dans la suite.
      \item \pt{2} calcul exact sinon maxi \pt{1.75} 
        \begin{itemize}
        \item \pt{0.5}  si utilisation d'un ou plusieurs if pour distinguer plusieurs
          cas pour les jours (cas du premier et deuxième jours du mois
          \emph{vs} autres jours du mois).
        \item \pt{0.5} on compte\pt{0.25} si utilisation de exactement une case  du
            tableau (pour les cas de début de mois) et \pt{+0.25} s'il s'agit de la case du mois précédent le
            mois courant 
\item \pt{0.25} Si calcul du jour toujours correct
\item \pt{0.25} Si calcul du mois toujours correct
\item \pt{0.25} Si calcul de l'année toujours correct, 
        \end{itemize}
      \item \pt{0.5} Affichage au bon format de demain et hier
        (décomposition en affichage de trois valeurs
        chacune).\pt{zéro} sinon.
   \end{itemize}
  \end{baremeenv}
\end{correction}

\section{Écriture de fonctions}
\begin{enumerate}
\item Déclarer et définir la procédure \verb|menu| qui n'a pas d'entrée et
  affiche :\label{item:menu}\bareme{1}
\begin{small}
\begin{verbatim}
********
* MENU *
********
\end{verbatim}
\end{small}
\item Déclarer et définir une fonction qui calcule la valeur absolue d'un réel.\label{valeurabsolue}\bareme{1}
\end{enumerate}

\begin{correction}
Difficultées diverses mais c'est voulu, on compte :
0.5 pt déclaration correcte et 0.5 pt code correct.

\end{correction}

\newpage
\thetotalpointsint