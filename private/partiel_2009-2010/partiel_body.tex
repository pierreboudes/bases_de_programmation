% -*- coding: utf-8 -*-

\newcounter{questioncount}
\setcounter{questioncount}{0}
\newcommand{\question}{\addtocounter{questioncount}{1}\paragraph{Question \Alph{questioncount}.}}
\newcommand{\commentaire}[1]{}

\entete{Éléments d'informatique : partiel de mi-semestre}
\vspace{-1cm}
\begin{description}
\item[Durée :] 3 heures.
\item[Documents autorisés :] Aucun.
\item[Recommandations :] Un barème vous est donné à
titre indicatif afin de vous permettre de gérer votre temps. La
notation prendra en compte à la fois la syntaxe et la sémantique de
vos programmes, c'est-à-dire qu'ils doivent compiler correctement. Une
fois votre programme écrit, il est recommandé de le faire tourner à la
main sur un exemple pour s'assurer de sa correction.
\end{description}


\section{Étude de programmes et questions de cours (7 points)}

\subsection{Trace d'un programme (4.5 points)}

\begin{small}
  \listinginput{1}{programmes/somme.c}
\end{small}

\question 
À quoi sert la ligne 6 et à quelle étape de la compilation sera t-elle traitée ?\bareme{0.5}

\begin{correction}
  La ligne 6 contient une directive préprocesseur qui définit une constante symbolique N comme étant équivalente à 2. Le préprocesseur du compilateur C traite ces directives avant la compilation proprement dite. Le traitement consistera ici en remplacer partout dans le programme C le texte N par le texte 2.
\end{correction}

\begin{baremeenv}
  Compter tout juste ou tout faux.
\end{baremeenv}

\question
Simulez l'exécution du programme, en réalisant sa
  \textbf{trace} : l'exécution du programme est représentée par un
  tableau à $n + 2$ colonnes; la première colonne étant le numéro de
  la ligne exécutée, les $n$ colonnes suivantes, les colonnes des $n$
  variables du programme ($n$ à déterminer) et la dernière colonne, la
  colonne servant à l'affichage à l'écran du programme. Seules les
  lignes modifiant l'état mémoire du programme sont à reporter dans le
  tableau. \bareme{3}
  \begin{correction}
La trace est donnée dans le tableau~\ref{simulation}.
\begin{table}[h]
  \begin{small}
    \begin{center}
      \begin{tabular}[h]{|r|c|c|c|c|c|l|}
        \hline
ligne & t[0] & t[1] & s & i & j & Affichage (sortie écran) \\ \hline
initialisation  & 3 & 0 & ? & ? & ? &  \\ \hline
%     t[0] & t[1] & s  & i  & j 
18 &      &       &    & 0 &    &  \\ \hline
21 &      &       & 0 &    &    &  \\ \hline
22 &      &       &    &    & 1 &  \\ \hline
25 &      &       & 1 &    &    &  \\ \hline
26 &      &       &    &    & 2 &  \\ \hline
25 &      &       & 3 &    &    &  \\ \hline
26 &      &       &    &    & 3 &  \\ \hline
25 &      &       & 6 &    &    &  \\ \hline
26 &      &       &    &    & 4 &  \\ \hline
28 &      &       &    &    &    & 6\verb+\n+ \\ \hline
29 &      &       &    & 1 &    &  \\ \hline
21 &      &       & 0 &    &    &  \\ \hline
22 &      &       &    &    & 1 &  \\ \hline
28 &      &       &    &    &    & 0\verb+\n+ \\ \hline
29 &      &       &    & 2 &    &  \\ \hline
31 & \multicolumn{6}{|c|}{renvoie \C{EXIT\_SUCCESS}}\\ \hline
		\end{tabular}
		\caption{Trace du programme du premier exercice.}
		\label{simulation}
	\end{center}
      \end{small}
    \end{table}
  \end{correction}


    \begin{baremeenv}
      Compter
      \begin{itemize}
      \item 0.5 pt pour une ligne d'en-tête juste, autrement dit pour
        les colonnes (0 sinon)
      \item 0.5 pt pour une ligne d'initialisation juste (0 sinon)
      \item aucune pénalité si la dernière ligne est absente (renvoie
        ....).
      \item 2 point pour le reste avec 0.5 point de pénalité par ligne
        erronée.
      \item Si des valeurs de variables non modifiées apparaissent sur
        certaines lignes compter également une pénalité de 0.5 pt.
      \end{itemize}
    \end{baremeenv}

\question
Que calcule et affiche ce programme ?\bareme{1}

\begin{correction}
   Pour chaque élément du tableau ce programme calcule et affiche la somme des $t[i]$ premiers entiers  : \[1 + \ldots + t[i] = \sum_{k = 1}^{t[i]} k.\]
\end{correction}

\begin{baremeenv}
  Compter 0.5 pt si il est question de cette somme sur la première case du tableau et 0.5 point si il est question de faire l'affichage d'un calcul (attention : toujours le même) pour chaque case du tableau.
\end{baremeenv}


\subsection{Une erreur classique (2.5 points)} 

Pippo a écrit un programme C. Celui-ci compile, mais une erreur survient à l'exécution, qu'il ne comprend pas.

\begin{verbatim}
$ gcc puissance.c -o puissance.exe
$ puissance.exe
Entrer un nombre reel : 2.3
Entrer son exposant (entier positif) : 2
Segmentation fault
\end{verbatim}
\question
Expliquer brièvement ce que signifie ce message d'erreur (dernière ligne).\bareme{1}
\begin{correction}
Le message d'erreur \C{Segmentation fault} signifie que le programme a tenté de lire ou d'écrire dans un espace mémoire qui ne lui était pas réservé, ce que le système a détecté et refusé, provoquant la terminaison prématurée du programme et l'affichage du message d'erreur. 
\end{correction}

\begin{baremeenv}
  Compter 0.25 pt si ça parle tout de suite de l'erreur du scanf (au lieu de donner la signification du message d'erreur), 0.5 pt si il est juste fait mention de la mémoire (sans réservation etc.).
\end{baremeenv}

\medskip
Voici quelques lignes choisies du programme.
\begin{listing}{10}
int main()
{
    /* Déclaration et initialisation des variables */
   double x;
   int n;
\end{listing}
\vdots
\begin{listing}{22}
    printf("Entrer un nombre reel : ");
    scanf("%lg", x);
    printf("Entrer son exposant (entier positif) : ");
    scanf("%d", n);
\end{listing}

\question
Que faut-il corriger ? \bareme{1}

\begin{correction}
Il faut mettre une esperluette devant le nom de la variable dans les deux \C{scanf}.
 Comme ceci :
  \begin{listing}{22}
    printf("Entrer un nombre reel : ");
    scanf("%lg", &x); /* <-- ici */
    printf("Entrer son exposant (entier positif) : ");
    scanf("%d", &n);  /* <-- ici */
\end{listing}

Complément de correction.
Le rôle de l'esperluette est de donner à \C{scanf} l'adresse de la variable dans laquelle écrire le résultat de la saisie utilisateur. Sans esperluette \C{scanf} récupère la valeur de la variable et l'utilise comme une adresse, ce qui provoque en général (si on a de la chance) une erreur de segmentation (\C{Segmentation fault}). 
\end{correction}

\begin{baremeenv}
  Compter 0.5 pt par scanf correctement corrigé (et oui c'est un piège). Si d'autres corrections, hors sujet, apparaissent diminuer la note de l'exercice (jusqu'à zéro).
\end{baremeenv}


\question
 Quelle option de compilation aurait dû utiliser Pippo pour obtenir de
  l'aide ?\bareme{0.5}

\begin{correction}
En utilisant l'option \verb+-Wall+ (afficher tous les avertissements) de la commande \C{gcc}, Pippo aurait eut à la compilation un message d'avertissement le prévenant d'un erreur probable à la ligne 23 et de même pour la ligne 25.
\end{correction}

\begin{baremeenv}
Compter juste si il est fait mention de \verb+-Wall+. Sinon zéro.  
\end{baremeenv}


\section{Faut-il aller au cinéma ? (4 points)}

Je viens de gagner une place de cinéma pour un séance ce soir. Je dispose des informations suivantes, codées dans des variables entières, pour décider si je vais me rendre à cette séance ou rester chez moi :
\begin{itemize}
\item \C{critique} une critique du film lui attribuant une appréciation parmi MAUVAIS, MOYEN, BON.
\item \C{acteurs} mon appréciation personnelle du choix des acteurs : MOYEN ou BON.
\item \C{distance} la distance approximative (en kilomètres) qui me sépare du cinéma.
\end{itemize}
Vous utiliserez des constantes symboliques pour coder les appréciations et vous initialiserez les trois variables, \C{critique}, \C{acteurs}, \C{distance} à des valeurs de votre choix. Pour prendre ma décision je dispose de l'arbre de décision suivant. 

\begin{figure}[!ht]
    \begin{center}
\small
% Define block styles
\tikzstyle{decision} = [diamond, draw, fill=blue!20, 
    text width=4.5em, text badly centered, inner sep=0pt]
\tikzstyle{block} = [rectangle, draw, fill=blue!20,  node distance = 2.5cm,
    text width=3.5em, text centered, rounded corners, minimum height=2em]
\tikzstyle{line} = [draw, -latex']
\tikzstyle{cloud} = [draw, ellipse,fill=red!20,  text width=6em, text centered, 
    minimum height=2em,]
\tikzstyle{choix} = [rectangle, inner sep=2pt,
    text centered]
    
\begin{tikzpicture}[node distance = 2.8cm, auto]
    % Place nodes
    \node [cloud] (critique) {Qu'en dit la critique ?};
    \node [cloud,  below of=critique] (acteurs) {Choix des acteurs ?};
    \node [left of=acteurs] (A) {};
   \node [right of=acteurs] (B) {};
    \node [cloud, left of=A] (distance1) {Distance $\leq 5$ km ?};
    \node [block, right of=B] (noncritique) {Je reste};
    \node [block, below left of=distance1] (ouidistance) {J'y vais};
    \node [block, below right of=distance1] (nondistance) {Je reste};
    \node [block, below left of=acteurs] (nonacteurs) {Je reste};
    \node [cloud, below right of=acteurs] (distance2) {Distance $\leq 1$ km ?};
    \node [block, below left of=distance2] (ouid2) {J'y vais};
    \node [block, below right of=distance2] (nond2) {Je reste};
    
   % Draw edges
\path [line] (critique) -- node [near start, left, choix] {bon} (distance1);
\path [line] (critique) -- node [near start, right, choix] {moyen} (acteurs);
\path [line] (critique) -- node [near start, right, choix] {mauvais} (noncritique);
\path [line] (distance1) -- node [near start, left, choix] {oui} (ouidistance);
\path [line] (distance1) -- node [near start, right, choix] {non} (nondistance);
\path [line] (acteurs) -- node [near start, left, choix] {moyen} (nonacteurs);
\path [line] (acteurs) -- node [near start, right, choix] {bon} (distance2);
\path [line] (distance2) -- node [near start, left, choix] {oui} (ouid2);
\path [line] (distance2) -- node [near start, right, choix] {non} (nond2);
\end{tikzpicture}
    \end{center}
    \caption{Décider si je vais au cinéma}
    \label{fig:ad}
\end{figure}

\question
Écrire un programme implantant cet arbre de décision et affichant la décision à prendre (quel que soit le choix d'initialisation des variables). \bareme{4}

\begin{correction}
 \listinginput{1}{programmes/cinema.c}
\end{correction}

\begin{baremeenv}
  On accepte le fait que la distance soit une variable réelle, sans pénalité. 

Compter :
\begin{itemize}
\item 
  0,5 points pour le codage BON, MOYEN, MAUVAIS à l'aide de define s'il est correct (pas d'erreur de syntaxe, comme un signe égal), retirer une pénalité de 0.25 s'il a trop de constantes symboliques. S'il y a par exemple une constante symbolique pour définir la valeur d'initialisation de la distance.
\item 0,5 points pour des
  déclarations avec initialisations correctes.  
\item 0,5 points pour l'indentation (si celle-ci est absente le sanctionner).
\item  Une pénalité de 0,25 points par oubli du stdlib, du  stdio ou du return final (on ne sanctionne pas l'utilisation d'un 0 à la place du \C{EXIT\_SUCCESS}).
\item On admet que les cas exclusifs soient traités dans des if en cascade (comme dans le corrigé à la racine de l'arbre) plutôt qu'avec des if else emboîtés. On admet les else if même si ça n'a pas été vu en cours. 
\item Pour le reste, dessiner l'arbre sous-jacent au programme de l'étudiant et calculer la distance d'édition entre cet arbre est celui demandé par les opérations suivantes (il ne s'agit pas d'un arbre ordonné) :
  \begin{itemize}
  \item étiquette à modifier sur un sommet ou un arc
\item inversion père-fils
\item sommet absent
  \end{itemize}
On part d'un total de 2.5 et on compte 0.5 point de pénalité par opération d'édition nécessaire.  
\end{itemize}
\end{baremeenv}

\section{Puissance (4 points)}

\question
\bareme{2}
Écrire un programme qui :
\begin{itemize}
\item   demande à l'utilisateur d'entrer un nombre réel $x$;
\item calcule et affiche la valeur de $x^2$.
\end{itemize}

Exemple d'exécution :
\begin{verbatim}
Entrer un nombre reel : 7.5e9 
7.5e+09 exposant 2 = 5.625e+19
\end{verbatim}

\begin{correction}
 \listinginput{1}{programmes/carre.c}
\end{correction}
\begin{baremeenv}
    \begin{itemize}
  \item Un point pour une saisie correcte
\item un point pour le calcul et l'affichage
\item des pénalités de 0.25 pt si il y a des erreurs : manque stdio, return \verb+EXIT_SUCCESS+ (on admet return 0)
  \end{itemize}
\end{baremeenv}

\question
Réecrire ce programme (ou indiquer clairement les modifications à apporter) de manière à ce qu'il demande également à l'utilisateur l'exposant $n$ (un entier positif) auquel doit être élevé $x$ et affiche le résultat du calcul de $x^n$. Il faut programmer ce calcul (ne pas faire appel à une bibliothèque offrant la fonction ad-hoc). L'algorithme utilisé devra apparaître clairement. \bareme{2}

Exemple d'exécution :
\begin{verbatim}
Entrer un nombre reel : 12.35
Entrer son exposant (entier positif) : 10
12.35 exposant 10 = 8.2541e+10
\end{verbatim}

\begin{correction}
  \listinginput{1}{programmes/puissance.c}
\end{correction}
\begin{baremeenv}
  On ne regarde que la partie modifiée depuis le programme précédent et on compte 1 points pour un algorithme correct :
\begin{itemize}
\item mettre le résultat à 1 
\item Répéter n fois
  \begin{itemize}
  \item multiplier le résultat par x
  \end{itemize}
\item afficher le résultat
\end{itemize}

Et 1 point pour son écriture en C.

Pénalités de 0.25 pt s'il manque l'initialisation à 1, ou la saisie, ou l'affichage, ou la déclaration de la variable de boucle.
\end{baremeenv}

\section{Histogrammes (5 points)}

\question
Soit un entier $n$ initialisé à une valeur positive de votre choix. Écrire un programme qui affiche une ligne contenant $n$ fois le caractère \verb+'#'+ et se terminant par un saut de ligne. Pour cette question, vous pouvez répondre en ne donnant que le code de la fonction principale du programme (\C{main}). \bareme{1}

\begin{correction}
\begin{verbatim}
int main()
{
  int n = 6;
  int i; /* var. de boucle */

  for (i = 0; i < n; i = i + 1) /* repeter n fois */
  {
    printf("#"); /* afficher # */
  }
  /* fin de ligne */
  printf("\n");

  return EXIT_SUCCESS;
}
\end{verbatim}
\end{correction}

\begin{baremeenv}
  On part de 1 point et on enlève 0.25 pt par erreur : oubli de déclarer la variable de boucle, absence du retour à la ligne etc. Zéro s'il n'y a pas de for.
\end{baremeenv}

\medskip
Soit un tableau d'entiers, initialisé à des valeurs de votre choix, toutes positives ou nulles, et dont la taille N sera définie par une constante symbolique. Dans les exemples suivants la taille du tableau est 10 et les valeurs contenues dans le tableau sont $3,0,2,8,9,10,3,5,5,2$.

On veut écrire un programme qui affiche sous forme d'histogramme (graphique en bâtons, ou graphique barres) les données du tableau : chaque barre de l'histogramme aura une longueur égale à la donnée représentée.

\question
Écrire un programme qui affiche sous forme d'histogramme les données du tableau. Les barres seront dessinées horizontalement, à l'aide du caractère \verb+'#'+. \bareme{3}

Exemple d'exécution :
\begin{small}
\begin{verbatim}
Graphique barre 1 :

###

##
########
#########
##########
###
#####
#####
##
\end{verbatim}
\end{small}


\begin{correction}
Voir le programme complet plus loin.
\end{correction}
\begin{baremeenv}
   On compte entre un et trois points si il y a une double boucle imbriquée : on part de 3 on enlève des pénalités jusqu'à 1 (ne pas descendre en dessous). Pénalité faible : il manque un include,  le N n'est pas dans un define et non utilisé dans la boucle, le tableau n'est pas déclaré, pas initialisé, il manque un saut de ligne, variable de boucle non déclarée. Cas particulier : on ne compte que 1 point si les deux boucles ont la même variable, sans regarder plus loin.

S'il n'y a pas de double boucle on ne corrige pas plus on met zéro.
\end{baremeenv}

\question
Indiquer comment modifier le programme de la question précédente, de manière à demander à l'utilisateur le caractère qui sera utilisé pour dessiner les barres (à la place du \verb+#+).\bareme{1}

\begin{correction}
Voir le programme complet plus loin.
\end{correction}
\begin{baremeenv}
    On ne compte pas de pénalité si le scanf est fait dans un format \verb+"%c"+ au lieu de \verb+" %c"+.  On compte un demi point pour le scanf correct et un demi point pour le printf correct. Pénailité de 0.25 si la variable n'est pas correctement déclarée. 
\end{baremeenv}

\paragraph{Question bonus (plus difficile).}
Ajouter à votre programme un nouvel affichage en histogramme des données dont les barres progressent cette fois verticalement. Indication : avant cela, votre programme devra trouver le maximum parmi les données du tableau.\bareme{2}

Exemple d'exécution :
\begin{small}
\begin{verbatim}
Graphique barre 2 :

                ##            
             ## ##            
          ## ## ##            
          ## ## ##            
          ## ## ##            
          ## ## ##    ## ##   
          ## ## ##    ## ##   
 ##       ## ## ## ## ## ##   
 ##    ## ## ## ## ## ## ## ##
 ##    ## ## ## ## ## ## ## ##
\end{verbatim}

% \paragraph{Question 3.}
% Expliquer les modifications à apporter votre programme pour ajouter en légende la valeur des données, et pour que l'utilisateur puisse choisir le caractère utilisé pour l'affichage. 
\end{small}

\begin{correction}
  \listinginput{1}{programmes/graphbar.c}
\end{correction}
\begin{baremeenv}
Deux points si c'est vraiment parfait, uniquement.

On donne plus d'un point uniquement lorsque la réponse est très proche d'une solution correcte. On admet une réponse à base de tableau bidimensionnel.  

On peut aller jusqu'à un point si des parties du programme ne fonctionnent pas (recherche du maximum par exemple), mais qu'il y a de bonnes idées (par exemple, penser à balayer la ligne courante). 

S'il n'y a que la recherche du maximum on ne compte qu'un quart de point.  
\end{baremeenv}