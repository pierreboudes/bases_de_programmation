
\subsection{Faut-il renoncer à rembourser la dette publique ?}

\begin{figure}[h!]
    \begin{center}
% Define block styles
\tikzstyle{decision} = [diamond, draw, fill=blue!20, 
    text width=4.5em, text badly centered, node distance=3cm, inner sep=0pt]
\tikzstyle{block} = [rectangle, draw, fill=blue!20, 
    text width=5em, text centered, rounded corners, minimum height=4em, node distance=3.5cm]
\tikzstyle{line} = [draw, -latex']
\tikzstyle{cloud} = [draw, ellipse,fill=red!20,  text width=5.7em, text centered, 
    minimum height=2em,  node distance=3.2cm]
\tikzstyle{choix} = [rectangle, inner sep=2pt,
    text centered]
    
\begin{tikzpicture}[node distance = 1.7cm, auto]
    % Place nodes
    \node [cloud] (temps) {Comment sont les marchés ?};
    \node [below of=temps] (fantome) {};
    \node [left of=fantome] (fantome2) {};
    \node [block, left of=fantome2] (panique) {Ne plus rembourser};
    \node [cloud, right of=fantome] (inquiets) {Dette $ > 0.5$
      PIB ?};
    \node [block,right of=inquiets] (confiants) {Ne rien changer};
    \node [cloud,  below left of=inquiets] (pop) {La population
      refuse l'austérité ?};
    \node [block, below left of=pop] (ouipop) {Ne plus
      rembourser};
    \node [block, below right of=pop] (nonpop) {Ne
      rien changer};
    \node [block, below right of=inquiets] (nondette) {Ne rien changer};

   % Draw edges
\path [line] (temps) -| node [near start, above, choix] {paniqués} (panique);
\path [line] (temps) -- node [near start, right, choix] {inquiets} (inquiets);
\path [line] (temps) -| node [near start, above, choix] {confiants}
(confiants);
\path [line] (inquiets) -- node [left, choix] {oui} (pop);
\path [line] (inquiets) -- node [right, choix] {non} (nondette);
\path [line] (pop) -- node [left, choix] {oui} (ouipop);
\path [line] (pop) -- node [right, choix] {non} (nonpop);
\end{tikzpicture}
    \end{center}
    \caption{Décider s'il faut continuer de rembourser la dette}
    \label{fig:ad}
\end{figure}



Un arbre de décision
est un graphe particulier où les n{\oe}uds sont 
des questions et les arêtes sont les réponses à ces questions.
Il se lit de haut en bas. On avance dans l'arbre en répondant aux
questions. Les n{\oe}uds les plus bas jouent le rôle particulier de
classes de réponse au problème initial. 

Ici, il y a deux classes de réponse : <<~Ne plus rembourser~>> et
<<~Ne rien changer~>>.
Par exemple, si les marchés sont inquiets, si la dette est
strictement supérieure à $0.5$ fois le PIB et que la population refuse
l'austérité, alors on ne rembourse plus.


Vous utiliserez quatre variables dont vous coderez les valeurs avec des
constantes symboliques. Les trois premières représentent les propriétés du jour
courant pour prendre la décision, la dernière représente la décision à
prendre :
  
\begin{itemize}
\item \verb|marches| est un entier représentant l'état des marchés,
  elle peut valoir un entier signifiant PANIQUE, ou un entier
  signifiant INQUIETUDE ou un entier signifiant CONFIANCE.
\item \verb|dette| est un nombre représentant la dette en
  pourcentage du PIB (une valeur de 50 représente 50\% du PIB).
\item \verb|refus| est un entier représentant le refus de l'austérité
  qui peut avoir la valeur usuelle
 TRUE (si la population refuse l'austérité) ou la valeur usuelle FALSE (sinon).
\item \verb|rembourser| est un entier représentant la décision à
  prendre et peut avoir la valeur TRUE (s'il faut continuer de
  rembourser) ou FALSE (s'il faut arrêter de rembourser). Sa valeur
  initiale est TRUE. 
\end{itemize}

Votre programme sera en deux parties. Une première partie concernera
la prise de décision sans affichage, en fonction
des valeurs de \verb|marches|, \verb|dette|, \verb|refus|. Vous
donnerez des valeurs de votre choix à ces trois variables mais le
programme doit fonctionner pour tous les autres choix possibles. À la fin
de cette première partie \verb|rembourser| contiendra la valeur
correcte pour la décision prise (TRUE ou FALSE). La seconde partie
exploitera la valeur de \verb|rembourser| pour effectuer l'affichage.

\question
Écrire le programme complet en distinguant bien les deux
parties.\bareme{4}

\begin{correction}
\begin{figure}
  \centering
\begin{small}
\listinginput{1}{marches.c}
\end{small} 
  \caption{Faut-il annuler la dette ? -- Corrigé.}
  \label{fig:dropdebt}
\end{figure}
Correction figure~\ref{fig:dropdebt}.
\end{correction}


