% -*- coding: utf-8 -*-

\renewcommand{\labelitemi}{$\bullet$}

\newcommand{\commentaire}[1]{}

\entete{Travaux pratiques 3 : affichages et structure de contrôle  \emph{if}}

\newcommand{\TPshortName}{TP3}

% -*- coding: utf-8 -*-

Vous allez mettre tous vos programmes écrits dans ce TP dans le
répertoire \TPshortName. 

\begin{newenu}
\item À partir du début de votre arborescence, créez le répertoire
\TPshortName : \C{mkdir \TPshortName} 
\item Allez dans ce répertoire pour y mettre des fichiers : 
  \C{cd \TPshortName}
\end{newenu}

L'étape suivante est à répéter pour chaque nouveau programme (exo1, exo2 etc..) :
\begin{lastenu}
\item Créez un nouveau fichier source pour le langage C ou une nouvelle copie d'un programme existant.
  \begin{description}
  \item[Création]   \verb|gedit exo1.c &| (vous pouvez utiliser
    \C{emacs} ou \C{kwrite} au lieu de \C{gedit})
\item[Copie] Il est plus rapide de repartir d'une copie de votre programme \C{bonjour.c} du TP2 pour éviter de retaper tout le squelette. Dans le terminal : \\ \C{cp ../TP2/bonjour.c exo1.c} \\ \verb|gedit exo1.c &|\\ Vous pouvez-aussi ouvrir \verb+bonjour.c+ et utiliser la fonction \emph{Enregistrer sous...} de votre éditeur mais attention à enregistrer la nouvelle copie dans le bon répertoire. 
\end{description}
\end{lastenu}

Vous pouvez utiliser à tout moment la commande \C{ls} (list directory)
pour voir la liste des fichiers d'un répertoire. 


Les trois étapes suivantes seront à répéter autant de fois que nécessaire pour la mise au point de chaque programme (apprenez à utiliser les raccourcis clavier). 
\begin{lastenu}
\item Après avoir fini d'écrire votre programme, enregistrez le.
\item Créez un programme exécutable à partir de votre fichier source :\\
  \verb|gcc -Wall exo1.c -o exo1.exe|
\item Quand l'étape précédente a réussi (il faut lire attentivement les
  messages affichés), exécutez le programme pour
  vérifier qu'il fonctionne : \verb|exo1.exe| (ou
  \verb|./exo1.exe|).
\end{lastenu}


\section{Affichage}

\begin{newenu}
\item Écrire un programme \C{exo1.c} qui affiche à l'écran << coucou >>.
\item Modifier ce programme pour qu'il affiche à l'écran << coucou >> sur cinq lignes de deux façons :
  \begin{itemize}
  \item avec cinq printf;
  \item avec un seul printf.
  \end{itemize}
\item Écrire un programme \C{exo2.c}  qui affiche à l'écran l'évaluation de l'expression \C{7 * 3 + 2}.
\item Modifier ce programme pour qu'il affiche à l'écran l'évaluation
  de l'expression \C{3 * x + 2}, avec la variable entière \verb|x|
  initialisée à une valeur quelconque.
\end{newenu}

\begin{correction}
Durée 1/2 heure ?

\begin{verbatim}
/* declaration de fonctionnalites supplementaires */
#include <stdlib.h> /* EXIT_SUCCESS */
#include <stdio.h> /* printf */

/* declaration constantes et types utilisateurs */

/* declaration de fonctions utilisateurs */

/* fonction principale */
int main()
{
    /* affiche coucou sur une ligne 5 fois */
    printf("coucou\ncoucou\ncoucou\ncoucou\ncoucou\n");
    
    /* valeur fonction */
    return EXIT_SUCCESS;
}

/* definitions des fonctions utilisateurs */

\end{verbatim}

\begin{verbatim}
/* declaration de fonctionnalites supplementaires */
#include <stdlib.h> /* EXIT_SUCCESS */
#include <stdio.h> /* printf */

/* declaration constantes et types utilisateurs */

/* declaration de fonctions utilisateurs */

/* fonction principale */
int main()
{
    /* declaration et initialisation variables */
    int x = 3;

    printf("x * 7 + 2 = %d\n",x * 7 + 2);

  /* valeur fonction */
  return EXIT_SUCCESS;
}

/* definitions des fonctions utilisateurs */
\end{verbatim}
\end{correction}

\section{Faut-il répudier la dette ?}

\begin{figure}[h!]
    \begin{center}
% Define block styles
\tikzstyle{decision} = [diamond, draw, fill=blue!20, 
    text width=4.5em, text badly centered, node distance=3cm, inner sep=0pt]
\tikzstyle{block} = [rectangle, draw, fill=blue!20, 
    text width=5em, text centered, rounded corners, minimum height=4em, node distance=3.5cm]
\tikzstyle{line} = [draw, -latex']
\tikzstyle{cloud} = [draw, ellipse,fill=red!20,  text width=5.7em, text centered, 
    minimum height=2em,  node distance=3.2cm]
\tikzstyle{choix} = [rectangle, inner sep=2pt,
    text centered]
    
\begin{tikzpicture}[node distance = 1.7cm, auto]
    % Place nodes
    \node [cloud] (temps) {Comment sont les marchés ?};
    \node [below of=temps] (chatfantome) {};
    \node [cloud,  right of=chatfantome] (chat) {Les taux montent ?};
    \node [left of=chatfantome] (chatfantome2) {};
    \node [cloud, left of=chatfantome2] (temperature) {Dette $ > 50\%$ du
      PIB ?};
    \node [block,right of=chat] (nontemps) {Ne rien changer};
    \node [block, below left of=temperature] (ouitemperature) {Ne plus
      rembourser};
    \node [block, below right of=temperature] (nontemperature) {Ne
      rien changer};
    \node [block, below  left of=chat] (ouichat) {Ne plus rembourser};
    \node [block, below right of=chat] (nonchat) {Ne rien changer};

   % Draw edges
\path [line] (temps) -| node [near start, above, choix] {paniqués} (temperature);
\path [line] (temps) -- node [near start, right, choix] {inquiets} (chat);
\path [line] (temps) -| node [near start, above, choix] {stables} (nontemps);
\path [line] (temperature) -- node [left, choix] {oui} (ouitemperature);
\path [line] (temperature) -- node [right, choix] {non} (nontemperature);
\path [line] (chat) -- node [left, choix] {oui} (ouichat);
\path [line] (chat) -- node [right, choix] {non} (nonchat);
\end{tikzpicture}
    \end{center}
    \caption{Décider s'il faut continuer de rembourser la dette}
    \label{fig:ad}
\end{figure}



Un arbre de décision\footnote{(cf.
  \texttt{http\hspace{-0.5em}://fr.wikipedia.org/wiki/Arbre\_de\_décision}) 
  Les arbres de décision
  sont très utilisés en informatique pour prendre des décisions
  automatiquement. Ils sont soit programmés par un humain soit appris
  automatiquement par un algorithme d'apprentissage automatique.} 
est un graphe particulier où les n\oe{}uds sont 
des questions et les arêtes sont les réponses à ces questions.
Il se lit de haut en bas. On avance dans l'arbre en répondant aux
questions. Les n{\oe}uds les plus bas jouent le rôle particulier de
classes de réponse au problème initial. 

Ici, il y a deux classes de réponse : \textit{<<~Ne plus rembourser~>>} et
\textit{<<~Ne rien changer~>>}.
Par exemple, si les marchés sont paniqués et si la dette est
strictement supérieure à $50\%$ du PIB alors on ne rembourse plus.



Soient 3 variables entières représentant les propriétés du jour
courant pour prendre la décision~:
  
\begin{itemize}
\item \verb|marches| est la variable représentant l'état des marchés, elle contient une valeur pour PANIQUE, une pour INQUIETUDE et
  une pour STABILITE.
\item \verb|dette| est la variable représentant la dette en points de
  PIB.
\item \verb|hausse| est la variable représentant le fait que les taux
  montent; elle contient une valeur pour
  NON et une pour OUI.
\end{itemize}

Écrire un programme \C{exo3.c} implantant l'arbre de décision pour proposer une
réponse étant donné un jour. Après chaque test effectué sur une
variable, vous afficherez sa valeur afin de suivre la progression dans
l'arbre. 



\paragraph{La méthode à suivre~:}
\begin{enumerate}
\item Se donner des exemples de problèmes et les résoudre à la main. 
\item Écrire un algorithme en français permettant de résoudre ces
  problèmes. 
\item Traduire l'algorithme en langage C, en utilisant l'algorithme
  pour les commentaires. 
\item Tester le programme sur les exemples pour s'assurer de sa
  correction.
\end{enumerate}

\begin{correction}
Durée 1 heure ?

Les printf indiquant les tests effectués sont à ajouter.

\begin{verbatim}
algo :
si panique alors
  si dette > 50 alors
    ne plus rembourser
  sinon
    ne rien changer
si inquietude alors
  si les taux montent alors
    ne plus rembourser
  sinon
    ne rien changer
si stabilite alors
  ne rien changer
\end{verbatim}

Les choix sont mutuellement exclusifs.

\begin{verbatim}
/* declaration de fonctionnalites supplementaires */
#include <stdlib.h> /* EXIT_SUCCESS */
#include <stdio.h> /* printf */

/* declaration constantes et types utilisateurs */
#define STABILITE 0
#define INQUIETUDE 1
#define PANIQUE 2
#define OUI 1
#define NON 0
/* declaration de fonctions utilisateurs */

/* fonction principale */
int main()
{
    /* declaration et initialisation variables */
    int marches = PANIQUE; 
    int dette = 54; /* pourcents de PIB */
    int hausse = OUI; 

/* 
   si marches paniques alors
     si dette > 50 alors
       ne plus rembourser
     sinon
       ne rien changer
   si marches inquiets alors
     si taux en hausse alors
       ne rembourse plus
     sinon
       ne rien changer
   si marches stables alors
     ne rien changer
*/

    /* cas mutuellement exclusif */

    if(marches == PANIQUE)
    {
        if(dette > 50)
        {
            /* ne plus rembourser */
            printf("Ne plus rembourser.\n");
        }
        else /* dette <= 50 */
        {
            /* ne rien changer */
            printf("Ne rien changer.\n");
        }
    }

    if(marches == INQUIETUDE)
    {
        if(hausse == OUI)
        {
            /* ne plus rembourser */
            printf("Ne plus rembourser.\n");
        }
        else /* NON pas de hausse des taux */
        {
            /* ne rien changer */
            printf("Ne rien changer.\n");
        }
    }

    if(marches == STABILITE)
    {
        /* ne rien changer */
        printf("Ne rien changer.\n");
    }
    
    return EXIT_SUCCESS;
}

/* definitions des fonctions utilisateurs */
\end{verbatim}
\end{correction}

%%% Local Variables: 
%%% mode: latex
%%% TeX-master: "tp3"
%%% End: 