% -*- coding: utf-8 -*-

% \renewcommand{\labelitemi}{$\bullet$}

% \newcommand{\commentaire}[1]{}

\entete{Travaux pratiques 6 : évaluation d'expressions booléennes; la structure de contrôle ``while''}

\vspace{-1em}L'objectif de ce TP est de vous familiariser avec les expressions booléennes et leur utilisation avec la structure de contrôle ``while''.

\section{Le nombre secret}

Nous voulons programmer le jeu du nombre à découvrir. Le joueur doit
deviner un nombre secret choisit par l'ordinateur entre 0 et \verb|NB_MAX| (une constante du programme). S'il propose un nombre trop
grand, l'ordinateur lui répond "Plus petit", s'il propose trop petit, l'ordinateur lui répond "Plus grand". Dans ces deux cas, il
est invité à proposer un autre nombre. Le jeu s'arrête quand il devine juste.
Un exemple d'exécution de ce jeu pourrait être :
\begin{verbatim}
Votre choix ?
8
Plus petit.
Votre choix ?
4
Plus petit.
Votre choix ?
2
Vous avez trouvé le nombre secret.
\end{verbatim}

\begin{enumerate}
\item Proposez un algorithme en français pour le jeu.
\item Traduisez-le en langage C et exécutez-le.
\item Pourquoi préférez-vous une boucle \verb|while| ici ?
\end{enumerate}

Pour rendre le jeu intéressant, l'ordinateur doit choisir le nombre secret \emph{au hasard}. La librairie C standard propose des fonctions renvoyant des nombres pseudo-aléatoires\footnote{http://fr.wikipedia.org/wiki/Générateur\_de\_nombres\_pseudo-aléatoires} déclarées dans \C{<stdlib.h>}. L'ordinateur va utiliser la fonction :
	\C{int rand();}
qui renvoie un nombre pseudo-aléatoire entier entre 0 et la constante
\verb|RAND_MAX| (égale à 2147483647) inclus. Pour renvoyer un nombre
pseudo-aléatoire entre 0 et \verb|NB_MAX|, \verb|NB_MAX| inclus
($\verb|NB_MAX| << \verb|RAND_MAX|$), il suffit de calculer le reste
de la division entière de \C{rand()} par \verb|(NB_MAX + 1)|,
c'est-à-dire le nombre renvoyé par rand()  modulo \verb|(NB_MAX + 1)|
(opérateur \% en C). rand vient de random qui veut dire aléatoire en anglais.

Un exemple de programme illustrant l'utilisation de rand pour
engendrer un nombre pseudo-aléatoire est le suivant :
\begin{small}
\begin{verbatim}
#include <stdlib.h> /* EXIT_SUCCESS, rand, srand */
#include <time.h> /* time */

#define NB_MAX 15 /* nombre secret entre 0 et NB_MAX inclus */

int main()
{
    int nombre_secret; /* nombre secret à deviner */

    /* initialisation du générateur de nombres pseudo-aléatoires */
    srand(time(NULL)); /* à ne faire qu'une fois */

    /* tirage du nombre secret */
    nombre_secret = rand() % (NB_MAX + 1); /* entre 0 et NB_MAX inclus */

    /* manche joueur ... */

    return EXIT_SUCCESS;
}
\end{verbatim}
\end{small}

\begin{correction}
\begin{verbatim}
#include <stdlib.h> /* EXIT_SUCCESS, rand, srand */
#include <stdio.h> /* printf, scanf */
#include <time.h> /* time */

#define TRUE 1
#define FALSE 0

#define NB_MAX 15 /* nombre secret entre 0 et NB_MAX inclus */

/* declaration de fonctions utilisateurs */

int main()
{
    int nombre_secret; /* nombre secret à deviner */
    int choix; /* choix de l'utilisateur pour le nombre secret */
    int pas_trouve = TRUE; /* TRUE si pas trouvé nombre secret */

    /* initialisation du générateur de nombres pseudo-aléatoires */
    srand(time(NULL)); /* à ne faire qu'une fois */

    /* tirage du nombre secret */
    nombre_secret = rand() % (NB_MAX + 1); /* entre 0 et NB_MAX inclus */

    /* manche joueur */
    while(pas_trouve) /* pas trouvé nombre secret */
    {
        /* demande nombre à l'utilisateur */
        printf("Votre choix (nombre entre 0 et %d) ?\n",NB_MAX);
        scanf("%d",&choix);
        
        if(choix == nombre_secret) /* trouvé */
        {
            pas_trouve = FALSE;
        }
        else /* pas trouvé */
        {
            /* donne indice */
            if(choix > nombre_secret)
            {
                printf("Plus petit.\n");
            }
            else
            {
                printf("Plus grand.\n");
            }
        }
    }
    /* trouvé nombre secret */

    printf("Vous avez trouvé le nombre secret.\n");

    return EXIT_SUCCESS;
}

/* Definition de fonctions utilisateurs */
\end{verbatim}
\end{correction}

\section{Évaluation d'expressions booléennes}
\label{sec:exo2}

Une table de vérité donne la valeur d'une ou de plusieurs expressions
booléennes, construits à partir de variables et d'opérateurs booléens,
en fonction des valeurs des variables booléennes. Un exemple de table
de vérité pour l'expression \textit{a OU b} est donné dans le tableau
ci-dessous (avec F : faux et V : vrai) :

\begin{center}

\begin{tabular}[h]{|c|c|c|}
\hline
a & b & a OU b\\
\hline
F & F & F\\
\hline
F & V & V\\
\hline
V & F & V\\
\hline
V & V & V\\
\hline
\end{tabular}
\end{center}

Écrire un programme qui demande à l'utilisateur les valeurs de deux variables booléennes \verb|a| et \verb|b| et qui affiche
à l'écran la ligne correspondante de la table de vérité de l'ensemble des
expressions : a ET b, a OU b, NON a, NON b, NON a ET b. Cette ligne à
afficher (qui aura 7 colonnes) est déterminée par la valeur des deux
variables a et b. Pour une indentation correcte des colonnes, vous pouvez utiliser dans printf
  \verb|``\t''| qui affiche une tabulation.

Voici deux exemples de sortie :
\begin{verbatim}
entrez deux valeurs booleennes : 1 0
a       b       a ET b  a OU b  NON a   NON b   NON a ET b
1       0       0       1       0       1       0

entrez deux valeurs booleennes : -12 0
a       b       a ET b  a OU b  NON a   NON b   NON a ET b
-12     0       0       1       0       1       0
\end{verbatim}

\begin{correction}
\begin{verbatim}
#include <stdlib.h> /* EXIT_SUCCESS */
#include <stdio.h> /* printf, scanf */

/* declaration constantes et types utilisateurs */

/* declaration de fonctions utilisateurs */

int main()
{
    int a; /* booléen */
    int b; /* booléen */

    printf("Entrez deux valeurs booléennes : ");
    scanf("%d",&a);
    scanf("%d",&b);

    printf("a\tb\ta ET b\ta OU b\tNON a\tNON b\tNON a ET b\n");
    printf("%d\t%d\t%d\t%d\t%d\t%d\t%d\n",a,b,a && b,a || b,!a,!b,!a && b);

    return EXIT_SUCCESS;
}

/* Definition de fonctions utilisateurs */
\end{verbatim}
\end{correction}

