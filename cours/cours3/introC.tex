\documentclass[12pt,a4paper]{article}
\usepackage{latexsym,amssymb,amsmath}
\usepackage[utf8]{inputenc}
\usepackage[T1]{fontenc}
\usepackage[francais]{babel}
%\usepackage[dvips]{graphicx}
%\usepackage{moreverb}
%\usepackage{version}
\usepackage{url}

\begin{document}
\section{Démo}
Démo d'un programme C, avec des mots clés pendant la
démo. Introduction du patron (template).

liste de mots clés :
\begin{itemize}
\item fichier texte
\item indentation, commentaires
\item Langage C, programmation structurée, impérative (les variables
  changent de valeur)
\item compilateur (analyser/traduire)
\item exécutable : fichier objet, binaire (encodage).
\end{itemize}

\section{Un peu d'histoire : compilateurs, programmation structurée,
  langage C}

Des années 50 aux années 70 (photos). Compilation, langages de haut niveau,
programmation structurée, bibliothèques.

\subsection{Comment les informaticiens pensent et écrivent les
  programmes informatiques ?}

\paragraph{Programmer.} Découper une tache complexe en taches faciles à coder
dans le langage de programmation. (Approche top-down).

\paragraph{Blocs.} On organise les séquence d'instructions en blocs.

\paragraph{Diagrammes de blocs.} Exécution conditionnelle et boucles. On peut
écrire n'importe quel programme en combinant des instructions en
séquences (concaténation) ou à l'aide des instructions de contrôle :
branchement conditionnel (sélection) et boucles (répétition). Faire un
dessin. Aujourd'hui pour le langage C, uniquement le branchement
conditionnel de blocs d'instructions.

\section{Langage C}

Contenu de ce programme :
\begin{itemize}
\item Variable
\item Déclaration et initialisation (type int).
\item Affectation. Évaluation d'expression.
\item préprocesseur : \verb|include| et \verb|EXIT_SUCCESS|. Le
  préprocesseur : disposer d'un fichier source pour différentes
  versions assez proches (la constante nommée \verb|EXIT_SUCCESS|
  vient de la bibliothèque standard).
\item Fonctions et bibliothèque : exemple du printf, non expliqué (on
  verra plus tard les fonctions).
\item indentation, commentaires.  Ne pas montrer un truc pas indenté.
\end{itemize}

\begin{quote}
  Tout ce qui peut nous éloigner du codage interne est bénéfique.
\end{quote}

\section{Traduction en assembleur et trace en C}

Que fait le compilateur (préprocesseur, traduction en assembleur,
codage, édition de lien).

Au tableau la traduction + la trace.

\section{Le langage C}

\begin{description}
\item[Déclaration] (type identificateur de variable point-virgule).
\item[Déclaration avec initialisation] (type identificateur de variable = valeur
  point-virgule)
  \begin{description}
  \item[Nom] une lettre suivie de lettres chiffres ou \_
  \end{description}

\item[Instruction] affectation, return [expression]
  (point-virgule). Se terminent par un point virgule. Ou instruction
  de contrôle.

\item[Affectation] L-value = expression (point-virgule). 

  \begin{description}
  \item[L-value] variables uniquement.

  \item[Expression] Construite à partir de variables déclarées,
    constantes, opérations arithmétiques (\verb|+,-,*,/,%|), moins
    unaire, parenthèses.
  \end{description}

\item[Instruction de contrôle] manipulent des blocs : if
 ( condition ) bloc et if ( condition ) bloc else bloc. (Dessins).

\begin{description}
\item[Condition] inégalité ou égalité (\verb|==, <, <=, >= , >, !=|).

\item[Blocs] sont encadrés par des accolades, contiennent éventuellement
des déclarations en préambule, et ensuite nécessairement une ou
plusieurs instructions.
\end{description}

\end{description}

\section{Le if du C}

Exemples (valeur absolue, courrier à Monsieur/Madame). Codage en
assembleur.

\section{Pour plus tard : compilation}
\begin{description}
\item[Analyse lexicale.]  (identification des lexèmes), espaces inutiles
  sauf comme séparateurs (int x, intx).
\item[Analyse syntaxique] trouve la structure syntaxique, (arbre
  syntaxique), et teste l'appartenance au langage. Exemple x + 1 est
  une sous-suite est-ce qu'elle correspond à une structure syntaxique.
\item[Analyse sémantique] trouver le sens des différentes actions
  voulues par le programmeur. Quelles sont les objets manipulés par le
  programme, quelles sont les propriétés de ces objets, quelles sont
  les actions du programme sur ces objets. Beaucoup d'erreurs peuvent
  apparaître durant cette phase : identificateur utilisé mais non
  déclaré (la réciproque génère un \emph{warning}), opération n'ayant
  aucun sens,
\item[Génération du code] encodage en assembleur.
\item[Édition de liens] le code objet des fonctions externes
  (bibliothèques) est ajouté à l'exécutable. Le point d'entrée dans le
  programme est choisi (main).
\end{description}
Des erreurs de compilation pour chacun de ces cas.


\end{document}

