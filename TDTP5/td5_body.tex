% -*- coding: utf-8 -*-

\newcommand{\commentaire}[1]{}

\entete{Travaux dirigés 9 : fonctions et
  procédures (2)}

\vspace{-2cm}

% \begin{correction}
%   Note aux chargés de TD : blabla.
% \end{correction}
\section{Trace de fonctions}

\begin{newenu}
\item  Faire la trace du programme suivant.
\end{newenu}
{
\footnotesize
\listinginput{1}{echange.c}
}


\begin{correction}
L'échange des valeurs de \C a et \C b a lieu, par contre
l'échange des valeurs de \C x et \c y n'a pas lieu, puisque
permute\_valeurs n'accède pas aux cases mémoires du main seulement à
ses propres cases mémoires.

  \paragraph{Trace.}

  \begin{table}[h]
        \setlength{\unitlength}{\tabcolsep}
       \begin{tabular}[t]{|r|c|c|l|}
          \multicolumn{4}{l}{\C{main()}}\\ \hline
          ligne & x & y & Affichage (sortie écran) \\ \hline
          initialisation  & 1 & 2 & \\ \hline
          17 &   & & \\ \hline
          \multicolumn{4}{r|}{
            \put(1,0){\noindent
              \begin{tabular}[t]{|r|c|c|c|l|}
                \multicolumn{5}{l}{\C{permute\_valeurs(1, 2)}}\\ \hline
                ligne & a & b & aux & Affichage \\ \hline
                initialisation  & 1 & 2 & ? & \\ \hline
                29 &  &  & 1 & \\ \hline
                30 & 2 &  &  & \\ \hline
                31 &  &  1  &  & \\ \hline
                32 &\multicolumn{4}{|l|}{ne renvoie rien}\\ \hline
              \end{tabular}
            }}\\ \hline
          18  &  &  & \C{x = 1 et y = 2} \\ \hline
          21 &\multicolumn{3}{|l|}{SORTIE AVEC SUCCÈS}\\ \hline
        \end{tabular}
\hfill
       \caption{Trace du programme de l'exercice 1.}
        \label{simulation}
   \end{table}                 
  \end{correction}

% \begin{lastenu}
% \item  Faire la trace du programme suivant et expliquer.
% \end{lastenu}
% {
% \footnotesize
% \listinginput{1}{echange_tableau.c}
% }
 
% \begin{lastenu}
% \item Modifier ce dernier programme de manière à ce qu'il utilise une
%   procédure \C{affichage\_tableau} pour l'affichage du tableau. Le
%   tableau \emph{et} la taille du tableau seront passé en paramètre de
%   cette procédure.
% \end{lastenu}

\section{Le menu avec fonctions et procédures}

\begin{minipage}[t]{.5 \linewidth}
  Dans cet exercice vous compléterez le menu écrit en TP. Vous
  travaillerez sur trois parties du programme :
  \begin{itemize}
  \item les déclarations du début du programme (fonctionnalités,
    constantes, fonctions),
  \item les définitions de fonctions,
  \item la fonction principale (\C{main}).
  \end{itemize}
\end{minipage}
\hfill
\begin{minipage}[t]{.4\linewidth}
{\footnotesize
\begin{verbatim}
****************** MENU ******************
*                                        *
*   1) Tester si un nombre est premier   *
*   2) Calcul de factorielle             *
*   3) Deviner un nombre                 *
*   4) Motif d'etoiles                   *
*                                        *
*   0) QUITTER                           *
*                                        *
************************** votre choix :
\end{verbatim}
}
\end{minipage}
%\pagebreak
\begin{lastenu}
\item Déclarer et définir une procédure qui affichera le menu.
  \begin{correction}
    C'est tout bête.
{\footnotesize
\begin{verbatim}
/* Declarations de fonctions utilisateurs */
void afficher_menu();
...
/* Definitions de fonctions utilisateurs */
void afficher_menu()
{
    printf("****************** MENU ******************\n");
    printf("*                                        *\n");
    printf("*   1) Tester si un nombre est premier   *\n");
    printf("*   2) Calcul de factorielle             *\n");
    printf("*   3) Deviner un nombre                 *\n");
    printf("*   4) Motif d'etoiles                   *\n");
    printf("*                                        *\n");
    printf("*   0) QUITTER                           *\n");
    printf("*                                        *\n");
    printf("************************** votre choix :");
}
\end{verbatim}
}
\end{correction}

\item Déclarer et définir une fonction \C{choix\_utilisateur}, sans paramètres qui renverra une valeur entière saisie par l'utilisateur.

  \begin{correction}
{\footnotesize
\begin{verbatim}
int choix_utilisateur()
{
    int choix;
    scanf("%d", &choix);
    return choix;
}
\end{verbatim}
}
  \end{correction}

\item Déclarer et définir une fonction \C{executer\_menu} qui :
  \begin{itemize}
  \item affichera le menu à l'utilisateur et réalisera la saisie de son choix;
  \item lorsque ce choix est 1, appelera une fonction non encore définie \C{menu\_premier};
    \item puis, lorsque ce choix est différent de $0$, renverra \C{TRUE} et lorsque ce choix est égal à zéro renverra \C{FALSE}.
  \end{itemize}

  \begin{correction}
    Les étudiants doivent se poser la question de ce dont pourrait
    avoir besoin \C{menu\_premier} pour effectuer le traitement et ce
    qu'elle doit renvoyer (réponse : rien dans les deux cas).

{\footnotesize
\begin{verbatim}
int executer_menu()
{
    int choix;

    /* Affichage du menu et choix de l'utilisateur */
    afficher_menu();
    choix = choix_utilisateur();

    if (1 == choix) /* ------- 1) Tester si un nombre est premier ----- */
    {
        menu_premier(); 
    }

    /* Valeur de retour */
    if (choix != 0)
    {
        return TRUE;
    }
    return FALSE;
}
\end{verbatim}
}
  \end{correction}
\item Déclarer et définir une procédure \C{menu\_premier} qui traitera
  le choix 1, et qui fera appel à la fonction \C{est\_premier}
  (dont vous rappellerez la définition et la déclaration) et à la
  fonction \C{choix\_utilisateur} pour le choix de l'entier.
  \begin{correction}
{\footnotesize
\begin{verbatim}
void menu_premier()
{
    int p;

    printf("Donner un nombre : ");
    p = choix_utilisateur();
    
    if (est_premier(p))
    {
        printf("Le nombre %d est premier\n", p);
    }
    else
    {
        printf("Le nombre %d n'est pas premier\n", p);      
    }
}

Rappel :
int est_premier(int n)
{
    int i = 2;
    int premier = TRUE;

    while (premier && i < n) 
    {   
        if (n % i == 0) 
        {
           premier = FALSE;
        }
        i = i + 1;
    }
    return premier;
}
\end{verbatim}
}
  \end{correction}
\item Écrire le \C{main} de telle sorte qu'il fasse appel à la fonction
  \C{executer\_menu} tant que celle-ci renvoie \C{TRUE}.


\begin{correction}
{\footnotesize
\begin{verbatim}
int main()
{
    /* Declarations et initialisation des variables */
    int encore = TRUE;
    
    /* Boucle d'interaction avec l'utilisateur */
    while (encore)
        {
            encore = executer_menu();
        }
    
    /* Greetings */
    printf("Bye bye\n");
    
    /* Valeur fonction */
    return EXIT_SUCCESS;
}
\end{verbatim}
}
  \end{correction}

\item Faire la trace de votre programme dans le cas où l'utilisateur
  saisit $1$ puis $3$ puis $0$.
\begin{correction}
Avant de faire la trace il nous faut des numéros de ligne :
{\footnotesize
\listinginput{1}{td5_menu_pour_trace.c}
}

On peut maintenant faire la trace (Table~\ref{trace2}
page~\pageref{trace2}). La trace est donnée en entier, mais 
inutile de tout faire. Disons qu'à partir du second appel à
\C{executer\_menu}  c'est un peu superflu. 
  \begin{table}[h]
       \scriptsize\hspace{-2cm}
        \setlength{\unitlength}{\tabcolsep}
       \begin{tabular}[t]{|r|c|l|}
          \multicolumn{3}{l}{\C{main()}}\\ \hline
          ligne & encore & Affichage\\ \hline
          initialisation  & 1 (vrai) & \\ \hline
          36 &  & \\ \hline
         \multicolumn{3}{r|}{
            \put(1,0){\noindent
              \begin{tabular}[t]{|r|c|l|}
                \multicolumn{3}{l}{\C{executer\_menu()}}\\ \hline
                ligne & choix & Affichage \\ \hline
                init.  & ? & \\ \hline
                75 &  & \\ \hline
                \multicolumn{3}{r|}{
                  \put(1,0){\noindent
                    \begin{tabular}[t]{|r|l|}
                      \multicolumn{2}{l}{\C{afficher\_menu()}}\\ \hline
                      ligne & Affichage \\ \hline
                      initialisation & \\ \hline
                      51 & \C{*}\ldots \\ \hline
                      \vdots & \vdots \\ \hline
                      60 & \ldots \C{* votre choix : } \\ \hline
                      61 &\multicolumn{1}{|l|}{ne renvoie rien}\\ \hline
                    \end{tabular}
                  }}\\ \hline
                76 &  & \\ \hline
                \multicolumn{3}{r|}{
                  \put(1,0){\noindent
                    \begin{tabular}[t]{|r|c|l|}
                      \multicolumn{3}{l}{\C{choix\_utilisateur()} saisie 1}\\ \hline
                      ligne & choix & Affichage \\ \hline
                      initialisation  & ? & \\ \hline
                      66  & 1 & \\ \hline
                      67 &\multicolumn{2}{|l|}{renvoie 1}\\ \hline
                    \end{tabular}
                  }}\\ \hline
                76 & 1 & \\ \hline
                80 & & \\ \hline
                \multicolumn{3}{r|}{
                  \put(1,0){\noindent
                         \begin{tabular}[t]{|r|c|l|}
                           \multicolumn{3}{l}{\C{menu\_premier()}}\\ \hline
                           ligne & p & Affichage \\ \hline
                           initialisation  & ? & \\ \hline
                           96 &  & \ldots un nombre : \\ \hline
                           97 &  & \\ \hline
                           \multicolumn{3}{r|}{
                             \put(1,0){\noindent
                               \begin{tabular}[t]{|r|c|l|}
                                 \multicolumn{3}{l}{\C{choix\_utilisateur()} saisie 1}\\ \hline
                                 ligne & choix & Affichage \\ \hline
                                 initialisation  & ? & \\ \hline
                                 66  & 3 & \\ \hline
                                 67 &\multicolumn{2}{|l|}{renvoie 3}\\ \hline
                               \end{tabular}                        
                             }} \\ \hline
                           97 & 3 & \\ \hline
                           98 & & \\ \hline
                           \multicolumn{3}{r|}{
                             \put(1,0){\noindent
                                 \begin{tabular}[t]{|r|c|c|c|l|}
                                   \multicolumn{5}{l}{\C{est\_premier(3)}}\\ \hline
                                   ligne & n & i & premier & Affichage \\ \hline
                                   init.  & 3 & 2 & 1 (vrai) & \\ \hline
                                   120 &  & 3 & & \\ \hline
                                   122 &\multicolumn{4}{|l|}{renvoie vrai (1)}\\ \hline
                                 \end{tabular}
                             }} \\ \hline                           
                           100 & & \ldots 3 est premier \\ \hline
                           106 &\multicolumn{2}{|l|}{ne renvoie rien}\\ \hline
                         \end{tabular}
                  }}\\ \hline
                86 &\multicolumn{2}{|l|}{renvoie vrai (1)}\\ \hline
              \end{tabular}              
            }
          }\\ \hline
          36 & 1 (vrai) & \\ \hline
          36 &  & \\ \hline
          \multicolumn{3}{r|}{
            \put(1,0){\noindent
        \begin{tabular}[t]{|r|c|l|}
          \multicolumn{3}{l}{\C{executer\_menu()}}\\ \hline
          ligne & choix & Affichage \\ \hline
          initialisation  & ? & \\ \hline
          75 &  & \\ \hline
                \multicolumn{3}{r|}{
                  \put(1,0){\noindent
                    \begin{tabular}[t]{|r|l|}
                      \multicolumn{2}{l}{\C{afficher\_menu()}}\\ \hline
                      ligne & Affichage \\ \hline
                      initialisation & \\ \hline
                      51 & \C{*}\ldots \\ \hline
                      \vdots & \vdots \\ \hline
                      60 & \ldots \C{* votre choix : } \\ \hline
                      61 &\multicolumn{1}{|l|}{ne renvoie rien}\\ \hline
                    \end{tabular}
                  }}\\ \hline
          76 &  & \\ \hline
                \multicolumn{3}{r|}{
                  \put(1,0){\noindent
                    \begin{tabular}[t]{|r|c|l|}
                      \multicolumn{3}{l}{\C{choix\_utilisateur()} saisie 0}\\ \hline
                      ligne & choix & Affichage \\ \hline
                      initialisation  & ? & \\ \hline
                      66  & 0 & \\ \hline
                      67 &\multicolumn{2}{|l|}{renvoie 0}\\ \hline
                    \end{tabular}
                  }}\\ \hline
          76  & 0 & \\ \hline
          86 &\multicolumn{2}{|l|}{renvoie faux (0)}\\ \hline
        \end{tabular}              
            }
          } \\ \hline
          36 & 0 (faux) & \\ \hline
          40 & & Bye bye \\ \hline
          43 &\multicolumn{2}{|l|}{SUCCÈS}\\ \hline
        \end{tabular}
        \caption{Trace du programme de l'exercice 2.}
        \label{trace2}
   \end{table}
 \end{correction}
\item Modifier votre fonction \C{choix\_utilisateur} de telle sorte
  que :
  \begin{itemize}
  \item elle prenne en argument deux paramètres entiers $a$ et $b$;
  \item si l'utilisateur saisit un nombre $n\in [a, b]$ la fonction
    retourne $n$ sans générer d'affichage;
\item si l'utilisateur saisit un nombre $n\not\in [a, b]$ l'intervalle
  de saisie soit affiché à l'utilisateur et la saisie redemandée,
  jusqu'à cinq fois.
  \end{itemize}

    \begin{correction}
Au delà de cinq essais, c'est la dernière saisie utilisateur qui est renvoyée.
{\footnotesize
\begin{verbatim}
int choix_utilisateur(int a, int b)
{
   int compteur = 5; /* compteur du nombre d'essais */
   int choix; /* choix de l'utilisateur */

   scanf("%d", &choix);
   while ((compteur > 0) &&  ((choix < a) || (choix > b)))
   {
       printf("Le nombre doit etre entre %d et %d (inclus) : ", a, b);
       scanf("%d", &choix);
       compteur = compteur - 1;
   }
   return choix;
}
\end{verbatim}
}
IL faut aussi modifier les appels à cette fonction. Pour la fonction \C{executer\_menu} on donne l'intervalle $[0, 3]$ (à supposer que ces quatres choix soient traités). Pour la procédure \C{menu\_premier}, on donne l'intervalle $[1, \text{\C{INT\_MAX}}]$ (charger \C{limits.h} pour \C{INT\_MAX}).
    \end{correction}

\end{lastenu}

On se donne la procédure suivante :
{\footnotesize
\begin{verbatim}
void afficher_motif(int cote)
{
    int ligne; /* numero de ligne, de bas en haut */
    int colonne; /* numero de colonne, de gauche a droite */
    for (ligne = cote - 1; ligne >= 0; ligne = ligne - 1) /* pour chaque ligne */
    {
        for (colonne = 0; colonne < cote; colonne = colonne + 1) /* pour chaque colonne */
        {
            if (motif(colonne, ligne)) /* le point appartient au motif */
            {
                printf("* ");  
            }
            else /* le point n'appartient pas au motif */
            {
                printf("  "); 
            }   
        }
        printf("\n"); /* ligne suivante */
    }
}
\end{verbatim}
}

\begin{lastenu}
  \item Définir la fonction motif de telle sorte qu'un appel à     \C{afficher\_motif(5)} affiche :
{\footnotesize
\begin{verbatim}
*       * 
*     *   
*   *     
* *       
* * * * * 
\end{verbatim}
}

\begin{correction}
Facile : 
{\footnotesize
\begin{verbatim}
int motif(int x, int y)
{
    return x == y
        || x == 0
        || y == 0;
}
\end{verbatim}
}
\end{correction}

  \item Écrire les fonctions nécessaires au traitement du choix 2 et
    du choix 3 du menu sur le modèle du traitement du choix 1.

\begin{correction}
Il faut écrire une procédure \C{menu\_deviner}, qui fera appel à une fonction \C{tirage\_aleatoire} puis en boucle à \C{choix\_utilisateur} sur l'intervalle dans lequel le nombre est tiré (de 0 à une constante symbolique).

Le code complet du programme à la fin du TD (mais personne ne va finir non ?) :
{\footnotesize
\listinginput{1}{td5_menu.c}
}

\end{correction}
\end{lastenu}

\section{En travaux pratiques}
Le programme de ce TD est à reprendre en TP.  Votre code doit être correctement indenté, sinon il sera
difficile d'y trouver vos erreurs. Vous pouvez utiliser la commande
\verb|astyle menu.c| ou \verb|~boudes/pub/bin/astyle menu.c| qui
indentera correctement votre code. Alternative:
l'éditeur de texte \verb|emacs|, beaucoup plus puissant que
\verb|gedit|, gère parfaitement l'indentation. 

Ajouter au menu une entrée \verb|calculette| et une 
procédure \verb|calculette| qui affichera le résultat d'une 
expression \verb|nombre opération nombre| entrée par l'utilisateur, où les nombres
sont des \verb|double| et l'opération un caractère parmi
\verb|+, -, *, /|.  Indication :
{\footnotesize
\begin{verbatim}
...
scanf("%lg %c %lg", &x, &op, &y);
if ('+' == op)/* faire une addition */
{
    printf("%lg\n", x + y); /* affichage du résultat */
}
\end{verbatim}
}
Des questions supplémentaires sont disponibles sur la
page web du cours.
