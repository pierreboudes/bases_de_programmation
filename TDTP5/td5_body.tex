% -*- coding: utf-8 -*-

\newcommand{\commentaire}[1]{}

\entete{Travaux dirigés 5 : fonctions et
  procédures (2)}


% \begin{correction}
%   Note aux chargés de TD : blabla.
% \end{correction}
% \section{Trace de fonctions}
% \begin{newenu}
% \item  Faire la trace du programme suivant.
% \end{newenu}
% {
% \footnotesize
% \listinginput{1}{echange.c}
% }


% \begin{correction}
% L'échange des valeurs de \C a et \C b a lieu, par contre
% l'échange des valeurs de \C x et \c y n'a pas lieu, puisque
% permute\_valeurs n'accède pas aux cases mémoires du main seulement à
% ses propres cases mémoires.

%   \paragraph{Trace.}

%   \begin{table}[h]
%         \setlength{\unitlength}{\tabcolsep}
%        \begin{tabular}[t]{|r|c|c|l|}
%           \multicolumn{4}{l}{\C{main()}}\\ \hline
%           ligne & x & y & Affichage (sortie écran) \\ \hline
%           initialisation  & 1 & 2 & \\ \hline
%           17 &   & & \\ \hline
%           \multicolumn{4}{r|}{
%             \put(1,0){\noindent
%               \begin{tabular}[t]{|r|c|c|c|l|}
%                 \multicolumn{5}{l}{\C{permute\_valeurs(1, 2)}}\\ \hline
%                 ligne & a & b & aux & Affichage \\ \hline
%                 initialisation  & 1 & 2 & ? & \\ \hline
%                 29 &  &  & 1 & \\ \hline
%                 30 & 2 &  &  & \\ \hline
%                 31 &  &  1  &  & \\ \hline
%                 32 &\multicolumn{4}{|l|}{ne renvoie rien}\\ \hline
%               \end{tabular}
%             }}\\ \hline
%           18  &  &  & \C{x = 1 et y = 2} \\ \hline
%           21 &\multicolumn{3}{|l|}{SORTIE AVEC SUCCÈS}\\ \hline
%         \end{tabular}
% \hfill
%        \caption{Trace du programme de l'exercice 1.}
%         \label{simulation}
%    \end{table}                 
%   \end{correction}

% \begin{lastenu}
% \item  Faire la trace du programme suivant et expliquer.
% \end{lastenu}
% {
% \footnotesize
% \listinginput{1}{echange_tableau.c}
% }
 
% \begin{lastenu}
% \item Modifier ce dernier programme de manière à ce qu'il utilise une
%   procédure \C{affichage\_tableau} pour l'affichage du tableau. Le
%   tableau \emph{et} la taille du tableau seront passé en paramètre de
%   cette procédure.
% \end{lastenu}

%\section{Des procédures au menu}


Durant les séances précédentes vous avez réalisé plusieurs programmes en C
effectuant chacun une tâche. Le but ici est d'utiliser des
fonctions pour commencer à réunir plusieurs de ces programmes en un
seul, dans lequel l'utilisateur choisira la tâche à effectuer dans un
menu. À la fin de l'exécution d'une tâche, le menu est à nouveau
affiché pour laisser le choix à l'utilisateur d'exécuter d'autres
tâche ou de quitter le programme. Un exemple d'exécution est donné
plus bas.

Comme vous allez écrire un gros programme dans un seul fichier source, il est
essentiel que vous structuriez bien votre fichier source, et que
celui-ci soit correctement indenté. Si vous n'aimez pas gérer vous
même l'indentation apprenez à utiliser l'éditeur de texte \C{emacs} et son
indentation automatique.

Vous travaillerez sur trois parties du programme :
  \begin{itemize}
  \item les déclarations du début du programme (fonctionnalités,
    constantes, fonctions),
  \item les définitions de fonctions,
  \item la fonction principale (\C{main}).
  \end{itemize}

Vous ferez en sorte de pouvoir \textbf{tester votre programme le plus tôt
possible et le plus souvent possible}, quitte à afficher à l'utilisateur que certaines choses ne
sont pas disponibles / pas terminées. 

\begin{multicols}{2} 
\footnotesize\setlength{\columnseprule}{0.3pt}
\begin{verbatim}
****************** MENU ******************
*                                        *
*   1) Tester si un nombre est premier   *
*   2) Deviner un nombre                 *
*   0) QUITTER                           *
*                                        *
************************** votre choix : 1
Donner un nombre entier positif : 34
Le nombre 34 n'est pas premier, 2 divise 34
\end{verbatim}
\begin{verbatim}
****************** MENU ******************
*                                        *
*   1) Tester si un nombre est premier   *
*   2) Deviner un nombre                 *
*   0) QUITTER                           *
*                                        *
************************** votre choix : 2
\end{verbatim}
\begin{verbatim}
Votre choix ?
8
Plus petit.
Votre choix ?
4
Plus petit.
Votre choix ?
2
Vous avez trouvé le nombre secret.
\end{verbatim}
\begin{verbatim}
****************** MENU ******************
*                                        *
*   1) Tester si un nombre est premier   *
*   2) Deviner un nombre                 *
*   0) QUITTER                           *
*                                        *
************************** votre choix : 0
\end{verbatim}
\begin{verbatim}
Sayonara
\end{verbatim}\\[2cm]
 ~
\end{multicols}

\begin{newenu}
\item Commencer par faire en sorte que   :
  \begin{enumerate}
  \item le programme affiche un menu proposant $3$ choix représentés
    par des entiers : ($1$) tester si un entier est premier, ($2$)
    deviner un nombre ou ($0$) quitter. L'utilisateur fera son choix
    en entrant un entier.
  \item Si cet entier est $0$, mettre fin au programme.
  \item Si cet entier est $1$ ou $2$ afficher <<  non
    disponible >>, puis boucler à l'étape $1$.
  \end{enumerate}
% \begin{correction}
% Correction avec d'autres fonctions ajoutées au menu. La correction est légérement différente de ce qui est demandé.

%   \listinginput{1}{menu3.c}
% \end{correction}

Il est recommandé de découper ce programme en fonctions et procédures.
\begin{itemize}
\item Déclarer et définir une procédure \verb+afficher_menu()+ qui
  affichera le menu. Tester là.
  \begin{correction}
    C'est tout bête.
{\footnotesize
\begin{verbatim}
/* Declarations de fonctions utilisateurs */
void afficher_menu();
...
/* Definitions de fonctions utilisateurs */
void afficher_menu()
{
    printf("****************** MENU ******************\n");
    printf("*                                        *\n");
    printf("*   1) Tester si un nombre est premier   *\n");
    printf("*   2) Calcul de factorielle             *\n");
    printf("*   3) Deviner un nombre                 *\n");
    printf("*   4) Motif d'etoiles                   *\n");
    printf("*                                        *\n");
    printf("*   0) QUITTER                           *\n");
    printf("*                                        *\n");
    printf("************************** votre choix :");
}
\end{verbatim} 
}
\end{correction}

\item Déclarer et définir une fonction \C{choix\_utilisateur()}, sans
  paramètres qui renverra une valeur entière saisie par
  l'utilisateur. Tester là.

  \begin{correction}
{\footnotesize
\begin{verbatim}
int choix_utilisateur()
{
    int choix;
    scanf("%d", &choix);
    return choix;
}
\end{verbatim}
}
  \end{correction}

\item Déclarer et définir une fonction \C{executer\_menu()} qui :
  \begin{itemize}
  \item affichera le menu à l'utilisateur et réalisera la saisie de
    son choix (avec les fonctions et procédures précédentes);
  \item lorsque ce choix est 1, appelera une procédure
    \C{menu\_premier()} (pendant la mise au point du programme, faîtes
    en sorte que cette procédure affiche "non disponible") ;
    \item puis, lorsque ce choix est différent de $0$, renverra \C{TRUE} et lorsque ce choix est égal à zéro renverra \C{FALSE}.
  \end{itemize}

  \begin{correction}
    Les étudiants doivent se poser la question de ce dont pourrait
    avoir besoin \C{menu\_premier} pour effectuer le traitement et ce
    qu'elle doit renvoyer (réponse : rien dans les deux cas).

{\footnotesize
\begin{verbatim}
int executer_menu()
{
    int choix;

    /* Affichage du menu et choix de l'utilisateur */
    afficher_menu();
    choix = choix_utilisateur();

    if (1 == choix) /* ------- 1) Tester si un nombre est premier ----- */
    {
        menu_premier(); 
    }

    /* Valeur de retour */
    if (choix != 0)
    {
        return TRUE;
    }
    return FALSE;
}
\end{verbatim}
}
  \end{correction}
\item Modifier  la fonction \C{main} de telle sorte qu'elle fasse appel à la fonction
  \C{executer\_menu} tant que celle-ci renvoie \C{TRUE}.
\begin{correction}
{\footnotesize
\begin{verbatim}
int main()
{
    /* Declarations et initialisation des variables */
    int encore = TRUE;
    
    /* Boucle d'interaction avec l'utilisateur */
    while (encore)
        {
            encore = executer_menu();
        }
    
    /* Greetings */
    printf("Bye bye\n");
    
    /* Valeur fonction */
    return EXIT_SUCCESS;
}
\end{verbatim}
}
  \end{correction}
\end{itemize}
\end{newenu}

\section{Ajouts de fonctionnalités}
Vous pouvez maintenant commencer à programmer quelques unes des
  tâches que propose le menu. Si vous possédez des programmes
  réalisant ces tâches, inspirez vous en (copier/coller).

\begin{lastenu}
    \item Redéfinir la procédure \C{menu\_premier()} de manière à tester
      si un nombre choisi par l'utilisateur est premier. Elle fera
      appel à une fonction \C{est\_premier()}
  (dont vous retrouverez la définition et la déclaration dans le TD 4) et à la
  fonction \C{choix\_utilisateur()} pour le choix de l'entier.
  \begin{correction}
{\footnotesize
\begin{verbatim}
void menu_premier()
{
    int p;

    printf("Donner un nombre : ");
    p = choix_utilisateur();
    
    if (est_premier(p))
    {
        printf("Le nombre %d est premier\n", p);
    }
    else
    {
        printf("Le nombre %d n'est pas premier\n", p);      
    }
}

/* Rappel : */
int est_premier(int n)
{
    int i = 2;
    int premier = TRUE;

    while (premier && i < n) 
    {   
        if (n % i == 0) 
        {
           premier = FALSE;
        }
        i = i + 1;
    }
    return premier;
}
\end{verbatim}
}
  \end{correction}
\item Modifier votre fonction \C{choix\_utilisateur()} de telle sorte
  que :
  \begin{itemize}
  \item elle prenne en argument deux paramètres entiers $a$ et $b$;
  \item si l'utilisateur saisit un nombre $n\in [a, b]$ la fonction
    retourne $n$ sans générer d'affichage;
\item si l'utilisateur saisit un nombre $n\not\in [a, b]$ l'intervalle
  de saisie est affiché à l'utilisateur et la saisie redemandée,
  jusqu'à cinq fois.
\item si au bout de cinq fois l'utilisateur n'a toujours pas donné un
  nombre dans l'intervalle, la procédure renvoie $a$.
\item Tester. Quel est le problème lorsque l'utilisateur saisie autre chose
  que des chiffres au clavier ? Nous corrigerons ça un peu plus loin.
\end{itemize}

    \begin{correction}
{\footnotesize
\begin{verbatim}
int choix_utilisateur(int a, int b)
{
   int compteur = 5; /* compteur du nombre d'essais */
   int choix; /* choix de l'utilisateur */

   scanf("%d", &choix);
   while ((compteur > 0) &&  ((choix < a) || (choix > b)))
   {
       printf("Le nombre doit etre entre %d et %d (inclus) : ", a, b);
       scanf("%d", &choix);
       compteur = compteur - 1;
   }
   if ( (choix < a) || (choix > b) ) {
       choix = a;
   }
   return choix;
}
\end{verbatim}
}
IL faut aussi modifier les appels à cette fonction. Pour la fonction
\C{executer\_menu()} on donne l'intervalle $[0, 3]$ (à supposer que
ces quatres choix soient traités). Pour la procédure
\C{menu\_premier()}, on donne l'intervalle $[1, \text{\C{INT\_MAX}}]$
(charger \C{limits.h} pour définir la constante symbolique \C{INT\_MAX}, la plus grande valeur que
peut prendre un entier).
    \end{correction}
\item Ajouter le nécessaire à votre programme pour pouvoir jouer à
  deviner un nombre.
\begin{correction}
Il faut écrire une procédure \C{menu\_deviner()}, qui fera appel à une fonction \C{tirage\_aleatoire} puis en boucle à \C{choix\_utilisateur} sur l'intervalle dans lequel le nombre est tiré (de 0 à une constante symbolique).

Un code complet du programme à la fin du TD (mais personne ne va finir non ?) :
{\footnotesize
\listinginput{1}{td5_menu.c}
}
\end{correction}
\item Ajouter une entrée dans le menu pour la simulation d'une
  population de lapins.
\item Comme vous l'aviez remarqué, la fonction \C{choix\_utilisateur()}
  échoue à redemander un nombre si l'utilisateur saisit autre chose
  que des chiffres.  Vous pouvez corriger ça en utilisant le code
  suivant à la place de \verb+scanf("%d", &choix).+
{\small
\begin{verbatim}
char saisie[256]; /* declaration d'un chaine qui nous servira de
                   * "tampon" */

fgets(saisie, 256, stdin); /* preleve la ligne tapee par
                              l'utilisateur et la place dans le
                                  tampon */
sscanf(saisie, "%d", &choix); /* tente de reconnaitre un entier
                                 dans le tampon */
\end{verbatim}
}

\item Ajouter au menu une entrée \verb|calculette| et écrire une 
procédure \verb|calculette()| qui affichera le résultat d'une 
expression \verb|nombre opération nombre| entrée par l'utilisateur, où les nombres
sont des \verb|double| et l'opération un caractère parmi
\verb|+, -, *, /|.  

Indication : vous pouvez vous inspirer du code suivant :
{\footnotesize
\begin{verbatim}
...
scanf("%lg %c %lg", &x, &op, &y);
if ('+' == op)/* faire une addition */
{
    printf("%lg\n", x + y); /* affichage du résultat */
}
\end{verbatim}
}
\end{lastenu}
% Des questions supplémentaires sont disponibles sur la
% page web du cours.

%%% Local Variables: 
%%% mode: latex
%%% TeX-master: "td5"
%%% End: 