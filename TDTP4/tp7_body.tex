% -*- coding: utf-8 -*-

% \renewcommand{\labelitemi}{$\bullet$}

% \newcommand{\commentaire}[1]{}

\entete{Travaux pratiques 7 : qu'y a-t-il au menu ?}
% -*- coding: utf-8 -*-

\begin{correction}
  Note aux chargés de TD.
  \begin{itemize}
  \item L'objectif du TP est de les familiariser avec les fonctions au
    travers d'un des points présenté en cours : la structuration du
    code. Ils créent un menu permettant l'exécution de différents
    sous-problèmes qu'il sont déjà résolus en TP, chaque sous-problème
    étant implanté dans une fonction. La refactorisation du code sera
    faite plus tard, pour le moment on s'en tient à des fonctions
    utilisateur qui
    sont toutes appelées dans le main et n'appellent pas d'autres
    fonctions utilisateur.
  \item On incite ici les étudiants à structurer, indenter et
    commenter correctement leur code, sans quoi ils ne viendront pas à
    bout de ce type d'exercice.
  \end{itemize}
\end{correction}

Dans les TP précédents vous avez réalisé plusieurs programmes en C
effectuant chacun une tâche. Le but de ce TP est d'utiliser des
fonctions pour commencer à réunir plusieurs de ces programmes en un
seul, dans lequel l'utilisateur choisira la tâche à effectuer dans un
menu. À la fin de l'exécution d'une tâche, le menu est à nouveau
affiché pour laisser le choix à l'utilisateur d'exécuter d'autres
tâche ou de quitter le programme. Un exemple d'exécution est donné
plus bas.

\section{Un menu}
% \begin{itemize}
% \item À partir du début de votre arborescence, créez le répertoire
%   TP8~: \verb|mkdir TP8|  
% \item Allez dans ce répertoire pour y mettre des fichiers : 
%   \verb|cd TP8| 
% \item Créez un nouveau fichier source pour le langage C : 
%   \verb|gedit menu1.c &| ou \verb|emacs menu1.c &| 
% \item Après avoir fini d'écrire votre programme, enregistrez le
%   fichier source sur le disque dur (la mémoire secondaire). 
% \item Créez un programme exécutable en lançant la compilation et
%   l'édition de liens : \verb|gcc -Wall menu1.c -o menu1| 
% \item Quand l'étape précédente a réussi, exécutez le programme pour
%   vérifier qu'il fonctionne : \verb|./menu1| ou \verb|menu1| 
% \end{itemize}

% Vous pouvez utiliser à tout moment la commande ``ls'' (list directory)
% pour voir la liste des fichiers d'un répertoire. 

 Dans un répertoire \verb|TP7|, écrire et
tester le programme \verb|menu1.c|(sans fonctions) de manière à ce que :
\begin{enumerate}\item 
  le programme affiche un menu proposant $3$ choix représentés par des
  entiers : ($1$) tester si un entier est premier, ($2$) deviner un nombre ou ($0$)
  quitter. L'utilisateur fera son choix en entrant un entier.
\item Si cet entier est $0$, mettre fin au programme. 
\item Sinon : si cet entier est $1$ afficher << premier : non
  disponible >>, si c'est $2$, <<deviner un nombre : non disponible >> sinon
  << choix non disponible >>, puis boucler à l'étape $1$.
\end{enumerate}

\section{Des programmes}
\begin{itemize}
\item Dans votre répertoire \verb|TP7|, créer le programme
  \verb|premier.c| qui teste si un nombre entré par l'utilisateur est
  premier (voir TD). Vérifier que votre programme fonctionne.
\item Dans votre répertoire \verb|TP6| (ou celui de votre binôme) doit se
  trouver le programme réalisant le jeu \emph{deviner un nombre}. Vérifier que celui-ci
  est correctement indenté et commenté, et qu'il fonctionne.
\end{itemize}

\section{Intégration des programmes dans le menu en utilisant les fonctions}
Remarque : pour copier/coller sous un système unix vous pouvez :
(copier) sélectionner le texte à copier à l'aide de la souris;
(coller) effectuer un clic du milieu (bouton-molette) à l'endroit où
vous souhaitez coller.


\begin{itemize}
\item Enregistrer le fichier \verb|menu1.c| sous le nom \verb|menu2.c|.
\item Dans \verb|premier.c|, extraire le code relatif au problème et
  intégrer-le à la bonne place dans votre \verb|menu2.c| : déclarer
  une fonction \C{int est\_premier(int n)} et la définir. Cette
  fonction renverra la constante symbolique \C{TRUE} si $n$ est
  premier et \C{FALSE} (valeur 0) sinon. Faire en sorte que le
  traitement du choix 1 de l'utilisateur utilise cette fonction pour
  déterminer si un nombre est premier.
  \item Faire la même chose avec \emph{deviner un nombre}.

  \item Tester \verb|menu2.c| et s'il vous reste du temps ajouter des
    choix dans le menu, inspirés par les différents problèmes que vous
    avez déjà résolus dans les TP précédents (<< factorielle >>,
   << rectangle d'étoiles >>, << majeur/mineur >>).
\end{itemize}


\begin{multicols}{2} 
\footnotesize\setlength{\columnseprule}{0.3pt}
\begin{verbatim}
****************** MENU ******************
*                                        *
*   1) Tester si un nombre est premier   *
*   2) Deviner un nombre                 *
*   0) QUITTER                           *
*                                        *
************************** votre choix : 1
Donner un nombre entier positif : 34
Le nombre 34 n'est pas premier, 2 divise 34
\end{verbatim}
\begin{verbatim}
****************** MENU ******************
*                                        *
*   1) Tester si un nombre est premier   *
*   2) Deviner un nombre                 *
*   0) QUITTER                           *
*                                        *
************************** votre choix : 2
\end{verbatim}
\begin{verbatim}
Votre choix ?
8
Plus petit.
Votre choix ?
4
Plus petit.
Votre choix ?
2
Vous avez trouvé le nombre secret.
\end{verbatim}
\begin{verbatim}
****************** MENU ******************
*                                        *
*   1) Tester si un nombre est premier   *
*   2) Deviner un nombre                 *
*   0) QUITTER                           *
*                                        *
************************** votre choix : 0
\end{verbatim}
\begin{verbatim}
Sayonara
\end{verbatim}\\[2cm]
 ~
\end{multicols}

\begin{correction}
Correction avec d'autres fonctions ajoutées au menu. La correction est légérement différente de ce qui est demandé.

  \listinginput{1}{menu3.c}
\end{correction}

%%% Local Variables: 
%%% mode: latex
%%% TeX-master: "tp7"
%%% End: 