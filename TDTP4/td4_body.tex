% -*- coding: utf-8 -*-

\newcommand{\commentaire}[1]{}

\entete{Travaux dirigés 4 : fonctions et
  procédures (1)}


\begin{correction}
  Note aux chargés de TD : voir cours fonction. L'appel de fonction
  leur est présenté comme une expression qui s'évalue comme la valeur
  retournée par cette fonction.  Pour cette semaine les appels de
  fonctions utilisateurs n'ont lieu que dans le \verb+main+, à
  l'exception d'une des dernières questions du TD sur le coefficient
  binomial. On dit spécifiquement procédures pour les
  fonctions de type de sortie void.
  
  La trace des programmes avec appel de fonction doit représenter la
  pile d'appel (voir tableau \ref{simulation}). Pour cela, au moment
  de l'appel d'une fonction, on fait la trace de cette fonction en
  indentant vers la droite. À la fin de l'exécution de la fonction,
  quand elle renvoie sa valeur, on poursuit la trace de la fonction
  appelante. La ligne responsable de l'appel est indiquée avant la
  trace de la fonction, mais son exécution proprement dite (si elle a
  sa place dans la trace) n'est réalisée qu'après l'appel, donc elle
  se retrouve après le retour de fonction. Notez que :
\begin{enumerate}
\item on ne trace que les fonctions utilisateurs (bien que l'on
  rappelle que les fonctions systèmes sont bien appelées, comme
  printf).
\item lorsque plusieurs fonctions utilisateurs sont appelées sur une
  même ligne (plusieurs fonctions, arguments d'une autre fonction),
  l'ordre d'évaluation ne doit pas jouer. On choisit les appels de
  gauche à droite, du plus imbriqué au moins imbriqué.
  \end{enumerate}
\end{correction}

\section{Trace de fonctions}

Faire la trace du programme suivant et dire ce que calcule la fonction \verb+est_xxx+.

{
\footnotesize
%\listinginput{1}{echange.c}
\listinginput{1}{premier.c}
}


\begin{correction}
  Cette fonction teste si son argument est
  un nombre premier : elle renvoie TRUE si son argument $n$ est
  premier (ou négatif ou nul -- remarque : zéro n'est pas premier), et FALSE
  sinon.
%% 24 & & &  & \\ \hline
%%           &  &  &  &\multicolumn{1}{r|}{
%%              \put(1,0){\noindent
%%             \begin{tabular}[h]{|c|c|c|c|l|}
  \paragraph{Trace.}
  \begin{table}[h]
      %\begin{center}
        \setlength{\unitlength}{\tabcolsep}
          \begin{tabular}[t]{|c|c|l|}
          \multicolumn{3}{l}{\C{main()}}\\ \hline
          ligne & n & Affichage (sortie écran) \\ \hline
          initialisation  & 9 & \\ \hline
          16 & & \\\hline
          & & \multicolumn{1}{|r|}{
            \put(1,0){\noindent
              \begin{tabular}[t]{|c|c|c|l|}
          \multicolumn{4}{l}{\C{est\_xxx(9)}}\\ \hline
          ligne & n & i & Affichage \\ \hline
          initialisation  & 9 & ? & \\ \hline
         34 &  & 2 & \\ \hline
         40  &  & 3 & \\ \hline
         38  &\multicolumn{3}{|l|}{renvoie FALSE}\\ \hline 
        \end{tabular}
              }
          }\\ \hline 
          22 & & L'entier 9 n'est pas xxx\\ \hline
          26 &\multicolumn{2}{|l|}{renvoie EXIT\_SUCCESS}\\ \hline
          \end{tabular}
      %\end{center}
  \end{table}
  %% \begin{table}[h]
  %%     \begin{center}
  %%       \begin{tabular}[t]{|c|c|l|}
  %%         \multicolumn{3}{l}{\C{main()}}\\ \hline
  %%         ligne & n & Affichage (sortie écran) \\ \hline
  %%         initialisation  & 9 & \\ \hline
  %%         22 & & L'entier 9 n'est pas xxx\\ \hline
  %%         26 &\multicolumn{2}{|l|}{SORTIE AVEC SUCCÈS}\\ \hline
  %%       \end{tabular}
  %%       \begin{tabular}[t]{|c|c|c|l|}
  %%         \multicolumn{4}{l}{\C{est\_xxx(9)}}\\ \hline
  %%         ligne & n & i & Affichage \\ \hline
  %%         initialisation  & 9 & ? & \\ \hline
  %%        34 &  & 2 & \\ \hline
  %%        40  &  & 3 & \\ \hline
  %%        38  &\multicolumn{3}{|l|}{renvoie faux (0)}\\ \hline
  %%       \end{tabular}
  %%       \caption{Trace du programme de l'exercice 1.}
  %%       \label{simulation}
  %%     \end{center}
  %%   \end{table}			
  \end{correction}
  
\section{Déclaration et définition de fonctions}

Les fonctions \verb+valeur_absolue()+, \verb+factorielle()+ et
\verb+minimum()+ ne sont pas fournies avec le programme
suivant. Compléter le programme avec le prototype et le code des
fonctions (le \verb+main+ doit rester inchangé) et
faire la trace du programme complet.

{\footnotesize
\listinginput{14}{min_max_abs.c}
}

\begin{correction}

\paragraph{Code.}

{\footnotesize
\listinginput{1}{min_max_abs_corr.c}
}


  \paragraph{Trace.}

  \begin{table}[h]
%      \begin{center}
        \setlength{\unitlength}{\tabcolsep}
        \begin{tabular}[h]{|c|c|c|c|l|}
         \multicolumn{5}{l}{\C{main()}}\\ \hline
          ligne & x & y & z & Affichage \\ 
          \hline
          initialisation  & -3 & 5 & ? & \\ \hline
          21 & & & & \\ \hline
          &&&& \multicolumn{1}{r|}{
            \put(1,0){\noindent
                \begin{tabular}[h]{|c|c|l|}
            \multicolumn{3}{l}{\C{valeur\_absolue(-3)}}\\ \hline
                  ligne & n & Affichage\\ \hline
                  initialisation  & -3 & \\ \hline
                  35 & \multicolumn{2}{|l|}{renvoie 3} \\ \hline
                \end{tabular}
           }% fin put
          }\\ \hline
          21 & 3 &  &  & \\ \hline
          22 & & & & \\ \hline
          &&&&\multicolumn{1}{r|}{
            \put(1,0){\noindent
                \begin{tabular}[h]{|c|c|c|l|}
                  \multicolumn{4}{l}{\C{minimum(3, 5)}} \\ \hline
                  ligne & a & b & Affichage \\ \hline
                  initialisation  & 3 & 5 & \\ \hline
                  49 & \multicolumn{3}{|l|}{renvoie 3} \\ \hline
                \end{tabular}
              }
            }\\ \hline
         22 &  &  & 3 & \\ \hline
         23 & & & & \\ \hline
         &&&&\multicolumn{1}{r|}{
            \put(1,0){\noindent
             \begin{tabular}[h]{|c|c|c|c|l|}
               \multicolumn{5}{l}{\C{factorielle(3)}} \\ \hline
               ligne & n & i & res & Affichage\\ \hline
               initialisation  & 3 & ? & 1 & \\ \hline
               63 &  & 1 &   & \\ \hline
               65 &  &   & 1 & \\ \hline
               66 &  & 2 &   & \\ \hline
               65 &  &   & 2 & \\ \hline
               66 &  & 3 &   & \\ \hline
               65 &  &   & 6 & \\ \hline
               66 &  & 4 &   & \\ \hline
               69 & \multicolumn{4}{|l|}{renvoie 6} \\ \hline
             \end{tabular}           
           }
         }\\ \hline
         23 &  &  & 6 & \\ \hline
         24 & & & & \\ \hline
         &&&&\multicolumn{1}{r|}{
            \put(1,0){\noindent
              \begin{tabular}[h]{|c|c|c|l|}
                \multicolumn{4}{l}{\C{minimum(5, 6)}}\\ \hline
               ligne & a & b & Affichage \\ \hline
               initialisation  & 5 & 6 & \\ \hline
               49 & \multicolumn{3}{|l|}{renvoie 5} \\ \hline
             \end{tabular}
           }
         }\\ \hline
         24 &  &  & 5 & \\ \hline
         27 & \multicolumn{4}{|l|}{renvoie EXIT\_SUCCESS} \\ \hline
       \end{tabular}
       \caption{Trace du programme de l'exercice 2.}
        \label{simulation2}
%      \end{center}
    \end{table}			

 \end{correction}
 
\section{Écriture de fonctions}

Pour les questions suivantes il faut donner la déclaration et la
définition de chaque fonction. Vous pouvez faire l'exercice une
première fois en donnant uniquement les déclarations, puis le
reprendre pour les définitions. Répondre dans un seul programme, dans
lequel vous écrirez une fonction principale (\C{main}) faisant appel à
ces fonctions pour les tester.
\begin{enumerate}
\item Écrire la fonction \verb|carre| qui prend en entrée un entier et qui renvoie le carré de cet entier.
\item Écrire la fonction \verb|cube| qui prend en entrée un entier et qui renvoie le cube de cet entier.
\item Écrire la fonction \verb|est_majeur| qui prend en entrée un entier représentant l'age en années d'une personne et renvoie \verb|TRUE| si cette personne est majeure et \verb|FALSE| sinon (on considérera les deux constantes utilisateurs bien déclarées).
\item Écrire la fonction \verb|somme| qui prend en entrée un entier
  $n$ et qui renvoie $\sum_{i=1}^{i=n} i$. Où faut-il déclarer la
  variable de boucle ?
\item Si vous ne l'avez pas fait à l'exercice précédent, écrire la fonction \verb|factorielle| qui prend en entrée un entier
  $n$ et qui renvoie $\Pi_{i=1}^{i=n} i$. 
\item Écrire la procédure \verb|afficher_rectangle| qui prend en
  entrée deux entiers, largeur et hauteur, et affiche un rectangle
  d'étoiles de ces dimensions.
\item Écrire la fonction \verb|saisie_utilisateur| sans argument, qui demande à l'utilisateur de saisir un nombre entier et le
  retourne.
\item Écrire la fonction \verb|binomial| qui prend en entrée un entier
  $n$ et un entier $p$ et retourne le nombre de tirages différents,
  non ordonnés et sans remise, de $p$ éléments parmi $n$, c'est à dire
  le coefficient binomial :
\[
\binom{n}{p} = \text{C}^p_n = \frac{n!}{p!(n-p)!}.
\]
Faire appel à la fonction \verb|factorielle|.
\item Optionnel. Écrire la fonction \verb|saisie_dans_intervalle| à deux
  paramètre entiers $a$ et $b$, qui demande à l'utilisateur de saisir
  un nombre entier jusqu'à ce qu'il soit dans l'intervalle $[a, b]$ et le
  retourne. On pourra fixer un nombre maximum de tentatives après quoi
  le nombre de l'intervalle le plus proche de la saisie utilisateur
  sera renvoyé.
\end{enumerate}

\begin{correction}
\begin{verbatim}
int cube(int x);
int est_majeur(int age);
int somme(int n);
void afficher_rectangle(int largeur, int hauteur);
int saisie_utilisateur();
int binomial(int n, int p);
int autre_somme(int n);


int carre(int x)
{
   return x*x;
}

int cube(int x)
{
   return x*x*x;
}

int est_majeur(int age)
{
    if (age < 18)
    {
        return FALSE;
    }
    /* sinon */
    return TRUE;
}

int somme(int n)
{
    int i;
    int somme = 0;

    for (i = 1; i <= n; i = i + 1)
    {
        somme = somme + i;
    }

    return somme; /* ou bien juste : return n * (n+1) / 2; */
}

void afficher_rectangle(int largeur, int hauteur)
{
    int i; /* lignes */
    int j; /* colonnes */

    for (i = 0; i < hauteur; i = i + 1) /* pour chaque ligne */
    {
        /* afficher largeur etoiles et un saut de ligne */
        for (j = 0; j < largeur; j = j + 1) /* pour chaque colonne */
        {
            printf("*");
        }
        printf("\n");
    }
}

int saisie_utilisateur()
{
    int n = 0;
    printf("entrer un entier : ");
    scanf("%d", &n);
    return n;
}

int binomial(int n, int p)
{
    return factorielle(n) / (factorielle(p) * factorielle(n - p));
}
\end{verbatim}
\end{correction}

%%% Local Variables: 
%%% mode: latex
%%% TeX-master: "td4"
%%% End: 