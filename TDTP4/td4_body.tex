% -*- coding: utf-8 -*-

\newcommand{\commentaire}[1]{}

\entete{Travaux dirigés 4 : fonctions et
  procédures (1)}


\begin{correction}
  Note aux chargés de TD : voir cours fonction. L'appel de fonction
  leur est présenté comme une expression qui s'évalue comme la valeur
  retournée par cette fonction.  Pour cette semaine les appels de
  fonctions utilisateurs n'ont lieu que dans le \verb+main+, à
  l'exception d'une des dernières questions du TD sur le coefficient
  binomial. On dit spécifiquement procédures pour les
  fonctions de type de sortie void.

  La trace des programmes avec appel de fonction doit représenter la
  pile d'appel (voir tableau \ref{simulation}). Pour cela, au moment
  de l'appel d'une fonction, on fait la trace de cette fonction en
  indentant vers la droite. À la fin de l'exécution de la fonction,
  quand elle renvoie sa valeur, on poursuit la trace de la fonction
  appelante. La ligne responsable de l'appel est indiquée avant la
  trace de la fonction, mais son exécution proprement dite (si elle a
  sa place dans la trace) n'est réalisée qu'après l'appel, donc elle
  se retrouve après le retour de fonction. Notez que :
\begin{enumerate}
\item on ne trace que les fonctions utilisateurs (bien que l'on
  rappelle que les fonctions systèmes sont bien appelées, comme
  printf).
\item lorsque plusieurs fonctions utilisateurs sont appelées sur une
  même ligne (plusieurs fonctions, arguments d'une autre fonction),
  l'ordre d'évaluation ne doit pas jouer. On choisit les appels de
  gauche à droite, du plus imbriqué au moins imbriqué.
  \end{enumerate}
\end{correction}

\section{Trace de fonctions}

Faire la trace du programme suivant et dire ce que calcule la fonction \verb+est_xxx+.

{
\footnotesize
%\listinginput{1}{echange.c}
\listinginput{1}{premier.c}
}


\begin{correction}
  Cette fonction teste si son argument est
  un nombre premier : elle renvoie true si son argument $n$ est
  premier (ou négatif ou nul -- remarque : zéro n'est pas premier), et false
  sinon.
%% 24 & & &  & \\ \hline
%%           &  &  &  &\multicolumn{1}{r|}{
%%              \put(1,0){\noindent
%%             \begin{tabular}[h]{|c|c|c|c|l|}
  \paragraph{Trace.}
  \begin{table}[h]
      %\begin{center}
        \setlength{\unitlength}{\tabcolsep}
          \begin{tabular}[t]{|c|c|l|}
          \multicolumn{3}{l}{\C{main()}}\\ \hline
          ligne & n & Affichage (sortie écran) \\ \hline
          initialisation  & 9 & \\ \hline
          15 & & \\\hline
          & & \multicolumn{1}{|r|}{
            \put(1,0){\noindent
              \begin{tabular}[t]{|c|c|c|l|}
          \multicolumn{4}{l}{\C{est\_xxx(9)}}\\ \hline
          ligne & n & i & Affichage \\ \hline
          initialisation  & 9 & ? & \\ \hline
         33 &  & 2 & \\ \hline
         37  &  & 3 & \\ \hline
         35  &\multicolumn{3}{|l|}{renvoie false}\\ \hline
        \end{tabular}
              }
          }\\ \hline
          21 & & L'entier 9 n'est pas xxx\\ \hline
          25 &\multicolumn{2}{|l|}{renvoie EXIT\_SUCCESS}\\ \hline
          \end{tabular}
          %\end{center}
          \caption{Trace du programme de l'exercice 1.}
          \label{simulation}
  \end{table}
\end{correction}

\section{Déclaration et définition de fonctions}

Les fonctions \verb+valeur_absolue()+, \verb+factorielle()+ et
\verb+minimum()+ ne sont pas fournies avec le programme (\C{fonctions1.c})
suivant. Compléter le programme avec le prototype et le code des
fonctions (le \verb+main+ doit rester inchangé) et
faire la trace du programme complet.

{\footnotesize
\listinginput{14}{min_max_abs.c}
}

\begin{correction}

\paragraph{Code.}

{\footnotesize
\listinginput{1}{min_max_abs_corr.c}
}


  \paragraph{Trace.}

  \begin{table}[h]
%      \begin{center}
        \setlength{\unitlength}{\tabcolsep}
        \begin{tabular}[h]{|c|c|c|c|l|}
         \multicolumn{5}{l}{\C{main()}}\\ \hline
          ligne & x & y & z & Affichage \\
          \hline
          initialisation  & -3 & 5 & ? & \\ \hline
          21 & & & & \\ \hline
          &&&& \multicolumn{1}{r|}{
            \put(1,0){\noindent
                \begin{tabular}[h]{|c|c|l|}
            \multicolumn{3}{l}{\C{valeur\_absolue(-3)}}\\ \hline
                  ligne & n & Affichage\\ \hline
                  initialisation  & -3 & \\ \hline
                  35 & \multicolumn{2}{|l|}{renvoie 3} \\ \hline
                \end{tabular}
           }% fin put
          }\\ \hline
          21 & 3 &  &  & \\ \hline
          22 & & & & \\ \hline
          &&&&\multicolumn{1}{r|}{
            \put(1,0){\noindent
                \begin{tabular}[h]{|c|c|c|l|}
                  \multicolumn{4}{l}{\C{minimum(3, 5)}} \\ \hline
                  ligne & a & b & Affichage \\ \hline
                  initialisation  & 3 & 5 & \\ \hline
                  49 & \multicolumn{3}{|l|}{renvoie 3} \\ \hline
                \end{tabular}
              }
            }\\ \hline
         22 &  &  & 3 & \\ \hline
         23 & & & & \\ \hline
         &&&&\multicolumn{1}{r|}{
            \put(1,0){\noindent
             \begin{tabular}[h]{|c|c|c|c|l|}
               \multicolumn{5}{l}{\C{factorielle(3)}} \\ \hline
               ligne & n & i & res & Affichage\\ \hline
               initialisation  & 3 & ? & 1 & \\ \hline
               63 &  & 1 &   & \\ \hline
               65 &  &   & 1 & \\ \hline
               66 &  & 2 &   & \\ \hline
               65 &  &   & 2 & \\ \hline
               66 &  & 3 &   & \\ \hline
               65 &  &   & 6 & \\ \hline
               66 &  & 4 &   & \\ \hline
               69 & \multicolumn{4}{|l|}{renvoie 6} \\ \hline
             \end{tabular}
           }
         }\\ \hline
         23 &  &  & 6 & \\ \hline
         24 & & & & \\ \hline
         &&&&\multicolumn{1}{r|}{
            \put(1,0){\noindent
              \begin{tabular}[h]{|c|c|c|l|}
                \multicolumn{4}{l}{\C{minimum(2, 6)}}\\ \hline
               ligne & a & b & Affichage \\ \hline
               initialisation  & 5 & 6 & \\ \hline
               49 & \multicolumn{3}{|l|}{renvoie 2} \\ \hline
             \end{tabular}
           }
         }\\ \hline
         24 &  &  & 2 & \\ \hline
         27 & \multicolumn{4}{|l|}{renvoie EXIT\_SUCCESS} \\ \hline
       \end{tabular}
       \caption{Trace du programme de l'exercice 2.}
        \label{simulation2}
%      \end{center}
    \end{table}

 \end{correction}

\section{Écriture de fonctions}

Pour les questions suivantes il faut donner la déclaration et la
définition de chaque fonction. Vous pouvez faire l'exercice une
première fois en donnant uniquement les déclarations, puis le
reprendre pour les définitions. Répondre dans un seul programme, \C{fonctions2.c}, dans
lequel vous écrirez une fonction principale (\C{main}) faisant appel à
toutes les fonctions pour les tester.
\begin{enumerate}
\item Écrire la fonction \verb|carre| qui prend en entrée un entier et qui renvoie le carré de cet entier.
\item Écrire la fonction \verb|cube| qui prend en entrée un entier et qui renvoie le cube de cet entier.
\item Écrire la fonction \verb|est_majeur| qui prend en entrée un entier représentant l'age en années d'une personne et renvoie \verb|true| si cette personne est majeure et \verb|false| sinon (on considérera les deux constantes utilisateurs bien déclarées).
\item Écrire la fonction \verb|somme| qui prend en entrée un entier
  $n$ et qui renvoie $\sum_{i=1}^{i=n} i$. Où faut-il déclarer la
  variable de boucle ?
\item Si vous ne l'avez pas fait à l'exercice précédent, écrire la fonction \verb|factorielle| qui prend en entrée un entier
  $n$ et qui renvoie $\Pi_{i=1}^{i=n} i$.
\item Écrire la procédure \verb|afficher_rectangle| qui prend en
  entrée deux entiers, largeur et hauteur, et affiche un rectangle
  d'étoiles de ces dimensions.
\item Écrire la fonction \verb|saisie_utilisateur| sans argument, qui demande à l'utilisateur de saisir un nombre entier et le
  retourne.
\item Écrire la fonction \verb|binomial| qui prend en entrée un entier
  $n$ et un entier $p$ et retourne le nombre de tirages différents,
  non ordonnés et sans remise, de $p$ éléments parmi $n$, c'est à dire
  le coefficient binomial :
\[
\binom{n}{p} = \text{C}^p_n = \frac{n!}{p!(n-p)!}.
\]
Faire appel à la fonction \verb|factorielle|.
\item Optionnel. Écrire la fonction \verb|saisie_dans_intervalle| à deux
  paramètre entiers $a$ et $b$, qui demande à l'utilisateur de saisir
  un nombre entier jusqu'à ce qu'il soit dans l'intervalle $[a, b]$ et le
  retourne. On pourra fixer un nombre maximum de tentatives après quoi
  le nombre de l'intervalle le plus proche de la saisie utilisateur
  sera renvoyé.
\end{enumerate}

\begin{correction}
\begin{verbatim}
int cube(int x);
int est_majeur(int age);
int somme(int n);
void afficher_rectangle(int largeur, int hauteur);
int saisie_utilisateur();
int binomial(int n, int p);

int carre(int x)
{
   return x*x;
}

int cube(int x)
{
   return x*x*x;
}

int est_majeur(int age)
{
    if (age < 18)
    {
        return false;
    }
    /* sinon */
    return true;
}

int somme(int n)
{
    int i;
    int somme = 0;

    for (i = 1; i <= n; i = i + 1)
    {
        somme = somme + i;
    }

    return somme; /* ou bien juste : return n * (n+1) / 2; */
}

void afficher_rectangle(int largeur, int hauteur)
{
    int i; /* lignes */
    int j; /* colonnes */

    for (i = 0; i < hauteur; i = i + 1) /* pour chaque ligne */
    {
        /* afficher largeur etoiles et un saut de ligne */
        for (j = 0; j < largeur; j = j + 1) /* pour chaque colonne */
        {
            printf("*");
        }
        printf("\n");
    }
}

int saisie_utilisateur()
{
    int n = 0;
    printf("entrer un entier : ");
    scanf("%d", &n);
    return n;
}

int binomial(int n, int p)
{
    return factorielle(n) / (factorielle(p) * factorielle(n - p));
}
\end{verbatim}
\end{correction}


\section{Programmer avec des fonctions et procédures}
Dans les exercices suivants, l'objectif est de mettre au point des programmes entiers en y apportant vos
propres fonctions et procédures. La méthode pour y arriver : \textbf{découper le
programmes en tâches simples à mettre en œuvre et
progresser par étapes en  testant régulièrement votre code, même s'il
est incomplet}.

Pour chaque exercice, vous pouvez faire plus ou moins de fonctions et
procédures selon si vous découpez le travail en de plus ou moins gros
morceaux. Lorsqu'on découpe une tâche en beaucoup de morceaux on parle
de \emph{code ravioli}. Par exemple, pour afficher des motifs
d'étoiles vous pouvez déclarer et définir les procédures suivantes :
\begin{small}
\begin{verbatim}
void afficher_ligne(int nb_etoiles); /* afficher_ligne(5) affiche "*****\n" */
void afficher_etoile();              /* affiche "*" */
void afficher_espace();              /* affiche " " */
void finir_ligne();                  /* affiche "\n" */
\end{verbatim}
\end{small}


Lorsque c'est possible, exercez vous à produire du code ravioli en
rajoutant des fonctions et procédures (par exemple, utiliser
\verb+afficher_ligne(n)+ au lieu d'écrire directement une boucle for
pour afficher $n$ étoiles et un retour à la ligne). Puis,
réciproquement, à supprimer les sous-tâches qui vous paraissent
exagérées (par exemple, remplacer un appel à \verb+afficher_espace()+
par un appel direct à \verb+printf(" ")+.)

\begin{correction}
À faire.
\end{correction}

\subsection{Encore un triangle d'étoiles}
Pippo a commencé à écrire un procédure qui permet d'afficher un
triangle d'étoiles. Comme c'est un peu compliqué, il progresse par étapes
et fait en sorte de pouvoir tester son code le plus tôt possible, même
s'il n'est pas terminé.  Compléter son programme \C{triangle3.c}, figure~\ref{fig:triangle}.

\begin{figure}[tb]
  \begin{small}
\begin{verbatim}
#include <stdlib.h>
#include <stdio.h>

void afficher_triangle(int cote);

int main() {
    int i; /* var de boucle */

    /* testons différentes tailles de triangles */
    for (i = 2; i <= 10; i += 2) {
        printf("\n\nafficher_triangle(%d)\n", i);
        afficher_triangle(i);
    }
    return EXIT_SUCCESS;
}

void afficher_triangle(int cote) {
    int i; /* var de boucle */

    /* il faudrait deja arriver a faire un carre ! */
    for (i = 0; i < cote; i += 1) {/* pour chaque ligne */
        /* afficher une ligne de cote etoiles. Faire une sous-procedure ? */
    }
}
\end{verbatim}
\vspace{-0.5cm}
  \end{small}
  \caption{Programme de Pippo pour mettre au point la fonction afficher triangle}
\label{fig:triangle}
\end{figure}

\begin{correction}
À faire.
\end{correction}


\subsection{Population de lapins}

Un couple de lapins a sa première portée à deux mois, puis une portée
tous les mois. Chaque portée est un couple de lapins. Tous les couples
ainsi obtenus se reproduisent de la même manière.\footnote{Merci à Laure Petrucci pour
    cet exercice et à  Lionel Allorge pour le dessin de lapin en cc-by-sa.}

\newcommand{\plapin}{\includegraphics[scale=0.05]{320px-Lapin01.png}}
\newcommand{\mlapin}{\includegraphics[scale=0.08]{320px-Lapin01.png}}
\newcommand{\glapin}{\includegraphics[scale=0.11]{320px-Lapin01.png}}

\begin{center}
  \begin{tabular}{l c l}
    1er septembre & \plapin \plapin &(deux jeunes
    lapins) \\
    1er octobre &\mlapin \mlapin& (deux lapins d'un mois)\\
    1er novembre &\plapin \plapin \quad \glapin \glapin& (deux vieux
    et leurs deux petits)\\
    1er décembre & \plapin \plapin   \quad  \mlapin \mlapin  \quad
    \glapin \glapin & \\
    1er janvier&\plapin \plapin \plapin \plapin   \quad  \mlapin
    \mlapin  \quad \glapin \glapin \glapin \glapin   &
  \end{tabular}
\end{center}
\begin{newenu}
\item   On distinguera le nombre de couples de nouveaux lapins
\verb+nouveaux+, le nombre de couples de lapins ayant un mois,
\verb+un_mois+, et le nombre de couples de lapins ayant 2 mois ou
plus, \verb+vieux+. Calculer (à la main) le nombre de couples de
lapins de chaque type, ainsi que leur nombre total, pour les 10
premiers mois.
\item Écrire un algorithme \verb+nb_lapins(nb_mois, nb_couples)+
  calculant le nombre de lapins obtenus au bout de \verb+nb_mois+ mois
  à partir de \verb+nb_couples+ couples jeunes, et renvoyant le
  résultat.
\item À combien s'élève la population au bout de 24 mois ?
\item Écrire un algorithme \verb+lapins_un_milliard+ calculant au bout
  de combien de temps les lapins sont plus d’un milliard (on supposera
  qu’aucun lapin ne meurt pendant cette période), en partant d’un
  couple de jeunes lapins.
\item Mettre en œuvre ces algorithmes dans un programme C, \C{lapins.c}.
\end{newenu}

\paragraph{Les lapins disparaissent !} Cinq couples de jeunes lapins se sont
installés sur une île déserte et ont fondé une communauté. Les premiers
mois tout se passe normalement, mais à partir du septième mois, des
lapins disparaissent après avoir donné naissance à leurs dernières portées
: exactement un cinquième des vieux lapins,
quatre cinquièmes des lapins venant d'avoir deux mois et neuf dixièmes des
nouveaux lapins survivent (par arrondi inférieur), tous les autres disparaissent. Ceci tous les mois, sans que la cause ne soit connue
(certains scientifiques lapins mettent en cause la nourriture
transgénique ou les antibiotiques large spectre, d'autres lapins évoquent une famille de renards, et d'autres encore pensent que les
services secrets humains ont découvert le plan d'invasion de la Terre
par les lapins et ont pris des contre-mesures).

\begin{lastenu}
  \item Tenir compte de ces nouvelles données dans la simulation. Que
    se passe t'il ?
\item Afficher la population de lapin obtenue à chaque nouveau mois
  sous forme graphique sur une ligne. Par exemple, un couple de
  nouveaux lapins sera affiché par le caractère point "\verb+.+'', un couple de lapins d'un
  mois par le caractère "\verb+o+", un vieux couple de lapins par le
  caractère "\verb+#+".
\item Combien de lapins sont nés sur l'île ?
\end{lastenu}

\begin{correction}
  \begin{small}
    \verbatiminput{lapins.c}
    \end{small}
\end{correction}

%%% Local Variables:
%%% mode: latex
%%% TeX-master: "td4"
%%% End:
