\begin{tabular}[c]{l||c|c|c|c|c|c|c|c|c|c|c|c|c|}
\hline
 \emph{Instructions} & Cycles & CP& r0& r1& r2& r3& r4& r5& 30& 31& 32& 33& 34\\ \hline
\hfill Initialisation & 0 & 1 & ? & ? & ? & ? & ? & ? & 3
 & 10
 & 100
 & 1
 & ?
 \\ \hline \commentaire{Lecture de la donnée d'adresse 30 dans le registre 1
} \C{lecture 30 r1
} & 1 & 2  & & 3 & & & & & & & & &\\ \hline
 \commentaire{Lecture de la donnée d'adresse 31 dans le registre 0
} \C{lecture 31 r0
} & 2 & 3  & 10 & & & & & & & & & &\\ \hline
 \commentaire{Initialisation du registre 2 à 2
} \C{valeur 2 r2
} & 3 & 4  & & & 2 & & & & & & & &\\ \hline
 \commentaire{Saut à l'adresse 12
} \C{saut 12
} & 4 & \textbf{12} & & & & & & & & & & &\\ \hline
 \commentaire{Initialisation du registre 3 à 0
} \C{valeur 0 r3
} & 5 & 13  & & & & 0 & & & & & & &\\ \hline
 \commentaire{Ajout de la valeur du registre 1 au registre 3
} \C{add r1 r3
} & 6 & 14  & & & & 3 & & & & & & &\\ \hline
 \commentaire{Soustrait la valeur du registre 2 au registre 3
} \C{soustr r2 r3
} & 7 & 15  & & & & 1 & & & & & & &\\ \hline
 \commentaire{Si la valeur (1) du registre 3 est positive, saute à l'adresse 5
} \C{sautpos r3 5
} & 8 & \textbf{5} & & & & & & & & & & &\\ \hline
 \commentaire{Initialisation du registre 4 à 30
} \C{valeur 30 r4
} & 9 & 6  & & & & & 30 & & & & & &\\ \hline
 \commentaire{Ajout de la valeur du registre 2 au registre 4
} \C{add r2 r4
} & 10 & 7  & & & & & 32 & & & & & &\\ \hline
 \commentaire{Lecture de la donnée d'adresse 32 dans le registre 5
} \C{lecture *r4 r5
} & 11 & 8  & & & & & & 100 & & & & &\\ \hline
 \commentaire{Soustrait la valeur du registre 0 au registre 5
} \C{soustr r0 r5
} & 12 & 9  & & & & & & 90 & & & & &\\ \hline
 \commentaire{Si la valeur (90) du registre 5 est positive, saute à l'adresse 11
} \C{sautpos r5 11
} & 13 & \textbf{11} & & & & & & & & & & &\\ \hline
 \commentaire{Ajout de la valeur 1 au registre 2
} \C{add 1 r2
} & 14 & 12  & & & 3 & & & & & & & &\\ \hline
 \commentaire{Initialisation du registre 3 à 0
} \C{valeur 0 r3
} & 15 & 13  & & & & 0 & & & & & & &\\ \hline
 \commentaire{Ajout de la valeur du registre 1 au registre 3
} \C{add r1 r3
} & 16 & 14  & & & & 3 & & & & & & &\\ \hline
 \commentaire{Soustrait la valeur du registre 2 au registre 3
} \C{soustr r2 r3
} & 17 & 15  & & & & 0 & & & & & & &\\ \hline
 \commentaire{Si la valeur (0) du registre 3 est positive, saute à l'adresse 5
} \C{sautpos r3 5
} & 18 & \textbf{5} & & & & & & & & & & &\\ \hline
 \commentaire{Initialisation du registre 4 à 30
} \C{valeur 30 r4
} & 19 & 6  & & & & & 30 & & & & & &\\ \hline
 \commentaire{Ajout de la valeur du registre 2 au registre 4
} \C{add r2 r4
} & 20 & 7  & & & & & 33 & & & & & &\\ \hline
 \commentaire{Lecture de la donnée d'adresse 33 dans le registre 5
} \C{lecture *r4 r5
} & 21 & 8  & & & & & & 1 & & & & &\\ \hline
 \commentaire{Soustrait la valeur du registre 0 au registre 5
} \C{soustr r0 r5
} & 22 & 9  & & & & & & -9 & & & & &\\ \hline
 \commentaire{Si la valeur (-9) du registre 5 est positive, saute à l'adresse 11
} \C{sautpos r5 11
} & 23 & 10  & & & & & & & & & & &\\ \hline
 \commentaire{Lecture de la donnée d'adresse 33 dans le registre 0
} \C{lecture *r4 r0
} & 24 & 11  & 1 & & & & & & & & & &\\ \hline
 \commentaire{Ajout de la valeur 1 au registre 2
} \C{add 1 r2
} & 25 & 12  & & & 4 & & & & & & & &\\ \hline
 \commentaire{Initialisation du registre 3 à 0
} \C{valeur 0 r3
} & 26 & 13  & & & & 0 & & & & & & &\\ \hline
 \commentaire{Ajout de la valeur du registre 1 au registre 3
} \C{add r1 r3
} & 27 & 14  & & & & 3 & & & & & & &\\ \hline
 \commentaire{Soustrait la valeur du registre 2 au registre 3
} \C{soustr r2 r3
} & 28 & 15  & & & & -1 & & & & & & &\\ \hline
 \commentaire{Si la valeur (-1) du registre 3 est positive, saute à l'adresse 5
} \C{sautpos r3 5
} & 29 & 16  & & & & & & & & & & &\\ \hline
 \commentaire{Initialisation du registre 4 à 31
} \C{valeur 31 r4
} & 30 & 17  & & & & & 31 & & & & & &\\ \hline
 \commentaire{Ajout de la valeur du registre 1 au registre 4
} \C{add r1 r4
} & 31 & 18  & & & & & 34 & & & & & &\\ \hline
 \commentaire{Écriture du registre 0 à l'adresse 34
} \C{ecriture r0 *r4
} & 32 & 19  & & & & & & & & & & & 1
\\ \hline
 \commentaire{Fin du processus.
} \C{stop
} & 33 & 20  & & & & & & & & & & &\\ \hline
\end{tabular}