\begin{tabular}[c]{l||c|c|c|c|c|c|}
\hline
 \emph{Instructions} & Cycles & CP& r0& r1& 15& 16\\ \hline
\hfill Initialisation & 0 & 1 & ? & ? & 3
 & ?
 \\ \hline \commentaire{Lecture de la donnée d'adresse 15 dans le registre 0
} \C{lecture 15 r0
} & 1 & 2  & 3 & & &\\ \hline
 \commentaire{Saut à l'adresse 6
} \C{saut 6
} & 2 & \textbf{6} & & & &\\ \hline
 \commentaire{Si la valeur (3) du registre 0 est positive, saute à l'adresse 3
} \C{sautpos r0 3
} & 3 & \textbf{3} & & & &\\ \hline
 \commentaire{Écriture du registre 0 à l'adresse 16
} \C{ecriture r0 16
} & 4 & 4  & & & & 3
\\ \hline
 \commentaire{Initialisation du registre 1 à -1
} \C{valeur -1 r1
} & 5 & 5  & & -1 & &\\ \hline
 \commentaire{Ajout de la valeur du registre 1 au registre 0
} \C{add r1 r0
} & 6 & 6  & 2 & & &\\ \hline
 \commentaire{Si la valeur (2) du registre 0 est positive, saute à l'adresse 3
} \C{sautpos r0 3
} & 7 & \textbf{3} & & & &\\ \hline
 \commentaire{Écriture du registre 0 à l'adresse 16
} \C{ecriture r0 16
} & 8 & 4  & & & & 2
\\ \hline
 \commentaire{Initialisation du registre 1 à -1
} \C{valeur -1 r1
} & 9 & 5  & & -1 & &\\ \hline
 \commentaire{Ajout de la valeur du registre 1 au registre 0
} \C{add r1 r0
} & 10 & 6  & 1 & & &\\ \hline
 \commentaire{Si la valeur (1) du registre 0 est positive, saute à l'adresse 3
} \C{sautpos r0 3
} & 11 & \textbf{3} & & & &\\ \hline
 \commentaire{Écriture du registre 0 à l'adresse 16
} \C{ecriture r0 16
} & 12 & 4  & & & & 1
\\ \hline
 \commentaire{Initialisation du registre 1 à -1
} \C{valeur -1 r1
} & 13 & 5  & & -1 & &\\ \hline
 \commentaire{Ajout de la valeur du registre 1 au registre 0
} \C{add r1 r0
} & 14 & 6  & 0 & & &\\ \hline
 \commentaire{Si la valeur (0) du registre 0 est positive, saute à l'adresse 3
} \C{sautpos r0 3
} & 15 & \textbf{3} & & & &\\ \hline
 \commentaire{Écriture du registre 0 à l'adresse 16
} \C{ecriture r0 16
} & 16 & 4  & & & & 0
\\ \hline
 \commentaire{Initialisation du registre 1 à -1
} \C{valeur -1 r1
} & 17 & 5  & & -1 & &\\ \hline
 \commentaire{Ajout de la valeur du registre 1 au registre 0
} \C{add r1 r0
} & 18 & 6  & -1 & & &\\ \hline
 \commentaire{Si la valeur (-1) du registre 0 est positive, saute à l'adresse 3
} \C{sautpos r0 3
} & 19 & 7  & & & &\\ \hline
 \commentaire{Fin du processus.
} \C{stop
} & 20 & 8  & & & &\\ \hline
\end{tabular}