% -*- coding: utf-8 -*-

\newcommand{\commentaire}[1]{}

\entete{Travaux dirigés 1 : Assembleur, affectation en C, if else.}

\vspace{-1cm}

\section{Premier programme C et affectation}


\begin{figure}[h]
  \centering
  \shortstack{Programme C\\
    \fbox{\quad
      \begin{minipage}[b]{.68\linewidth}
        \small \listinginput{1}{affectations.c}
      \end{minipage}}}\hfill
  \shortstack{Traduction\\
    \fbox{\quad
      \begin{minipage}[b]{.22\linewidth}
        \small \listinginput{1}{affectations.ail}
      \end{minipage}}}
\caption{Un programme C et sa traduction machine}
\label{fig:programmes}
\end{figure}


Voici, figure~\ref{fig:programmes}, un premier programme C.
% Un programme en langage C est donné sous
% forme de texte, \emph{le source}, et
% il doit passer par une étape de
% traduction avant de pouvoir être
% exécuté par le processeur.
Nous donnons ici la traduction de ce code source en amil. Il s'agit d'un artifice pédagogique, la traduction réelle en code binaire exécutable est plus compliquée.  Par analogie avec la musique, le source est la partition, et le fichier exécutable est le morceau musical (codé sur le support adapté au système de lecture : un fichier mp3, un CD, etc.). La traduction est effectuée par un ensemble de programmes, le source doit donc obéir à des règles syntaxiques précises.

\begin{correction}
Pour la correction, voici également (figure~\ref{fig:traces}) les traces de ces deux programmes,
auxquelles se référer si les étudiants ne comprennent pas les
programmes en eux même.
\begin{figure}[htbp]
  \centering
  \subtable[Trace en C]{
    \begin{tabular}[h]{|r|c|c| l |}
      \hline
      Ligne & x & y & sortie \\ \hline
      initialisation & 5 & ? & \\ \hline
      15 & & 2 & \\ \hline
      16 & 2 &  & \\ \hline
19 & \multicolumn{3}{|c|}{renvoie \C{EXIT\_SUCCESS}}\\ \hline
   \end{tabular}
    \label{tab:traceC}
  }
  \subtable[Trace amil]{
    \begin{tabular}[c]{l||c|c|c|c|c|}
\hline
 \emph{Instructions} & Cycles & CP& r0& 10& 11\\ \hline
\hfill Initialisation & 0 & 1 & ? & 5
 & ?
 \\ \hline \commentaire{Initialisation du registre 0 à 2
} \C{valeur 2 r0
} & 1 & 2  & 2 & &\\ \hline
 \commentaire{Écriture du registre 0 à l'adresse 11
} \C{ecriture r0 11
} & 2 & 3  & & & 2
\\ \hline
 \commentaire{Lecture de la donnée d'adresse 11 dans le registre 0
} \C{lecture 11 r0
} & 3 & 4  & 2 & &\\ \hline
 \commentaire{Écriture du registre 0 à l'adresse 10
} \C{ecriture r0 10
} & 4 & 5  & & 2
 &\\ \hline
 \commentaire{Fin du processus.
} \C{stop
} & 5 & 6  & & &\\ \hline
\end{tabular}
    \label{tab:traceamil}
  }
  \caption{Traces}
\label{fig:traces}
\end{figure}
\end{correction}

Les textes entre \C{/*} et \C{*/} sont des \emph{commentaires}, ils ne feront pas partie du programme exécutable, ils servent aux humains qui manipulent les programmes. Les commentaires du programme figure~\ref{fig:programmes} vous serviront au cours du semestre à structurer tous vos programmes C.

Tout programme C comporte une fonction principale, le \C{main()}, qui sert de point d'entrée au programme. Cette fonction doit se terminer par l'instruction \verb+return EXIT_SUCCESS+. En renvoyant cette valeur, le \C{main} signale au système d'exploitation la terminaison correcte du programme.


% Une \emph{variable} est un symbole (habituellement un nom) qui renvoie à une position de mémoire dont le contenu peut prendre successivement différentes valeurs pendant l'exécution d'un programme.

L'instruction \C{int x = 5} déclare une variable \C{x} et fixe sa valeur initiale à 5. Le mot clé \C{int} signifie que cette variable contiendra un entier. Dans le code amil \C{x} correspond à l'adresse \C{10} où se trouve intialement la valeur \C{5}.

L'instruction \C{int y} déclare une variable entière \C{y} sans l'initialiser. L'effet de cette déclaration est de réserver un espace mémoire pour y stocker un entier.

Le signe égal (\C{=}) a un sens bien particulier, il dénote une
\emph{affectation}. L'objectif de cette partie du TD est de bien comprendre
l'affectation.  La partie à gauche du signe égal doit désigner une case mémoire, c'est typiquement une variable. La partie à droite du signe égal est une expression dont la valeur sera évaluée et écrite à l'adresse à laquelle renvoie la partie gauche. Par exemple, \C{y = 2} a été traduit en code machine par une instruction évaluant l'expression \C{2} dans un registre (ici \C{valeur 2 r0}), et par une instruction d'écriture de la valeur trouvée dans la mémoire réservée à \C{y}. Une variable s'évalue comme sa valeur (celle contenue dans la mémoire correspondante, au moment de l'évaluation).


\begin{enumerate}
\item Quel espace mémoire a été réservé pour \C{y} dans le code amil ?
  \begin{correction}
L'adresse 11.
  \end{correction}
\item Comment a été traduite l'instruction d'affectation \C{x = y} en amil ?
  \begin{correction}
Lignes 3 et 4 :
\begin{verbatim}
lecture 11 r0
ecriture r0 10
\end{verbatim}
  \end{correction}
\item Si il y avait \C{y = x + 2}, ligne 15 dans le programme C, à la
  place de \C{y = 2}, quel serait le code amil correspondant ? Et \C{x
  = x + 1} ligne 16 ?
  \begin{correction}
Ne pas se préoccuper du décalage des lignes de code amil. Pour \C{y =
  x + 2} :
\begin{verbatim}
    lecture 10 r0
    valeur 2 r1
    add r1 r0
    ecriture r0 11
\end{verbatim}
Pour \C{x =
  x + 1} :
\begin{verbatim}
    lecture 10 r0
    valeur 1 r1
    add r1 r0
    ecriture r0 10
\end{verbatim}

  \end{correction}
\end{enumerate}


\subsection{Échange des valeurs de deux variables en C}

\begin{enumerate}
\item
  Écrire un programme C qui déclare et initialise deux variables
  entières \C{x} et \C{y} et effectue la permutation de ces deux
  valeurs. Commencer par écrire un algorithme, à l'aide de phrases
  telles que << Copier la valeur de la variable ... dans la variable
  ...>> (indication: vous pouvez utiliser plus de deux variables).
\begin{correction}
L'agorithme doit être utilisé pour structurer le code. On ne donne pas ici de valeur à $x$ et $y$, mais n'hésitez pas à en donner (par initialisation par exemple).

{\small
\listinginput{1}{echange.c}
}
\end{correction}

\item Traduire ce programme C en un programme amil. On supposera que les deux variables sont stockées aux adresses \C{10} et \C{11}.
  \begin{correction}
    À nouveau, l'algorithme est à utiliser en commentaires pour
      structurer le code.
\begin{verbatim}
# copie de la case d'adresse 10 dans une case auxillaire (12)
 1 lecture 10 r0
 2 ecriture r0 12
# copie de la case d'adresse 11 à l'adresse 10
 3 lecture 11 r0
 4 ecriture r0 10
# copie de la case auxillaire à l'adresse 11
 5 lecture 12 r0
 6 ecriture r0 11
 7 stop
 9
10 4
11 5
12 ?
\end{verbatim}
  \end{correction}
%\item Exécuter ces programmes sur un exemple (faire les traces).
\begin{correction}
 \begin{figure}
  \centering
  \begin{tabular}[c]{|c|c|c|c|c|c|c|}
\hline
Cycles & CP & instruction & r0& 10& 11& 12\\ \hline
INIT & 1 & & ? & 4
 & 5
 & ?
 \\ \hline1 & 2 & \commentaire{Lecture de la donnée d'adresse 10 dans le registre 0
} lecture 10 r0
 & 4 & & & \\ \hline
2 & 3 & \commentaire{Écriture du registre 0 à l'adresse 12
} ecriture r0 12
 & & & & 4
 \\ \hline
3 & 4 & \commentaire{Lecture de la donnée d'adresse 11 dans le registre 0
} lecture 11 r0
 & 5 & & & \\ \hline
4 & 5 & \commentaire{Écriture du registre 0 à l'adresse 10
} ecriture r0 10
 & & 5
 & & \\ \hline
5 & 6 & \commentaire{Lecture de la donnée d'adresse 12 dans le registre 0
} lecture 12 r0
 & 4 & & & \\ \hline
6 & 7 & \commentaire{Écriture du registre 0 à l'adresse 11
} ecriture r0 11
 & & & 4
 & \\ \hline
7 & 8 & \commentaire{Fin du processus.
} stop
 & & & & \\ \hline
\end{tabular}
  \caption{Simulation de l'échange de deux valeurs en mémoire (4 et 5)}
  \label{simsw1}
\end{figure}
  \end{correction}
\item (Facultatif). Donner d'autres solutions en assembleur à ce problème (la permutation des contenus des adresses \C{10} et \C{11}).
  \begin{correction}
Par exemple :
\begin{verbatim}
# copie de la case d'adresse 10 dans un registre pour sauvegarde
1    lecture 10 r1
# copie de la case d'adresse 11 à l'adresse 10
2    lecture 11 r0
3    ecriture r0 10
# copie de a dans registre de sauvegarde à l'adresse 11
4    ecriture r1 11
5    stop
\end{verbatim}
\begin{figure}
  \centering
  \begin{tabular}[c]{|c|c|c|c|c|c|c|}
\hline
Cycles & CP & instruction & r0& r1& 10& 11\\ \hline
INIT & 1 & & ? & ? & 4
 & 5
 \\ \hline1 & 2 & \commentaire{Lecture de la donnée d'adresse 10 dans le registre 1
} lecture 10 r1
 & & 4 & & \\ \hline
2 & 3 & \commentaire{Lecture de la donnée d'adresse 11 dans le registre 0
} lecture 11 r0
 & 5 & & & \\ \hline
3 & 4 & \commentaire{Écriture du registre 0 à l'adresse 10
} ecriture r0 10
 & & & 5
 & \\ \hline
4 & 5 & \commentaire{Écriture du registre 1 à l'adresse 11
} ecriture r1 11
 & & & & 4
 \\ \hline
5 & 6 & \commentaire{Fin du processus.
} stop
 & & & & \\ \hline
\end{tabular}
  \caption{Simulation de l'échange de deux valeurs en mémoire (4 et 5), avec un second registre}
  \label{simsw2}
\end{figure}
  \end{correction}

\item (Facultatif). Mêmes questions que précédemment mais pour faire une permutation
  circulaire de 3 valeurs.

  \begin{correction}


{\small
\listinginput{1}{echange3.c}
}

Sans le second registre (traduction) :
\begin{verbatim}
lecture 10 r0
ecriture r0 13
lecture 11 r0
ecriture r0 10
lecture 12 r0
ecriture r0 11
lecture 13 r0
ecriture r0 12
stop
\end{verbatim}
Avec le second registre :
\begin{verbatim}
lecture 10 r1
lecture 11 r0
ecriture r0 10
lecture 12 r0
ecriture r0 11
ecriture r1 12
stop
\end{verbatim}
\end{correction}
\end{enumerate}


\begin{correction}
  Note aux chargés de TD.
  \begin{itemize}
  \item En cours, les étudiants ont vu les variables impératives et
    le if. Leur sémantique a été donnée par leur traduction en langage
    amil.
  \item On applique la procédure de résolution :
    \begin{itemize}
    \item on se donne des exemples
    \item on trouve un algorithme en francais
    \item on traduit l'algorithme en C, en s'aidant de commentaires
    \item on teste sur les exemples
    \end{itemize}
  \item Le code de la fonction main vide en C a ete présenté rapidement, commenté avec les differents points qu'ils vont voir au semestre. Il est long et peut etre raccourci mais il faut s'assurer qu'ils sachent écrire ce genre de préambule du C avant d'écrire leurs programmes.
  \item Le code leur est toujours donné avec les commentaires, en suivant scrupuleusement l'indentation choisie. Ils doivent bien comprendre que le code et les commentaires sont indissociables. N'hesitez pas a ajouter des commentaires en fonction des difficultés rencontrées dans votre groupe.
  \end{itemize}
\end{correction}

%\subsection{Pour essayer en TP}

\section{Exécution conditionnelle d'instructions :  \emph{if}}
Les programmes suivants réalisent des affichages, pour cela nous
utiliserons la fonction \C{printf}, disponible après avoir inséré
\verb|#include <stdio.h>| en début de programme (ligne~3 dans le
programme figure~\ref{fig:programmes}).

Testez vos programmes sur
machine (avec l'aide de votre enseignant).

\subsection{Majeur ou mineur ?}

Soit la variable \verb|age|, contenant l'âge d'une personne. Écrire un
programme qui affiche si cette personne est majeure ou
mineure. Indication : \verb|printf("Vous êtes majeur !\n");| affiche
\C{Vous êtes majeur !} et un saut de ligne.

\begin{correction}
\begin{verbatim}
algorithme :
si age >= 18 alors
  affiche majeur
sinon
  affiche mineur
\end{verbatim}
\begin{listing}{1}
/* Declaration de fonctionnalites supplementaires */
#include <stdlib.h> /* EXIT_SUCCESS */
#include <stdio.h> /* printf */

/* Declaration des constantes et types utilisateurs */

/* Declaration des fonctions utilisateurs */

/* Fonction principale */
int main()
{
    /* Declaration et initialisation des variables */
    int age = 16; /* age de la personne */

    if(age >= 18) /* majeur */
    {
	/* affiche majeur */
	printf("Vous êtes majeur.\n");
    }
    else /* mineur (age < 18) */
    {
	/* affiche mineur */
	printf("Vous êtes mineur.\n");
    }

    /* valeur fonction */
    return EXIT_SUCCESS; /* renvoie OK */
}

/*Definitions des fonctions utilisateurs */
\end{listing}
\end{correction}

\subsection{Exercice type :  le minimum de 3 valeurs}

Soient 3 variables \verb|a|, \verb|b|, \verb|c|, initialisées à des
valeurs quelconques. Écrire un programme qui calcule et affiche à
l'écran le minimum des 3 valeurs.

Indication : si \C{x} est une variable contenant l'entier 42,
\verb|printf("Solution : %d\n", x);| affichera \C{Solution : 42} et
un saut de ligne.

\begin{correction}
\begin{verbatim}
algorithme :
soit min = a /* valeur par defaut */
si b < min /* b plus petit que min courant */
  /* b est le min courant */
  min = b
/* min contient min(a,b) */
si c < min /* c plus petit que min courant */
  /* c est le min courant */
  min = c
/* min contient min(min(a,b),c) = min(a,b,c) */
affiche min
\end{verbatim}
\begin{listing}{1}
\end{listing}
\end{correction}

\subsection{Exercice type : Dans une seconde, il sera exactement\ldots}

Écrire un programme qui, étant donnée une heure représentée
sous la forme de trois variables : une pour les heures, \verb|h|, une pour
les minutes,
\verb|m| et une pour les secondes, \verb|s|, affiche l'heure qu'il sera une
seconde plus tard. Il faudra envisager tous les
cas possibles pour le changement d'heure. Deux exemples de sortie sont~:

\begin{verbatim}
L'heure actuelle est : 23h12m12s
Dans une seconde, il sera exactement : 23h12m13s

L'heure actuelle est : 23h59m59s
Dans une seconde, il sera exactement : 00h00m00s
\end{verbatim}

Pour l'affichage :
\verb|printf("L'heure actuelle est : %dh%dm%ds\n", h, m, s);|.

\begin{correction}
Mieux (hors programme) :
  \verb|printf("L'heure actuelle est : %02dh%02dm%02ds\n", h, m, s);|.
\end{correction}

\begin{correction}
\begin{verbatim}
algo :
- affiche l'heure actuelle
- ajoute une seconde
- si tour du cadran des secondes alors
    - remise a 0 des secondes
    - il est une minute de plus
    - si tour du cadran des minutes alors
        - remise a 0 des minutes
        - il est une heure de plus
        - si tour du cadran des heures alors
    	    - remise a zero des heures
- affiche la nouvelle heure
\end{verbatim}
  \begin{listing}{1}
/* Declaration de fonctionnalites supplementaires */
#include <stdlib.h> /* EXIT_SUCCESS */
#include <stdio.h> /* printf */

/* Declaration des constantes et types utilisateurs */

/* Declaration des fonctions utilisateurs */

/* Fonction principale */
int main()
{
    /* soient 23h59m59s */
    int h = 23;
    int m = 59;
    int s = 59;

    /* affiche l'heure actuelle */
    printf("L'heure actuelle est : %dh%dm%ds\n",h,m,s);

    /* une seconde de plus */
    s = s + 1;

    if(s == 60) /* tour du cadran des secondes */
    {
	/* remise a 0 */
	s = 0;

	/* une minute de plus */
	m = m + 1;

	if(m == 60) /* tour du cadran des minutes */
	{
	    /* remise a 0 */
	    m = 0;

	    /* une heure de plus */
	    h = h + 1;

	    if(h == 24) /* tour du cadran des heures */
	    {
		/* remise a zero */
		h = 0;
	    }
	}
    }
    /* h,m,s contiennent l'heure une seconde plus tard */

    /* affiche l'heure */
    printf("Dans une seconde, il sera exactement : %dh%dm%ds\n",h,m,s);

    /* valeur fonction */
    return EXIT_SUCCESS;
}

/* Definitions des fonctions utilisateurs */
  \end{listing}
\end{correction}


%%% Local Variables:
%%% mode: latex
%%% TeX-master: "td1"
%%% End:
