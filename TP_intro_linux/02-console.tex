\documentclass[12pt,a4paper]{article}
\usepackage[utf8]{inputenc} 
\usepackage{latexsym,amssymb,amsmath}
%\usepackage{xltxtra} % charge aussi fontspec et xunicode,
                     % nécessaires...
% \usepackage[latin1]{inputenc}
\usepackage[T1]{fontenc}
\usepackage[francais]{babel}

\usepackage{listings} \lstset{language=C, showstringspaces=false}
\usepackage{graphicx}
\usepackage{pst-tree}

\topmargin -1cm 
\oddsidemargin -10mm 
%\evensidemargin 0mm 
\textheight 27cm 
\textwidth 18cm 
\columnsep 4.1mm

\parindent 1.0em 
\headsep 0mm 
\headheight 0pt 
\lineskip 0pt
\parskip 0.33em

\normallineskip 0pt 
\def\baselinestretch{1}

\sloppy \hbadness=10000

% arbres
\newcommand{\n}[2]{
  \pstree{\Tr{#1}}{#2}}
\newcommand{\f}[1]{\TR{#1}}
\newcommand{\x}[1]{\pstree[linestyle=none,arrows=-,levelsep=1ex]
  {\Tfan}{\f{#1}}}
\psset{treemode=R,levelsep=20ex,treesep=1em,nodesep=1em}


\begin{document}
\newcommand{\entete}[1]{
        \noindent
        \rule{\linewidth}{0.5mm}\\
        \noindent
        Universit\'e Paris-Nord \hfill CP2I 1\\
        \noindent
        Institut Galil\'ee \hfill Ann\'ee 2008-2009\\
        Eléments d'Informatique - $1^{\text{er}}$ semestre
        \begin{center} {\bf\Large {#1}}
        \end{center}
        \rule{\linewidth}{0.5mm}
}

\entete{Linux et la console}

\newcommand{\puis}{\ensuremath{\rightarrow\ }}


\paragraph{Objectif de TP :\\}

Familiarisation avec la console, l'interpréteur de commandes, les processus.

\section{Les commandes}

   On a vu comment manipuler (création, copie, déplacement) les
   fichiers et les répertoires, lancer des applications, ... avec le
   gestionnaire de fichiers Konqueror. Si vous êtes arrivés jusque là,
   vous avez pu constater que la souris seule ne suffit pas toujours,
   même pour se déplacer dans les répertoires.

   Sous Linux, le clavier a longtemps eu une place prépondérente, et
   de fait, toutes ces opérations peuvent
   aussi s'effectuer par l'intermédiaire de commandes qui sont
   interprétées au fur et à mesure par un \emph{interpréteur de commande} que
   l'on nomme aussi \emph{ligne de commande} ou \emph{shell}. Il existe
   plusieurs types de shells pour lesquels les commandes peuvent
   varier. Nous utilisons ici le shell BASH, qui devrait être votre
   shell par défaut.



   \paragraph{Définitions :\\}

   \begin{itemize}
   \item Une {\bf commande} est un programme exécutable.

   \item L'{\bf interpréteur de commandes} permet d'exécuter les
     différentes commandes tapées par l'utilisateur.
   \end{itemize}


\subsection{L'interpréteur de commandes}
   Lorsque vous lancez un shell, l'ordinateur affiche un prompt qui
   vous invite à taper une commande. Ici, le prompt est de la forme :
   ``login@machine:repertoire\_courant\$''. Pour les personnes qui
   connaissent, le Shell Linux est l'équivalent du mode Ms-Dos sous
   Windows. Évidemment, les possibilités offertes par le Shell BASH
   sont sans commune mesure avec celles du DOS.

\paragraph{Exercice :} lancement d'un interpréteur de commandes (Shell)

\begin{enumerate}
\item passer en mode console (touches CTRL+ALT+F1) ;
\item Tapez votre login et votre mot de passe à l'invite ;
\item C'est fait : vous êtes sur l'interpréteur de commandes
\item Pour terminer l'interpréteur de commandes, utilisez la
  combinaison de touches CTRL+D.
\end{enumerate}

Vous pouvez utiliser jusqu'à 6 consoles à la fois, en passant de
l'une à l'autre avec les touches CTRL+ALT+F1 à CTRL+ALT+F6.

Vous pouvez également ouvrir des consoles à l'intérieur de KDE. Pour
retourner à l'écran graphique de KDE~: CTRL+ALT+F7. Ensuite, trouvez
un programme nommé {\tt konsole}, ou encore {\tt xterm} dans le
menu~K, et lancez-le !




\subsection{Les commandes}

Les commandes ont la forme générique suivante :\\
\mbox{}\hspace{10em}{\bf commande [-option(s)] [paramètre(s)]}

Pour exécuter une commande, on tape le nom de la commande, suivie de
ses options (les crochets signalent que ces options sont facultatives)
et de ses paramètres (facultatifs). Le nombre d'options disponibles et
de paramètres possibles pour une commande dépend de la commande. Les
options commencent par le signe {\bf \tt -}, parfois doublé ({\bf \verb+--+}). 


Il existe deux types de commandes :
\begin{itemize}
\item des commandes internes (qui font partie du shell) : {\bf cd},
  {\bf fg}, {\bf echo}, {\bf kill}, etc.

\item des commandes externes (ce sont des programmes exécutables, qui
  se trouvent dans la hiérarchie des répertoires)~:
  {\bf ls}, {\bf rm}, etc.
\end{itemize}




\subsection{Deux compagnons indispensables~: man et internet}

\subsubsection{man (manual)}
Sous ces 3 lettres se cache une commande qui sera votre première source d'informations durant ces TPs. Cette commande permet d'afficher le manuel de la commande dont le nom est passé en paramètre.

Pour savoir comment fonctionne \verb+man+, affichez son manuel~:
\begin{lstlisting}[frame=single]
 	man man
\end{lstlisting}

Appuyez sur \verb+espace+ pour allez à la page suivante et sur la
touche \verb+q+ pour quitter le manuel et retourner sur le
terminal. Prenez l'habitude de regarder le manuel des commandes que
vous ne connaissez pas avant d'appeler le prof~! Cette commande est
très importante pour retrouver comment utiliser une commande qu'on a
pas utilisé depuis longtemps et/ou retrouver l'option de cette
commande qui nous intéresse. 

\subsubsection{Internet}

Prenez également l'habitude d'utiliser votre moteur de recherche
internet favori (\emph{firefox}~!)et \emph{wikipédia} pour trouver une
information. Lancez une recherche sur \og \textit{les commandes unix}
\fg{}, cherchez une page web qui explique briévement comment utiliser
les commandes unix et gardez cette page précieusement pour la suite. 

Pour lancer \emph{firefox} depuis la ligne de commande, tapez juste
{\tt firefox} suivi de la touche {\tt Entrée}.



\subsection{Quelques astuces}

\subsubsection{L'esperluette ou ``et commercial''}

Vous avez lancé firefox depuis la ligne de commande, vous
pouvez remarquer. Que se passe t'il dans la \textit{konsole}~? 
Essayez d'ouvrir un nouveau firefox à partir du terminal. Fermez le
premier que vous avez lancé, puis entrez~: \verb+firefox &+ dans
\textit{konsole}. L’esperluette 
(\verb+&+) permet de lancer une application en tâche de fond et de
laisser la main sur le terminal à l'utilisateur. Vous pouvez entrer de
nouveau \verb+firefox &+ dans le terminal pour ouvrir un deuxième
navigateur. 

\subsection{Complétion automatique}

Dans une console, la touche de tabulation est magique. Elle permet de
compléter automatiquement la commande ou le chemin que vous êtes en
train de taper au clavier. Elle vous fera gagner énormément de temps
et vous évitera de trop user vos doigts~! (habituez vous à
l'utiliser).

\subsection{Historique des commandes}

Dans un terminal, on peut voir les dernières commandes entrées en
appuyant sur la flèche haut (la flèche bas permet d'aller en sens
inverse dans l'historique). Cette astuce est très pratique pour ne pas
avoir à retaper une commande ou un chemin lorsqu'on a fait une faute
de frappe (ce qui ne devrait pas arriver très souvent si vous utilisez
la complétion automatique~!). Pour voir le contenu de l'historique,
utilisez la commande {\tt history}.




\subsection{Quelques commandes utiles}
\begin{itemize}
\item \textbf{man} \textit{nom\_commande} : affiche le manuel d'utilisation de la
  commande \textit{nom\_commande}.
\item \textbf{info} \textit{nom\_commande} : affiche une version plus complète du manuel
  d'utilisation.
\item \textbf{which} \textit{ nom\_de\_commande} : affiche le répertoire dans lequel se
  trouve l'application ou la commande \textit{ nom\_commande}.
\item \textbf{pwd} : affiche le répertoire courant.
\item \textbf{ls} : affiche la liste des fichiers et répertoires contenus dans
  le répertoire courant. Cette commande possède des options et accepte
  des paramètres.  
  Ex:
  \begin{itemize}
  \item \texttt{ls -l} : liste détaillée des fichiers

  \item \texttt{ls -a} : affiche les fichiers « cachés » (ceux dont le
    nom commence par un point)
  \item \texttt{ls -l a*} : affichage détaillé de la liste
    des fichiers commençant par la lettre {\tt a}
  \end{itemize}
\end{itemize}

Exercice :
Avec la ligne de commande faire les opérations suivantes :
\begin{itemize}
\item trouver les options possibles pour la commande
  \texttt{date} ;
\item trouver le répertoire courant (dans lequel on se trouve) ;
\item trouver l'emplacement des commandes \textit{which},
  \textit{man}, \textit{ls}, et \textit{date} ;
\item afficher la liste détaillée des fichiers ;
\item en utilisant le manuel : afficher la liste détaillée des
  fichiers et répertoires (y compris les fichiers cachés) triée en
  fonction de leur date de création (du plus ancien au plus récent).
\end{itemize}

\subsection{Commandes sur les fichiers et les répertoires}

\subsubsection{chemin relatif vs. chemin absolu}
On utilise la notation suivante :
\begin{itemize}
\item ``{\bf /}'' désigne le répertoire racine (de plus haut niveau) ;
\item ``{\bf .}'' désigne le répertoire courant ;
\item ``{\bf ..}'' désigne le répertoire père.
\end{itemize}

Le chemin d'accès à un fichier est la liste des répertoires à
traverser pour atteindre ledit fichier. Si le point de départ est la
racine (/), on parle de chemin absolu, si le point de départ est le
répertoire courant, on parle de chemin relatif. Par exemple, vous êtes
dans votre répertoire personnel, le chemin absolu du répertoire tp1
est {\tt
  /export/home/users/licence/licence1/cp2i12009/TMP/votre\_login/info/tp1},
alors que son chemin relatif est {\tt info/tp1}.






\subsubsection{Commandes sur les répertoires}
{\it nom\_rep} désigne un nom de répertoire :

\begin{itemize}
\item {\bf pwd} : affiche le répertoire courant ;
\item {\bf ls} : affiche la liste des fichiers et des répertoires
  contenus dans le répertoire courant ;
\item {\bf cd} {\it nom\_rep} : fait de nom\_rep le répertoire courant ;
\item {\bf mkdir} {\it nom\_rep} : crée le répertoire nom\_rep dans le
  répertoire courant ;
\item {\bf rmdir} {\it nom\_rep} : détruit nom\_rep à condition qu'il soit
  vide.  

\subsubsection{Commandes sur les fichiers}
{\it f1} et {\it f2} désignent des chemins d'accès à des fichiers.
\begin{itemize}
\item {\bf cp} {\it f1 f2} : copie f1 en f2. Si f2 est un répertoire, copie
  f1 dans f2 en gardant le même nom ;
\item {\bf rm} {\it f1} : détruit f1 ;
\item {\bf mv} {\it f1 f2} : déplace f1 en f2 ;
\item {\bf cat} {\it f1} : affiche le contenu du fichier f1 (en une seule fois) ;
\item {\bf less} {\it f1} : affiche le contenu du fichier f1 (en plusieurs fois -
  q pour quitter).
\end{itemize}
\end{itemize}

\paragraph{Exercice}
\begin{itemize}
\item vérifier que les répertoires créés avec konqueror
  existent. Si le fichier {\tt blah2.txt} a
  disparu, le créer dans le répertoire {\tt tp1} ;
\item se placer dans le répertoire tp2 ;
\item créer les sous répertoires {\tt tp21} et {\tt tp22} ;
\item copier le fichier {\tt blah2.txt} dans le répertoire tp21 ;
\item se placer dans tp21 et visualiser le contenu du fichier
  blah2.txt ;
\item déplacer ce fichier dans tp22 ;
\item effacer tp21 et tp22.
\end{itemize}

\subsubsection{Permissions sur les fichiers et les répertoires}
    Comme nous l'avons vu, l'utilisateur n'a qu'un droit limité sur
    les fichiers et les répertoires du système (il n'a le droit
    d'écriture que sur son répertoire personnel). La commande {\tt ls -l}
    (liste détaillée des fichiers et des répertoires) permet de
    visualiser (entre autre) les droits d'accès sur les fichiers et
    les répertoires. 

\begin{verbatim}
Ex : ls -l
       drwxrw-rw-             ... TP1
       drwxrw-rw-             .... TP2
       -rw-r--r--             .... police2.txt
\end{verbatim}

Détaillons la première séquence de caractères en partant de la gauche :
\begin{itemize}

\item le premier champ vous indique le type de fichier :
  \begin{itemize}
  \item - c'est un fichier normal,
  \item d c'est un répertoire,
  \item l c'est un lien symbolique
  \end{itemize}

\item les trois champs suivants correspondent aux droits du
  propriétaire (user u) :
  \begin{itemize}
  \item r droit en lecture (read), - pas de droit en lecture,
  \item w droit en écriture (write), - pas de droit en écriture,
  \item x droit en exécution si c'est un fichier, droit d'accès si
    c'est un répertoire.
  \end{itemize}

\item les trois champs suivants correspondent aux droits du groupe. On
  retrouve r,w,x

\item les trois derniers correspondent aux droits des autres (others),
  c'est-à-dire aux personnes qui ne sont pas le propriétaire et qui
  n'appartiennent pas au groupe.  
\end{itemize}

La commande {\bf chmod} permet de modifier
les droits d'accès d'un fichier (ou répertoire). Pour pouvoir
l'utiliser sur un fichier ou un répertoire, il faut en être le
propriétaire.  La syntaxe est la suivante : 


\begin{tabular}[c]{rllll}
{\bf chmod} & {\bf utilisateur} & {\bf opération} & {\bf droit} & {\bf
  fichier}\\
& {\bf \emph{u}} proriétaire & {\bf +} autorisation & {\bf r} lecture
& \\
& {\bf \emph{g}} groupe & {\bf -} interdiction & {\bf w} modification
& \\
& {\bf \emph{o}} les autres &  & {\bf x} exécution/traversée
& \\
\end{tabular}

Pour un fichier contenant un programme, le droit {\bf x} permet
de lancer le programme. Pour un répertoire, c'est le droit de
traverser ce répertoire, ou de se déplacer dedans. Vous pouvez vous
déplacer dans le répertoire {\tt
  /export/home/users/Enseignants/rozenknop} car vous avez le droit
{\bf x} dessus, mais vous n'avez pas le droit de lecture {\bf
  r}. C'est pourquoi {\tt Konqueror} refuse de de s'y rendre, et que
vous avez dû taper une partie du chemin au clavier pour télécharger ce
sujet. 


Exemples :
\begin{itemize}
\item Pour donner le droit en écriture pour le groupe sur le
  fichier mon-fichier, il faut taper la commande \texttt{chmod g+w mon-fichier}
\item Pour supprimer le droit d'accès du répertoire mon-répertoire aux
  autres utilisateurs (autres que propriétaire et utilisateurs du
  groupe), il faut taper : \texttt{chmod o-x mon-repertoire}
\item En tapant : \texttt{chmod u+x,g-w mon-fichier}, vous ajoutez le droit en
  exécution pour le propriétaire, et enlevez le droit en écriture pour
  le groupe du fichier.
\end{itemize}


\paragraph{}Vous avez une autre méthode pour vous servir de la commande chmod. On
considère que r=4, w=2 et x=1, si vous avez un fichier avec les droits
suivants : \texttt{-rw-rw-rw-}, pour les droits utilisateurs vous avez
\textit{(r=)4+(w=)2=6}, de même pour le groupe et les autres. Donc \texttt{-rw-rw-rw-}
est équivalent à 666. Suivant la même règle \texttt{rwxrw-r--} est équivalent à
764. Pour mettre un fichier avec les droits \texttt{-r--r--r--} vous pouvez
taper~: \texttt{chmod 444 mon-fichier}





Exercice :
\begin{itemize}
\item Placez-vous dans le votre répertoire personnel ;
\item Enlevez le droit en lecture pour le fichier \texttt{blah2.txt} (pour
  vous aussi);
\item Affichez les droits sur ce fichier pour vérifier que l'opération
  a fonctionné ;
\item Essayez de visualiser le contenu du fichier ;
\item Remettez les droits originaux ;
\item Interdisez le droit en accès pour le répertoire \texttt{tp1} (pour vous
  aussi) ;
\item Essayez de vous placer dedans ;
\item Remettez les droits originaux.
\end{itemize}

\section{Les processus}
A chaque application ou commande exécutée sous Linux correspond un
processus et le système attribue automatiquement un numéro à chaque
processus (appelé le \textit{PID}). Le système Linux donne la possibilité de
gérer de manière avancée ces processus, par exemple en arrêtant des
processus bloqués, en changeant la priorité d'exécution de ces
processus, etc. 

Nous ne donnons qu'un aperçu des nombreuses possibilités offertes par le système.

\subsection{Tâche de fond vs premier plan}
Lorsque vous exécutez une commande depuis le Shell, celui-ci se bloque
(i.e. vous ne pouvez plus rien faire), jusqu'à ce que la commande se
termine. Vous avez alors lancé la commande au premier plan.

Il est possible d'exécuter une commande sans bloquer le Shell,
c'est-à-dire que vous reprenez la main avant que la commande ait fini
de s'exécuter. On dit alors que la commande s'exécute en tâche de
fond. Pour exécuter une commande en tâche de fond, on ajoute \& après
la commande.

Remarquez que si vous fermez la console depuis laquelle vous avez
lancé \texttt{firefox} en tâche de fond, \texttt{firefox} se termine
en même temps. Pour éviter cela, c'est-à-dire pour qu'une application
continue en tâche de fond même après la fermeture du shell qui l'a
lancée, il faut faire précéder la commande par \texttt{nohup}~:
essayez par exemple de taper \texttt{nohup firefox \&} dans une
console, puis de fermer la console !

\subsection{Commandes sur les processus}
\begin{itemize}
\item \textbf{ps} affiche la liste des processus lancés depuis le
  \textit{shell}. Utiliser \textbf{-ux} en option pour avoir une liste
  complète de tous les processus tournant sur la machine. 
\item \textbf{kill} \textit{PID} : tue le processus de numéro
  \textit{PID}. Utiliser \textbf{-9} en option pour un meurtre sauvage et
  définitif (sinon, ce serait plutôt une requête du genre <<~dis, tu
  veux bien t'arrêter, stp ? Quand tu pourra...~>>)
\item \textbf{bg} : Met en tâche de fond un processus qui vient d'être
  arrêté par un CTRL+Z.
\item \textbf{fg} : remet un processus interrompu par CTRL+Z au
  premier plan.
\end{itemize}

Exercice :
\begin{itemize}
\item lancer \textbf{kwrite} par le shell ;
\item constater que l'on ne peut plus rien faire (Shell bloqué...) ;
\item ouvrir une autre console ;
\item lister les processus actifs ;
\item repérer le numéro de processus correspondant à la commande
  \texttt{kwrite} ;
\item tuer ce processus (pas trop sauvagement au début) ;
\item revenir à la première console (vous devez avoir retrouvé la
  main) ;
\item relancer \textbf{kwrite} par le shell ;
\item cliquer sur la console, puis appuyez sur CTRL+Z
\item essayez d'utiliser ls \textit{kwrite} que vous venez de lancer
\item revenez à la console et tapez \textbf{bg}, et réessayez
  \textit{kwrite} ;
\item lister les processus depuis la console ;
\end{itemize}




Cette méthode de gestion des processus (numéro associé à un processus
et meurtre de processus bloqués) vous permet de récupérer votre
système chaque fois que celui-ci se bloque (enfin, presque...). Vous
aurez probablement l'occasion d'expérimenter ceci au cours des
prochains travaux pratiques.






\section{L'enchaînement des commandes : les tubes}
\subsection{Les sorties}

Sous UNIX, la sortie d'une commande est partagée en deux canaux
différents : la \textbf{sortie standard} (=\textit{stdout}) qui affiche les
résultats de la commande, et la \textbf{sortie erreur}
(=\textit{stderr}), qui affiche 
les messages d'erreur de la commande. La raison pour cette séparation
est le système d'enchaînement des commandes, qui permet de
récupérer la sortie d'une commande et de l'utiliser comme entrée
pour une autre commande. 

Ceci se fait avec des structures qui
s'appellent "\textbf{tubes}" (anglais : \textit{pipes}), écrites avec
"|". Nous allons y revenir plus tard. 

Tout d'abord, nous allons faire quelques essais
pour identifier les deux différents canaux de sortie. La commande
"\textbf{echo}" nous sert pour créer des sorties : \\
\texttt{echo "un text"} \\
Cette
commande écrit le texte donné comme argument et l'envoie vers la
sortie standard. Comme, par défaut, tout les canaux de sortie sous
UNIX sont dirigés vers l'écran (c.à.d. le terminal ou shell ou
console), le texte
est affiché dans le terminal. Avec le caractère \textbf{>} ont peut diriger la
sortie d'une commande vers un fichier : \\
\begin{verbatim}
echo "un text" > nomfichier 
ls > nomfichier
\end{verbatim}
Le contenu du fichier peut ensuite être affiché
avec un éditeur~:
\begin{verbatim}
kwrite nomfichier &
\end{verbatim}
Une autre façon de vérifier le contenu
d'un fichier texte est l'utilisation de la commande "\texttt{cat}"
pour l'afficher : 
\begin{verbatim}
cat nomfichier
\end{verbatim}
Bien sûr on peut
également diriger la sortie de la commande "\texttt{cat}" vers un autre
fichier. Cette opération donne le même résultat que l'utilisation
de la commande "cp" pour faire une copie du fichier : 
\begin{verbatim}
cat fichiersource > fichierdestination
\end{verbatim}
fait la même chose que
\begin{verbatim}
cp fichiersource fichierdestination
\end{verbatim}

La séparation des sorties en deux
canaux devient claire dans le cas ou une erreur se produit. Comme
les messages d'erreur sont envoyés vers la sortie erreur, qui n'était
pas dirigée vers un fichier, le message d'erreur est écrit sur
l'écran : 
\begin{verbatim}
cat nomfichierquinexistepas > fichiersortie
\end{verbatim}
Si le fichier à afficher n'existe pas, le fichier "fichiersortie" est vide
(l'opération a échoué), et le message d'erreur ("\texttt{cat:
  nomfichierquinexistepas: Aucun fichier ou répertoire de ce type}")
est visible sur l'écran. Par contre, nous pouvons également
rediriger les messages d'erreur, donc la sortie erreur, vers un
fichier de la façon suivante en faisant précéder le signe \textbf{>}
par un \textbf{2}: 
\begin{verbatim}
cat nomfichierquinexistepas 2> fichiererreur > fichiersortie
\end{verbatim}
Dans ce cas, le fichier "fichiersortie" est vide, le fichier
"fichiererreur" contient le message d'erreur, et
aucun message n'est écrit sur l'écran.

Notez enfin que l'on utilise ``\textbf{$>>$}'' au lieu de ``>'' si l'on
ne veut pas effacer le fichier de sortie, mais seulement
\textit{ajouter} le texte à la fin.


\subsection{Les tubes}

Dans la suite nous allons nous servir de la commande "wc" (=word
count) pour compter le nombre de lignes dans un fichier : 
\begin{verbatim}
wc nomfichier
\end{verbatim}
Lancé sans argument, "\texttt{wc}" compte le nombre de lignes, de mots
et de caractères dans l'entrée standard, qui est par défaut le
clavier~: lancez \texttt{wc}, tapez
quelques lignes, finissez en suite par la combinaison
\texttt{CONTROL+D}. 

A quoi sert-il de compter les nombre
de lignes tapées dans un terminal~? La commande "wc", lancée sans
argument, compte les lignes de son entrée standard, qui n'est pas
toujours le clavier. On peut appliquer une commande à la sortie d'une
autre commande de la façon suivante~: 
\begin{verbatim}
commande1 | commande2
\end{verbatim}
Ainsi,
pour compter les fichiers dans le répertoire courant on peut connecter les
commandes "\texttt{ls}" et "\texttt{wc}" : 
\begin{verbatim}
ls | wc
\end{verbatim}
La commande "\texttt{ls}" crée une liste de
fichiers, qui est envoyé vers la sortie standard. Cette sortie est
dirigée cette fois vers l'entrée standard de la commande "wc", qui
compte le nombre de 
ligne dans cette liste, donc le nombre de fichiers, et l'envoie vers
sa sortie standard. Comme la sortie standard de la commande "\texttt{wc}" n'est
pas redirigée, le résultat du comptage est affiché à l'écran.

Voici une liste de commandes utiles :
\begin{itemize}
\item \textbf{ps} affiche le nombre de processus démarré dans le
  terminal.
\item \textbf{ps aux} affiche tous les processus qui tournent sur la machine
  (c.à.d les processus de tous les utilisateurs, lancés depuis tous
  les terminaux).
\item \textbf{sort} ordonne les données reçues sur son entrée standard et les
  envoie vers la sortie standard.
\item \textbf{grep} filtre les données reçues sur l'entrée par ligne (voir
  l'aide).
\item \textbf{uniq} enlève des lignes adjacentes en double (voir l'aide)
\item \textbf{cut} filtre les données reçues sur l'entrée par colonne (voir
  l'aide).
\item \textbf{head} n'affiche que les premières n lignes de l'entrée.
\item \textbf{tail} voir "\textit{head}", n'affiche que les derniers lignes.
\end{itemize}

Essayez les commandes suivantes pour voir
leur effet (Utilisez "\texttt{man commande}" pour accéder à l'aide) :

\begin{verbatim}
ps 
ps aux 
ps aux | sort +1 -2 
ps aux | grep root
cat /etc/passwd 
cat /etc/passwd | head -1 
cat /etc/passwd | cut -d: -f6 
cat /etc/passwd | cut -c1-3
\end{verbatim}

\paragraph{Exercices} :
\begin{itemize}

\item Ecrivez une ligne de commandes qui affiche le nom du
  processus qui prend le plus de mémoire. 

\item Ecrivez une ligne de commandes
  qui affiche le nombre de personnes qui sont propriétaires de

  processus tournant sur la machine.
\end{itemize}

\end{document}
