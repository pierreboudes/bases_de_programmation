% -*- coding: utf-8 -*-
\newcounter{questioncount}
\setcounter{questioncount}{0}
\newcommand{\question}{\addtocounter{questioncount}{1}\paragraph{Question \Alph{questioncount}.}}
\newcommand{\commentaire}[1]{}
\newcommand{\pt}[1]{\fbox{$#1 \operatorname{pt}$}}
\newcommand{\bareme}[1]{}
%\excludeversion{correction}
%\excludeversion{baremeenv}


\entete{Travaux dirigés 7}

\section{For ou while ?}

Il est demandé de résoudre les questions suivantes sans définir de
fonctions utilisateurs, directement dans le \C{main}, et en faisant le meilleur choix entre \C{for}
et \C{while}.

\subsection{Puissances successives d'un entier}

\question Écrire un programme qui demande un entier $n$ à l'utilisateur et
calcule puis affiche $n^{10}$ ($n$ puissance dix). \bareme{2}
\paragraph{Exemple d'exécution.} 
\begin{small}
\begin{verbatim}
Donner un entier : 2
2 exposant 10 = 1024
\end{verbatim}
\end{small}


\begin{correction}
Voir figure~\ref{fig:puissance}.
  \begin{figure}[htbp]
    \begin{small}
      \begin{verbatim}
#include <stdlib.h> /* EXIT_SUCCESS */
#include <stdio.h> /* printf, scanf */
/* declarations de constantes et types utilisateurs */

/* declarations de fonctions utilisateurs */

/* fonction principale */
int main()
{
    int n = 0; /* base */
    int produit = 1;
    int i; /* var de boucle */

    /* saisie */
    printf("Donner un entier : ");
    scanf("%d", &n);

    /* calcul */
    for (i = 0; i < 10; i = i + 1) /* faire 10 fois 
                                     la multiplication par n */
    {
        produit = produit * n;
    }

    /* affichage */
    printf("%d exposant %d = %d\n", n, 10, produit);

    return EXIT_SUCCESS;
}
\end{verbatim}
    \end{small}
    \caption{Puissance dix}
    \label{fig:puissance}
  \end{figure}
  \begin{baremeenv}
    \begin{itemize}
    \item \pt{0.5} pour la saisie meme si oublie du \&
    \item \pt{1.5} pour le reste, zéro si pas de boucle sinon :
      \begin{itemize}
      \item enlever \pt{0.5} si c'est un while
      \item enlever \pt{0.5} par problème de déclaration, d'initialisation,
        ou d'include manquant. 
\item enlever \pt{0.5} si erreur sur le nombre d'étapes de boucle.
      \item on ne demande pas un affichage élaboré.
      \end{itemize}
    \end{itemize}
  \end{baremeenv}
\end{correction}

\question Comment modifier simplement votre programme pour qu'il
affiche le calcul des puissances successives de l'entier $n$ saisi,
de $n^0$ jusqu'à $n^{10}$ ?\bareme{1}


\paragraph{Exemple d'exécution.} 
\begin{small}
\begin{verbatim}
Donner un entier : 2
2 exposant 0 = 1
2 exposant 1 = 2
2 exposant 2 = 4
...
2 exposant 10 = 1024
\end{verbatim}
\end{small}
\begin{correction}
  Il suffit de placer l'affichage dans la boucle, au début du bloc.
\begin{small}
\begin{verbatim}
    /* calcul et affichage */
    for (i = 0; i < 10; i = i + 1) /* faire 10 fois 
                                     la multiplication par n */
    {
        /* affichage */
        printf("%d exposant %d = %d\n", n, i, produit);
        produit = produit * n;
    }
    /* affichage */
    printf("%d exposant %d = %d\n", n, 10, produit);
\end{verbatim}
\end{small}
  \begin{baremeenv}
    Compter \pt{0} pour toute autre solution. Pénalité de \pt{-0.5} si
    n'affiche pas l'exposant ou pas la bonne valeur
    d'exposant. Ne pas enlever de point si l'affichage ne va que
    jusqu'à l'exposant 9.
  \end{baremeenv}
\end{correction}


\question Comment modifier simplement votre programme pour qu'il
affiche le calcul des puissances successives de l'entier $n$
uniquement à partir de l'exposant $5$ ?\bareme{1}

\paragraph{Exemple d'exécution.} 
\begin{small}
\begin{verbatim}
Donner un entier : 2
2 exposant 5 = 32
2 exposant 6 = 64
...
2 exposant 10 = 1024
\end{verbatim}
\end{small}

\begin{correction}
il suffit d'encapsuler l'affichage de l'intérieur de boucle dans un
test :
\begin{small}
\begin{verbatim}
    /* calcul et affichage */
    for (i = 0; i < 10; i = i + 1) /* faire 10 fois 
                                     la multiplication par n */
    {
        /* affichage */
        if (i > 4) 
        {
            printf("%d exposant %d = %d\n", n, i, produit);
        }
        produit = produit * n;
    }
    /* affichage */
    printf("%d exposant %d = %d\n", n, 10, produit);
\end{verbatim}
\end{small}
  \begin{baremeenv}
    Autre solution admise : deux boucles (i allant de 0 à 4 puis i
    allant de 5 à 9)  à la suite l'une de l'autre. Sinon zéro.
 \end{baremeenv}
\end{correction}

\subsection{Unicité des éléments d'un tableau}

Nous disposons d'un tableau $t$ de $N$ entiers. Utiliser une constante
symbolique pour $N$. Nous souhaitons
savoir si chaque entier apparaissant dans le tableau n'y apparaît
qu'une seule fois, autrement dit on veut savoir si chaque entier
est unique dans le tableau.

\question Écrire un programme qui, étant donné un tableau initalisé $t$,
  teste si le premier élément du tableau est unique et affiche
  \C{Vrai} si c'est le cas, \C{Faux} sinon.\bareme{2}
  \begin{correction} 
Voir figure~\ref{fig:unicite1}.
    \begin{figure}[htbp]
      \begin{small}
        \begin{listing}{1}
#include <stdlib.h> /* EXIT_SUCCESS */
#include <stdio.h> /* printf, scanf */

/* déclaration constantes et types utilisateurs */
#define N 3
#define TRUE 1
#define FALSE 0

/* Fonction principale */
int main()
{
  int t[N] = {10,10,15};
  int unique = TRUE;
  int i; /* var de boucle */

  i = 1;
  while (unique && (i < N)) /* tant que t[0] est unique */
  {
    if (t[i] == t[0]) /* t[0] n'est pas unique */
    {
      unique = FALSE;
    }
    i = i + 1; /* case suivante */
  }

  /* affichage du résultat */
  if (unique)
  {
    printf("Vrai\n");
  }
  else
  {
    printf("Faux\n");
  }
  /* valeur fonction */
  return EXIT_SUCCESS;
}
\end{listing}
      \end{small}
      \caption{Unicité des éléments d'un tableau}
      \label{fig:unicite1}
    \end{figure}

    \begin{baremeenv}
    \pt{2} si arrive au résultat avec une seule boucle (for ou
    while). L'affichage doit être correct mais on accepte les
    variantes : ''machin n'est
    pas unique / "machin est unique'' etc.

    Sinon maximum \pt{1,5} : \pt{0,5} s'il y a une seule boucle; \pt{0,5}
    de plus si c'est un while; ajouter \pt{0,5} si utilise une
    variable booléenne (déclarée et initialisée, et les define TRUE
    et FALSE).      
    \end{baremeenv}
  \end{correction}


\question Écrire un programme qui étant donné un tableau initialisé $t$, teste
  si tous les éléments sont uniques et affiche \C{Vrai} si c'est le
  cas, \C{Faux} sinon.\bareme{2}

  \begin{correction} 
Voir figure~\ref{fig:unicite2}.

    \begin{figure}[htbp]
      \begin{small}
        \begin{listing}{1}
#include <stdlib.h> /* EXIT_SUCCESS */
#include <stdio.h> /* printf, scanf */

/* déclaration constantes et types utilisateurs */
#define N 4
#define TRUE 1
#define FALSE 0

/* Fonction principale */
int main()
{
  int t[N] = {10,12,15,12};
  int unique = TRUE;
  int i; /* var de boucle */
  int j; /* var de boucle */

  j = 0; /* premiere case */
  while (unique && (j < N)) /* pour chaque case, verifier que la case
  est unique */
  { 
    i = j + 1; /* case suivant t[j] */
    while (unique && (i < N)) /* tant que t[j] est unique */
    {
      if (t[i] == t[j]) /* t[j] n'est pas unique */
      {
        unique = FALSE;
      }
      i = i + 1; /* case suivante */
    }
    j = j + 1; /* case suivante */
  }
  /* affichage du résultat */
  if (unique)
  {
    printf("Vrai\n");
  }
  else
  {
    printf("Faux\n");
  }
  /* valeur fonction */
  return EXIT_SUCCESS;
}
\end{listing}
      \end{small}
      \caption{Unicité de tous les éléments du tableau}
\label{fig:unicite2}
    \end{figure}

   \begin{baremeenv}
   \pt{2} si arrive au résultat avec deux boucles (avec affichage mais
   variantes acceptée). 

    Sinon maximum \pt{1,5} : \pt{0,5} s'il y a deux boucles avec
   variables de boucles différentes (quelles
   que soient les boucles et les bornes); \pt{0,5} si elles sont
   toutes deux des while; \pt{0,5} s'il a un algorithme correct en
   français en commentaire ou dans le texte; \pt{0,5} si utilise une
   variable boolénnee ou deux variables booléennes. 
   \end{baremeenv}
  \end{correction}

\section{Fractions}

\question Une fraction $\frac{p}{q}$ est définie par deux entiers $p$ et $q$. Le nombre
$q$ appelé dénominateur est nécessairement non nul et sera toujours positif, le
nombre $p$, appelé numérateur, peut être négatif ou nul. Définir un
type utilisateur pour les fractions (sans tenir compte des questions
de signe qui n'ont d'importance que pour l'affichage).
\bareme{1}
\begin{correction}
\begin{verbatim}
struct fraction_s
{
  int num; /* nunérateur */
  int den; /* dénominateur */
}; /* <- n'oublie pas le point-virgule ! */
\end{verbatim}
 \begin{baremeenv}
  Soit tout juste (noms arbitraires) soit faux.
    \begin{description}
    \item[1 pt] même si oublie du point-virgule final, ou si typedef
    \item[zéro] sinon
   \end{description}
  \end{baremeenv}
\end{correction}

\question Déclarer et définir une fonction \C{multiplier\_fractions} qui
    prend deux fractions en argument et renvoie la fraction obtenue par multiplication des deux
    fractions (ne pas chercher à simplifier la fraction
    obtenue). \emph{Répondre en faisant bien apparaître d'une part la
      déclaration, d'autre part la définition.}
\bareme{1.5}
\begin{correction}
Déclaration :
\begin{verbatim}
struct fraction_s multiplier_fractions(struct fraction_s x, 
                                       struct fraction_s y);
\end{verbatim}
Définition :
\begin{verbatim}
struct fraction_s multiplier_fractions(struct fraction_s x, 
                                       struct fraction_s y)
{
  struct fraction_s res;
  res.num = x.num * y.num; /* produit des numérateurs "sur" */
  res.den = x.den * y.den; /* produit des dénominateurs */
  return res;
}
\end{verbatim}

  \begin{baremeenv}
    Maximum \pt{1,5}. Sinon :
    \begin{description}
    \item[0,5] pour la déclaration exacte.
    \item[1] pour la définition.
    \item[0,5] si la définition retourne effectivement un point mais
      qui n'a pas la bonne valeur.
    \end{description}
  \end{baremeenv}
\end{correction}

\question Même question pour la somme de deux fractions
    \C{additionner\_fractions}.
\bareme{1}

\begin{correction}
Déclaration :
\begin{verbatim}
struct fraction_s additionner_fractions(struct fraction_s x, 
                                        struct fraction_s y);
\end{verbatim}
Définition :
\begin{verbatim}
struct fraction_s additionner_fractions(struct fraction_s x, 
                                        struct fraction_s y)
{
  struct fraction_s res;
  res.num = x.num * y.den + y.num * x.den;
  res.den = x.den * y.den; /* produit des dénominateurs */
  return res;
}
\end{verbatim}

\begin{baremeenv}
Maximum \pt{1}.
  \begin{description}
  \item[0,25] pour la déclaration exacte.
  \item[0,75] pour la définition, dont \pt[0,5] pour un dénominateur exact.
\end{description}
\end{baremeenv}  
\end{correction}


\question Déclarer et définir une procédure affichant une fraction
passée en argument exactement comme dans l'exemple suivant où les
fractions $\frac{34}{26}$, $\frac{-34}{26}$, $\frac{34}{1}$,
$\frac{0}{1}$ sont affichées tour à tour :
\begin{verbatim}
34/26
-34/26
34 
0
\end{verbatim}
Attention à bien respecter les deux derniers affichages.
\bareme{1.5}

\begin{correction}
Déclaration :
\begin{verbatim}
void afficher_fraction(struct fraction_s x);
\end{verbatim}
Définition :
\begin{verbatim}
void afficher_fraction(struct fraction_s x);
{
  if (x.den == 1)
  {
    printf("%d\n", x.num);
  }
  else
  {
    printf("%d/%d\n", x.num, x.den);
  }
}
\end{verbatim}
\begin{baremeenv}
Maximum \pt{1,5}.
  \begin{description}
  \item[0,5] pou la déclaration exacte.
  \item[1] pour la définition, dont \pt{0,5} pour la simple présence d'un if.
 \end{description}
\end{baremeenv}  
\end{correction}

\question Comment feriez-vous pour simplifier automatiquement les
fractions ? Proposer une fonction \verb|normaliser_fraction| qui
pourra faire appel à des fonctions arithmétiques usuelles. 
\section{Fonctions}
\question Déclarer et définir :
\begin{enumerate}
\item une fonction \verb|minimum| qui prend en entrée deux nombres à virgule
  et retourne leur minimum;\bareme{1}
\item Une procédure \verb|afficher_ligne| qui prend en entrée un
  entier \C{n} et un caractère \C{c} et affiche une ligne contenant
  \C{n} fois la caractère \C{c};\bareme{1}
\item Une fonction \verb|neper| qui prend en entrée un entier \C{n} et
  retourne la valeur de la somme suivante :
\[
1 + \sum^{k = n}_{k = 1} \frac{1}{k!}.
\]
Vous pouvez faire appel à une
  fonction \verb|int factorielle(int n)| calculant la factorielle de
  son argument.\bareme{1}
\item Une fonction \verb|saisie_utilisateur| sans paramètre qui
  demande un entier à l'utilisateur et retourne cet entier.
\item Une procédure \verb|tester_mes_fonctions| sans paramètre, qui
  appelle chacune des fonctions et procédures précédentes sur des
  paramètres de votre choix.
\end{enumerate}


\begin{correction}
Déclarations :
\begin{verbatim}
double valeur_absolue(double x);
void afficher_ligne(int n, char c);
double neper(int n);
\end{verbatim}

Définitions :
\begin{verbatim}
double valeur_absolue(double x)
{
  if (x < 0)
  {
    return -x;
  }
  return x;
}

void afficher_ligne(int n, char c)
{
  int i;
  for (i = 0; i < n; i = i + 1)
  {
    printf("%c", c);
  }
  printf("\n"); /* optionnel */
}

double neper(int n) /* récursive */
{
  if (n > 1)
  {
    return neper(n - 1) + 1.0/factorielle(n);
  }
  return 1.0;
}
/* alternative plus classique */
double neper_it(int n) 
{
  double somme = 1.0;
  int i;
  for (i = 1; i <= n; i = i + 1)
  {
    somme = somme + 1.0/factorielle(i);
  }
  return somme;
}

int saisie_utilisateur() {
  int res = 0;
  printf("saisir un entier\n");
  scanf("%d", &res);
  return res;
}


void tester_mes_fonctions() {/* au moins un appel à chaque fonction */
  printf("Test de la fonction valeur_absolue\n");
  printf("|%lg| = %lg\n", -3.16, valeur_absolue(-3.16));
  printf("|%lg| = %lg\n", 0.16, valeur_absolue(0.16));
  printf("|%lg| = %lg\n", 0, valeur_absolue(0));
  printf("Test de la fonction afficher_ligne\n");
  afficher_ligne(1, 'a');
  afficher_ligne(2, 'B');
  afficher_ligne(3, 'c');
  afficher_ligne(3, '.');
  afficher_ligne(26, 'z');
  printf("Test de la fonction neper()\n");
  printf("neper(%d) = %lg\n", 1, neper(1));
  printf("neper(%d) = %lg\n", 2, neper(2));
  printf("neper(%d) = %lg\n", 10, neper(10));
  printf("neper(%d) = %lg\n", 1000, neper(1000));
  printf("neper(%d) = %lg\n", 100000, neper(100000));
  printf("Vous avez saisi %d\n", saisie_utilisateur()); /* <-- expliquer */
}
\end{verbatim}

  \begin{baremeenv}
    On note \pt{0,5} par déclaration exacte et \pt{0,5} par définition
    correcte (en dehors de la question de typage) en admettant comme
    erreurs l'oublie de l'utilisation de \C{\%c} pour l'affichage d'un
    caractère, une erreur d'indice à un près dans les boucles, un
    problème d'arrondi entier.

   On ajoute \pt{0,5} si il est fait correctement appel à  la fonction factorielle,
   tant que la syntaxe d'appel est correct et quelle que soit les
   paramètres et l'usage de cet appel.
  \end{baremeenv}
\end{correction}

\section*{Bonus}
\paragraph{Question bonus.} Au choix.  Déclarer et définir une
fonction qui calcule le pgcd de deux entiers positifs ou nuls. Ou
bien, déclarer et définir une procédure \verb|afficher_disque| qui
prend en paramètre un entier \C{rayon} et affiche un disque d'étoiles
de ce rayon.\bareme{2}


\begin{correction}
Voici trois corrigés, le pgcd récursif, un pgcd itératif (avec
boucles au lieu de récursion), la procédure
\C{afficher\_disque}.
Déclarations :
\begin{verbatim}
int pgcd_r(int a, int b);
int pgcd_i(int a, int b);
void afficher_disque(int rayon);
\end{verbatim}

Définitions :
\begin{verbatim}
int pgcd_r(int a, int b)
{
    if (a < b)
    {
	return pgcd(b, a);
    }
    if (b == 0)
    {
	return a;
    }
    return pgcd(b,  a % b);
}

int pgcd_i(int a, int b)
{
    int i;
    int min;
    int d;
    /* cas particuliers */
    if (a == 0)
    {
	return b;
    }
    if (b == 0)
    {
	return a;
    }
    /* min = minimum(a, b) */
    if (a < b)
    {
	min = a;
    }
    else
    {
	min = b;
    }
    /* diviseurs */
    for (i = 1; i <= min; i = i + 1)
    {
	/* i diviseur commun ? */
	if ( (0 == a % i) && (0 == b % i) )
	{
	    d = i;
	}
    } /* d plus grand diviseur commun */
    return d;
}

void afficher_disque(int rayon)
{
    /* on balaie le carré contenant le disque, de coté = 2 rayons + 1,
       et on affiche une étoile si la distance au centre est
       inférieure au rayon, sinon un blanc */
    int i;
    int j;
    for (i = -rayon; i <= rayon; i = i + 1)
    {
	for (j = -rayon; j <= rayon; j = j + 1)
	{
	    if ( (i*i+j*j) <= rayon*rayon )
	    {
		printf("*");
	    }
	    else
	    {
		printf(" ");
	    }
	}
	printf("\n");
    }	
}
\end{verbatim}
\end{correction}

%%% Local Variables: 
%%% mode: latex
%%% TeX-master: "td7"
%%% End: 