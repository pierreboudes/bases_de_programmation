% -*- coding: utf-8 -*-

\newcommand{\commentaire}[1]{}

\entete{Travaux dirigés 7 : pour aller plus loin}



\section{PGCD (récursivité)}
Rappel : pour $a$ et $d$ deux entiers naturels, on dit que $d$ divise $a$, et on note $d\mid a$, s'il existe un entier $q$ tel que $a = dq$. En particulier $d\mid 0$ quel que soit $d$. Le plus grand diviseur commun (PGCD) de deux entiers naturels $a$ et $b$ est le plus grand entier $d$ tel que $d\mid a$ et $d\mid b$.  Cette définition donne un moyen effectif de trouver le PGCD de deux entiers (un algorithme) : il suffit de chercher tous les diviseurs de $a$ et de $b$ et de prendre le plus grand d'entre eux.

\begin{newenu}
\item Que vaut $\operatorname{PGCD}(0, 0)$ ?
  \begin{correction}
    $\operatorname{PGCD}(0, 0)$ est mal défini car il n'existe pas de plus grand entier naturel.
  \end{correction}
\item Si seulement un des entiers $a$ et $b$ est nul, que vaut $\operatorname{PGCD}(a, b)$ ?
  \begin{correction}
    L'autre entier. C'est à dire que  $\operatorname{PGCD}(0, x) =  \operatorname{PGCD}(x, 0) = x$.
  \end{correction}
\item Si aucun des deux entiers $a$ et $b$ n'est nul, jusqu'a quelle valeur doit-on chercher des diviseurs de $a$ et de $b$ ?

  \begin{correction}
    Lorsque $x$ est non nul un diviseur de $x$ est nécessairement plus petit que $x$.  On peut donc certainement s'arrêter au minimum de $a$ et $b$.
  \end{correction}
\item Comment tester si un nombre en divise un autre, en C ?
  \begin{correction}
Il faut les inciter à formuler leur réponse complétement, puis à essayer de l'écrire en C, même si ça n'est pas optimal.

Une réponse possible.
    Si $a, d > 0$, pour tester si $d \mid a$ on peut multiplier tour à tour $d$ par tous les entiers en commençant à $1$ tant que le résultat est strictement inférieur $a$. Lorsqu'on s'arrête soit on a trouvé $a$ comme résultat de la multiplication et dans ce cas $d$ divise $a$ soit on a trouvé un nombre strictement supérieur à $a$ et $d$ ne divise pas $a$. (On peut aussi faire des additions successives de $d$ dans un accumulateur).

\begin{verbatim}
int divise(int d, int a)
{
    int q = 1;
    if (a == 0) /* cas dégénéré */
    {
        return true;
    }
    if (d == 0) /* autre cas dégénéré */
    {
        return false;
    }
    /* cas général (ici on en fait trop) */
    while (q * d < a)
    {
         q = q + 1;
    }
    /* q est le premier entier tel que q * d >= a */
    if (q * d == a)
    {
        return true;
    }
    return false; /* on peut aussi leur montrer return q * d == a. */
}
\end{verbatim}

Autre solution plus économe : tester si $a \mod d$ vaut zéro (\verb+a % d == 0+ ou bien \verb+a == d *(a/d)+).

 \end{correction}
\item Écrire une fonction non récursive qui calcule le PGCD de deux entiers naturels.

  \begin{correction}
    \begin{verbatim}
int pgcd(int a, int b)
{
    int min;     /* minimum entre a et b */
    int d = 1;   /* var. de boucle */
    int res = 0; /* resultat */

    if (a == 0) /* premier cas dégénéré */
    {
        return b;
    }
    if (b == 0) /* second cas dégénéré */
    {
        return a;
    }
    /* a > 0 et b > 0*/

    /* calcul de minimum(a, b) */
    if (a > b)
    {
        min = b;
    }
    else
    {
        min = a;
    }
    /* min est le minimum entre a et b */

    for (d = 1; d < min; d = d + 1)
    {
         if ( (a % d == 0) && (b % d == 0) ) /* d | a et d | b */
         {
              res = d;
         }
    }
    /* res est le plus grand d tq d | a et d | b */

    return res;
}


\end{verbatim}
\end{correction}

\item Rappeler l'algorithme d'Euclide pour le calcul du PGCD.
  \begin{correction}
    \begin{quote}
  << Étant donnés deux entiers naturels $a$ et $b$, on
commence par tester si b est nul. Si oui, alors le PGCD est égal à
$a$. Sinon, on calcule $c$, le reste de la division de $a$ par $b$. On
remplace $a$ par $b$, et $b$ par $c$, et on recommence le procédé. Le
dernier reste non nul est le PGCD. >> (wikipedia.fr, dans le passé).
\end{quote}

Autrement dit, lorsque $a \geq b$ on utilise la division euclidienne, $a = bq + r$ et on recommence avec le couple $(b, r)$ jusqu'à tomber sur un reste nul. C'est justifié car si $d \mid a$ et $d \mid b$  alors $d \mid r$ (et $d \mid b$) et réciproquement si $d \mid b$ et $d \mid r$ alors $d \mid a = bq  + r$ (et $d\mid b$).

En résumé le PGCD de deux entiers positifs $a$ et $b$
est défini par : $\operatorname{PGCD}(a, b) = a$ si $ b = 0$ et
$\operatorname{PGCD}(a, b) = \operatorname{PGCD}(b, a \mod b)$.

 \end{correction}
\item  Écrire une fonction récursive pour le calcul du PGCD de deux
  entiers.
  \begin{correction}
\begin{verbatim}
int pgcd(int a, int b)
{
    if (b == 0)
    {
        return a;
    }
    return pgcd(b, a % b);
}
\end{verbatim}
Remarque : si l'entier le plus grand est $b$, la fonction échange ses arguments au premier appel récursif.
  \end{correction}
\item À votre avis quelle est la méthode la plus rapide ?
  \begin{correction}
La fonction récursive sera toujours au moins aussi rapide que la fonction itérative. On peut par exemple remarquer que si les deux entiers sont égaux dans la fonction récursive le deuxième appel se fait avec zéro comme second argument et renvoie $a$ immédiatemment tandis que l'algortihme précédent va dérouler la boucle jusqu'à $a$.
  \end{correction}
\end{newenu}


\section{Les \C{struct} ne sont pas si complexes et point difficiles}

\begin{lastenu}
\item Proposer un type utilisateur pour les nombres complexes en
  notation algébrique ($a + \mathrm{i} b$, $a,b\in\mathbb{R}$).

  \begin{correction}
\begin{verbatim}
struct complexe_s
{
    double re; /* partie reelle */
    double im; /* partie imaginaire */
}
\end{verbatim}
  \end{correction}

\item Donner une fonction d'addition et une fonction de multiplication
  entre ces nombres complexes (les tester).

  \begin{correction}
\begin{verbatim}
struct complexe_s addition_complexe(struct complexe_s z1, struct complexe_s z2)
{
    z1.re = z1.re + z2.re;
    z1.im = z1.im + z2.im;
    return z1;
}

struct complexe_s multiplication_complexe(struct complexe_s z1, struct complexe_s z2)
{
    struct complexe_s res;
    res.re = z1.re * z2.re - z1.im * z2.im;
    res.im = z2.re * z1.im + z1.re * z2.im;
    return res;
}
\end{verbatim}
  \end{correction}
\end{lastenu}

%\subsection{Point}

\begin{lastenu}
\item Proposer un type utilisateur pour les points de l'espace
  en coordonnées réelles : $\mathbb{R}^3$.
  \begin{correction}
\begin{verbatim}
struct point_s
{
    double x; /* abscisses */
    double y; /* ordonnees */
    double z; /* hauteur */
}
\end{verbatim}
\end{correction}

\item Écrire une fonction \C{est\_dans\_sphere} prenant en argument un
  point $c$, un réel $r$, et un point $p$, qui
  renvoie vrai si le point $p$ appartient à la sphère de centre $c$ et
  de rayon $r$, faux sinon.
  \begin{correction}
\begin{verbatim}
est_dans_sphere(struct point_s c, double r, struct point_s p)
{
    return ((p.x - c.x)*(p.x - c.x)
          + (p.y - c.y)*(p.y - c.y)
          + (p.z - c.z)*(p.z - c.z))
         <= r*r;
}
\end{verbatim}

\end{correction}
\item Définir une fonction \C{distance} qui retourne la distance entre
  deux points de l'espace passés en arguments. Simplifier la fonction
  \verb|est_dans_sphere| en faisant appel à cette fonction distance.
\begin{correction}
\begin{verbatim}
#include <math.h> /* pour sqrt() */

double distance(struct point_s p1, struct point_s p2)
{
   return sqrt((p1.x - p2.x)*(p1.x - p2.x)
             + (p1.y - p2.y)*(p1.y - p2.y)
             + (p1.z - p2.z)*(p1.z - p2.z));
}


est_dans_sphere(struct point_s c, double r, struct point_s p)
{
    return distance(c, p) <= r;
}
\end{verbatim}
\end{correction}

\item Proposer un type utilisateur pour représenter une sphère dans
  l'espace.
  \begin{correction}
Une sphère est donnée par son centre (un point) et son rayon. Le type serait donc le suivant :
\begin{verbatim}
struct sphere_s
{
    struct point_s centre; /* centre */
    double          rayon; /* rayon  */
}
\end{verbatim}
\end{correction}
\item Définir une fonction \verb+collision_spheres+ qui prend en
  entrée deux sphères et retourne \C{true} si les deux sphères ont une
  intersection non vide, \C{false} sinon.
  \begin{correction}
\begin{verbatim}
int collision_spheres(struct sphere_s s1, struct sphere_s s2)
{
    return distance(s1.centre, s2.centre) <= (s1.rayon + s2.rayon);
}
\end{verbatim}
\end{correction}
\end{lastenu}


\section{Le jeu de la tablette de chocolat}

Une tablette (ou plaquette) de chocolat est une matrice de carreaux de
chocolat que l'ont peut couper selon les lignes ou les colonnes qui
séparent les carreaux. La hauteur et la largeur se comptent en nombre
de carreaux. On a peint en vert un carreau dans le coin d'une
tablette. Deux joueurs coupent tour à tour la tablette, au choix,
selon une ligne ou une colonne (pas nécessairement près du
bord). L'objectif est de ne pas être le joueur qui se retrouve avec le
carreau vert (la tablette de hauteur 1 et de largeur 1).

Nous allons programmer ce jeu en faisant jouer à l'ordinateur le rôle
de l'opposant.

\subsection{Type utilisateur (optionnel)}

Définir un type utilisateur qui servira à représenter la tablette
(faire simple !).

\begin{correction}
une tablette = paire d'entiers positifs (ses dimensions).
Soit ils programment fonctionnel et il leur faut retourner une paire
et donc utiliser un struct soient ils programment argument-résultat et
ils utilisent deux entiers et deux pointeurs sur ces entiers en entrée
des fonctions de jeu. Parler des deux !
\begin{verbatim}
struct tablette_s
{
  int l; /* largeur (nombre de colonnes) */
  int c; /* hauteur (nombre de lignes) */
}
\end{verbatim}
\end{correction}

\subsection{Le tour de jeu et l'arbitre}
Le joueur commencera la partie, puis l'opposant (l'ordinateur) jouera
et ainsi de suite. À chaque fois l'une des deux dimensions de la tablette sera
modifiée, jusqu'à ce qu'il y ait un perdant. Vous adopterez la
structure suivante pour le programme :
\begin{itemize}
\item Une variable sera utilisée pour déterminer à qui vient le tour de
  jouer (joueur ou opposant);
\item Une boucle réalisera le tour de jeu;
\item Une fonction \C{partie\_perdue()} prenant en argument la tablette
  et renvoyant vrai lorsque celle-ci se réduit au carré vert, servira
  à construire la condition de sortie de la boucle;
\item Après quoi un affichage annoncera au joueur s'il a gagné ou
  perdu.
\item À l'intérieur de la boucle, selon le tour, il sera fait appel à
  une fonction \C{joueur()} ou bien à une fonction \C{opposant()} (ces
  fonctions doivent modifier la tablette).
\end{itemize}

Vous êtes libre du choix de la taille initiale de la tablette.

\subsection{Affichage}

Réaliser une fonction d'affichage de la tablette qui sera appelée au
début de chaque tour de jeu. Vous pouvez dans un premier temps vous
contenter d'un affichage numérique de ses dimensions, et par la suite
améliorer votre programme avec un affichage plus graphique.

\subsection{Les joueurs}

Le joueur étant l'humain entre le clavier et l'écran, il faudra lui
faire saisir le choix de son coup, en deux temps : couper des colonnes
ou couper des lignes ? Combien ? À vous de déterminer la manière de
réaliser cette saisie.

Vous pouvez commencer à tester le fonctionnement de votre programme en
utilisant la fonction \C{joueur()} également pour l'opposant.

L'opposant jouera au hasard. Pour cela vous pouvez vous aider d'une
fonction \C{nombre\_aleatoire} qui tire au hasard un nombre entre
zéro et son argument (un entier positif). Cette fonction utilisera
\C{rand}. N'oubliez pas d'initialiser
le générateur de nombres aléatoires dans le \C{main()} ni d'inclure
les bons fichiers d'en-tête (fonctionnalités supplémentaires).

Tester votre programme. Avez-vous prévu les cas où l'un des deux
joueurs n'a pas le choix ? Est-ce que les coups invalides sont rendus
impossibles ? Vous pouvez réutiliser et améliorer des fonctions vues
précédemment (\C{choix\_utilisateur()}, par exemple).

\section{Améliorations}

\subsection{Tablette aléatoire}
Dans votre programme la tablette est initialisée au
départ. Pouvez-vous utiliser une fonction sans argument,
\C{tablette\_aleatoire} qui renvoie une tablette aléatoire ?

\subsection{L'opposant parfait}
Pouvez-vous trouver la meilleure manière de jouer (cela dépend de la
tablette de départ) ? Si oui, faîtes en profiter votre opposant. Vous
pouvez également le faire se tromper de temps en temps, en introduisant
pour cela une dose de hasard...

\subsection{Plusieurs tablettes}
On peut utiliser plusieurs tablettes pour ce jeu : chaque joueur
choisit sur quelle tablette jouer son tour et le perdant sera celui
recevant pour son tour toutes les tablettes à la dimension $1\times
1$.

\begin{correction}
{
\footnotesize
\listinginput{1}{chocolat.c}
}
\end{correction}



%%% Local Variables:
%%% mode: latex
%%% TeX-master: "td7"
%%% End: