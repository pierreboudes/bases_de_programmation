\begin{tabular}[c]{l||c|c|c|c|c|c|c|c|c|c|c|c|}
\hline
 \emph{Instructions} & Cycles & CP& r0& r1& r2& r4& r5& 30& 31& 32& 33& 34\\ \hline
\hfill Initialisation & 0 & 1 & ? & ? & ? & ? & ? & 3
 & 100
 & 1
 & 10
 & ?
 \\ \hline \commentaire{Initialisation du registre 0 à 0
} \C{valeur 0 r0
} & 1 & 2  & 0 & & & & & & & & &\\ \hline
 \commentaire{Initialisation du registre 2 à 1
} \C{valeur 1 r2
} & 2 & 3  & & & 1 & & & & & & &\\ \hline
 \commentaire{Saut à l'adresse 9
} \C{saut 9
} & 3 & \textbf{9} & & & & & & & & & &\\ \hline
 \commentaire{Lecture de la donnée d'adresse 30 dans le registre 1
} \C{lecture 30 r1
} & 4 & 10  & & 3 & & & & & & & &\\ \hline
 \commentaire{Soustrait la valeur du registre 2 au registre 1
} \C{soustr r2 r1
} & 5 & 11  & & 2 & & & & & & & &\\ \hline
 \commentaire{Si la valeur (0) du registre 3 est positive, saute à l'adresse 4
} \C{sautpos r3 4
} & 6 & \textbf{4} & & & & & & & & & &\\ \hline
 \commentaire{Initialisation du registre 4 à 30
} \C{valeur 30 r4
} & 7 & 5  & & & & 30 & & & & & &\\ \hline
 \commentaire{Ajout de la valeur du registre 2 au registre 4
} \C{add r2 r4
} & 8 & 6  & & & & 31 & & & & & &\\ \hline
 \commentaire{Lecture de la donnée d'adresse 31 dans le registre 5
} \C{lecture *r4 r5
} & 9 & 7  & & & & & 100 & & & & &\\ \hline
 \commentaire{Ajout de la valeur du registre 5 au registre 0
} \C{add r5 r0
} & 10 & 8  & 100 & & & & & & & & &\\ \hline
 \commentaire{Ajout de la valeur 1 au registre 2
} \C{add 1 r2
} & 11 & 9  & & & 2 & & & & & & &\\ \hline
 \commentaire{Lecture de la donnée d'adresse 30 dans le registre 1
} \C{lecture 30 r1
} & 12 & 10  & & 3 & & & & & & & &\\ \hline
 \commentaire{Soustrait la valeur du registre 2 au registre 1
} \C{soustr r2 r1
} & 13 & 11  & & 1 & & & & & & & &\\ \hline
 \commentaire{Si la valeur (0) du registre 3 est positive, saute à l'adresse 4
} \C{sautpos r3 4
} & 14 & \textbf{4} & & & & & & & & & &\\ \hline
 \commentaire{Initialisation du registre 4 à 30
} \C{valeur 30 r4
} & 15 & 5  & & & & 30 & & & & & &\\ \hline
 \commentaire{Ajout de la valeur du registre 2 au registre 4
} \C{add r2 r4
} & 16 & 6  & & & & 32 & & & & & &\\ \hline
 \commentaire{Lecture de la donnée d'adresse 32 dans le registre 5
} \C{lecture *r4 r5
} & 17 & 7  & & & & & 1 & & & & &\\ \hline
 \commentaire{Ajout de la valeur du registre 5 au registre 0
} \C{add r5 r0
} & 18 & 8  & 101 & & & & & & & & &\\ \hline
 \commentaire{Ajout de la valeur 1 au registre 2
} \C{add 1 r2
} & 19 & 9  & & & 3 & & & & & & &\\ \hline
 \commentaire{Lecture de la donnée d'adresse 30 dans le registre 1
} \C{lecture 30 r1
} & 20 & 10  & & 3 & & & & & & & &\\ \hline
 \commentaire{Soustrait la valeur du registre 2 au registre 1
} \C{soustr r2 r1
} & 21 & 11  & & 0 & & & & & & & &\\ \hline
 \commentaire{Si la valeur (0) du registre 3 est positive, saute à l'adresse 4
} \C{sautpos r3 4
} & 22 & \textbf{4} & & & & & & & & & &\\ \hline
 \commentaire{Initialisation du registre 4 à 30
} \C{valeur 30 r4
} & 23 & 5  & & & & 30 & & & & & &\\ \hline
 \commentaire{Ajout de la valeur du registre 2 au registre 4
} \C{add r2 r4
} & 24 & 6  & & & & 33 & & & & & &\\ \hline
 \commentaire{Lecture de la donnée d'adresse 33 dans le registre 5
} \C{lecture *r4 r5
} & 25 & 7  & & & & & 10 & & & & &\\ \hline
 \commentaire{Ajout de la valeur du registre 5 au registre 0
} \C{add r5 r0
} & 26 & 8  & 111 & & & & & & & & &\\ \hline
 \commentaire{Ajout de la valeur 1 au registre 2
} \C{add 1 r2
} & 27 & 9  & & & 4 & & & & & & &\\ \hline
 \commentaire{Lecture de la donnée d'adresse 30 dans le registre 1
} \C{lecture 30 r1
} & 28 & 10  & & 3 & & & & & & & &\\ \hline
 \commentaire{Soustrait la valeur du registre 2 au registre 1
} \C{soustr r2 r1
} & 29 & 11  & & -1 & & & & & & & &\\ \hline
 \commentaire{Si la valeur (0) du registre 3 est positive, saute à l'adresse 4
} \C{sautpos r3 4
} & 30 & \textbf{4} & & & & & & & & & &\\ \hline
 \commentaire{Initialisation du registre 4 à 30
} \C{valeur 30 r4
} & 31 & 5  & & & & 30 & & & & & &\\ \hline
 \commentaire{Ajout de la valeur du registre 2 au registre 4
} \C{add r2 r4
} & 32 & 6  & & & & 34 & & & & & &\\ \hline
 \commentaire{add r2 r4
} \C{lecture *r4 r5
} & 33 & 7  & & & & 34 & & & & & &\\ \hline
\end{tabular}