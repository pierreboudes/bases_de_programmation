\begin{tabular}[c]{lp{11.3cm}}
  \C{stop} & Arrête l'exécution du programme.\\
  \C{noop} & N'effectue aucune opération.\\
  \C{saut i} & Met le compteur de programme à la valeur $i$.\\
  \C{sautpos ri j} & Si la valeur contenue dans le registre $i$ est positive ou nulle, met le compteur de programme à la valeur $j$.\\
  \C{valeur x ri} & Initialise le registre $i$ avec la valeur $x$.\\
  \C{lecture i rj} & Charge, dans le registre $j$, le contenu de la mémoire d'adresse $i$.\\
  \C{ecriture ri j} & Écrit le contenu du registre $i$ dans la mémoire d'adresse $j$.\\
  \C{inverse ri} & Inverse le signe du contenu du registre $i$.\\
  \C{add ri rj} & Ajoute la valeur du registre $i$ à celle du registre $j$ (la somme obtenue est placée dans le registre $j$).\\
  \C{soustr ri rj} & Soustrait la valeur du registre $i$ à celle du registre $j$ (la différence obtenue est placée dans le registre $j$).\\
  \C{mult ri rj} & Multiplie par la valeur du registre $i$ celle du registre $j$ (le produit obtenu est placé dans le registre $j$).\\
  \C{div ri rj} & Divise par la valeur du registre $i$ celle du registre $j$ (le quotient obtenu, arrondi à la valeur entière inférieure, est placé dans le registre $j$).\\
  \multicolumn{2}{l}{\textbf{Instructions plus avancées}}\\
  \C{et ri rj} & Effectue le et bit à bit de la valeur du registre $i$ et de celle du registre $j$. Le résultat est placé dans le registre $j$.\\
  \C{lecture *ri rj} & Charge, dans le registre $j$, le contenu de la mémoire dont l'adresse est la valeur du registre $i$\\
  \C{ecriture ri *rj} & Écrit le contenu du registre $i$ dans la mémoire dont  l'adresse est la valeur du registre $j$.
\end{tabular}
