% -*- coding: utf-8 -*-

% \renewcommand{\labelitemi}{$\bullet$}

% \newcommand{\commentaire}[1]{}

\entete{Travaux pratiques 4 : la structure de contrôle ``for''}

\vspace{-1em}L'objectif de ce TP est de vous familiariser avec les notions d'itération et de boucles imbriquées.

\begin{correction}
  Il y a deux exos type qu'il faut corriger au tableau.
  Le dernier exo du TD4 (Affichage de n fois ``Bonjour'') peut être fait en intro si pas fini pendant le TD.
\end{correction}

\newcommand{\TPshortName}{TP4}
% -*- coding: utf-8 -*-

Vous allez mettre tous vos programmes écrits dans ce TP dans le
répertoire \TPshortName. 

\begin{newenu}
\item À partir du début de votre arborescence, créez le répertoire
\TPshortName : \C{mkdir \TPshortName} 
\item Allez dans ce répertoire pour y mettre des fichiers : 
  \C{cd \TPshortName}
\end{newenu}

L'étape suivante est à répéter pour chaque nouveau programme (exo1, exo2 etc..) :
\begin{lastenu}
\item Créez un nouveau fichier source pour le langage C ou une nouvelle copie d'un programme existant.
  \begin{description}
  \item[Création]   \verb|gedit exo1.c &| (vous pouvez utiliser
    \C{emacs} ou \C{kwrite} au lieu de \C{gedit})
\item[Copie] Il est plus rapide de repartir d'une copie de votre programme \C{bonjour.c} du TP2 pour éviter de retaper tout le squelette. Dans le terminal : \\ \C{cp ../TP2/bonjour.c exo1.c} \\ \verb|gedit exo1.c &|\\ Vous pouvez-aussi ouvrir \verb+bonjour.c+ et utiliser la fonction \emph{Enregistrer sous...} de votre éditeur mais attention à enregistrer la nouvelle copie dans le bon répertoire. 
\end{description}
\end{lastenu}

Vous pouvez utiliser à tout moment la commande \C{ls} (list directory)
pour voir la liste des fichiers d'un répertoire. 


Les trois étapes suivantes seront à répéter autant de fois que nécessaire pour la mise au point de chaque programme (apprenez à utiliser les raccourcis clavier). 
\begin{lastenu}
\item Après avoir fini d'écrire votre programme, enregistrez le.
\item Créez un programme exécutable à partir de votre fichier source :\\
  \verb|gcc -Wall exo1.c -o exo1.exe|
\item Quand l'étape précédente a réussi (il faut lire attentivement les
  messages affichés), exécutez le programme pour
  vérifier qu'il fonctionne : \verb|exo1.exe| (ou
  \verb|./exo1.exe|).
\end{lastenu}


\section{Affichage de figures géométriques}

Les exercices suivants utilisent le caractère \verb|*| (étoile) pour dessiner des figures géométriques simples, appelées figures d'étoiles.

\subsection{Exercice type : affichage d'un rectangle d'étoiles}

Écrire un programme qui, étant données deux variables,
\verb|longueur| et \verb|largeur|, initialisées à des valeurs
strictement positives quelconques,affiche  un rectangle d'étoiles ayant pour
longueur \verb|longueur| étoiles et largeur \verb|largeur|
étoiles. Deux exemples d'exécution, avec deux initialisations
différentes, sont les suivants :
\begin{verbatim}
Affichage d'un rectangle d'etoiles de longueur 10 et largeur 5.
**********
**********
**********
**********
**********

Affichage d'un rectangle d'etoiles de longueur 6 et largeur 3.
******
******
******
\end{verbatim}

\begin{correction}
Durée 3/4 d'heure ?

Les algos sont à faire (les extraire du code).

Vous pouvez dans un premier temps supprimer la boucle la plus imbriquée en leur demandant d'afficher un rectangle de longueur exactement ``************''.

\begin{verbatim}
/* declaration de fonctionnalites supplementaires */
#include <stdlib.h> /* EXIT_SUCCESS */
#include <stdio.h> /* printf */

/* declaration constantes et types utilisateurs */

/* declaration de fonctions utilisateurs */

/* fonction principale */
int main()
{
    /* declaration et initialisation variables */
    int largeur = 3; /* largeur du rectangle en nb d'etoiles */
    int longueur = 6; /* longueur du rectangle en nb d'etoiles */
    int i; /* var. de boucle */
    int j; /* var. de boucle */

    printf("Affichage d'un rectangle d'etoiles de longueur %d et largeur %d.\n",longueur,largeur);

    for(i = 0;i < largeur;i = i + 1) /* chaque ligne d'étoiles */
    {
        /* affiche longueur etoiles */
        for(j = 0;j < longueur;j = j + 1) /* chaque colonne d'etoiles */
        {
            /* affiche une etoile */
            printf("*");
        }
        /* j >= longueur */

        /* passe a la ligne suivante */
        printf("\n");
    }
    /* i >= largeur */

    return EXIT_SUCCESS;
}

/* definitions des fonctions utilisateurs */
\end{verbatim}
\end{correction}

\subsection{Exercice type : affichage d'un demi-carré d'étoiles}

Écrire un programme qui affiche, étant donnée la variable,
\verb|cote|, initialisée à une valeur quelconque, un demi-carré d'étoiles (triangle rectangle isocèle) ayant pour
longueur de côté \verb|cote| étoiles. Deux exemples d'exécution, avec deux initialisations
différentes, sont les suivants :
\begin{verbatim}
Affichage d'un demi-carre d'etoiles de cote 6.
*
**
***
****
*****
******

Affichage d'un demi-carre d'etoiles de cote 2.
*
**
\end{verbatim}

\begin{correction}
Durée 3/4 d'heure ?

Les algos sont à faire (les extraire du code).

\begin{verbatim}
/* declaration de fonctionnalites supplementaires */
#include <stdlib.h> /* EXIT_SUCCESS */
#include <stdio.h> /* printf */

/* declaration constantes et types utilisateurs */

/* declaration de fonctions utilisateurs */

/* fonction principale */
int main()
{
    /* declaration et initialisation variables */
    int cote = 2; /* cote du demi-carré en nb d'etoiles */
    int i; /* var. de boucle */
    int j; /* var. de boucle */

    printf("Affichage d'un demi-carre d'etoiles de cote %d.\n",cote);

    for(i = 1;i <= cote;i = i + 1) /* chaque numero de ligne d'étoiles */
    {
        /* affiche autant d'etoiles que le numero de ligne */
        for(j = 0;j < i;j = j + 1) /* chaque colonne d'etoiles */
        {
            /* affiche une etoile */
            printf("*");
        }
        /* j >= i */

        /* passe a la ligne suivante */
        printf("\n");
    }
    /* i > cote */

    return EXIT_SUCCESS;
}

/* definitions des fonctions utilisateurs */
\end{verbatim}
\end{correction}


\subsection{Affichage d'un demi-carré droit d'étoiles (optionnel)}


Écrire un programme qui affiche un demi-carré droit d'étoiles de côté spécifié par l'utilisateur. Exemple d'exécution :
\begin{small}
\begin{verbatim}
Entrer la taille du demi-carré :
5
Affichage d'un demi-carre droit d'etoiles de cote 5.
    *
   **
  ***
 ****
*****
\end{verbatim}
\end{small}

\begin{correction}
\begin{verbatim}
/* declaration de fonctionnalites supplementaires */
#include <stdlib.h> /* EXIT_SUCCESS */
#include <stdio.h> /* printf, scanf */

/* declaration constantes et types utilisateurs */

/* declaration de fonctions utilisateurs */

/* fonction principale */
int main()
{
    int cote; /* cote du demi-carré droit en nb d'etoiles a saisir par l'utilisateur*/
    int i; /* var. de boucle */
    int j; /* var. de boucle */

    /* saisie cote */
    printf("Entrer la taille du demi-carré :\n");
    scanf("%d",&cote);

    /* affichage du demi-carre droit */
    printf("Affichage d'un demi-carre droit d'etoiles de cote %d.\n",cote);

    for(i = 1;i <= cote;i = i + 1) /* chaque numero de ligne d'étoiles */
    {
        /* affiche les blancs */
        for(j = 0;j < cote - i;j = j + 1) /* chaque colonne de blancs */
        {
            /* affiche un blanc */
            printf(" ");
        }
        /* j >= cote - i */

        /* affiche autant d'etoiles que le numero de ligne */
        for(j = 0;j < i;j = j + 1) /* chaque colonne d'etoiles */
        {
            /* affiche une etoile */
            printf("*");
        }
        /* j >= i */

        /* passe a la ligne suivante */
        printf("\n");
    }
    /* i > cote */

    return EXIT_SUCCESS;
}

/* definition de fonctions utilisateurs */

\end{verbatim}
\end{correction}

%%% Local Variables: 
%%% mode: latex
%%% TeX-master: "tp4"
%%% End: 