% -*- coding: utf-8 -*-

Vous allez mettre tous vos programmes écrits dans ce TP dans le
répertoire \TPshortName. 

\begin{newenu}
\item À partir du début de votre arborescence, créez le répertoire
\TPshortName : \C{mkdir \TPshortName} 
\item Allez dans ce répertoire pour y mettre des fichiers : 
  \C{cd \TPshortName}
\end{newenu}

L'étape suivante est à répéter pour chaque nouveau programme (exo1, exo2 etc..) :
\begin{lastenu}
\item Créez un nouveau fichier source pour le langage C ou une nouvelle copie d'un programme existant.
  \begin{description}
  \item[Création]   \verb|gedit exo1.c &| (vous pouvez utiliser
    \C{emacs} ou \C{kwrite} au lieu de \C{gedit})
\item[Copie] Il est plus rapide de repartir d'une copie de votre programme \C{bonjour.c} du TP2 pour éviter de retaper tout le squelette. Dans le terminal : \\ \C{cp ../TP2/bonjour.c exo1.c} \\ \verb|gedit exo1.c &|\\ Vous pouvez-aussi ouvrir \verb+bonjour.c+ et utiliser la fonction \emph{Enregistrer sous...} de votre éditeur mais attention à enregistrer la nouvelle copie dans le bon répertoire. 
\end{description}
\end{lastenu}

Vous pouvez utiliser à tout moment la commande \C{ls} (list directory)
pour voir la liste des fichiers d'un répertoire. 


Les trois étapes suivantes seront à répéter autant de fois que nécessaire pour la mise au point de chaque programme (apprenez à utiliser les raccourcis clavier). 
\begin{lastenu}
\item Après avoir fini d'écrire votre programme, enregistrez le.
\item Créez un programme exécutable à partir de votre fichier source :\\
  \verb|gcc -Wall exo1.c -o exo1.exe|
\item Quand l'étape précédente a réussi (il faut lire attentivement les
  messages affichés), exécutez le programme pour
  vérifier qu'il fonctionne : \verb|exo1.exe| (ou
  \verb|./exo1.exe|).
\end{lastenu}
