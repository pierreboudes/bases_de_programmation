% -*- coding: utf-8 -*-

\renewcommand{\labelitemi}{$\bullet$}

\newcommand{\commentaire}[1]{}

\entete{Travaux pratiques 2 : premiers pas en langage C}

\vspace{-1em}
\begin{correction}
  C'est leur premiere manip. de shell pour utiliser gcc, lancer le
  prog etc. On donne les points à suivre mais il vaut mieux les leur
  expliquer.
\end{correction}

Vous allez mettre tous vos programmes écrits dans ce TP dans le
répertoire TP2~: 

\begin{newenu}
\item À partir du début de votre arborescence, créez le répertoire
  TP2~: \verb|mkdir TP2|  
\item Allez dans ce répertoire pour y mettre des fichiers : 
  \verb|cd TP2| 
\item Créez un nouveau fichier source pour le langage C : 
  \verb|gedit bonjour.c &|
\end{newenu}


Ce premier programme devra afficher \C{Bonjour !}, vous le composerez en recopiant le programme donné en exemple dans le TD, sauf les lignes 12 à 16. Gardez tous les commentaires vous en aurez besoin plus tard. Votre éditeur de texte vous assistera en colorisant automatiquement le code saisi. 

Pour réaliser l'affichage vous utiliserez l'instruction :\\
\verb+printf("Bonjour !\n");+, dans votre fonction principale. Le \C{printf} est une \emph{fonctionnalité supplémentaire} d'entrée sortie, pour qu'il fonctionne il faut  insérer la ligne suivante après la ligne 2 : \\
\verb+#include <stdio.h> /* pour printf */+

Le \verb+\n+ représente un saut de ligne. 

\begin{lastenu}
\item Après avoir fini d'écrire votre programme, enregistrez le.
\item Créez un programme exécutable à partir de votre fichier source, cette étape, dite de compilation, sera expliquée au prochain cours :
  \verb|gcc -Wall bonjour.c -o bonjour.exe|
\item Quand l'étape précédente a réussi, exécutez le programme pour
  vérifier qu'il fonctionne : \verb|bonjour.exe| (ou
  \verb|./bonjour.exe|).
\end{lastenu}

Vous répéterez ces trois dernières étapes (écrire/sauvegarder/compiler exécuter), très souvent ce semestre et le suivant. Exercez-vous à tout faire avec les raccourcis clavier \emph{sans utiliser la souris}.  


\begin{lastenu} 
\item Modifiez le programme de manière à ce qu'il affiche
  votre prénom après le <<Bonjour>> (compiler/exécuter).
\end{lastenu}


