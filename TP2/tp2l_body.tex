% -*- coding: utf-8 -*-

\renewcommand{\labelitemi}{$\bullet$}

\newcommand{\commentaire}[1]{}

\entete{Travaux pratiques 2 (version longue) : premiers pas en langage
  C et déboguage}

\vspace{-1em}
\begin{correction}
  C'est leur premiere manip. de shell pour utiliser gcc, lancer le
  prog etc. On donne les points à suivre mais il vaut mieux les leur
  expliquer.
\end{correction}

Vous allez mettre tous vos programmes écrits dans ce TP dans le
répertoire TP2~: 

\begin{newenu}
\item À partir du début de votre arborescence, créez le répertoire
  TP2~: \verb|mkdir TP2|  
\item Allez dans ce répertoire pour y mettre des fichiers : 
  \verb|cd TP2| 
\item Créez un nouveau fichier source pour le langage C : 
  \verb|kwrite bonjour.c &|
\end{newenu}


Ce premier programme devra afficher \C{Bonjour !}, vous le composerez en recopiant le programme donné en exemple dans le TD, sauf les lignes 12 à 16. Gardez tous les commentaires vous en aurez besoin plus tard. Votre éditeur de texte vous assistera en colorisant automatiquement le code saisi. 

Pour réaliser l'affichage vous utiliserez l'instruction : \verb+printf("Bonjour !\n");+, dans votre fonction principale. Le \C{printf} est une \emph{fonctionnalité supplémentaire} d'entrée sortie, pour qu'il fonctionne il faut  insérer la ligne suivante après la ligne 2 : 
\verb+#include <stdio.h> /* pour printf */+

Le \verb+\n+ représente un saut de ligne. 

\begin{lastenu}
\item Après avoir fini d'écrire votre programme, enregistrez le.
\item Créez un programme exécutable à partir de votre fichier source, cette étape, dite de compilation, sera expliquée au prochain cours :
  \verb|gcc -Wall bonjour.c -o bonjour.exe|
\item Quand l'étape précédente a réussi, exécutez le programme pour
  vérifier qu'il fonctionne : \verb|bonjour.exe| (ou
  \verb|./bonjour.exe|).
\end{lastenu}

Vous répéterez ces trois dernières étapes (écrire/sauvegarder/compiler exécuter), très souvent ce semestre et le suivant. Exercez-vous à tout faire avec les raccourcis clavier \emph{sans utiliser la souris}.  


\begin{lastenu} 
\item Modifiez le programme de manière à ce qu'il affiche
  votre prénom après le << Bonjour >> (compiler/exécuter).
\end{lastenu}

\section*{Partie optionnelle : déboguage}

Pour ausculter vos programmes pendant leur exécution et en trouver les bugs éventuels vous pourrez faire appel à un débogueur. En salles de TP nous utilisons DDD (data display debugger). S'il vous reste du temps, voici comment faire vos premiers pas sous DDD.

\begin{lastenu}
  \item En utilisant la fonction \emph{enregistrer sous\ldots} de votre éditeur de texte sauvegardez votre programme sous le nom \C{echange.c}, compilez le sous le nom \C{echange.exe} et exécutez le (\C{echange.exe} ou \C{./echange.exe}).
\item Déclarez une variable entière \C{x} dans votre programme et après le \C{printf}, affectez lui la valeur 10.
 \end{lastenu}

Pour que le débogueur fonctionne au mieux il faut ajouter des informations à l'exécutable, au moment de sa création, à l'aide de l'option \C{-g}. 
\begin{lastenu}
\item Recommencez la compilation mais en ajoutant l'option \C{-g} à la commande gcc :
  \verb|gcc -g -Wall echange.c -o echange.exe|
\item Ouvrez le débogueur sur ce nouvel exécutable :
\verb|ddd echange.exe|
\item Lancez une exécution complète de votre programme (bouton \emph{Run}). 
\item Les points d'arrêt (\emph{breakpoint}) permettent de suspendre temporairement l'exécution du programme. À l'aide d'un clic-droit poser un point d'arrêt, sur la ligne avant le \emph{printf}. Lancez l'exécution avec \emph{Run}. Lorsque le programme atteint le point d'arrêt, le débogueur suspend son exécution. Profitez en pour ouvrir la visualisation des variables locales (menu \emph{Data} puis \emph{Display local variables}). Progressez dans l'exécution du programme pas à pas à l'aide du bouton \emph{Step}. 
\item Quittez le débogueur, modifiez votre programme de manière à ce qu'il effectue l'échange des valeurs entre deux variables \C{x} et \C{y} que vous initialiserez à des valeurs de votre choix (comme en TD), et lancez l'exécution pas à pas de votre programme dans le débogueur, comme aux étapes précédentes. Cela fonctionne t'il correctement ?
\end{lastenu}
